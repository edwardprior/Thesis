\documentclass{amsart}
\setcounter{tocdepth}{3}
\setcounter{secnumdepth}{3}

\usepackage{amssymb}
\usepackage{amsmath}
\usepackage{amsthm}
\usepackage{eucal}
\usepackage{tikz-cd}
\usepackage{pdfsync}
\usepackage{relsize}
\usepackage{afterpage}

\usepackage[unicode=true, pdfusetitle,
 bookmarks=true,bookmarksnumbered=false,
 breaklinks=false,
 backref=false,
 colorlinks=true,
 %linkcolor=blue!70!black,
 citecolor=black,
 urlcolor=blue!78!red,
 final
]{hyperref}

\usepackage[capitalise]{cleveref}

\usepackage[color=orange!80,bordercolor=black,textwidth=3cm,textsize=small,colorinlistoftodos]{todonotes}
\makeatletter \providecommand\@dotsep{5}
\makeatother
\newcommand{\amsartlistoftodos}{\makeatother \listoftodos\relax}

\newenvironment{eq}{\begin{equation}}{\end{equation}}
\newenvironment{eq*}{\begin{equation*}}{\end{equation*}}

\numberwithin{equation}{section}

\newtheorem{thm}[equation]{Theorem}
\newtheorem{prop}[equation]{Proposition}
\newtheorem{lem}[equation]{Lemma}
\newtheorem{cor}[equation]{Corollary}

\newtheoremstyle{example}{\topsep}{\topsep}{}{}{\bfseries}{.}{2pt}{\thmname{#1}\thmnumber{ #2}\thmnote{ #3}}
\theoremstyle{example}

\newtheorem{nota}[equation]{Notation}
\newtheorem{example}[equation]{Example}
\newtheorem{defn}[equation]{Definition}
\newtheorem{rem}[equation]{Remark}
\newtheorem{comment}[equation]{Comment}

%% E.P. notes
\newcommand{\epnote}[1]{\todo[color=blue!40,linecolor=blue!40!black,size=\tiny]{#1}}
\newcommand{\epmpar}[1]{\todo[noline,color=blue!40,linecolor=blue!40!black,size=\tiny]{#1}}
\newcommand{\epnoteil}[1]{\todo[inline,color=blue!40,linecolor=blue!40!black,size=\normalsize]{#1}}

%% N.G. notes
\newcommand{\ngnote}[1]{\todo[color=red!40,linecolor=red!40!black,size=\tiny]{#1}}
\newcommand{\ngmpar}[1]{\todo[noline,color=red!40,linecolor=red!40!black,size=\tiny]{#1}}
\newcommand{\ngnoteil}[1]{\todo[inline,color=red!40,linecolor=red!40!black,size=\normalsize]{#1}}

% Cleveref definitions
\crefname{prop}{Proposition}{Propositions}
\crefname{thm}{Theorem}{Theorems}
\crefname{defn}{Definition}{Definitions}
\crefname{notn}{Notation}{Notations}
\crefname{construction}{Construction}{Constructions}
\crefname{lem}{Lemma}{Lemmas}
\crefname{rem}{Remark}{Remarks}
\crefname{cor}{Corollary}{Corollaries}
\crefname{scholium}{Scholium}{Scholia}
\crefname{figure}{Figure}{Figures}
\crefname{equation}{Display}{Displays}
\crefname{eq}{Display}{Displays}
\crefname{eqn}{Display}{Displays}

\newcommand{\quotient}[2]{{\raisebox{.2em}{$#1$}\left/\raisebox{-.2em}{$#2$}\right.}}



\title[Classification of 3-groups]{Action operads, free algebras on invertible objects, and the classification of 3-groups} 
\author{Edward G. Prior}

\begin{document}

\begin{titlepage}
\maketitle
\pagenumbering{gobble}
\end{titlepage}

\pagenumbering{arabic}

\tableofcontents
\pagebreak

\chapter{Operads and their algebras}

\section{Operads}

\begin{defn} Operads $O$ \end{defn}

\begin{defn} Action operads $G$ \end{defn}

\begin{example} The symmetric operad $S$ \end{example}

\begin{example} The braid operad $B$ \end{example}

\begin{defn} $G$-operads \end{defn}

\section{Operad algebras}

\begin{defn} Operad algbras \end{defn}

\begin{defn} $G$-operad algebras \end{defn}

\section{$\mathrm{E}G$-algebras}

\begin{defn} The $G$-operad $\mathrm{E}G$ \end{defn}

\begin{defn}\label{monaddef} The monad $\mathrm{E}G$ \end{defn}

\begin{defn} $\mathrm{E}G$-algebras \end{defn}

\begin{prop} $G$-operad algebras are monoidal categories with permutation-like structure \end{prop} 

\begin{cor} Braided monoidal categories are $G$-operad algebras \end{cor}

\begin{defn} A strict monoidal category $X$ is said to be \emph{spacial} if, for any object $x \in \mathrm{Ob}(X)$ and any endomorphism of the unit object $f: I \to I$, 
\begin{eq*} f \otimes \mathrm{id}_x = \mathrm{id}_x \otimes f \end{eq*}
\end{defn}

The motivation for the name `spacial' comes from the context of string diagrams \cite{graphicalmon}. In a string diagram, the act of tensoring two strings together is represented by placing those strings side by side. Since the defining feature of the unit object is that tensoring it with other objects should have no effect, the unit object is therefore represented diagrammatically by the absense of a string. An endomorphism of the unit thus appears as an entity with no input or output strings, detached from the rest of the diagram. In a real-world version of these diagrams, made out of physical strings arranged in real space, we could use this detachedness to grab these endomorphisms and slide them over or under any strings we please, without affecting anything else in the diagram. This ability is embodied algebraically by the equation above, and hence categories which obey it are called `spacial'.

\begin{lem}\label{spacial} All $\mathrm{E}G$-algebras are spacial. \end{lem}
\begin{proof}
Let $X$ be an $\mathrm{E}G$-algebra, and fix $x \in \mathrm{Ob}(X)$ and \( f: I \to I \). From the surjectivity of \( \pi : G(2) \to S_2 \) we know that the set $\pi^{-1}( \, (1 \, 2) \, )$ is non-empty, and from the rules for composition of action morphisms we see that for any such $g \in \pi^{-1}( \, (1 \, 2) \, )$,
\begin{eq*}\begin{array}{rll}
		\alpha( \, g \, ; \, \mathrm{id}_x, \, \mathrm{id}_I \, ) \circ \alpha( \, e_2 \, ; \, \mathrm{id}_x, \, f \, ) & = & \alpha( \, g \, ; \, \mathrm{id}_x, \, f \, ) \\
		& = & \alpha( \, e_2 \, ; \, f, \, \mathrm{id}_x \, ) \circ \alpha( \, g \, ; \, \mathrm{id}_x, \, \mathrm{id}_I \, ) \\
		\end{array}
\end{eq*}
Thus in order to obtain the result we're after, it will suffice to find a particular $g \in \pi^{-1}( \, (1 \, 2) \, )$ for which
\begin{eq*}\alpha( \, g \, ; \, \mathrm{id}_x, \, \mathrm{id}_I \, ) = \mathrm{id}_x \end{eq*}
However, since
\begin{eq*}\begin{array}{rll}
		\alpha( \, g \, ; \, \mathrm{id}_x, \, \mathrm{id}_I \, ) & = & \alpha( \, g \, ; \, \mathrm{id}_x, \, \alpha( e_0; - ) \, ) \\
		& = & \alpha( \, \mu(g; e_1, e_0) \, ; \, \mathrm{id}_x \, )
		\end{array}
\end{eq*}
all we really need is to find a $g \in \pi^{-1}( \, (1 \, 2) \, )$ for which
\begin{eq*} \mu(g; e_1, e_0) = e_1 \end{eq*}
To this end, choose an arbtrary element $h \in \pi^{-1}( \, (1 \, 2) \, )$. This $h$ probably won't obey the above equation, but we can use it to construct a new element $g$ which does. Specifically, define
\begin{eq*} k \, := \, \mu( \, h \ ; \, e_1, \, e_0 \, ) \end{eq*}
and then consider
\begin{eq*} g \, := \, h \cdot \mu(e_2; k^{-1}, e_1) \end{eq*}
To see that this is the correct choice of $g$, first note that we must have \( \pi(k) = e_1 \), since this is the only element of $S_1$. Following from that, we have 
\begin{eq*}\begin{array}{rll}
		\pi \big( \, \mu(e_2; k^{-1}, e_1) \, \big) & = & \mu \big( \, \pi(e_2) \ ; \, \pi(k^{-1}), \, \pi(e_1) \, \big) \\
		& = & \mu \big( \, e_2  \ ; \, e_1, \, e_1 \, \big) \\
		& = & e_2
		\end{array}
\end{eq*}
and hence
\begin{eq*}\begin{array}{rll}
		\pi(g) & = & \pi \big( h \cdot \mu(e_2; k^{-1}, e_1) \big) \\
		& = & \pi(h) \cdot \pi \big(\mu(e_2; k^{-1}, e_1) \big) \\
		& = & (1 \, 2) \cdot e_2 \\
		& = & (1 \, 2)
		.\end{array}
\end{eq*}
So $g$ is indeed in $\pi^{-1}( \, (1 \, 2) \, )$, and furthermore
\begin{eq*}\begin{array}{rll}
		\mu(g; e_1, e_0) & = & \mu \big( \, h \cdot \mu(e_2; k^{-1}, e_1) \ ; \, e_1, \, e_0 \, \big) \\
		& = & \mu( \, h \ ; \, e_1, \, e_0 \, ) \cdot \mu \big( \, \mu(e_2; k^{-1}, e_1) \ ; \, e_1, \, e_0 \, \big) \\
		& = & \mu( \, h \ ; \, e_1, \, e_0 \, ) \cdot \mu \big( \, e_2 \ ; \, \mu(k^{-1}; e_1), \, \mu(e_1; e_0) \, \big) \\
		& = & \mu( \, h \ ; \, e_1, \, e_0 \, ) \cdot \mu( \, e_2 \ ; \, k^{-1}, e_0 \, ) \\
		& = & k \cdot k^{-1} \\
		& = & e_1
		\end{array}
\end{eq*}
Therefore, $h \cdot \mu(e_2; k^{-1}, e_1)$ is exactly the $g$ we were looking for, and so working backwards through the proof we obtain the required result:
\begin{eq*} \begin{array}{rll}
		\mu(g; e_1, e_0) & = & e_1 \\
		\implies \quad \alpha( \, g \, ; \, \mathrm{id}_x, \, \mathrm{id}_I \, ) & = & \mathrm{id}_x \\
		& & \\
		\alpha( \, g \, ; \, \mathrm{id}_x, \, \mathrm{id}_I \, ) \circ \alpha( \, e_2 \, ; \, \mathrm{id}_x, \, f \, ) & = & \alpha( \, e_2 \, ; \, f, \, \mathrm{id}_x \, ) \circ \alpha( \, g \, ; \, \mathrm{id}_x, \, \mathrm{id}_I \, ) \\
		\implies \quad \alpha( \, e_2 \, ; \, \mathrm{id}_x, \, f \, ) & = & \alpha( \, e_2 \, ; \, f, \, \mathrm{id}_I \, )
		\end{array}
\end{eq*}
\end{proof}

\section{The free $\mathrm{E}G$-algebra on $n$ objects} 

Our goal for the next few chapters will be to understand the free braided monoidal category on an finite number of invertible objects. Thus, now that we have a firm grasp on action operads and their algebras, we should begin to think about the simpler free constructions they can form. We will use this extensively when calculating the invertible case later on. 

In the paper \cite{operadborel}, Gurski establishes how to contruct free $G$-operad algebras through the use of the monad $\mathrm{E}G$. What follows in this section is a quick summary of the results which will be useful for our purposes. For a more detailed treatment please refer to \cite{operadborel}.

\begin{prop}\label{freealg} There exists a free $\mathrm{E}G$-algebra on $n$ objects. That is, there is an $\mathrm{E}G$-algebra $Y$ such that for any other $\mathrm{E}G$-algebra $X$, we have an isomorphism of categories
\begin{eq*} \mathrm{E}G\mathrm{Alg}_S(Y, X) \cong X^n \end{eq*}
\end{prop}
\begin{proof}
There is an obvious forgetful 2-functor \( U: \mathrm{E}G\mathrm{Alg}_S \to \mathrm{Cat}\) sending $\mathrm{E}G$-algebras to their underlying categories. $U$ has a left adjoint, which we call the free 2-functor \( F : \mathrm{Cat} \to \mathrm{E}G\mathrm{Alg}_S \) adjoint to it. It follows immediately that
\begin{eq*}\begin{array}{rll}
		U(X)^n & = & \mathrm{Cat}(\{z_1, ..., z_n\}, U(X) ) \\
		& \cong & \mathrm{E}G\mathrm{Alg}_S( F(\{z_1, ..., z_n\}), X) 
		\end{array}
\end{eq*}
where $\{z_1, ..., z_n\}$ is any set with $n$ distinct elements. Since $X$ and $U(X)$ are obviously isomorphic as categories, this shows that $F(\{z_1, ..., z_n\})$ is the free algebra on $n$ objects as required. 
\end{proof}

\begin{defn}\label{Gndef} Let $\{ z_1, ..., z_n \}$ be an $n$-object set, which we will also consider as a discrete category. Then we will denote by $\mathbb{G}_n$ the $\mathrm{E}G$-algebra whose underlying category is $\mathrm{E}G( \{ z_1, ..., z_n \})$ and whose action
\begin{eq*} \alpha : \mathrm{E}G\big( \, \mathrm{E}G( \{ z_1, ..., z_n \}) \, \big) \to \mathrm{E}G( \{ z_1, ..., z_n \}) \end{eq*}
is the appropriate component of the multiplication natural transformation $\mu: \mathrm{E}G \circ \mathrm{E}G \to \mathrm{E}G$ of the 2-monad $\mathrm{E}G$.
\end{defn}

\begin{thm} $\mathbb{G}_n$ is the free $\mathrm{E}G$-algebra on $n$ objects. That is,
\begin{eq*}  F(\{z_1, ..., z_n\}) = \mathbb{G}_n \end{eq*}
\end{thm}
\begin{proof}
\end{proof}

\cref{Gndef} is a fairly opaque definition, so we'll spend a little time upacking it. Recall from \cref{monaddef} that $\mathrm{E}G( \{ z_1, ..., z_n \})$ is the coequalizer of the maps
\begin{eq*} \begin{tikzcd}
\coprod_{m \geq 0} \mathrm{E}G(m) \times G(m) \times \{ z_1, ..., z_n \}^m \ar[r, shift left] \ar[r, shift right] & \coprod_{m \geq 0} \mathrm{E}G(m) \times \{ z_1, ..., z_n \}^m
\end{tikzcd} \end{eq*}
that comes from the action of $G(m)$ on $\mathrm{E}G(m)$ by multiplication on the right,
\begin{eq*} \begin{array}{rll}
		\mathrm{E}G(m) \times G(m) & \to & \mathrm{E}G(m) \\
		(g, h) & \mapsto & gh \\
		( \, !: g \to g', \mathrm{id}_h \, ) & \mapsto & !: gh \to g'h
		\end{array}
\end{eq*}
and the action of $G(m)$ on $\{ z_1, ..., z_n \}^m$ by permutation,
\begin{eq*} \begin{array}{rll}
		G(m) \times \{ z_1, ..., z_n \}^m & \to & \{ z_1, ..., z_n \}^m \\
		( \, h \, ; \, x_1, ..., x_m \, ) & \mapsto & (x_{\pi(h^{-1})(1)}, ..., x_{\pi(h^{-1})(m)}) \\
		 \, (\mathrm{id}_h \, ; \, \mathrm{id}_{(x_1, ..., x_m)} \, ) & \mapsto & \mathrm{id}_{(x_{\pi(h^{-1})(1)}, ..., x_{\pi(h^{-1})(m)})}
		\end{array}
\end{eq*}

First, objects in this algebra are equivalence classes of tuples $(g; x_1, ..., x_m)$, for $g \in G(m)$ and $x_i \in \{z_1, ..., z_n\}$, under the relation
\begin{eq*} ( \, gh \, ; \, x_1, \, ..., \, x_m \, ) \sim ( \, g \, ; \, x_{\pi(h)^{-1}(1)}, \, ..., \, x_{\pi(h)^{-1}(m)} \, )\end{eq*}
Notice that using this relation we can rewrite any object uniquely in the form $[e; x_1, ..., x_m]$ for some $m \in \mathbb{N}$ and $x_i \in \{z_1, ..., z_n\}$. This means that each equivalence class is just the tensor product $x_1 \otimes ... \otimes x_m$ in the underlying monoidal category of $\mathbb{G}_n$, for some unique sequence of generators. That is, we can view the objects of $\mathbb{G}_n$ as elements of the monoid freely generated by each of the $z_i$, or in other words:

\begin{lem} \label{Gnobj} $\mathrm{Ob}(\mathbb{G}_n)$ is the free monoid on $n$ generators, $\mathbb{N}^{\ast n}$, the free product of $n$ copies of $\mathbb{N}$. \end{lem}

Similarly, the morphisms of $\mathbb{G}_n$ are the maps
\begin{eq*} (! ; \mathrm{id}_{x_1}, ..., \mathrm{id}_{x_m}) : ( g ; x_1, ..., x_m ) \to ( g' ; x_1, ..., x_m )\end{eq*}
with $g, g' \in G(m)$ and $x_i \in \{z_1, ..., z_n \}$. Using the relation $\sim$ on objects we can rewrite each of these morphisms in the form
\begin{eq*} [h ; \mathrm{id}_{y_1},...,\mathrm{id}_{y_m}] \, : \, y_1 \otimes ... \otimes y_m \, \to \, y_{\pi(h^{-1})(1)} \otimes ... \otimes y_{\pi(h^{-1})(m)} \end{eq*}
where
\begin{eq*} h = g' g^{-1}, \quad \quad y_i = x_{\pi(g^{-1})(i)} \end{eq*}
 The $\mathrm{E}G$-action of $\mathbb{G}_n$ is permutation and tensor product, and the action on morphisms is given by
\begin{eq*} \alpha( \, g \, ; \, [h_1; \mathrm{id}_{x_1}, ..., \mathrm{id}_{x_{m_1}}], \, ..., \, [h_k; \mathrm{id}_{x_1}, ..., \mathrm{id}_{x_{m_k}}] \, ) = [ \, \mu(g;h_1, .., h_k) \, ; \, \mathrm{id}_{x_1}, \, ..., \, \mathrm{id}_{x_{m_k}} \, ] \end{eq*}
Notice that using tensor product notation the object $[e; x]$ is simply $x$, and so $[e; \mathrm{id}_x] = \mathrm{id}_{[e;x]}$ should be written as $\mathrm{id}_x$. Hence by the above $[g; \mathrm{id}_{x_1}, ..., \mathrm{id}_{x_m}]$ is really just $\alpha(g; \mathrm{id}_{x_1}, ..., \mathrm{id}_{x_m})$, and so we have the following:

\begin{lem} \label{Gnmapsaction} Each morphism of $\mathbb{G}_n$ can be expressed uniquely as an action morphism $\alpha(g; \mathrm{id}_{x_1}, ..., \mathrm{id}_{x_m})$, for some $g, g' \in G(m)$ and $x_i \in \{z_1, ..., z_n \}$. \end{lem}

From this, we can also determine $\mathbb{G}_n$'s connected components. 

\begin{prop}\label{Gnconcomp} The connected components of $\mathbb{G}_n$ are $\mathbb{N}^n$, with the assignment $[ \,\, ] : \mathbb{N}^{*n} \to \mathbb{N}^n$ of objects to their component being the quotient map of abelianisation.
\end{prop}
\begin{proof}
By \cref{Gnmapsaction}, all morphisms in $\mathbb{G}_n$ can be written uniquely as $\alpha(g; \mathrm{id}_{x_1}, ..., \mathrm{id}_{x_m})$, for some $g \in G(m)$ and $x_i \in \{z_1, ..., z_n \}$, the set of generators of $\mathbb{N}^{*n}$. Since maps of this form have source $x_1 \otimes ... \otimes x_m$ and target $x_{\pi(g^{-1})(1)} \otimes ... \otimes x_{\pi(g^{-1})(m)}$, we see that there can only exist a morphism between two objects if they can expanded as tensor products which are permutations of one another. Moreover, for any two objects where this is true --- say source $x_1 \otimes ... \otimes x_m$ and target $x_{\sigma^{-1}(1)} \otimes ... \otimes x_{\sigma^{-1}(m)}$ --- we can always find a map $\alpha(g; \mathrm{id}_{x_1}, ..., \mathrm{id}_{x_m})$ between them, because $\pi: G(m) \to S_m$ is surjective and so there must exist at least one $g$ with $\pi(g) = \sigma$. Thus two objects of $\mathbb{G}_n$ share a connected component if and only if they are tensor products that differ by a permutation, and therefore the canonical map $[ \,\, ] : \mathrm{Ob}(\mathbb{G}_n) \to \pi_0(\mathbb{G}_n)$ sending each object to its connected component is just the map which forgets about these permutations, making the free product on $\mathbb{N}^{*n}$ commutative. That is, $[ \,\, ]$ is the quotient map for the abelianisation $\mathrm{ab} : \mathbb{N}^{*n} \to (\mathbb{N}^{*n})^{\mathrm{ab}}$, and so $\pi_0(\mathbb{G}_n) = \mathbb{N}^n$.
\end{proof} 

\chapter{Free invertible algebras as initial objects}
\label{initialalgebra}

In this chapter we will start to consider how to construct free $\mathrm{E}G$-algebras on some number of invertible objects. Specifically, we will begin by showing that such algebras are the initial objects of a particular comma category, in accordance with some well known properties of adjunctions and their units. Using this initial object prespective will allow us to recover all of the data associated with the objects of a given free invertible algebra --- what those objects are, how they act under tensor product, and which pairs of objects form the source and target of at least one morphism. Unfortunately, a concrete description of the morphisms themselves will ultimately remain elusive. We can get tantalisingly closer though, and an examiniation of the exact way that this method fails will provide the neccessary insight to motivate a more successful approach in \cref{coeqalgebra}.

\section{The free algebra on $n$ invertible objects}

We saw in \cref{freealg} that the existence of a free $\mathrm{E}G$-algebra on $n$ objects can be proven by taking the left adjoint of a 2-functor which forgets about the algebra structure. Now we want to extend this idea into the realm of algebras on invertible objects. For the analogous approach, we will need to find a new 2-functor that lets us forget about non-invertible objects, and then hopefully we can find its left adjoint too, and use it to freely add inverses to $\mathbb{G}_n$. First though, we need to make this concept of `forgetting non-invertible objects' a little more precise.

\begin{defn} Given an $\mathrm{E}G$-algebra $X$, we'll denote by $X_{\mathrm{inv}}$ the sub-$\mathrm{E}G$-algebra containing all invertible objects in $X$ and the isomorphisms between them. \end{defn}

Note that this is indeed a well-defined $\mathrm{E}G$-algebra. If $f_1, ..., f_m$ are isomorphisms from invertible objects $x_1, ..., x_m$ to invertible objects $y_1, ..., y_m$, then $\alpha(g; f_1, ..., f_m)$ is a map from the invertible object $\alpha(g; x_1, ..., x_m)$ to the invertible object $\alpha(g; y_1, ..., y_m)$, and it has an inverse $\alpha(g^{-1}; f_{\pi(g)(1)}^{-1}, ..., f_{\pi(g)(m)}^{-1})$, since
\begin{eq*} \begin{array}{ll}
		& \alpha\big( \, g^{-1} \, ; \, f_{\pi(g)(1)}^{-1}, \, ..., \, f_{\pi(g)(m)}^{-1} \, \big) \, \circ \, \alpha( \, g \, ; \, f_1, ..., f_m \,) \\[\medskipamount]
		= & \alpha\big( \, g^{-1}g \, ; \, f_1^{-1} f_1, \, ..., \, f_m^{-1} f_m \, \big) \\[\medskipamount]
		= & \mathrm{id}_{x_1 \otimes ... \otimes x_m} \\
		& \\
		& \alpha( \, g \, ; \, f_1, ..., f_m \,) \, \circ \, \alpha\big( \, g^{-1} \, ; \, f_{\pi(g)(1)}^{-1}, \, ..., \, f_{\pi(g)(m)}^{-1} \, \big) \\[\medskipamount]
		= & \alpha\big( \, gg^{-1} \, ; \, f_{\pi(g)(1)} f_{\pi(g)(1)}^{-1}, \, ..., \, f_{\pi(g)(m)} f_{\pi(g)(m)}^{-1} \, \big) \\[\medskipamount]
		= & \mathrm{id}_{y_{\pi(g)(1)} \otimes ... \otimes y_{\pi(g)(m)}}
		\end{array}
\end{eq*}
Clearly then, $X_{\mathrm{inv}}$ is the correct algebra for our new forgetful 2-functor to send $X$ to. Knowing this, we can contruct the rest of the functor fairly easily.

\begin{prop} \label{invprop} The assignment $X \mapsto X_{\mathrm{inv}}$ can be extended to a 2-functor $(\_)_{\mathrm{inv}}: \mathrm{E}G_G\mathrm{Alg} \to \mathrm{E}G_G\mathrm{Alg}$.
\end{prop}
\begin{proof}
Let $F: X \to Y$ be a (strict) map of $\mathrm{E}G$-algebras. If $x$ is an invertible object in $X$ with inverse $x^*$, then $F(x)$ is an invertible object in $Y$ with inverse $F(x^*)$, by
\begin{eq*} \begin{array}{rcccccl}
			F(x) \otimes F(x^*) & = & F(x \otimes x^*) & = & F(I) & = & I \\
			 F(x^*) \otimes F(x) & = & F(x^* \otimes x) & = & F(I) & = & I 
		\end{array}
\end{eq*}
Since $F$ sends invertible objects to invertible objects, it will also send isomorphisms of invertible objects to isomorphisms of invertible objects. In other words, the map $F: X \to Y$ can be restricted to a map $F_{\mathrm{inv}} : X_{\mathrm{inv}} \to Y_{\mathrm{inv}}$. Moreover, we have that
\begin{eq*} \begin{array}{rcccl}
			(F \circ G)_{\mathrm{inv}}(x) & = & F \circ G(x) & = & F_{\mathrm{inv}} \circ G_{\mathrm{inv}}(x) \\
			(F \circ G)_{\mathrm{inv}}(f) = F \circ G(f) = F_{\mathrm{inv}} \circ G_{\mathrm{inv}}(f) 
		\end{array}
\end{eq*}
and so the assignment $F \mapsto F_{\mathrm{inv}}$ is clearly functorial. Next, let $\theta : F \Rightarrow G$ be a monoidal natural transformation. Choose an invertible object $x$ from $X$, and consider the component map of its inverse, $\theta_{x^*} : F(x^*) \to G(x^*)$. Since $\theta$ is monoidal, we have $\theta_{x^*} \otimes \theta_x = \theta_I = I$ and $\theta_x \otimes \theta_{x^*} = I$, or in other words that $\theta_{x^*}$ is the monoidal inverse of $\theta_x$. We can use this fact to construct a compositional inverse as well, namely $\mathrm{id}_{F(x)} \otimes \theta_{x^*} \otimes \mathrm{id}_{G(x)}$, which can be seen as follows:
\begin{eq*}  \begin{array}{rll}
		\big( \mathrm{id}_{F(x)} \otimes \theta_{x^*} \otimes \mathrm{id}_{G(x)} \big)  \circ \theta_x & = & \theta_x \otimes \theta_{x^*} \otimes \mathrm{id}_{G(x)} \\
		& = &  \mathrm{id}_{G(x)} \\
		&& \\
		\theta_x \circ  \big( \mathrm{id}_{F(x)} \otimes \theta_{x^*} \otimes \mathrm{id}_{G(x)} \big) & = & \mathrm{id}_{F(x)} \otimes \theta_{x^*} \otimes \theta_x \\
		& = &  \mathrm{id}_{F(x)} \\
		\end{array} 
\end{eq*}
Therefore, we see that all the components of our transformation on invertible objects are isomorphisms, and hence we can define a new transformation $\theta_{\mathrm{inv}}: F_{\mathrm{inv}} \Rightarrow G_{\mathrm{inv}}$ whose components are just $(\theta_{\mathrm{inv}})_x = \theta_x$. The assignment $\theta \mapsto \theta_{\mathrm{inv}}$ is also clearly functorial, and thus we have a complete 2-functor $(\_)_{\mathrm{inv}}: \mathrm{E}G_G\mathrm{Alg} \to \mathrm{E}G_G\mathrm{Alg}$.
\end{proof}

\begin{prop} \label{invadj} The 2-functor $(\_)_{\mathrm{inv}}: \mathrm{E}G_G\mathrm{Alg} \to \mathrm{E}G_G\mathrm{Alg}$ has a left adjoint, $L: \mathrm{E}G_G\mathrm{Alg} \to \mathrm{E}G_G\mathrm{Alg}$.
\end{prop}
\begin{proof} To begin, consider the 2-monad $\mathrm{E}G(\_)$. This is a finitary monad --- that is it preserves all filtered colimits --- and it is a 2-monad over $\mathrm{Cat}$, which is locally finitely presentable. It follows from this that $\mathrm{E}G_G\mathrm{Alg}$ is itself locally finitely presentable. Thus if we want to prove $(\_)_{\mathrm{inv}}$ has a left adjoint, we can use the Adjoint Functor Theorem for locally finitely presentable categories, which amounts to showing that $(\_)_{\mathrm{inv}}$ preserves both limits and filtered colimits.
\begin{itemize}
\item Given an indexed collection of $\mathrm{E}G$-algebras $X_i$, the $\mathrm{E}G$-action of their product $\prod X_i$ is defined componentwise. In particular, this means that the tensor product of two objects in $\prod X_i$ is just the collection of the tensor products of their components in each of the $X_i$. An invertible object in $\prod X_i$ is thus simply a family of invertible objects from the $X_i$ --- in other words, $(\prod X_i)_{\mathrm{inv}} = \prod (X_i)_{\mathrm{inv}}$.
\item Given maps of $\mathrm{E}G$-algebras $F: X \to Z$, $G : Y \to Z$, the $\mathrm{E}G$-action of their pullback $X \times_Z Y$ is also defined componentwise. It follows that an invertible object in $X \times_Z Y$ is just a pair of invertible objects $(x, y)$ from $X$ and $Y$, such that $F(x) = G(y)$. But this is the same as asking for a pair of objects $(x, y)$ from $X_{\mathrm{inv}}$ and $Y_{\mathrm{inv}}$ such that $F_{\mathrm{inv}}(x) = G_{\mathrm{inv}}(y)$, and hence $(X \times_Z Y)_{\mathrm{inv}} = X_{\mathrm{inv}} \times_{Z_{\mathrm{inv}}} Y_{\mathrm{inv}}$.
\item Given a filtered diagram $D$ of $\mathrm{E}G$-algebras, the $\mathrm{E}G$-action of their colimit $\mathrm{colim}(D_n)$ is defined in the following way: use filteredness to find an algebra which contains (representatives of the classes of) all the things you want to act on, then apply the action of that algebra. In the case of tensor products this means that $[x]\otimes[y] = [x \otimes y]$, and thus an invertible object in $\mathrm{colim}(D_n)$ is just (the class of) an invertible object in one of the algebras of $D$. In other words, $\mathrm{colim}(D_n)_{\mathrm{inv}} = \mathrm{colim}(D_{\mathrm{inv}})$.
\end{itemize}
Preservation of products and pullbacks gives preservation of limits, and preservation of limits and filtered colimits gives the result.
\end{proof}

With this new 2-functor $L: \mathrm{E}G_G\mathrm{Alg} \to \mathrm{E}G_G\mathrm{Alg}$, we now have the ability to `freely add inverses to objects' in any $\mathrm{E}G$-algebra we want. The algebra $L\mathbb{G}_n$ is then a clear candidate for our free algebra on $n$ invertible objects, and indeed the proof of this is very simple.

\begin{thm} There exists a free $\mathrm{E}G$-algebra on $n$ invertible objects. Specifically, the algebra $L\mathbb{G}_n$ is such that for any other $\mathrm{E}G$-algebra $X$, we have an isomorphism of categories
\begin{eq*} \mathrm{E}G_G\mathrm{Alg}(L\mathbb{G}_n, X) \quad \cong \quad (X_{\mathrm{inv}})^n \end{eq*}
\end{thm}
\begin{proof}
Using the adjunction from \cref{invadj} along with the one from \cref{freealg}, we see that
\begin{eq*}\begin{array}{rll}
		 U(X_{\mathrm{inv}})^n & = & \mathrm{Cat}(\{z_1, ..., z_n\}, U(X_{\mathrm{inv}}) ) \\
		& \cong & \mathrm{E}G_G\mathrm{Alg}( F(\{z_1, ..., z_n\}), X_{\mathrm{inv}}) \\
		& \cong & \mathrm{E}G_G\mathrm{Alg}( LF(\{z_1, ..., z_n\}), X)
\end{array}
 \end{eq*}
As before, $X_{\mathrm{inv}}$ and $U(X_{\mathrm{inv}})$ are obviously isomorphic as categories, and so \( LF(\{z_1, ..., z_n\}) = L\mathbb{G}_n \) satisfies the requirements for the free algebra on $n$ invertible objects.
\end{proof}

\section{$L\mathbb{G}_n$ as an initial algebra}

We have now proven that a free $\mathrm{E}G$-algebra on $n$ invertible objects indeed exists. But this fact on its own is not very helpful. To be able to actually use the free algebra $L\mathbb{G}_n$, we need to know how to contruct it explicitly, in terms of its objects and morphisms. We could do this by finding a detailed characterisation of the 2-functor $L$, and then applying this to our explicit description of $\mathbb{G}_n$ from \cref{Gndef}. However, this would probably take far more effort than is required, since it would involve determining the behaviour of $L$ in many situtations that we aren't interested in. Also, we wouldn't be leveraging $\mathbb{G}_n$'s status as a free algebra to make the calculations any easier. We will try a different strategy instead, one that begins by noticing a special property of the functor $L$.

\begin{prop} \label{linveql} For any $\mathrm{E}G$-algebra $X$, we have $L(X)_{\mathrm{inv}} = L(X)$.
\end{prop}
\begin{proof}
From the definition of adjunctions, the isomorphisms
\begin{eq*}\mathrm{E}G_G\mathrm{Alg}(LX , Y) \quad \cong \quad \mathrm{E}G_G\mathrm{Alg}(X, Y_{\mathrm{inv}}) \end{eq*}
are subject to certain naturality conditions. Specifically, given $F: X' \to X$ and $G: Y \to Y'$ we get a commutative diagram
\begin{eq*} \begin{tikzcd}
\mathrm{E}G_G\mathrm{Alg}(LX , Y) \ar[dd, "G \circ \_ \circ LF"'] \ar[r, "\sim"] & \mathrm{E}G_G\mathrm{Alg}(X, Y_{\mathrm{inv}}) \ar[dd, "G_{\mathrm{inv}} \circ \_ \circ F"] \\
& \\
\mathrm{E}G_G\mathrm{Alg}(LX' , Y') \ar[r, "\sim"] & \mathrm{E}G_G\mathrm{Alg}(X', Y'_{\mathrm{inv}})
\end{tikzcd} \end{eq*}
Consider the case where $F$ is the identity map $\mathrm{id}_X : X \to X$ and $G$ is the inclusion $j: L(X)_{\mathrm{inv}} \to L(X)$. Note that because $j$ is an inclusion, the restriction $j_{\mathrm{inv}}: (L(X)_{\mathrm{inv}})_{\mathrm{inv}} \to L(X)_{\mathrm{inv}}$ is also an inclusion, but since $((\_)_{\mathrm{inv}})_{\mathrm{inv}} = (\_)_{\mathrm{inv}}$, we have that $j_{\mathrm{inv}} = \mathrm{id}$. It follows that
\begin{eq*} \begin{tikzcd}
\mathrm{E}G_G\mathrm{Alg}(LX , LX_{\mathrm{inv}}) \ar[dd, "j \circ \_"'] \ar[r, "\sim"] & \mathrm{E}G_G\mathrm{Alg}(X, LX_{\mathrm{inv}}) \ar[dd, equal] \\
& \\
\mathrm{E}G_G\mathrm{Alg}(LX , LX) \ar[r, "\sim"] & \mathrm{E}G_G\mathrm{Alg}(X, LX_{\mathrm{inv}})
\end{tikzcd} \end{eq*}
Therefore, for any map $f: LX \to LX$ there exists a unique $g: LX \to LX_{\mathrm{inv}}$ such that $j \circ g =f$. But this means that for any such $f$, we must have $\mathrm{im}(f) \subseteq L(X)_{\mathrm{inv}}$, and so in particular $L(X) = \mathrm{im}(\mathrm{id}_{LX}) \subseteq L(X)_{\mathrm{inv}}$. Since $L(X)_{\mathrm{inv}} \subseteq L(X)$ by definition, we obtain the result.
\end{proof}

This result is not especially surprising. Intuitively, it just says that when you freely add inverses to an algebra, every object ends up with an inverse. But the upshot of this is that we now have another way of thinking about $L(X)$: as the target object of the unit of our adjunction, $\eta_X: X \to L(X)_{\mathrm{inv}}$. This means that we don't really need to know the entirety of $L$ in order to determine the free algebra $L\mathbb{G}_n$, just its unit. To find this unit directly, we can turn to the following fact about adjunctions, for which a proof can be found in Lemma 2.3.5 of Leinster's \textit{Basic Category Theory} \cite{bct}.

\begin{prop}\label{initial} Let $F \dashv G: A \to B$ be an adjunction with unit $\eta$. For any object $a$ in $A$, let $(a \downarrow G)$ denote the comma category whose objects are pairs $(b, f)$ consisting of an object $b$ from $B$ and a morphism $f: a \to G(b)$ from $A$, and whose morphisms $h: (b, f) \to (b', f')$ are morphisms $f: b \to b'$ from $B$ such that $G(f) \circ f = f'$. Then the pair $\big(F(a), \eta_a: a \to GF(a) \big)$ is an initial object of $(a \downarrow G)$.
\end{prop}

\begin{cor} $\eta_{\mathbb{G}_n}: \mathbb{G}_n \to (L\mathbb{G}_n)_{\mathrm{inv}} = L\mathbb{G}_n$ is an initial object of $(\mathbb{G}_n \downarrow \mathrm{inv})$.
\end{cor}

Being able to view $L\mathbb{G}_n$ as the initial object in the comma category $(\mathbb{G}_n \downarrow \mathrm{inv})$ will prove immensely useful in the coming sections. This is because it lets us think about the properties of $L\mathbb{G}_n$ in terms of maps $\psi: \mathbb{G}_n \to X_{\mathrm{inv}}$, and this is exactly the context where we can exploit $\mathbb{G}_n$'s status as a free algebra. As a result, its worth taking some time to think about what exactly this map $\eta_{\mathbb{G}_n}$ is.

\begin{lem} The initial object $\eta_{\mathbb{G}_n}: \mathbb{G}_n \to L\mathbb{G}_n$ is the obvious map from the free $\mathrm{E}G$-algebra on $n$ objects into the free $\mathrm{E}G$-algebra on $n$ \emph{invertible} objects. That is, $\eta_{\mathbb{G}_n}$ is the algebra map defined by
\begin{eq*} \begin{array}{rrrcl}
			\eta_{\mathbb{G}_n} & : & \mathbb{G}_n & \to & L\mathbb{G}_n \\
			& : & F(\{z_1, ..., z_n\}) & \to & LF(\{z_1, ..., z_n\}) \\
			& : & z_i & \mapsto & z_i
		\end{array}
\end{eq*}
\end{lem}
\begin{proof}
Consider the $n$-tuple $(z_1, ..., z_n)$ in $(\mathbb{G}_n)^n$. Clearly the image of $(z_1, ..., z_n)$ under the functor $L$ is just the object $(z_1, ..., z_n)$ in the algebra 
\begin{eq*} L\big( \, (\mathbb{G}_n)^n \, \big) \quad = \quad (L\mathbb{G}_n)^n \quad = \quad LF(\{z_1, ..., z_n\})^n \end{eq*}
But the image of $(z_1, ..., z_n) \in (\mathbb{G}_n)^n$ under the isomorphism
\begin{eq*} \mathrm{E}G_G\mathrm{Alg}( \, \mathbb{G}_n , \mathbb{G}_n \, ) \quad \cong \quad (\mathbb{G}_n)^n \end{eq*}
is just the identity map $\mathrm{id}_{\mathbb{G}_n}$. Thus by functoriality of $L$, the map $L(\mathrm{id}_{\mathbb{G}_n}) = \mathrm{id}_{L\mathbb{G}_n}$ must be the one which corresponds to the $n$-tuple $(z_1, ..., z_n) \in (L\mathbb{G}_n)^n$ image via the isomorphism
\begin{eq*} \mathrm{E}G_G\mathrm{Alg}( \, L\mathbb{G}_n , L\mathbb{G}_n \, ) \quad \cong \quad (L\mathbb{G}_n)^n \end{eq*}
Furthermore, the $\mathbb{G}_n$ component of the unit $\eta$ is by definition the image of the identity map $\mathrm{id}_{L\mathbb{G}_n}$ under the isomorphism
\begin{eq*}\mathrm{E}G_G\mathrm{Alg}( \, L\mathbb{G}_n , L\mathbb{G}_n \, ) \quad \cong \quad \mathrm{E}G_G\mathrm{Alg}( \, \mathbb{G}_n, L\mathbb{G}_n \, ) \end{eq*}
Hence it follows that $\eta_{\mathbb{G}_n}$ is the map that corresponds to $(z_1, ..., z_n)$ via
\begin{eq*} \mathrm{E}G_G\mathrm{Alg}( \, \mathbb{G}_n, L\mathbb{G}_n \, ) \quad \cong \quad (L\mathbb{G}_n)^n \end{eq*}
which is exactly the definition given in the statement of the lemma.
\end{proof}

This increadibly simple description makes the map $\eta$ very easy to work with. For example, we immediately obtain the following property, one which we will use frequently throughout the rest of the paper:

\begin{cor} \label{epi} $\eta$ is an epimorphism in $\mathrm{E}G_G\mathrm{Alg}$.
\end{cor}
\begin{proof}
Let $\phi, \psi: L\mathbb{G}_n \to X$ be a pair of algebra maps for which $\phi \circ \eta = \psi \circ \eta$. Then on the generators of $L\mathbb{G}_n$ we have
\begin{eq*} \begin{array}{rcccccl}
			\phi(z_i) & = & \phi\eta(z_i) & = & \psi\eta(z_i) & = & \psi(z_i) \\
			\implies & & \phi_{\mathrm{inv}}(z_i) & = & \psi_{\mathrm{inv}}(z_i) 
		\end{array}
\end{eq*}
But $L\mathbb{G}_n$ is the free $\mathrm{E}G$-algebra on $n$ invertible objects, so maps $L\mathbb{G}_n \to X_{\mathrm{inv}}$ are determined uniquely by where they those generating objects. It follows that $\phi_{\mathrm{inv}} = \psi_{\mathrm{inv}}$, and if $i: X_{\mathrm{inv}} \to X$ is the obvious inclusion,
\begin{eq*} \phi \quad = \quad i \phi_{\mathrm{inv}} \quad = \quad i \psi_{\mathrm{inv}} \quad = \quad \psi \end{eq*}
\end{proof}

Before moving on, we'll make a small change in notation. From now on, rather than writing objects in $(\mathbb{G}_n \downarrow \mathrm{inv})$ as maps $\psi: \mathbb{G}_n \to Y_{\mathrm{inv}}$, we will instead just let $X = Y_{\mathrm{inv}}$ and speak of maps $\psi: \mathbb{G}_n \to X$. This is purely to prevent the notation from becoming cluttered, and shouldn't be a problem so long as we always remember that the targets of these maps only ever contain invertible objects and morphisms. We'll also drop the subscript from $\eta_{\mathbb{G}_n}$, since it is the only component of the unit we'll ever use.

\section{The objects of $L\mathbb{G}_n$}

So now we know that $L\mathbb{G}_n$ is an initial object in the category $(\mathbb{G}_n \downarrow \mathrm{inv})$. But what does this actually tell us? After all, we do not currently have a method for finding initial objects in an arbitrary collection of $\mathrm{E}G$-algebra maps. Because of this, we'll have to approach the problem step-by-step, using the initiality of $\eta$ to extract different pieces of information about the algebra $L\mathbb{G}_n$ as we go. We'll begin by tring to find its objects.

\begin{defn}\label{Obdef} Denote by $\mathrm{Ob}: \mathrm{E}G_G\mathrm{Alg} \to \mathrm{Mon}$ be the functor that sends $\mathrm{E}G$-algebras $X$ to their monoid of objects $\mathrm{Ob}(X)$, and algebra maps $F: X \to Y$ to their underlying monoid homomorphism $\mathrm{Ob}(F): \mathrm{Ob}(X) \to \mathrm{Ob}(Y)$. \end{defn}

In order to find $\mathrm{Ob}(L\mathbb{G}_n)$, we'll need to make use of an important result about the nature of $\mathrm{Ob}$.

\begin{defn}\label{Edef2} Recall from \cref{Edef} that given a monoid $M$, the monoidal category $\mathrm{E}M$ is the one whose monoid of objects is $M$ and which has a unique isomorphism between any two objects. We can view $\mathrm{E}M$ as not just a category but an $\mathrm{E}G$-algebra, by letting the action on morphisms take the only possible values it can, given the required source and target. Then for any monoid homomorphisms $h: M \to M'$, the definition of $\mathrm{E}h: \mathrm{E}M \to \mathrm{E}M'$ given in \cref{Edef} must be a well-defined map of $\mathrm{E}G$-algebras, by functoriality. Thus we can also view $\mathrm{E}$ as a functor $\mathrm{Mon} \to \mathrm{E}G_G\mathrm{Alg}$.
 \end{defn}

\begin{prop}\label{Obadj} $\mathrm{E}$ is a right adjoint to the functor $\mathrm{Ob}$. 
\end{prop}
\begin{proof}
For any $\mathrm{E}G$-algebra $X$, a map $F: X \to \mathrm{E}M$ is determined entirely by its restriction to objects, the monoid homomorphism $\mathrm{Ob}(F) : \mathrm{Ob}(X) \to M$. This is because functorality of $F$ ensures that any map $x \to x'$ in $X$ must be sent to a map $F(x) \to F(x')$ in $\mathrm{E}M$, and by the definition of $\mathrm{E}$ there is always exactly one of these to choose from. In other words, we have an isomorphism between the homsets
\begin{eq*} \mathrm{E}G_G\mathrm{Alg}( \, X, \, \mathrm{E}M \, ) \quad \cong \quad \mathrm{Mon}( \, \mathrm{Ob}(X), \, M \, ) \end{eq*}
Additionally, this isomorphism is natural in both coordinates. That is, for any $G: X \to X'$ in $\mathrm{E}G_G\mathrm{Alg}$ and $h : M \to M'$ in $\mathrm{Mon}$, the diagram
\begin{eq*} \begin{tikzcd}
\mathrm{E}G_G\mathrm{Alg}(X, \mathrm{E}M) \ar[dd, "\mathrm{E}h \circ \_ \circ G"'] \ar[r, "\sim"] & \mathrm{Mon}(\mathrm{Ob}(X), M) \ar[dd, "h \circ \_ \circ \mathrm{Ob}(G)"] \\
& \\
\mathrm{E}G_G\mathrm{Alg}(X', \mathrm{E}M') \ar[r, "\sim"] & \mathrm{Mon}(\mathrm{Ob}(X'), M')
\end{tikzcd} \end{eq*}
commutes, because
\begin{eq*} \mathrm{Ob}( \, \mathrm{E}h \circ F \circ G \, ) \quad = \quad \mathrm{Ob}(Eh) \circ \mathrm{Ob}(F) \circ \mathrm{Ob}(G) \quad = \quad h \circ \mathrm{Ob}(F) \circ \mathrm{Ob}(G) \end{eq*}
Therefore, $\mathrm{Ob} \dashv \mathrm{E}$.
\end{proof}

What \cref{Obadj} is essentially saying is that the functor $\mathrm{Ob}$ provides a way for us to move back and forth between the categories $\mathrm{E}G_G\mathrm{Alg}$ and $\mathrm{Mon}$. By applying this reasoning to the universal property of the initial object $\eta$, we can then determine the value of $\mathrm{Ob}(L\mathbb{G}_n)$ in terms of a new universal property of $\mathrm{Ob}(\eta)$ in the category $\mathrm{Mon}$. In particular, the algebras in $(\mathbb{G}_n \downarrow \mathrm{inv})$ are those whose objects are all invertible, and so the induced property of $\mathrm{Ob}(\eta)$ will end up saying something about the relationship between $\mathrm{Ob}(\mathbb{G}_n)$ and groups --- those monoids whose elements are all invertible.

\begin{defn} Let $M$ be a monoid, $M^{\mathrm{gp}}$ a group, and $i: M \to M^{\mathrm{gp}}$ a monoid homomorphism between them. Then we say that $M^{\mathrm{gp}}$ is the \emph{group completion} of $M$ if for any other group $H$ and homomorphism $h: M \to H$, there exists a unique homomorphism $u: M^{\mathrm{gp}} \to H$ such that $u \circ i = h$.
\end{defn}

There are several different ways to actually calculate the group completion of a monoid. One is to use that fact that $M^{\mathrm{gp}}$ is the group whose group presentation is the same as the monoid presentation of $M$. That is, if $M$ is the quotient of the free monoid on generators $\mathcal{G}$ by the relations $\mathcal{R}$, then $M^{\mathrm{gp}}$ is the quotient of the free \emph{group} on generators $\mathcal{G}$ by relations $\mathcal{R}$. This makes finding the completion of free monoids particularly simple.

\begin{prop}\label{Zobj} The object monoid of $L\mathbb{G}_n$ is $\mathbb{Z}^{*n}$, the group completion of the object monoid of $\mathbb{G}_n$. The restriction of $\eta$ on objects, $\mathrm{Ob}(\eta)$, is then the obvious inclusion $\mathbb{N}^{*n} \hookrightarrow \mathbb{Z}^{*n}$.
\end{prop}
\begin{proof}
Let $H$ be a group, and $h: \mathrm{Ob}(\mathbb{G}_n) \to H$ a monoid homomorphism. By \cref{Obadj} we have an isomorphism of homsets
\begin{eq*} \mathrm{E}G_G\mathrm{Alg}( \, \mathbb{G}_n, \, \mathrm{E}H \, ) \quad \cong \quad \mathrm{Mon}( \, \mathrm{Ob}(\mathbb{G}_n), \, H \, ) \end{eq*}
Denote by $h': \mathbb{G}_n \to \mathrm{E}H$ the map of $\mathrm{E}G$-algebras corresponding to $h$ under this isomorphism. Since $H$ is a group, every object in $\mathrm{E}H$ is invertible, and so $h'$ is an object of $(\mathbb{G}_n \downarrow \mathrm{inv})$. Thus, by initiality of $\eta$, there must exist a unique map $u: L\mathbb{G}_n \to \mathrm{E}G$ making the lefthand traingle below commute:
\begin{eq*} \begin{tikzcd}
\mathbb{G}_n \ar[dd, "\eta"'] \ar[ddrr, "h'"] & & & & \mathrm{Ob}(\mathbb{G}_n) \ar[dd, "\mathrm{Ob}(\eta)"'] \ar[ddrr, "h"] & & \\
& & & & & & \\
L\mathbb{G}_n \ar[rr, "u"'] & & \mathrm{E}H & & \mathrm{Ob}(L\mathbb{G}_n) \ar[rr, "\mathrm{Ob}(u)"'] & & H
\end{tikzcd} \end{eq*}
It follows that the righthand triangle --- which is the image of the first under $\mathrm{Ob}$ --- also commutes. Hence for any group $H$ and homomorphism $h: \mathrm{Ob}(\mathbb{G}_n) \to H$, there is at least one map which factors $h$ through $\mathrm{Ob}(\eta)$.

But now recall from \cref{epi} that $\eta$ is an epimorphism. Left adjoint functors preseve epimorphisms, which means that $\mathrm{Ob}(\eta)$ is one too, and so for any $v: \mathrm{Ob}(L\mathbb{G}_n) \to H$,
\begin{eq*} \begin{array}{rllcrll}
			v \circ \mathrm{Ob}(\eta) & = & h & \implies & v \circ \mathrm{Ob}(\eta) & = & \mathrm{Ob}(u) \circ \mathrm{Ob}(\eta) \\
			& & & \implies & v & = & \mathrm{Ob}(u)
		\end{array}
\end{eq*}
Thus there is actually only one possible map which factors $h$ through $\mathrm{Ob}(\eta)$, and therefore every homomorphism from $\mathrm{Ob}(\mathbb{G}_n)$ onto a group factors uniquely through the group $\mathrm{Ob}(L\mathbb{G}_n)$. In other words, $\mathrm{Ob}(L\mathbb{G}_n)$ is the group completion $\mathrm{Ob}(\mathbb{G}_n)^{\mathrm{gp}}$. Since by \cref{Gnobj} the object monoid of $\mathbb{G}_n$ is $\mathbb{N}^{\ast n}$, the free monoid on $n$ generators, we can conclude that
\begin{eq*} \mathrm{Ob}(L\mathbb{G}_n) \quad = \quad \mathrm{Ob}(\mathbb{G}_n)^{\mathrm{gp}} \quad = \quad (\mathbb{N}^{\ast n})^{\mathrm{gp}} \quad = \quad \mathbb{Z}^{\ast n} \end{eq*}
the free group on $n$ generators. Moreover, the map $\mathrm{Ob}(\eta)$ is then the inclusion of $\mathrm{Ob}(\mathbb{G}_n)$ into its completion, which is just $\mathbb{N}^{*n} \hookrightarrow \mathbb{Z}^{*n}$.
\end{proof}

\section{The connected components of $L\mathbb{G}_n$}

The core result of \cref{Zobj} --- that $\mathrm{Ob}(L\mathbb{G}_n)$ is the group completion of $\mathrm{Ob}(\mathbb{G}_n)$ --- makes concrete the sense in which the functor $L$ represents `freely adding inverses' to objects. Extending this same logic to connected components as well, it would seem reasonable to expect that $\pi_0(L\mathbb{G}_n)$ is the group completion of $\pi_0(\mathbb{G}_n)$ as well. This is indeed the case, and the proof proceeds in a way completely analagous to \cref{Zobj}. 

First, we want to show that the process of taking connected components forms part of an adjunction. To do this we are going to need a category from which we can draw the kind of structures that can act as the components of an $\mathrm{E}G$-algebra. Exactly which category this should be will depend on our choice of action operad $G$, or more precisely its underlying permutations.

\begin{defn} For a given action operad $G$, denote by $\mathrm{im}(\pi)\mathrm{-Mon}$ the full subcategory of $\mathrm{Mon}$ on those monoids whose multiplication is invariant under the permutations in $\mathrm{im}(\pi)$. That is, a monoid $M$ is in $\mathrm{im}(\pi)\mathrm{-Mon}$ if and only if
\begin{eq*} m_1, ..., m_n \in M, \, g \in G(n) \quad \implies \quad m_1 ... m_n \, = \, m_{\pi(g)^{-1}(1)} ... m_{\pi(g)^{-1}(n)} \end{eq*}
\end{defn}

Of course, by \cref{surjortriv} there are really only two examples of such a $\mathrm{im}(\pi)\mathrm{-Mon}$. If the underlying permutations of $G$ are trivial, then $\mathrm{im}(\pi)\mathrm{-Mon}$ is just the whole of the category $\mathrm{Mon}$; if  instead $G$ is crossed then we are asking for monoids whose multiplication is invariant under arbitrary permutations from $\mathrm{S}$, and so $\mathrm{im}(\pi)\mathrm{-Mon}$ is just the category of \emph{commutative} monoids, $\mathrm{CMon}$. Regardless, when we are working with an arbitrary action operad $G$, the category $\mathrm{im}(\pi)\mathrm{-Mon}$ is exactly the collection of possible connected components that we were looking for.

\begin{lem}\label{pi0} Let $G$ be an action operad and $\mathrm{im}(\pi)$ its underlying permutation action operad. Then there is a functor
\begin{eq*} \pi_0: \mathrm{E}G_G\mathrm{Alg} \to \mathrm{im}(\pi)\mathrm{-Mon} \end{eq*}
which sends each algebra $X$ to its monoid of connected components $\pi_0(X)$, and sends each map of algebras $F: X \to Y$ to its restriction to connected components $\pi_0(F): \pi_0(X) \to \pi_0(Y)$.
\end{lem}
\begin{proof}
Let $x_1, ..., x_n$ be an arbitrary collection of objects from the algebra $X$, and $g$ an element of the group $G(n)$. Then the action of $G$ guarantees the existence of a morphism
\begin{eq*} \alpha(g; \mathrm{id}_{x_1}, ..., \mathrm{id}_{x_n}) \, : \, x_1 \otimes ... \otimes x_n \to x_{\pi(g^{-1})(1)} \otimes ... \otimes x_{\pi(g^{-1})(n)} \end{eq*}
By definition the source and target of this morphism belong to the same connected component, and hence
\begin{eq*} \begin{array}{rll}
			[ \, x_1 \otimes ... \otimes x_n \, ] & = & [ \, x_{\pi(g^{-1})(1)} \otimes ... \otimes x_{\pi(g^{-1})(n)} \, ] \\
			\implies \quad [x_1] \otimes ... \otimes [x_n] & = & [x_{\pi(g^{-1})(1)}] \otimes ... \otimes [x_{\pi(g^{-1})(n)}]
		\end{array} 
\end{eq*}
But since the $x_i$ are just arbitrary objects of $X$, the components $[x_i]$ are an arbitrary collection of elements from $\pi_0(X)$, and likewise for the group element $g$ and the permutation $\pi(g)$. Therefore multiplication in the monoid $\pi_0(X)$ is invariant under all permutations in the images of the homomorphisms $\pi_n: G(n) \to S_n$, and thus $\pi_0(X)$ is an object of $\mathrm{im}(\pi)\mathrm{-Mon}$, as required. Well-definedness of the functor $\pi_0$ on morphisms then follows immediately from the fullness of $\mathrm{im}(\pi)\mathrm{-Mon}$.
\end{proof}

Now that we have a functor which represents the act of finding the connected component monoid of an algebra, we need another functor heading in the opposite direction, so that we can construct an adjunction between them.

\begin{defn} There exists an inclusion of 2-categories $\mathrm{D}: \mathrm{Set} \hookrightarrow \mathrm{Cat}$ which allows us to view any set $S$ as a \emph{discrete category}, one whose objects are just the elements of $S$ and whose morphisms are all identities. If the given set also happens to be a monoid $M$, then there is an obvious way to see the discete category $\mathrm{D}M$ as a monoidal category, and so we have a similar inclusion $\mathrm{D}: \mathrm{Mon} \hookrightarrow \mathrm{MonCat}$. Finally, for any action operad $G$ and object $M$ of the category $\mathrm{im}(\pi)\mathrm{-Mon}$, there is a unique way to assign an $\mathrm{E}G$-action to the discete category $\mathrm{D}M$. This works because for any elements $m_1, ..., m_n \in M$ and $g \in G(n)$, the morphism $\alpha(g; \mathrm{id}_{m_1}, ..., \mathrm{id}_{m_n})$ must have source and target 
\begin{eq*} m_1 \otimes ... \otimes m_n  \quad = \quad m_{\pi(g^{-1})(1)} \otimes ... \otimes m_{\pi(g^{-1})(m)} \end{eq*}
and therefore it can only be the morphism $\mathrm{id}_{m_1 \otimes ... \otimes m_n}$. This choice of action yields one last inclusion $\mathrm{CMon} \hookrightarrow \mathrm{E}G_G\mathrm{Alg}$, which we shall also call $\mathrm{D}$. \end{defn}

\begin{prop}\label{concompadj} $\mathrm{D}$ is a right adjoint to the functor $\pi_0$. 
\end{prop}
\begin{proof}
Consider a map of $F: X \to \mathrm{D}C$ from some $\mathrm{E}G$-algebra $X$ onto the discrete $\mathrm{E}G$-algebra for a monoid $M$ in $\mathrm{im}(\pi)\mathrm{-Mon}$. For any $f: x \to x'$ in $X$, the morphism $F(f)$ must be an identity map in $\mathrm{D}M$, since these are the only morphisms that $\mathrm{D}M$ has. It follows that $x$ and $x'$ being in the same connected component will imply $F(x) = F(x')$, and so $F$ is determined entirely by its restriction to connected components, the monoid homomorphism $\pi_0(F) : \pi_0(X) \to M$. In other words, we have an isomorphism between the homsets
\begin{eq*} \mathrm{E}G_G\mathrm{Alg}( \, X, \mathrm{D}M \, ) \quad \cong \quad \mathrm{im}(\pi)\mathrm{-Mon}( \, \pi_0(X), M \, ) \end{eq*}
This isomorphism is natural in both coordinates, since for any $G: X \to X'$ in $\mathrm{E}G_G\mathrm{Alg}$ and $h : M \to M'$ in $\mathrm{im}(\pi)\mathrm{-Mon}$, 
\begin{eq*} \pi_0( \, \mathrm{D}h \circ F \circ G \, ) \quad = \quad \pi_0(\mathrm{D}h) \circ \pi_0(F) \circ \pi_0(G) \quad = \quad h \circ \pi_0(F) \circ \pi_0(G) \end{eq*}
and so the diagram
\begin{eq*} \begin{tikzcd}
\mathrm{E}G_G\mathrm{Alg}(X, \mathrm{D}M) \ar[dd, "\mathrm{D}h \circ \_ \circ G"'] \ar[rr, "\sim"] & & \mathrm{im}(\pi)\mathrm{-Mon}\big( \, \pi_0(X), M \, \big) \ar[dd, "h \circ \_ \circ \pi_0(G)"] \\
& & \\
\mathrm{E}G_G\mathrm{Alg}(X', \mathrm{D}M') \ar[rr, "\sim"] & & \mathrm{im}(\pi)\mathrm{-Mon}\big( \, \pi_0(X'), M' \, \big) 
\end{tikzcd} \end{eq*}
commutes. Therefore, $\pi_0 \dashv \mathrm{D}$.
\end{proof}

Now we can utilise \cref{concompadj} to draw out a universal property of $\pi_0(L\mathbb{G}_n)$, just as we did with $\mathrm{Ob}(L\mathbb{G}_n)$ in \cref{Obadj}.

\begin{prop}\label{Zconcomp} The connected components of $L\mathbb{G}_n$ are the group completion of the connected components of $\mathbb{G}_n$. Also, the restriction of $\eta$ onto connected components, $\pi_0(\eta)$, is the canonical map $\pi_0(\mathbb{G}_n) \to \pi_0(\mathbb{G}_n)^{\mathrm{gp}}$ associated with that group completion.
\end{prop}
\begin{proof}
Let $H$ be a group which is also an object of $\mathrm{im}(\pi)\mathrm{-Mon}$, and let $h: \pi_0(\mathbb{G}_n) \to H$ be a monoid homomorphism. By \cref{concompadj} there is a homset isomorphism
\begin{eq*} \mathrm{E}G_G\mathrm{Alg}( \, \mathbb{G}_n, \, \mathrm{D}H \, ) \quad \cong \quad \mathrm{im}(\pi)\mathrm{-Mon}( \, \pi_0(\mathbb{G}_n), \, H \, ) \end{eq*}
and thus some $\mathrm{E}G$-algebra map $h': \mathbb{G}_n \to \mathrm{D}H$ corresponding to $h$. As $H$ is a group, every object of $\mathrm{D}H$ is invertible, and so $h'$ is an object of $(\mathbb{G}_n \downarrow \mathrm{inv})$. It follows that there exists a unique map $u: L\mathbb{G}_n \to \mathrm{D}M$ which factors $h'$ through the initial object $\eta$:
\begin{eq*} \begin{tikzcd}
\mathbb{G}_n \ar[dd, "\eta"'] \ar[ddrr, "h'"] & & & & \pi_0(\mathbb{G}_n) \ar[dd, "\pi_0(\eta)"'] \ar[ddrr, "h"] & & \\
& & & & & & \\
L\mathbb{G}_n \ar[rr, "u"'] & & \mathrm{D}H & \quad & \pi_0(L\mathbb{G}_n) \ar[rr, "\pi_0(u)"'] & & H
\end{tikzcd} \end{eq*}
Applying the functor $\pi_0$ everywhere, we see that $\pi_0(u)$ must also factor $h$ through the homomorphism $\pi_0(\eta)$. Moreover, since $\eta$ is an epimorphism and $\pi_0$ a left adjoint functor, $\pi_0(\eta)$ is an epimorphism too, and so $\pi_0(u)$ is the only map with this property. Therefore, any monoid homomorphism $\pi_0(\mathbb{G}_n) \to H$ will factor uniquely through $\pi_0(L\mathbb{G}_n)$, so long as $H$ is in $\mathrm{im}(\pi)\mathrm{-Mon}$.  

Now consider another monoid homomorphism $k: \pi_0(\mathbb{G}_n) \to K$, where this time $K$ is still a group but not neccessarily in $\mathrm{im}(\pi)\mathrm{-Mon}$. From \cref{pi0}, we know that $\pi_0(\mathbb{G}_n)$ is still an object of $\mathrm{im}(\pi)\mathrm{-Mon}$, and from this we can conclude that the image $\mathrm{im}(k)$ will be too:
\begin{eq*} \begin{array}{rcrcl}
			 x_1, ..., x_m \in \pi_0(\mathbb{G}_n), \, g \in G(n) & \implies & x_1 \otimes ... \otimes x_m & = & x_{\pi(g)(1)} \otimes ... \otimes x_{\pi(g)(m)} \\
			& \implies & k( \, x_1 \otimes ... \otimes x_m \, ) & = & k( \, x_{\pi(g)(1)} \otimes ... \otimes x_{\pi(g)(m)} \, ) \\
			& \implies & k(x_1) \otimes ... \otimes k(x_m) & = & k(x_{\pi(g)(1)}) \otimes ... \otimes k(x_{\pi(g)(m)})
		\end{array}
\end{eq*}
Also, since $\mathrm{im}(k)$ is a submonoid of the group $K$, it is a group as well. Thus if we denote by $k_{\mathrm{im}}: \mathrm{Ob}(\mathbb{G}_n) \to \mathrm{im}(k)$ the restriction of $k$ to it image, then $k_{\mathrm{im}}$ is a map in $\mathrm{im}(\pi)\mathrm{-Mon}$ out of $\mathrm{Ob}(\mathbb{G}_n)$ and onto a group, and therefore by what we showed earlier there exists a unique homomorphism $v: \mathrm{Ob}(L\mathbb{G}_n) \to \mathrm{im}(k)$ with the property $v \circ \pi_0(\eta) = k_{\mathrm{im}}$. Composing this $v$ with the inclusion $i: \mathrm{im}(k) \hookrightarrow K$, we see that
\begin{eq*} i \circ v \circ \pi_0(\eta) \, = \, i \circ k_{\mathrm{im}} \, = \, k \end{eq*}
and $i \circ v$ must be the only map for which this is true, for restricting this equation back on $\mathrm{im}(k)$ yields the unique property of $v$ again. Thus $\pi_0(\eta)$ will actually take any homomorphism from $\mathrm{Ob}(\mathbb{G}_n)$ onto a group and factor it through $\pi_0(L\mathbb{G}_n)$ in a unique way, not just those homomorphisms in $\mathrm{im}(\pi)\mathrm{-Mon}$. In other words, 
\begin{eq*} \pi_0(L\mathbb{G}_n) \quad = \quad \pi_0(\mathbb{G}_n)^{\mathrm{gp}} \end{eq*}
and $\pi_0(\eta)$ is the canonical map of this group completion.
\end{proof}

As we've said before, this result is a reflection of the fact that the functor $L$ is trying to add inverses the objects of $\mathbb{G}_n$ freely, that is, with as little effect on the rest of the algebra as possible. Indeed, if we happen to know whether or not our action operad $G$ is crossed then we can now calculate exactly what the effect on the components will be.

\begin{cor}\label{crossconcomp} If $G$ is a crossed action algebra then
\begin{itemize} \itemsep0em
\item the connected components of $L\mathbb{G}_n$ are the monoid $\mathbb{Z}^n$
\item the restriction of $\eta$ to components is the obvious inclusion $\mathbb{N}^n \hookrightarrow \mathbb{Z}^n$
\item the assignment of objects to their component is given by the quotient map of abelianisation $\mathrm{ab}: \mathbb{Z}^{\ast n} \to \mathbb{Z}^n$
\end{itemize}
If instead $G$ is non-crossed, then
\begin{itemize} \itemsep0em
\item the connected components of $L\mathbb{G}_n$ are the monoid $\mathbb{Z}^{\ast n}$
\item the restriction of $\eta$ to components is the obvious inclusion $\mathbb{N}^{\ast n} \hookrightarrow \mathbb{Z}^{\ast n}$
\item the assignment of objects to their component is $\mathrm{id}_{\mathbb{Z}^{\ast n}}$
\end{itemize}
\end{cor}
\begin{proof}
Combining \cref{Zconcomp,Gnconcomp}, we see that
\begin{eq*} \pi_0(L\mathbb{G}_n) \quad = \quad \pi_0(\mathbb{G}_n)^{\mathrm{gp}} \quad = \quad \begin{cases}
													\quad (\mathbb{N}^n)^{\mathrm{gp}} \quad = \quad \mathbb{Z}^n & \text{if $G$ is crossed} \\
													\quad (\mathbb{N}^{\ast n})^{\mathrm{gp}} \quad = \quad \mathbb{Z}^{\ast n} & \text{otherwise}
														\end{cases}
\end{eq*}
Moreover, \cref{Zconcomp} says that restriction of $\eta$ to connected components, $\pi_0(\eta)$, will be the homomorphism associated with these group completion, which means the inclusion $\mathbb{N}^n \hookrightarrow \mathbb{Z}^n$ when $G$ is crossed and $\mathbb{N}^{\ast n} \hookrightarrow \mathbb{Z}^{\ast n}$ when it is not.

Next, by \cref{Gnconcomp} we know that the map $[ \, \_ \, ] : \mathrm{Ob}(\mathbb{G}_n) \to \pi_0(\mathbb{G}_n)$ sending objects of $\mathbb{G}_n$ to their connected component is either the quotient map of abelianisation $\mathbb{N}^{\ast n} \to \mathbb{N}^n$ or the identity on $\mathbb{N}^{\ast n}$, depending on whether or not it is crossed. If we also use $[ \, \_ \, ]$ to denote the map sending objects of $L\mathbb{G}_n$ to their components, it then follows from functoriality of $\eta$ that the corresponding choice of the followings two diagrams will commute:
\begin{eq*} \begin{tikzcd}
\mathbb{N}^{\ast n} \ar[dd, hookrightarrow, "\lbrack \, \_ \, \rbrack"'] \ar[rr, hookrightarrow, "\mathrm{Ob}(\eta)"] & & \mathbb{Z}^{\ast n} \ar[dd, "\lbrack \, \_ \, \rbrack"] & \quad & \mathbb{N}^{\ast n} \ar[dd, equals, "\lbrack \, \_ \, \rbrack"'] \ar[rr, hookrightarrow, "\mathrm{Ob}(\eta)"] & & \mathbb{Z}^{\ast n} \ar[dd, "\lbrack \, \_ \, \rbrack"] \\
& & & & \\
\mathbb{N}^n \ar[rr, hookrightarrow, "\pi_0(\eta)"'] & & \mathbb{Z}^n & & \mathbb{N}^{\ast n} \ar[rr, hookrightarrow, "\pi_0(\eta)"'] & & \mathbb{Z}^{\ast n}
\end{tikzcd} \end{eq*}
Using the values of $[ \, \_ \, ]$ from \cref{Gnconcomp}, $\mathrm{Ob}(\eta)$ from \cref{Zobj}, and $\pi_0(\eta)$ from earlier in this proof, it follows that for any generator $z_i$ of $\mathbb{Z}^{\ast n}$, 
\begin{eq*} [z_i] \quad = \quad [\mathrm{Ob}(\eta)(z_i)] \quad = \quad \pi_0(\eta)([z_i]) \quad = \quad \pi_0(\eta)(z_i) \quad = \quad z_i \end{eq*}
But this description of $[ \, \_ \, ]: \mathrm{Ob}(L\mathbb{G}_n) \to \pi_0(L\mathbb{G}_n)$ on generators is either the definition of the quotient map $\mathrm{ab}: \mathbb{Z}^{\ast n} \to (\mathbb{Z}^{\ast n})^{\mathrm{ab}}$ or the identity $\mathrm{id}: \mathbb{Z}^{\ast n} \to \mathbb{Z}^{\ast n}$, depending on the value of target monoid, as required.
\end{proof}

\section{The collapsed morphisms of $L\mathbb{G}_n$}  

Now that we understand the objects and connected components of the algebra $L\mathbb{G}_n$, the next most obvious thing to look for are its morphisms, $\mathrm{Mor}(L\mathbb{G}_n)$. It would be nice to construct this collection in the same way we constructed $\mathrm{Ob}(L\mathbb{G}_n)$ and $\pi_0(L\mathbb{G}_n)$, by applying the left adjoint of some adjunction to the initial map $\eta$. Before we can do this however, we need to ask ourselves a question. What sort of mathematical object is $\mathrm{Mor}(L\mathbb{G}_n)$, exactly?

Given a pair of morphisms $f: x \to y, f': y' \to z$ in an $\mathrm{E}G$-algebra $X$, there are two basic binary operations we can perform. First, we can take their tensor product $f \otimes f'$, and this together with the unit map $\mathrm{id}_{I}$ imbues $\mathrm{Mor}(X)$ with the structure of a monoid. Second, if we have $y = y'$ then we can form the composite morphism $f' \circ f$. However, these two operations are not as different as they first appear.

\begin{lem} \label{tenscomp} Let $f: x \to y$ and $f': y \to z$ be morphisms in some monoidal category, and $y$ is an invertible object of that category. Then
\begin{eq*} f' \circ f \quad = \quad f' \otimes \mathrm{id}_{y*} \otimes f \end{eq*}
\end{lem}
\begin{proof}
By the interchange law for monoidal categories,
\begin{eq*}\begin{array}{rll}
			f' \circ f & = & (f' \otimes \mathrm{id}_I) \circ (\mathrm{id}_I \otimes f) \\
			& = & (f' \otimes \mathrm{id}_{y*} \otimes \mathrm{id}_y) \circ (\mathrm{id}_y \otimes \mathrm{id}_{y*} \otimes f) \\
			& = & (f' \circ \mathrm{id}_y) \otimes (\mathrm{id}_{y*} \circ \mathrm{id}_{y*}) \otimes (\mathrm{id}_y \circ f) \\
			& = & f' \otimes \mathrm{id}_{y*} \otimes f 
		\end{array}
\end{eq*}
\end{proof} 

In other words, composition along invertible objects in $X$ can always be restated in terms of the tensor product. Thus in cases where every object of $X$ is invertible, the monoidal structure together with knowledge of each morphisms source and target will be enough to determine $X$ uniquely. Since all objects in $L\mathbb{G}_n$ are invertible, this means that we could choose to ignore composition of elements of $\mathrm{Mor}(L\mathbb{G}_n)$ for the time being, and focus on its status as a monoid under tensor product.

However, we are trying to extract information about the morphisms of $L\mathbb{G}_n$ by building some sort of left adjoint functor. Presumably we will also be able to apply it to other $\mathrm{E}G$-algebras, some of which won't have all of their objects invertible, and so we can't just use $\mathrm{Mor}(-): \mathrm{E}G_G\mathrm{Alg} \to \mathrm{Mon}$. What we need is a way to modify the morphism monoid of a category so that both composition and tensor product are recoverable from a single operation. Of course, there is one very easy method for achieving this --- simply force $\otimes$ and $\circ$ to be equal.

\begin{defn} Let $\mathrm{M} : \mathrm{MonCat} \to \mathrm{Mon}$ be the functor which sends monoidal categories $X$ to the quotient of their monoid of morphisms by the relation that sets $\otimes = \circ$.
\begin{eq*} \mathrm{M}X \quad = \quad \bigquotient{\mathrm{Mor}(X)}{f' \circ f \sim f' \otimes f}\end{eq*}
Each monoidal functors $F: X \to Y$ is then sent to the monoid homomorphism
\begin{eq*} \begin{array}{rlrll}
			\mathrm{M}(F) & : & \mathrm{M}X & \to & \mathrm{M}Y \\
			& : & \mathrm{M}(f) & \mapsto & \mathrm{M}\big( \, F(f) \, \big) \\
		\end{array}
\end{eq*}
where $\mathrm{M}(f)$ refers to the equivalence class of the map $f$ under the quotient $\mathrm{Mor}(X) \to \mathrm{M}(X)$. This homomorphism is well-defined, since it respects the relation $\otimes = \circ$:
\begin{eq*} \begin{array}{rll}
			\mathrm{M}(F)( \, f' \circ f \, ) & = & \mathrm{M}\big( \, F(f' \circ f) \, \big) \\
			& = & \mathrm{M}\big( \, F(f') \circ F(f) \, \big) \\
			& = & \mathrm{M}\big( \, F(f') \, \big) \circ \mathrm{M}\big( \, F(f) \, \big) \\
			& = & \mathrm{M}\big( \, F(f') \, \big) \otimes \mathrm{M}\big( \, F(f) \, \big) \\
			& = & \mathrm{M}\big( \, F(f') \otimes F(f) \, \big) \\
			& = & \mathrm{M}\big( \, F(f' \otimes f) \, \big) \\
			& = & \mathrm{M}(F)( \, f' \otimes f \, )
		\end{array}
\end{eq*}
We will call $\mathrm{M}X$ the \emph{collapsed} morphisms of the $X$.
\end{defn}

From now on we will generally refer to the single operation in $\mathrm{M}X$ as $\otimes$ rather than $\circ$, unless we are focusing on some aspect best understood using compositon. This convention makes it easier to remember that because the tensor product is defined between all pairs of morphisms in $X$,  the equivalence class $\mathrm{M}(f') \otimes \mathrm{M}(f)$ will always contain the morphism $f' \otimes f$, but not neccessarily $f' \circ f$, as it might fail to exist.

Now we need a candidate for the right adjoint to the functor $\mathrm{M}$.

\begin{defn} For a given monoid $M$, let $\mathrm{B}M$ represent the one-object category whose morphisms are the elements of $M$, with monoid multiplication as composition. This is known as the \emph{delooping} of $M$, for reasons that come from homotopy theory. Likewise, for any monoid homomorphism $h: M \to M'$ between abelian groups, denote by $\mathrm{B}h : \mathrm{B}M \to \mathrm{B}M'$ the obvious monoidal functor which acts like $h$ on morphisms. This defines a delooping functor $\mathrm{B}: \mathrm{Mon} \to \mathrm{Cat}$ from the category of monoids onto the category of small categories.

Moreover, let $C$ be a commutative monoid. Then we can view $\mathrm{B}C$ as a monoidal category, with the tensor product also given by the multiplication in $C$, and the sole object as the unit $I$. Clearly for any homomorphism between commutative monoids $h : C \to C'$ the corresponding functor $\mathrm{B}h : \mathrm{B}C \to \mathrm{B}C'$ will preserve this monoidal structure, as it is already preserving it as compositon. Thus the restriction of $\mathrm{B}$ to commutative monoids also gives a functor $\mathrm{CMon} \to \mathrm{MonCat}$, which we will still call $\mathrm{B}$.
\end{defn}

Commutativity is required in order for $\mathrm{B}C$ to be a well-defined monoidal category because we need its operations $\circ$ and $\otimes$ to obey the interchange law for monoidal categories:
\begin{eq*}\begin{array}{rrll}
			& (\mathrm{id}_I \circ f) \otimes (f' \otimes \mathrm{id}_I) & = & (\mathrm{id}_I \otimes f') \circ (f \otimes \mathrm{id}_I) \\
			\implies & \mathrm{id}_I \cdot f \cdot f' \cdot \mathrm{id}_I & = & \mathrm{id}_I \cdot f' \cdot f \cdot \mathrm{id}_I \\
			\implies & f \cdot f' & = & f' \cdot f
		\end{array}
\end{eq*}

\begin{prop}\label{Moradj} $\mathrm{B}$ is a right adjoint to the functor $\mathrm{M}(\, \_ \,)^{\mathrm{ab}} : \mathrm{MonCat} \to \mathrm{CMon}$.
\end{prop}
\begin{proof}
Let $X$ be a monoidal category, $C$ a commutative monoid, and $F: X \to \mathrm{B}C$ a monoidal functor. For any $f: x \to x'$ in $X$, the morphism $F(f)$ is just an element of the monoid $C$, and so $F$ can be used to define a function
\begin{eq*} \begin{array}{rlrll}
			F' & : & \mathrm{M}(X)^{\mathrm{ab}} & \to & C \\
			& : & \mathrm{ab} \circ \mathrm{M}(f) & \mapsto & F(f) \\
		\end{array}
\end{eq*}
where $\mathrm{ab}$ is the quotient map of abelianisation $\mathrm{M}(X) \to \mathrm{M}(X)^{\mathrm{ab}}$. This $F'$ is a well-defined monoid homomorphism; it preserves multiplication and respects the relation $\otimes = \circ$ because the monoid multiplication of $C$ is acts as both tensor product and composition in $\mathrm{B}C$.
\begin{eq*} \begin{array}{rll}
			F'\big( \, \mathrm{ab}\mathrm{M}(f' \circ f ) \, \big) & = & F(f' \circ f) \\
			& = & F(f') \circ F(f)  \\
			& = & F(f') \cdot F(f) \\
			& = & F(f') \otimes F(f) \\
			& = & F(f' \otimes f) \\
			& = & F' \big( \, \mathrm{ab}\mathrm{M}(f' \otimes f) \, \big)
		\end{array}
\end{eq*}
Conversely, if $h: \mathrm{M}(X)^{\mathrm{ab}} \to C$ is a monoid homomorphism, we can define from it a monoidal functor
\begin{eq*} \begin{array}{rlrll}
			h' & : & X & \mapsto & \mathrm{B}C \\
			& : & x & \mapsto & I \\
			& : & f: x \to y & \mapsto & h\big( \, \mathrm{ab}\mathrm{M}(f) \, \big) : I \to I
		\end{array}
\end{eq*}
Yet again, the monoidal functor $h'$ is well-defined because the fact that $\otimes = \circ$ in $\mathrm{B}C$ forces $h'$ to respect that relation.
\begin{eq*} \begin{array}{rll}
			h'(f' \circ f ) & = & h\big( \, \mathrm{ab}\mathrm{M}(f' \circ f ) \, \big) \\
			& = & h\big( \, \mathrm{ab}\mathrm{M}(f') \circ \mathrm{M}(f') \, \big) \\
			& = & h\big( \, \mathrm{ab}\mathrm{M}(f') \, \big) \circ h\big( \, \mathrm{ab}\mathrm{M}(f') \, \big) \\
			& = & h\big( \, \mathrm{ab}\mathrm{M}(f') \, \big) \cdot h\big( \, \mathrm{ab}\mathrm{M}(f') \, \big) \\
			& = & h\big( \, \mathrm{ab}\mathrm{M}(f') \, \big) \otimes h\big( \, \mathrm{ab}\mathrm{M}(f') \, \big) \\
			& = & h\big( \, \mathrm{ab}\mathrm{M}(f') \otimes \mathrm{ab}\mathrm{M}(f') \, \big) \\
			& = & h\big( \, \mathrm{ab}\mathrm{M}(f' \otimes f') \, \big) \\
			& = & h'(f' \otimes f)
		\end{array}
\end{eq*}
But these assignments $F \mapsto F'$ and $h \mapsto h'$ are clearly inverse to one another. For any $F: X \to \mathrm{B}C$ applying them twice gives
\begin{eq*} \begin{array}{rlrllll}
			F'' & : & X & \to & \mathrm{B}C & &\\
			& : & x & \mapsto & I & & \\
			& : & f: x \to y & \mapsto & F'\big( \, \mathrm{ab}\mathrm{M}(f) \, \big) : I \to I & = & F(f)
		\end{array}
\end{eq*}
and similarly for $h: \mathrm{M}X \to C$ we get
\begin{eq*} \begin{array}{rlrllll}
			h'' & : & \mathrm{M}(X)^{\mathrm{ab}} & \to & C & & \\
			& : & \mathrm{ab}\mathrm{M}(f) & \mapsto & h'(f) & = & h\big( \, \mathrm{ab}\mathrm{M}(f) \, \big)
		\end{array}
\end{eq*}
In other words, we have an isomorphism between the homsets
\begin{eq*} \mathrm{MonCat}( \, X, \mathrm{B}C \, ) \quad \cong \quad \mathrm{CMon}( \, \mathrm{M}(X)^{\mathrm{ab}}, C \, ) \end{eq*}
This isomorphism is natural in both coordinates, as for any monoidal functor $G: X \to X'$ and homomorphism $h : C \to C'$ between commutative monoids,
\begin{eq*} \mathrm{ab}\mathrm{M}( \, \mathrm{B}h \circ F \circ G \, ) \quad = \quad \mathrm{ab}\mathrm{M}(\mathrm{B}h) \circ \mathrm{ab}\mathrm{M}(F) \circ \mathrm{ab}\mathrm{M}(G) \quad = \quad h \circ \mathrm{ab}\mathrm{M}(F) \circ \mathrm{ab}\mathrm{M}(G) \end{eq*}
and so the diagram
\begin{eq*} \begin{tikzcd}
\mathrm{MonCat}(X, \mathrm{B}C) \ar[dd, "\mathrm{B}h \circ \_ \circ G"'] \ar[rr, "\sim"] & & \mathrm{CMon}\big( \, \mathrm{M}(X)^{\mathrm{ab}}, C \, \big) \ar[dd, "h \circ \_ \circ \mathrm{ab}\mathrm{M}G"] \\
& & \\
\mathrm{MonCat}(X', \mathrm{B}C') \ar[rr, "\sim"] & & \mathrm{CMon}\big( \, \mathrm{M}(X')^{\mathrm{ab}}, M' \, \big) 
\end{tikzcd} \end{eq*}
commutes. Therefore, $\mathrm{M}(\, \_ \,)^{\mathrm{ab}} \dashv \mathrm{B}$.
\end{proof}

\cref{Moradj} seems at first glance very similar to \cref{Obadj,concompadj}. However, our goal was to discover the relationship between the morphisms of $\mathbb{G}_n$ and $L\mathbb{G}_n$, paralleling what we did in \cref{Zobj,Zconcomp}, and in that regard $\mathrm{M}$ falls short in two very important ways. 

\begin{enumerate}
\item What we really wanted to have was an adjunction involving $\mathrm{E}G_G\mathrm{Alg}_{S}$, not $\mathrm{MonCat}$. This is because our previous methodology involved applying our left adjoint functors to $\eta$ and then using its initial property to factor various maps through $L\mathbb{G}_n$. But $\eta$ is an initial object in $(\mathbb{G}_n \downarrow \mathrm{inv})$, and so we only know how to use it to factor \emph{algebra} maps $\mathbb{G}_n \to X_{\mathrm{inv}}$, and not general monoidal functors. 
\item Even if we do find a way to use this adjunction to extract information about $L\mathbb{G}_n$, it will not be the monoid $\mathrm{Mor}(L\mathbb{G}_n)$ we were originally after, only a strange abelianised version where tensor product and composition coincide.  
\end{enumerate}

Unfortunately, this adjunction seems to be the best we can do. The only general method for assigning an $\mathrm{E}G$-action to the monoidal category $\mathrm{B}C$ for all $C$ is to set all of its action morphisms $\alpha(g; \mathrm{id}_I, ..., \mathrm{id}_I)$ to be $\mathrm{id}_I$. This would then cause the homomorphism $\mathrm{M}X \to C$ corresponding to any algebra map $X \to \mathrm{B}C$ to be the zero map if $X$ has only action morphisms. Given \cref{Gnmapsaction}, this is clearly no use. However, it turns out that this approach is fixable. To that end, we will spend the bulk of the next two chapters directly addressing problems 1 and 2. 

For now though, we will make one last small alteration to our plan going forward. Instead of working directly with the functor $\mathrm{M}(\, \_ \,)^{\mathrm{ab}}: \mathrm{MonCat} \to \mathrm{CMon}$, we will instead focus on its composite with the group completion functor, $( \, \_ \, )^{\mathrm{gp}} : \mathrm{CMon} \to \mathrm{Ab}$. It may not be clear yet why we would choose to do this, but over the next couple of chapters we will frequently find ourselves having to forming quotients of certain algebraic objects. If we were to stick with the functor $\mathrm{M}$ these would all be commutative monoid quotients, whereas by making the switch to $\mathrm{M}(\, \_ \,)^{\mathrm{gp},\mathrm{ab}}$ they will be abelian groups instead, which are far easier to work with. Also, notice that since the process of group completion is left adjoint to the forgetful functor $\mathrm{Ab} \to \mathrm{CMon}$, its composite with the left adjoint $\mathrm{M}(\, \_ \,)^{\mathrm{ab}}$ will be a left adjoint functor too. Thus with this new functor we will be able use all of the same important properties that we would have done with $\mathrm{M}(\, \_ \,)^{\mathrm{ab}}$, such as the preservation of colimits. Moreover, while we won't prove this for some time, it turns out that the morphisms of $L\mathbb{G}_n$ actually form a group under tensor product. This means that whatever method we would have used to recover $\mathrm{Mor}(L\mathbb{G}_n)$ from $\mathrm{M}(L\mathbb{G}_n)^{\mathrm{ab}}$ will still let us recover $\mathrm{Mor}(L\mathbb{G}_n) = \mathrm{Mor}(L\mathbb{G}_n)^{\mathrm{gp}}$ from $\mathrm{M}(L\mathbb{G}_n)^{\mathrm{gp},\mathrm{ab}}$.

Before we move on, we should spend a little time thinking about this new functor $\mathrm{M}(\, \_ \,)^{\mathrm{gp},\mathrm{ab}}$. Specifically, we might ask in what order we have to carry out its constituent parts: the collapsing of $\circ$ and $\otimes$ into a single operation, group completion, and abelianisation. It is a well known fact that group completion and abelianisation commute:
\begin{eq*} \begin{tikzcd}
\mathrm{Mon} \ar[rr, "(\, \_ \,)^{\mathrm{gp}}"] \ar[d, "(\, \_ \,)^{\mathrm{ab}}"'] & & \mathrm{Grp} \ar[d, "(\, \_ \,)^{\mathrm{ab}}"] \\
\mathrm{CMon} \ar[rr, "(\, \_ \,)^{\mathrm{gp}}"] & & \mathrm{Ab}
\end{tikzcd} \end{eq*}
Indeed, we already assume this when talking of `the' canonical map $\mathrm{M}(X)^{\mathrm{gp},\mathrm{ab}}$. But a more interesting question is whether it matters if we choose to group complete or abelianise the tensor product of a monoidal category before or after we collapse its morphisms.

\begin{lem}\label{Morder} For any monoidal category $X$, define
\begin{eq*} \begin{array}{rll} 
			\mathrm{M}_{\mathrm{gp}}(X) & \cong & \bigquotient{\mathrm{Mor}(X)^{\mathrm{gp}}}{\mathrm{gp}(f' \circ f) \sim \mathrm{gp}(f' \otimes f)} \\[\bigskipamount]
			\mathrm{M}_{\mathrm{ab}}(X) & \cong & \bigquotient{\mathrm{Mor}(X)^{\mathrm{ab}}}{\mathrm{ab}(f' \circ f) \sim \mathrm{ab}(f' \otimes f)}
		\end{array}
\end{eq*} 
Then
\begin{eq*} \mathrm{M}_{\mathrm{gp}}(X) \quad = \quad \mathrm{M}(X)^{\mathrm{gp}}, \quad \quad \quad \mathrm{M}_{\mathrm{ab}}(X) \quad = \quad \mathrm{M}(X)^{\mathrm{ab}} \end{eq*}
\end{lem}
\begin{proof}
Consider the following commutative diagram
\begin{eq*} \begin{tikzcd}
& \mathrm{M}(X) \ar[rr, "\mathrm{gp}"] \ar[ddrr, dashed, "v", near start] & & \mathrm{M}(X)^{\mathrm{gp}} \ar[dd, shift left, dashed, "u'"] \\
\mathrm{Mor}(X) \ar[ru, "\mathrm{M}"] \ar[rd, "\mathrm{gp}"'] & & & \\
& \mathrm{Mor}(X)^{\mathrm{gp}} \ar[rr, "\mathrm{M}"'] \ar[rruu, dashed, "u"', near start] & & \mathrm{M}_{\mathrm{gp}}(X) \ar[uu, shift left, dashed, "v'"]
\end{tikzcd} \end{eq*}
Here all of the solid arrows are the respective canonical homomorphisms.

Starting from the left, the top edge of the diagram is a map coming out of $\mathrm{Mor}(X)$ and going into a group, and so by the universal property of the group completion there is a unique homomorphism $u$ factoring it through $\mathrm{Mor}(X)^{\mathrm{gp}}$. But now this $u$ is a map out of $\mathrm{Mor}(X)^{\mathrm{gp}}$ and into group where tensor product and compositon are equal, and so by the universal property of the quotient this factors once more through the map $u'$. On the other hand, the bottom edge of the diagram will factor through the map $v$ because of the collapsed morphisms property, and then through the map $v'$ due to the group completion property. Then this diagram says that
\begin{eq*} \begin{array}{rll}
			v' \circ u' \circ \mathrm{gp} \circ \mathrm{M} & = & v' \circ u' \circ u \circ \mathrm{gp} \\
			& = & v' \circ \mathrm{M} \circ \mathrm{gp} \\
			& = & u \circ \mathrm{gp} \\
			& = & \mathrm{gp} \circ \mathrm{M}
		\end{array}
\end{eq*}
But $\mathrm{M}: \mathrm{Mor}(X) \to \mathrm{M}(X)$ is the map associated with a quotient, and so it is an epimorphism. Thus we can cancel it out on the right, leaving just
\begin{eq*} v' \circ u' \circ \mathrm{gp} \quad = \quad \mathrm{gp} \end{eq*}
Then from this we can conclude that for any $\mathrm{M}(f) \in \mathrm{M}(X)$,
\begin{eq*} \begin{array}{rcccl}
			v'u'\big( \, \mathrm{gp}\mathrm{M}(f) \, \big) & = & \mathrm{gp}\mathrm{M}(f) \\
			v'u'\big( \, \mathrm{gp}\mathrm{M}(f)^* \, \big) & = & v'u'\big( \, \mathrm{gp}\mathrm{M}(f) \, \big)^* & = & \mathrm{gp}\mathrm{M}(f)^*
		\end{array}
\end{eq*} 
All elements of $\mathrm{M}(X)^{\mathrm{gp}}$ can be written as $\mathrm{gp}\mathrm{M}(f)$ or $\mathrm{gp}\mathrm{M}(f)^*$ for at least one $f$, so this really says that $v' \circ u'$ is the identity homomorphisms on $\mathrm{M}(X)^{\mathrm{gp}}$. 

A completely analagous argument can also be by made starting from the bottom edge of the diagram instead, and then concluding that $u' \circ v' = \mathrm{id}_{\mathrm{M}_{\mathrm{gp}}(X)}$. Furthermore, we can construct another diagram using the universal property of the abelianisation,
\begin{eq*} \begin{tikzcd}
& \mathrm{M}(X) \ar[rr, "\mathrm{ab}"] \ar[ddrr, dashed, "v''", near start] & & \mathrm{M}(X)^{\mathrm{ab}} \ar[dd, shift left, dashed, "u'''"] \\
\mathrm{Mor}(X) \ar[ru, "\mathrm{M}"] \ar[rd, "\mathrm{ab}"'] & & & \\
& \mathrm{Mor}(X)^{\mathrm{ab}} \ar[rr, "\mathrm{M}"'] \ar[rruu, dashed, "u''"', near start] & & \mathrm{M}_{\mathrm{ab}}(X) \ar[uu, shift left, dashed, "v'''"]
\end{tikzcd} \end{eq*}
and then through a series of analagous arguments conclude that $v''' \circ u''' = \mathrm{id}_{\mathrm{M}(X)^{\mathrm{ab}}}$ and $u''' \circ v''' = \mathrm{id}_{\mathrm{M}_{\mathrm{ab}}(X)}$. All together, these yield the two isomorphisms given in the statement of the proposition.
\end{proof}

In other words, we do not need to worry about order of operations when using the left adjoint functor $\mathrm{M}(\, \_ \,)^{\mathrm{gp},\mathrm{ab}}$. This is very convenient, and later on when we actually need to evalute particular $\mathrm{M}(X)^{\mathrm{gp},\mathrm{ab}}$, we will use this fact to carry out the calculation in whichever order proves easiest. 

\chapter{Free invertible algebras as cokernels}
\label{cokeralgebra}

In the previous chapter, we made progress towards understanding the structure of $L\mathbb{G}_n$ by showing that the algebra was an initial object in a certain comma category. Specifically, we saw that the map $\eta: \mathbb{G}_n \to L\mathbb{G}_n$ is initial among all $\mathrm{E}G$-algebra maps $\mathbb{G}_n \to X_{\mathrm{inv}}$. This fact is the rigourous way of expressing a fairly obvious intuition about $L\mathbb{G}_n$ --- that we should expect the free algebra on $n$ invertible objects to be like the free algebra on $n$ objects, except that its objects are invertible.

However, this not the only way of thinking about $L\mathbb{G}_n$. Consider for a moment the free $\mathrm{E}G$-algebra on $2n$ objects, $\mathbb{G}_{2n}$. Intuitively, if we were to take this algebra and then enforce upon it the extra relations $z_{n+1} = z_1^*, ..., z_{2n} = z_n^*$, then we would be changing it from a structure with $2n$ independent generators into one with $n$ indepedent generators and their inverses. That is, there seems to be a natural way to think about $L\mathbb{G}_n$ as a quotient of the larger algebra $\mathbb{G}_{2n}$. In this chapter we will work towards making this idea precise, and then examine some of its consequences. Together with the information we have already gleaned from the initial object persepctive, this will then provide us with a complete description of the algebra $L\mathbb{G}_n$.

\section{$L\mathbb{G}_n$ as a cokernel} 

We'll begin with some definitions.

\begin{defn}\label{qdef} Let $\delta$ be the map of $\mathrm{E}G$-algebras defined on generators by
\begin{eq*} \begin{array}{rlrlll}
			\delta & : & \mathbb{G}_{2n} & \to & \mathbb{G}_{2n} \\
			& : & z_{i} & \mapsto & z_i \otimes z_{n+i} \\
			& : & z_{n+i} & \mapsto & z_{n+i} \otimes z_i			
		\end{array}
\end{eq*}
for $1 \le i \le n$. We will also denote by $q: \mathbb{G}_{2n} \to \mathrm{coker}(\delta)$ the cokernel this map.
\end{defn}

Note that the above definition does actually make sense. The given descriptions of $\delta$ is enough to specify it uniquely because $\mathbb{G}_{2n}$ is the free $\mathrm{E}G$-algebra on $2n$ objects, and hence algebra maps $\mathbb{G}_{2n} \to \mathbb{G}_{2n}$ are canonically isomorphic to functions $\{z_1, ..., z_{2n}\} \to \mathrm{ob}(\mathbb{G}_{2n})$. Also we can be sure that the map $q$ exists, because $\mathrm{E}G\mathrm{Alg}_S$ is a locally finitely presentable category and thus has all finite colimits.

The goal of this approach will be show that $\mathrm{coker}(\delta)$ is in fact that same algebra as $L\mathbb{G}_n$. In order to do this, it would help if we could easily compare $q: \mathbb{G}_{2n} \to \mathrm{coker}(\delta)$ to our initial object $\eta: \mathbb{G}_{2n} \to L\mathbb{G}_n$. In other words, we really want to show that $q$ is an object of $(\mathbb{G}_n \downarrow \mathrm{inv})$ --- that $\mathrm{coker}(\delta)$ has only invertible objects. This can be done using the adjunction we found in \cref{Obadj}.

\begin{prop}\label{Qobj} The object monoid of $\mathrm{coker}(\delta)$ is $\mathbb{Z}^{*n}$, and the restriction of $q$ to objects $\mathrm{Ob}(q): \mathrm{Ob}(\mathbb{G}_{2n}) \to \mathrm{Ob}(\mathrm{coker}(\delta))$ is the monoid homomorphism defined on generators as
\begin{eq*} \begin{array}{rlrlll}
			\mathrm{Ob}(q) & : & \mathbb{N}^{\ast 2n} & \to & \mathbb{Z}^{\ast n} \\
			& : & z_i & \mapsto & z_i  \\
			& : & z_{n+i} & \mapsto & z_i^*		
		\end{array}
\end{eq*}
\end{prop}
\begin{proof}
Consider $\mathrm{Ob}(\delta)$, the restrictions on objects of the algebra maps $\delta: \mathbb{G}_{2n} \to \mathbb{G}_{2n}$. By \cref{Gnobj}, this is a monoid homomorphism $\mathbb{N}^{\ast 2n} \to \mathbb{N}^{\ast 2n}$, and since $\mathrm{Mon}$ is cocomplete it too must have a cokernel. This will be a new homomorphism whose source is $\mathbb{N}^{\ast 2n}$ and whose target is the quotient of $\mathbb{N}^{\ast 2n}$ by the relations $\mathrm{Ob}(\delta)(x) = I$. Remembering \cref{qdef}, and that $\mathbb{N}^{\ast 2n}$ is the free monoid on $2n$ generators, this quotient monoid will have the following presentation:
\begin{eq*}\begin{array}{ll}
			\text{Generators:} & z_1, \, ..., \, z_{2n} \\
			\text{Relations:} & z_i \otimes z_{n+i} = I, \\
			& z_{n+i} \otimes z_i = I
		\end{array}
\end{eq*}
This is just the same as
\begin{eq*}\begin{array}{ll}
			\text{Generators:} & z_1, \, ..., \, z_{2n} \\
			\text{Relations:} & z_{n+i} = z_i^*, \\
		\end{array}
\end{eq*}
which is the presentation of $\mathbb{Z}^{\ast n}$. 

But by \cref{Obadj}, $\mathrm{Ob}$ is a left adjoint and hence preserves all colimits. Thus the cokernel of $\mathrm{Ob}(\delta)$ is just the underlying homomorphism of the cokernel of $\delta$. Therefore $\mathrm{Ob}(\mathrm{coker}(\delta)) = \mathbb{Z}^{\ast n}$, and $\mathrm{Ob}(q)$ is the quotient map $\mathbb{N}^{\ast 2n} \to \mathbb{Z}^{\ast n}$ sending $z_i \mapsto z_i$ and $z_{n+i} \mapsto z_i^*$ for $1 \le i \le n$.
\end{proof}

An immediate corollary of \cref{Qobj} is that every object of the cokernel algebra $\mathrm{coker}(\delta)$ is invertible. Thus $q: \mathbb{G}_{2n} \to \mathrm{coker}(\delta)$ is an object of the category $(\mathbb{G}_n \downarrow \mathrm{inv})$, and hence we can use the initiality of $\eta$ to determine the following result:

\begin{prop}\label{coker} Let $i: \mathbb{G}_n \to \mathbb{G}_{2n}$ be the inclusion of $\mathrm{E}G$-algebras defined on generators by $i(z_i) = z_i$. Then $i \circ q$ is an initial object of $(\mathbb{G}_n \downarrow \mathrm{inv})$. In particular, this means that
\begin{eq*} \mathrm{coker}(\delta) \quad \cong \quad L\mathbb{G}_n \end{eq*}
\end{prop}
\begin{proof}
Let $\psi: \mathbb{G}_n \to X$ be an arbitrary object of $(\mathbb{G}_n \downarrow \mathrm{inv})$. Since $\mathbb{G}_n$ is the free $\mathrm{E}G$-algebra on $n$ objects, we can use it and $\psi$ to define a new map, $\psi^*: \mathbb{G}_n \to X$, which takes the values
\begin{eq*} \psi^*(z_i) \quad := \quad \psi(z_i)^* \end{eq*}
on generators. Clearly $psi^*$ is also an object of $(\mathbb{G}_n \downarrow \mathrm{inv})$, and so using these two we can define a new map, $\psi + \psi^*$, via the universal property of the colimit:
\begin{eq*} \begin{tikzcd}
& \mathbb{G}_n + \mathbb{G}_n \ar[dd, dashed, "\psi + \psi^*"] & \\
\mathbb{G}_n \ar[ur, hookrightarrow, "i"] \ar[dr, "\psi"'] & & \mathbb{G}_n \ar[ul, hookrightarrow, "i'"'] \ar[dl, "\psi^*"] \\
& X & 
\end{tikzcd} \end{eq*}
But because $\mathbb{G}_n$ is the free algebra on $n$ objects, and the free functor $F : \mathrm{Cat} \to \mathrm{E}G\mathrm{Alg}_S$ is a left adjoint and thus preserves colimits, we must have
\begin{eq*} \begin{array}{rll}
		\mathbb{G}_n + \mathbb{G}_n & = & F(\{ z_1, ..., z_n\}) + F(\{ z'_1, ..., z'_n\}) \\
		& = & F( \, \{ z_1, ..., z_n\} + \{ z'_1, ..., z'_n\} \, ) \\
		& = & F(\{ z_1, ..., z_{2n} \}) \\
		& = & \mathbb{G}_{2n} 
		\end{array}
\end{eq*}
This means that we can compose $\psi + \psi^*: \mathbb{G}_{2n} \to X$ with the maps $\delta: \mathbb{G}_{2n} \to  \mathbb{G}_{2n}$, though we need to be careful to specify exactly which inclusions we used in the definition of $\psi + \psi^*$. Suppose that the lefthand inclusion is $i$, the one given in the statement of the proposition, and the other is defined by the assignment $z_i \mapsto z_{i+n}$. Then for $1 \leq i \leq n$,
\begin{eq*} \begin{array}{rll}
			(\psi + \psi^*)\delta(z_i) & = & (\psi + \psi^*)(z_i \otimes z_{n+i}) \\
			& = & \psi(z_i) \otimes \psi(z_i)^* \\
			& = & I \\
			& & \\
			(\psi + \psi^*)\delta(z_{n+i}) & = & (\psi + \psi^*)(z_{n+i} \otimes z_i) \\
			& = & \psi(z_i)^* \otimes \psi(z_i) \\
			& = & I
		\end{array}
\end{eq*}
That is, $(\psi + \psi^*) \circ \delta = I$. But we've already defined $q: \mathbb{G}_{2n} \to \mathrm{coker}(\delta)$ to be the cokernel of $\delta$, the universal map with this property, and so there must exist a unique $\mathrm{E}G$-algebra map $u: \mathrm{coker}(\delta) \to X$ making the righthand triangle below diagram commute:
\begin{eq*} \begin{tikzcd}
\mathbb{G}_n \ar[rr, hookrightarrow, "i"] \ar[ddrr, "\psi"'] & & \mathbb{G}_{2n} \ar[rr, "q"] \ar[dd, "\psi + \psi^*", near start] & & \mathrm{coker}(\delta) \ar[ddll, "u"] \\
& & & & \\ 
& & X & &
\end{tikzcd} \end{eq*}
The other triangle commutes by the definition of $\psi + \psi^*$, and so together the diagram tells us that for any object $\psi$ of $(\mathbb{G}_n \downarrow \mathrm{inv})$, there exists at least one morphism $u$ in $(\mathbb{G}_n \downarrow \mathrm{inv})$ going from $q \circ i$ to $\psi$. 

Next, let $v: \mathrm{coker}(\delta) \to X$ be an arbitrary morphism $q \circ i \to \psi$ in $(\mathbb{G}_n \downarrow \mathrm{inv})$. By definition, this means that
\begin{eq*}\begin{array}{rll}
			\psi & = & vqi \\
			\implies \quad \psi + \psi^* & = & vqi + (vqi)^* 
		\end{array}
\end{eq*}
Also, for $1 \leq i \leq n$ we have
\begin{eq*}\begin{array}{rcrllcccl}
			q(z_i) \otimes q(z_{n+i}) & = & q(z_{i-n} \otimes z_i) & = & q\delta(z_i) & = &  I \\
			q(z_{n+i}) \otimes q(z_i) & = & q(z_i \otimes z_{n+i}) & = & q\delta(z_{n+i}) & = & I \\
			& \implies & q(z_{n+i}) & = & q(z_i)^* & & & &
		\end{array}
\end{eq*}
Therefore,
\begin{eq*}\begin{array}{rll}
			(\psi + \psi^*)(z_i) & = & \big( vqi + (vqi)^* \big)(z_i) \\
			& = & vqi(z_i) \\
			& = & vq(z_i) \\
		\end{array}
\end{eq*}
\begin{eq*} \begin{array}{rll}
			(\psi + \psi^*)(z_{n+i}) & = & \big( vqi + (vqi)^* \big)(z_{n+i}) \\
			& = & vqi(z_i)^* \\
			& = & v \big( q(z_i)^* \big) \\
			& = & vq(z_{n+i})
		\end{array}
\end{eq*}
or in other words $\psi + \psi^* = v \circ q$ for any morphism $v: q \circ i \to \psi$ in $(\mathbb{G}_n \downarrow \mathrm{inv})$. But this is the property that the map $u$ was supposed to satisfy uniquely, and thus it must be the only morphism $q \circ i \to \psi$ in $(\mathbb{G}_n \downarrow \mathrm{inv})$. Therefore $q \circ i$ is an initial object, and hence it is isomorphic in $(\mathbb{G}_n \downarrow \mathrm{inv})$ to any other initial object, such as $\eta$. It follows that the targets of these two maps, $\mathrm{coker}(\delta)$ and $L\mathbb{G}_n$ respectively, are isomorphic as $\mathrm{E}G$-algebras.
\end{proof}

It's worth noting that we have not given a method for actually taking cokernels in $\mathrm{E}G\mathrm{Alg}_S$, and so \cref{coker} doesn't immediately provide an explicit description for the whole of $L\mathbb{G}_n$. However, it does offer us another way to extract partial information, like what we were doing in \cref{initialalgebra}. Consider \cref{Qobj}; now that we know that $\mathrm{coker}(\delta)$ is actually $L\mathbb{G}_n$, the statement of this proposition is just the same as that of \cref{Zobj}. But the proof of the former uses the ability of cokernels to preserve left adjoint functors, rather than any of the initial algebra and group completion properties that appear in the latter.

Of course, by \cref{coker} the fact that $q$ is a cokernel is equivalent to it being initial, and so while they may not look it at first glance, these two approaches are secretly the same. Thus from now on whenever we are trying to determine some aspect of $L\mathbb{G}_n$, we will make sure to take a look at both methods, just in case there are some properties of our free algebra which are more readily apparent from one description than another.

\section{$L\mathbb{G}_n$ as a surjective cokernel}

One immediate consequence our new cokernel perspective of $L\mathbb{G}_n$ is that, since left adjoint functor all preserve colimits, \cref{Obadj,concompadj} now both imply results about the partial surjectivity of this new map $q$. The former says that since $\mathrm{Ob}(q)$ is a cokernel map of monoids, and hence that every object of $L\mathbb{G}_n$ is the image under $q$ of some object of $\mathbb{G}_{2n}$; the latter says a similar thing for connected components. From this one might guess that $q$ is will just turn out to be a surjective map of $\mathrm{E}G$-algebras, and indeed this is the case. Moreover, much as \cref{Zobj} is an analogue of \cref{Gnobj}, the fact that $q$ is surjective on morphisms means that there will be a result analagous to \cref{Gnmapsaction} as well. That is, since every morphism of $\mathbb{G}_{2n}$ is an action morphism, and since $\mathrm{E}G$-algebra maps always send action morphisms to action morphisms, if $q$ is surjective then every morphism of $L\mathbb{G}_n$ is also an action morphism. 

Unfortunately, we can not go about proving that $q$ is surjective on morphisms by a similar adjunction technique, since this best we have is the one from \cref{Moradj} and it will only tell us about the map $\mathrm{Mor}(q)^{\mathrm{gp}, \mathrm{ab}}$. However, there is a general result about the coequalisers of $\mathrm{E}G$-algebras that we can prove to get us around this.

\begin{prop}\label{coeqsurj} Let $\phi, \phi' : X \to Y$ be a pair of parallel $\mathrm{E}G$-algebra maps, and $k: Y \to Z$ their coequalizer in $\mathrm{E}G\mathrm{Alg}_S$. If the monoid $\mathrm{Ob}(Z)$ is also a group, then the functor $k$ is surjective.
\end{prop}
\begin{proof}
We begin by mirroring the proof of \cref{Qobj}. We know that the functor $\mathrm{Ob} : \mathrm{E}G\mathrm{Alg}_S \to \mathrm{Mon}$ is a left adjoint, by \cref{Obadj}, and thus preserves all colimits. It follows that the monoid homomorphism $\mathrm{Ob}(k): \mathrm{Ob}(Y) \to \mathrm{Ob}(Z)$ is the coequaliser of the parallel pair $\mathrm{Ob}(\phi), \mathrm{Ob}(\phi') : \mathrm{Ob}(X) \to \mathrm{Ob}(Y)$ in $\mathrm{Mon}$, or in other words
\begin{eq*} \mathrm{Ob}(Z) \quad = \quad \bigquotient{\mathrm{Ob}(Y)}{\sim}\end{eq*}
where $\sim$ is the relation defined by
\begin{eq*}\mathrm{Ob}(\phi)(y) \sim \mathrm{Ob}(\phi')(y), \quad \quad \quad a \sim a', b \sim b' \implies ab \sim a'b' \end{eq*}
The map $\mathrm{Ob}(k): \mathrm{Ob}(Y) \to \mathrm{Ob}(Y)/\sim$ is then clearly surjective.

Next, let $f: v \to w$ and $f' : w' \to v'$ be any two morphisms of the algebra $Y$ for which $k(f)$ and $k(f')$ are composable in $Z$. Since these maps are composable we know that $k(w)$ and $k(w')$ must be the same object of $Z$, and since $Z$ is a group we know this object has an inverse $k(w)^* = k(w')^*$. So by the surjectivity of $k$ we can find another object $y$ of $Y$ for which $k(y) = k(w)^*$. Using this, define the morphism $h: x \to x'$ to be the tensor product $f' \otimes \mathrm{id}_y \otimes f$. Then
\begin{eq*} \begin{array}{rll}
		k(h) & = & k(f' \otimes \mathrm{id}_y \otimes f) \\
		& = & k(f') \otimes \mathrm{id}_{k(y)} \otimes k(f) \\
		& = & k(f') \otimes \mathrm{id}_{k(w)^*} \otimes k(f)
		\end{array}
\end{eq*}
But by \cref{tenscomp}, this is really just the composite $k(f') \circ k(f)$. Thus the set of morphisms of $Z$ which are images of morphisms of $Y$ is closed under composition. 

So now consider $k(Y)$, the subcategory of $Z$ that contains every object $x'$ for which there exists $x$ in $Y$ with $k(x) = x'$, and every morphism $f'$ for which there exists $f$ in $Y$ with $q(f) = f'$. We know that the morphisms of $k(Y)$ are closed under composition, and so this is indeed a well-defined category. Moreover, for any collection of morphisms $f'_1, ..., f'_m$ of $k(Y)$ we'll have
\begin{eq*} \begin{array}{rll}
 			\alpha_{Z}(g; f'_1, ..., f'_m) & = & \alpha_Z\big( \, g \, ; \, k(f_1), ..., k(f_m) \, \big) \\
			& = & k \big( \, \alpha_{Y}(g; f_1, ..., f_m) \, \big) \\
			& \in & k(Y) 
		\end{array}
\end{eq*}
for some $f_1, ..., f_m$, since $k$ is a map of $\mathrm{E}G$-algebras. Thus $k(Y)$ is also a well-defined sub-$\mathrm{E}G$-algebra of $Z$. There is also clearly a canonical map $k': Y \to k(Y)$, the unique surjective map of $\mathrm{E}G$-algebras with the property that $k'(x) = k(x)$ for any object $x$ and $k'(f) = k(f)$ for any morphism $f$. If we denote by $i$ the evident inclusion of algebras $i: k(Y) \hookrightarrow Z$, then these maps are related by the fact that $i \circ k' = k$.
\begin{eq*} \begin{tikzcd}
& & X \ar[dd, bend right, "\phi"'] \ar[dd, bend left, "\phi'"] & & \\
& & & & \\
& & Y \ar[ddll, "k'"'] \ar[dd, "k"] \ar[ddrr, "j"] & & \\ 
& & & & \\
k(Y) \ar[rr, hookrightarrow, "i"] & & Z \ar[rr, "u"] & & U
\end{tikzcd} \end{eq*}
Given all of this, let $j: Y \to U$ be any map of $\mathrm{E}G$-algebras with the property that $j \circ \phi = j \circ \phi'$. Since $h$ is the coequaliser of $\phi$ and $\phi'$, it follows that there exists a unique map $u:  Y \to U$ such that $j = u \circ k$. This means that $j = u \circ i \circ k'$, and hence there is obviously at least one map, $u \circ i$, which lets us factors $j$ through $k'$. But for any other map $v: k(Y) \to U$ that factors $j$ like this, we'll have
\begin{eq*} \begin{array}{rrll}
			& v \circ k' & = & j \\
			& & = & u \circ i \circ k' \\
			\implies \quad & v & = & u \circ i
		\end{array}
\end{eq*}
because $k'$ is surjective, and thus $u \circ i$ is the unique map with this property. That is, $k'$ is also a coequaliser of $\phi$ and $\phi'$. But colimits are always unique up to a unique isomorphism, and so there should be a unique invertible map $k(Y) \to Z$ factoring $k$ through $k'$. This is clearly just the inclusion $i$, and as a result $k(Y) = Z$ and $k' = k$. In other words, the map coequaliser mapm $k$ is surjective. 
\end{proof}

Because a cokernel of a morphism is just its coequaliser with the zero map, and since we know that the objects of $L\mathbb{G}_n$ form a group, we can immediately apply this result to the functor $q$.

\begin{cor}\label{qsurj} The cokernel map $q: \mathbb{G}_{2n} \to L\mathbb{G}_n$ is surjective.
\end{cor}

As we noted earlier, knowing that $q$ is surjective will now allow use to take \cref{Gnmapsaction}, a statement about the morphisms $\mathbb{G}_{2n}$, and extend the result onto $L\mathbb{G}_n$:

\begin{lem} \label{allmapsaction} Every morphism in $L\mathbb{G}_n$ can be expressed as $\alpha_{L\mathbb{G}_n}(g; \mathrm{id}_{x_1}, ..., \mathrm{id}_{x_m})$, for some $g \in G(m)$ and $x_i \in \{z_1, ..., z_n, z_1^*, ..., z_n^* \}$.
\end{lem}
\begin{proof}
Let $f$ be an arbitrary morphism in $L\mathbb{G}_n$. By surjectivity of $q$, there must exist at least one morphism $f'$ in $\mathbb{G}_{2n}$ such that $q(f') = f$, and from \cref{Gnmapsaction} we know that this $f'$ can be expressed uniquely as $\alpha(g; \mathrm{id}_{x'_1}, ..., \mathrm{id}_{x'_m})$ for some $g \in G(m)$ and $x'_i \in \{z_1, ..., z_{2n} \}$. Thus, because $q$ is a map of $\mathrm{E}G$-algebras, we will have
\begin{eq*}\begin{array}{rll}
			f & = & q(f') \\
			& = & q\big( \, \alpha_{\mathbb{G}_{2n}}( \, g \, ; \, \mathrm{id}_{x'_1}, ..., \mathrm{id}_{x'_m} \, ) \, \big) \\
			& = & \alpha_{L\mathbb{G}_n}( \, g \, ; \, \mathrm{id}_{q(x'_1)}, ..., \mathrm{id}_{q(x'_m)} \, ) 
		\end{array}
\end{eq*}
Therefore there is at least one collection of $x_i = q(x'_i)$ for which the statement of the proposition holds.
\end{proof}

\cref{allmapsaction} formalises a certain intuition about how the functor $L$ should act on algebras, the idea that a `free' structure really shouldn't have any `superfluous' components, only whatever data is absolutely required for it to be well-defined. In the case of $L\mathbb{G}_n$, we have proven that the only morphisms contained in the free $\mathrm{E}G$-algebra on invertible objects are $\mathrm{E}G$-action morphisms. However, while this is very similar to what we have in the non-invertible case it should be stressed that \cref{allmapsaction} does \emph{not} prove that the morphisms of $L\mathbb{G}_n$ have \emph{unique} representations $\alpha(g; \mathrm{id}_{w_1}, ..., \mathrm{id}_{w_m})$, as morphisms of $\mathbb{G}_n$ do.

Before moving on, its worth noting that in the proof of \cref{coeqsurj} we never needed to use the fact that $\phi$, $\phi'$ and $k$ were maps of $\mathrm{E}G$-algebras, only that they were monoidal functors. Because we had assumed from the beginning that we were working in $mathrm{E}G\mathrm{Alg}_S$, we did at one point have to show that the category $k(Y)$ was an algebra, so that we could then use the universal property of $k$ in $\mathrm{E}G\mathrm{Alg}_S$, but if $k$ had just been a coequaliser in $\mathrm{MonCat}$ from the start then this part would not have been necessary. We also had to invoke \cref{Obadj} --- which says that $\mathrm{Ob}: \mathrm{E}G\mathrm{Alg}_S \to \mathrm{Mon}$ is a left adjoint --- so that we could exploit preservation of colimits. But since $\mathrm{Ob}$ clearly doesn't care about the morphisms of an algebra, it doesn't really matter whether we are applying it to an algebra in the first place. The actions of $X$, $Y$ and $Z$ just never came into play.

With that in mind, we can co-opt all of these previous proofs about $\mathrm{E}G$-algebra maps to prove the analagous statements about monoidal functors.

\begin{prop}\label{Obadjmon} Let the functors 
\begin{eq*} \mathrm{Ob} \, : \, \mathrm{MonCat} \to \mathrm{Mon}, \quad \quad \quad \mathrm{E} \, : \, \mathrm{Mon} \to \mathrm{MonCat} \end{eq*}
be defined exactly as those from \cref{Obdef,Edef}, except without the requirement that the monoidal categories be $\mathrm{E}G$-algebras. Then $\mathrm{E}$ is a right adjoint to the functor $\mathrm{Ob}$. 
\end{prop}
\begin{proof}
The same as the proof of \cref{Obadj}.
\end{proof}

\begin{prop} \label{coeqsurjmon} Let $\phi, \phi' : X \to Y$ be a pair of parallel monoidal functors, and $k: Y \to Z$ their coequalizer in $\mathrm{Mon}$. If the monoid $\mathrm{Ob}(Z)$ is also a group, then the functor $k$ is surjective.
\end{prop}
\begin{proof}
The same as the proof of \cref{coeqsurj}, but with \cref{Obadjmon} in place of \cref{Obadj}, and no reference to $k(Y)$ being a sub-$\mathrm{E}G$-algebra.
\end{proof}

\section{The classification of 2-groups}

See \cite{hda5} for more detail. 

\subsection{$n$-groups}

\begin{defn} Weak $n$-categories \end{defn}

\begin{defn} $n$-groupoids \end{defn}

\begin{defn} $n$-groups \end{defn}

\begin{example} Examples \end{example}

\subsection{Classifying 2-groups}

\begin{defn} 2-groups \end{defn}

\begin{defn} Coherent 2-groups \end{defn}

\begin{thm} \label{2gpclass} The classification of 2-groups \end{thm}

\subsection{Group cohomology}

\begin{defn} Cochain complex of groups \end{defn}

\begin{defn} Cocycles, coboundaries, and cohomology classes \end{defn}

\begin{defn} Group cohomology \end{defn}

\subsection{Proof of the classification theorem}

\begin{proof}[Proof of \cref{2gpclass}] \end{proof}

\subsection{Generalizing to 3-groups}

\section{2-group cohomology}

See \cite{picard} and \cite{kkg} for more details.

\subsection{Cohomology of categories}

\begin{defn} Cochain complex of categories \end{defn}

\begin{defn} Cocycles, coboundaries, and cohomology classes \end{defn}

\subsection{Symmetric 2-group cohomology}

\begin{defn} Symmetric 2-group cohomology \end{defn}

\begin{prop} Coherence of symmetric 2-group cohomology \end{prop}

\subsection{Braided 2-group cohomology}

\begin{defn} Braided 2-group cohomology \end{defn}

\begin{prop} Coherence of braided 2-group cohomology \end{prop} 
 

\input{3groupclass.tex}



\begin{thebibliography}{9}

\bibitem{hda5}
John C. Baez; Aaron D. Lauda.
\textit{Higher-Dimensional Algebra V: 2-Groups}.
https://arxiv.org/pdf/math/0307200.pdf 

\bibitem{picard}
K. H. Ulbrich
\textit{Group cohomology for Picard categories}
Journal of Algebra, 91 (1984), pp. 464-498 
http://dx.doi.org/10.1016/0021-8693(84)90114-5

\bibitem{kkg}
K. H. Ulbrich
\textit{Kohärenz in Kategorien mit Gruppenstruktur}
Journal of Algebra, 72 (1981), pp. 279-295
http://dx.doi.org/10.1016/0021-8693(81)90295-7

\bibitem{graphicalmon}
Peter Selinger.
\textit{A survey of graphical languages for monoidal categories}.
http://www.mscs.dal.ca/~selinger/papers/graphical.pdf

\bibitem{operadborel} 
Nick Gurski. 
\textit{Operads, tensor products, and the categorical Borel construction}. 
 arXiv:1508.04050 [math.CT].

\bibitem{bct}
Tom Leinster.
\textit{Basic Category Theory}
Cambridge Studies in Advanced Mathematics, Vol. 143, Cambridge University Press, Cambridge, 2014.
https://arxiv.org/pdf/1612.09375v1.pdf

\bibitem{eckhil}
Eckmann, B.; Hilton, P. J. 
\textit{Group-like structures in general categories. I. Multiplications and comultiplications}
Mathematische Annalen, 145 (3), pp. 227–255

\end{thebibliography}

\end{document}