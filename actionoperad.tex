\section{Operads and their algebras}

\subsection{Operads}

\begin{defn} Operads $O$ \end{defn}

\begin{defn} Action operads $G$ \end{defn}

\begin{example} The symmetric operad $S$ \end{example}

\begin{example} The braid operad $B$ \end{example}

\begin{defn} $G$-operads \end{defn}

\subsection{Operad algebras}

\begin{defn} Operad algbras \end{defn}

\begin{defn} $G$-operad algebras \end{defn}

\subsection{$\mathrm{E}G$-algebras}

\begin{defn} The $G$-operad $\mathrm{E}G$ \end{defn}

\begin{defn}\label{monaddef} The monad $\mathrm{E}G$ \end{defn}

\begin{defn} $\mathrm{E}G$-algebras \end{defn}

\begin{prop} $G$-operad algebras are monoidal categories with permutation-like structure \end{prop} 

\begin{cor} Braided monoidal categories are $G$-operad algebras \end{cor}

\begin{defn} A strict monoidal category $X$ is said to be \emph{spacial} if, for any object $x \in \mathrm{Ob}(X)$ and any endomorphism of the unit object $f: I \to I$, 
\begin{eq*} f \otimes \mathrm{id}_x = \mathrm{id}_x \otimes f \end{eq*}
\end{defn}

The motivation for the name `spacial' comes from the context of string diagrams \cite{graphicalmon}. In a string diagram, the act of tensoring two strings together is represented by placing those strings side by side. Since the defining feature of the unit object is that tensoring it with other objects should have no effect, the unit object is therefore represented diagrammatically by the absense of a string. An endomorphism of the unit thus appears as an entity with no input or output strings, detached from the rest of the diagram. In a real-world version of these diagrams, made out of physical strings arranged in real space, we could use this detachedness to grab these endomorphisms and slide them over or under any strings we please, without affecting anything else in the diagram. This ability is embodied algebraically by the equation above, and hence categories which obey it are called `spacial'.

\begin{lem}\label{spacial} All $\mathrm{E}G$-algebras are spacial. \end{lem}
\begin{proof}
Let $X$ be an $\mathrm{E}G$-algebra, and fix $x \in \mathrm{Ob}(X)$ and \( f: I \to I \). From the surjectivity of \( \pi : G(2) \to S_2 \) we know that the set $\pi^{-1}( \, (1 \, 2) \, )$ is non-empty, and from the rules for composition of action morphisms we see that for any such $g \in \pi^{-1}( \, (1 \, 2) \, )$,
\begin{eq*}\begin{array}{rll}
		\alpha( \, g \, ; \, \mathrm{id}_x, \, \mathrm{id}_I \, ) \circ \alpha( \, e_2 \, ; \, \mathrm{id}_x, \, f \, ) & = & \alpha( \, g \, ; \, \mathrm{id}_x, \, f \, ) \\
		& = & \alpha( \, e_2 \, ; \, f, \, \mathrm{id}_x \, ) \circ \alpha( \, g \, ; \, \mathrm{id}_x, \, \mathrm{id}_I \, ) \\
		\end{array}
\end{eq*}
Thus in order to obtain the result we're after, it will suffice to find a particular $g \in \pi^{-1}( \, (1 \, 2) \, )$ for which
\begin{eq*}\alpha( \, g \, ; \, \mathrm{id}_x, \, \mathrm{id}_I \, ) = \mathrm{id}_x \end{eq*}
However, since
\begin{eq*}\begin{array}{rll}
		\alpha( \, g \, ; \, \mathrm{id}_x, \, \mathrm{id}_I \, ) & = & \alpha( \, g \, ; \, \mathrm{id}_x, \, \alpha( e_0; - ) \, ) \\
		& = & \alpha( \, \mu(g; e_1, e_0) \, ; \, \mathrm{id}_x \, )
		\end{array}
\end{eq*}
all we really need is to find a $g \in \pi^{-1}( \, (1 \, 2) \, )$ for which
\begin{eq*} \mu(g; e_1, e_0) = e_1 \end{eq*}
To this end, choose an arbtrary element $h \in \pi^{-1}( \, (1 \, 2) \, )$. This $h$ probably won't obey the above equation, but we can use it to construct a new element $g$ which does. Specifically, define
\begin{eq*} k \, := \, \mu( \, h \ ; \, e_1, \, e_0 \, ) \end{eq*}
and then consider
\begin{eq*} g \, := \, h \cdot \mu(e_2; k^{-1}, e_1) \end{eq*}
To see that this is the correct choice of $g$, first note that we must have \( \pi(k) = e_1 \), since this is the only element of $S_1$. Following from that, we have 
\begin{eq*}\begin{array}{rll}
		\pi \big( \, \mu(e_2; k^{-1}, e_1) \, \big) & = & \mu \big( \, \pi(e_2) \ ; \, \pi(k^{-1}), \, \pi(e_1) \, \big) \\
		& = & \mu \big( \, e_2  \ ; \, e_1, \, e_1 \, \big) \\
		& = & e_2
		\end{array}
\end{eq*}
and hence
\begin{eq*}\begin{array}{rll}
		\pi(g) & = & \pi \big( h \cdot \mu(e_2; k^{-1}, e_1) \big) \\
		& = & \pi(h) \cdot \pi \big(\mu(e_2; k^{-1}, e_1) \big) \\
		& = & (1 \, 2) \cdot e_2 \\
		& = & (1 \, 2)
		.\end{array}
\end{eq*}
So $g$ is indeed in $\pi^{-1}( \, (1 \, 2) \, )$, and furthermore
\begin{eq*}\begin{array}{rll}
		\mu(g; e_1, e_0) & = & \mu \big( \, h \cdot \mu(e_2; k^{-1}, e_1) \ ; \, e_1, \, e_0 \, \big) \\
		& = & \mu( \, h \ ; \, e_1, \, e_0 \, ) \cdot \mu \big( \, \mu(e_2; k^{-1}, e_1) \ ; \, e_1, \, e_0 \, \big) \\
		& = & \mu( \, h \ ; \, e_1, \, e_0 \, ) \cdot \mu \big( \, e_2 \ ; \, \mu(k^{-1}; e_1), \, \mu(e_1; e_0) \, \big) \\
		& = & \mu( \, h \ ; \, e_1, \, e_0 \, ) \cdot \mu( \, e_2 \ ; \, k^{-1}, e_0 \, ) \\
		& = & k \cdot k^{-1} \\
		& = & e_1
		\end{array}
\end{eq*}
Therefore, $h \cdot \mu(e_2; k^{-1}, e_1)$ is exactly the $g$ we were looking for, and so working backwards through the proof we obtain the required result:
\begin{eq*} \begin{array}{rll}
		\mu(g; e_1, e_0) & = & e_1 \\
		\implies \quad \alpha( \, g \, ; \, \mathrm{id}_x, \, \mathrm{id}_I \, ) & = & \mathrm{id}_x \\
		& & \\
		\alpha( \, g \, ; \, \mathrm{id}_x, \, \mathrm{id}_I \, ) \circ \alpha( \, e_2 \, ; \, \mathrm{id}_x, \, f \, ) & = & \alpha( \, e_2 \, ; \, f, \, \mathrm{id}_x \, ) \circ \alpha( \, g \, ; \, \mathrm{id}_x, \, \mathrm{id}_I \, ) \\
		\implies \quad \alpha( \, e_2 \, ; \, \mathrm{id}_x, \, f \, ) & = & \alpha( \, e_2 \, ; \, f, \, \mathrm{id}_I \, )
		\end{array}
\end{eq*}
\end{proof}

\subsection{The free $\mathrm{E}G$-algebra on $n$ objects} 

Our goal for the next few chapters will be to understand the free braided monoidal category on an finite number of invertible objects. Thus, now that we have a firm grasp on action operads and their algebras, we should begin to think about the simpler free constructions they can form. We will use this extensively when calculating the invertible case later on. 

In the paper \cite{operadborel}, Gurski establishes how to contruct free $G$-operad algebras through the use of the monad $\mathrm{E}G$. What follows in this section is a quick summary of the results which will be useful for our purposes. For a more detailed treatment please refer to \cite{operadborel}.

\begin{prop}\label{freealg} There exists a free $\mathrm{E}G$-algebra on $n$ objects. That is, there is an $\mathrm{E}G$-algebra $Y$ such that for any other $\mathrm{E}G$-algebra $X$, we have an isomorphism of categories
\begin{eq*} \mathrm{E}G\mathrm{Alg}_S(Y, X) \cong X^n \end{eq*}
\end{prop}
\begin{proof}
There is an obvious forgetful 2-functor \( U: \mathrm{E}G\mathrm{Alg}_S \to \mathrm{Cat}\) sending $\mathrm{E}G$-algebras to their underlying categories. $U$ has a left adjoint, which we call the free 2-functor \( F : \mathrm{Cat} \to \mathrm{E}G\mathrm{Alg}_S \) adjoint to it. It follows immediately that
\begin{eq*}\begin{array}{rll}
		U(X)^n & = & \mathrm{Cat}(\{z_1, ..., z_n\}, U(X) ) \\
		& \cong & \mathrm{E}G\mathrm{Alg}_S( F(\{z_1, ..., z_n\}), X) 
		\end{array}
\end{eq*}
where $\{z_1, ..., z_n\}$ is any set with $n$ distinct elements. Since $X$ and $U(X)$ are obviously isomorphic as categories, this shows that $F(\{z_1, ..., z_n\})$ is the free algebra on $n$ objects as required. 
\end{proof}

\begin{defn}\label{Gndef} Let $\{ z_1, ..., z_n \}$ be an $n$-object set, which we will also consider as a discrete category. Then we will denote by $\mathbb{G}_n$ the $\mathrm{E}G$-algebra whose underlying category is $\mathrm{E}G( \{ z_1, ..., z_n \})$ and whose action
\begin{eq*} \alpha : \mathrm{E}G\big( \, \mathrm{E}G( \{ z_1, ..., z_n \}) \, \big) \to \mathrm{E}G( \{ z_1, ..., z_n \}) \end{eq*}
is the appropriate component of the multiplication natural transformation $\mu: \mathrm{E}G \circ \mathrm{E}G \to \mathrm{E}G$ of the 2-monad $\mathrm{E}G$.
\end{defn}

\begin{thm} $\mathbb{G}_n$ is the free $\mathrm{E}G$-algebra on $n$ objects. That is,
\begin{eq*}  F(\{z_1, ..., z_n\}) = \mathbb{G}_n \end{eq*}
\end{thm}
\begin{proof}
\end{proof}

\cref{Gndef} is a fairly opaque definition, so we'll spend a little time upacking it. Recall from \cref{monaddef} that $\mathrm{E}G( \{ z_1, ..., z_n \})$ is the coequalizer of the maps
\begin{eq*} \begin{tikzcd}
\coprod_{m \geq 0} \mathrm{E}G(m) \times G(m) \times \{ z_1, ..., z_n \}^m \ar[r, shift left] \ar[r, shift right] & \coprod_{m \geq 0} \mathrm{E}G(m) \times \{ z_1, ..., z_n \}^m
\end{tikzcd} \end{eq*}
that comes from the action of $G(m)$ on $\mathrm{E}G(m)$ by multiplication on the right,
\begin{eq*} \begin{array}{rll}
		\mathrm{E}G(m) \times G(m) & \to & \mathrm{E}G(m) \\
		(g, h) & \mapsto & gh \\
		( \, !: g \to g', \mathrm{id}_h \, ) & \mapsto & !: gh \to g'h
		\end{array}
\end{eq*}
and the action of $G(m)$ on $\{ z_1, ..., z_n \}^m$ by permutation,
\begin{eq*} \begin{array}{rll}
		G(m) \times \{ z_1, ..., z_n \}^m & \to & \{ z_1, ..., z_n \}^m \\
		( \, h \, ; \, x_1, ..., x_m \, ) & \mapsto & (x_{\pi(h^{-1})(1)}, ..., x_{\pi(h^{-1})(m)}) \\
		 \, (\mathrm{id}_h \, ; \, \mathrm{id}_{(x_1, ..., x_m)} \, ) & \mapsto & \mathrm{id}_{(x_{\pi(h^{-1})(1)}, ..., x_{\pi(h^{-1})(m)})}
		\end{array}
\end{eq*}

First, objects in this algebra are equivalence classes of tuples $(g; x_1, ..., x_m)$, for $g \in G(m)$ and $x_i \in \{z_1, ..., z_n\}$, under the relation
\begin{eq*} ( \, gh \, ; \, x_1, \, ..., \, x_m \, ) \sim ( \, g \, ; \, x_{\pi(h)^{-1}(1)}, \, ..., \, x_{\pi(h)^{-1}(m)} \, )\end{eq*}
Notice that using this relation we can rewrite any object uniquely in the form $[e; x_1, ..., x_m]$ for some $m \in \mathbb{N}$ and $x_i \in \{z_1, ..., z_n\}$. This means that each equivalence class is just the tensor product $x_1 \otimes ... \otimes x_m$ in the underlying monoidal category of $\mathbb{G}_n$, for some unique sequence of generators. That is, we can view the objects of $\mathbb{G}_n$ as elements of the monoid freely generated by each of the $z_i$, or in other words:

\begin{lem} \label{Gnobj} $\mathrm{Ob}(\mathbb{G}_n)$ is the free monoid on $n$ generators, $\mathbb{N}^{\ast n}$, the free product of $n$ copies of $\mathbb{N}$. \end{lem}

Similarly, the morphisms of $\mathbb{G}_n$ are the maps
\begin{eq*} (! ; \mathrm{id}_{x_1}, ..., \mathrm{id}_{x_m}) : ( g ; x_1, ..., x_m ) \to ( g' ; x_1, ..., x_m )\end{eq*}
with $g, g' \in G(m)$ and $x_i \in \{z_1, ..., z_n \}$. Using the relation $\sim$ on objects we can rewrite each of these morphisms in the form
\begin{eq*} [h ; \mathrm{id}_{y_1},...,\mathrm{id}_{y_m}] \, : \, y_1 \otimes ... \otimes y_m \, \to \, y_{\pi(h^{-1})(1)} \otimes ... \otimes y_{\pi(h^{-1})(m)} \end{eq*}
where
\begin{eq*} h = g' g^{-1}, \quad \quad y_i = x_{\pi(g^{-1})(i)} \end{eq*}
 The $\mathrm{E}G$-action of $\mathbb{G}_n$ is permutation and tensor product, and the action on morphisms is given by
\begin{eq*} \alpha( \, g \, ; \, [h_1; \mathrm{id}_{x_1}, ..., \mathrm{id}_{x_{m_1}}], \, ..., \, [h_k; \mathrm{id}_{x_1}, ..., \mathrm{id}_{x_{m_k}}] \, ) = [ \, \mu(g;h_1, .., h_k) \, ; \, \mathrm{id}_{x_1}, \, ..., \, \mathrm{id}_{x_{m_k}} \, ] \end{eq*}
Notice that using tensor product notation the object $[e; x]$ is simply $x$, and so $[e; \mathrm{id}_x] = \mathrm{id}_{[e;x]}$ should be written as $\mathrm{id}_x$. Hence by the above $[g; \mathrm{id}_{x_1}, ..., \mathrm{id}_{x_m}]$ is really just $\alpha(g; \mathrm{id}_{x_1}, ..., \mathrm{id}_{x_m})$, and so we have the following:

\begin{lem} \label{Gnmapsaction} Each morphism of $\mathbb{G}_n$ can be expressed uniquely as an action morphism $\alpha(g; \mathrm{id}_{x_1}, ..., \mathrm{id}_{x_m})$, for some $g, g' \in G(m)$ and $x_i \in \{z_1, ..., z_n \}$. \end{lem}

From this, we can also determine $\mathbb{G}_n$'s connected components. 

\begin{prop}\label{Gnconcomp} The connected components of $\mathbb{G}_n$ are $\mathbb{N}^n$, with the assignment $[ \,\, ] : \mathbb{N}^{*n} \to \mathbb{N}^n$ of objects to their component being the quotient map of abelianisation.
\end{prop}
\begin{proof}
By \cref{Gnmapsaction}, all morphisms in $\mathbb{G}_n$ can be written uniquely as $\alpha(g; \mathrm{id}_{x_1}, ..., \mathrm{id}_{x_m})$, for some $g \in G(m)$ and $x_i \in \{z_1, ..., z_n \}$, the set of generators of $\mathbb{N}^{*n}$. Since maps of this form have source $x_1 \otimes ... \otimes x_m$ and target $x_{\pi(g^{-1})(1)} \otimes ... \otimes x_{\pi(g^{-1})(m)}$, we see that there can only exist a morphism between two objects if they can expanded as tensor products which are permutations of one another. Moreover, for any two objects where this is true --- say source $x_1 \otimes ... \otimes x_m$ and target $x_{\sigma^{-1}(1)} \otimes ... \otimes x_{\sigma^{-1}(m)}$ --- we can always find a map $\alpha(g; \mathrm{id}_{x_1}, ..., \mathrm{id}_{x_m})$ between them, because $\pi: G(m) \to S_m$ is surjective and so there must exist at least one $g$ with $\pi(g) = \sigma$. Thus two objects of $\mathbb{G}_n$ share a connected component if and only if they are tensor products that differ by a permutation, and therefore the canonical map $[ \,\, ] : \mathrm{Ob}(\mathbb{G}_n) \to \pi_0(\mathbb{G}_n)$ sending each object to its connected component is just the map which forgets about these permutations, making the free product on $\mathbb{N}^{*n}$ commutative. That is, $[ \,\, ]$ is the quotient map for the abelianisation $\mathrm{ab} : \mathbb{N}^{*n} \to (\mathbb{N}^{*n})^{\mathrm{ab}}$, and so $\pi_0(\mathbb{G}_n) = \mathbb{N}^n$.
\end{proof}