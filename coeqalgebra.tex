\section{Free invertible algebras as coequalizers}

In order to describe an $\mathrm{E}G$-algebra like $L\mathbb{G}_n$ in full, we must have an understanding three things: its objects, its morphisms, and its $\mathrm{E}G$-action. We achieved the first of these in \cref{Zobj}, where we learnt that $\mathrm{Ob}(L\mathbb{G}_n) = \mathbb{Z}^{\ast n}$, and we just found the second in \cref{Zmor}. In order to find the last part, the action $\alpha_{L\mathbb{G}_n}$, we can build upon a trick we used last chapter.

Recall that in \cref{Zfactor}, we were able to use the surjectivity of a monoidal functor $\psi: \mathbb{G}_n \to X$ to factor it through our initial map $\eta: \mathbb{G}_n \to L\mathbb{G}_n$. We did this by defining supposed `action morphisms' in $X$, so that we had something to map the genuine action morphisms of $L\mathbb{G}_n$ onto. But if the functor $\psi$ had actually been a map of $\mathrm{E}G$-algebras, then this method would have allowed us to recover the real action on $X$:
\begin{eq*} \alpha_X(g; \mathrm{id}_{\psi(x_1)}, ..., \mathrm{id}_{\psi(x_m)}) \, = \, \psi \big( \, \alpha_{\mathbb{G}_n}(g; \mathrm{id}_{x_1}, ..., \mathrm{id}_{x_m}) \, \big)  \end{eq*}
Therefore our aim going forward will be to find a surjective map of $\mathrm{E}G$-algebras whose target is $L\mathbb{G}_n$, from which we will be able to recover the whole of its $\mathrm{E}G$-algebra structure.

\subsection{$L\mathbb{G}_n$ as a coequalizer} 

In the previous chapter, we proved that the algebra $L\mathbb{G}_n$ is an initial objects in a certain comma category under $\mathbb{G}_n$. However, this not the only way of thinking about $L\mathbb{G}_n$. Consider for a moment the free $\mathrm{E}G$-algebra on $2n$ objects, $\mathbb{G}_{2n}$. Intuitively, if we were to take this algebra and then enforce upon it the extra relations $z_{n+1} = z_1^*, ..., z_{2n} = z_n^*$, we would be changing it from a structure with $2n$ independent generators into one with $n$ indepedent generators and their inverses. That is, there seems to be a natural way to think about $L\mathbb{G}_n$ as a quotient of the larger algebra $\mathbb{G}_{2n}$. In this section we will work towards making this idea precise, and then examine some of its consequences. We'll begin with some definitions.

\begin{defn}\label{qdef} Let $\delta$ be the map of $\mathrm{E}G$-algebras defined on generators by
\begin{eq*} \begin{array}{rlrlll}
			\delta & : & \mathbb{G}_{4n} & \to & \mathbb{G}_{2n} \\
			& : & z_i & \mapsto & z_i  \\
			& : & z_{n+i} & \mapsto & z_{n+i} \\
			& : & z_{2n+i} & \mapsto & z_i \otimes z_{n+i} \\
			& : & z_{3n+i} & \mapsto & z_{n+i} \otimes z_i			
		\end{array}
\end{eq*}
for $1 \le i \le n$. Similarly, let $\zeta$ be the $\mathrm{E}G$-algebra map defined by
\begin{eq*} \begin{array}{rlrlll}
			\zeta & : & \mathbb{G}_{4n} & \to & \mathbb{G}_{2n} \\
			& : & z_i & \mapsto & z_i  \\
			& : & z_{n+i} & \mapsto & z_{n+i} \\
			& : & z_{2n+i} & \mapsto & I \\
			& : & z_{3n+i} & \mapsto & I
		\end{array}
\end{eq*}
We will denote by $q: \mathbb{G}_{2n} \to Q$ the coequalizer these two maps.
\end{defn}

Note that the above definitions do actually make sense. The given descriptions of $\delta$ and $\zeta$ are enough to specify those maps uniquely because $\mathbb{G}_k$ is the free $\mathrm{E}G$-algebra on $k$ objects, and hence algebra maps $\mathbb{G}_k \to \mathbb{G}_{k'}$ are canonically isomorphic to functors $\{z_1, ..., z_k\} \to \mathbb{G}_{k'}$. Also we can be sure that the map $q$ exists, because $\mathrm{E}G\mathrm{Alg}_S$ is a locally finitely presentable category and thus has all finite colimits.

The goal of this approach will be show that $Q$ is in fact that same algebra as $L\mathbb{G}_n$. In order to do this, it would help if we could easily compare $q: \mathbb{G}_{2n} \to Q$ to our initial object $\eta: \mathbb{G}_{2n} \to L\mathbb{G}_n$. In other words, we really want to show that $q$ is an object of $(\mathbb{G}_n \downarrow \mathrm{inv})$ --- that $Q$ has only invertible objects. This can be done using the adjunction we found in \cref{Obadj}.

\begin{lem}\label{Qobj} The object monoid of $Q$ is $\mathbb{Z}^{*n}$, and the restriction of $q$ to objects $\mathrm{Ob}(q): \mathrm{Ob}(\mathbb{G}_{2n}) \to \mathrm{Ob}(Q)$ is the monoid homomorphism defined on generators as
\begin{eq*} \begin{array}{rlrlll}
			\mathrm{Ob}(q) & : & \mathbb{N}^{*2n} & \to & \mathbb{Z}^{*n} \\
			& : & z_i & \mapsto & z_i  \\
			& : & z_{n+i} & \mapsto & z_i^*		
		\end{array}
\end{eq*}
\end{lem}
\begin{proof}
Consider $\mathrm{Ob}(\delta)$ and $\mathrm{Ob}(\zeta)$, the restrictions on objects of the algebra maps $\delta, \zeta: \mathbb{G}_{4n} \to \mathbb{G}_{2n}$. By \cref{Gnobj}, these are monoid homomorphisms $\mathbb{N}^{\ast 4n} \to \mathbb{N}^{\ast 2n}$, and since $\mathrm{Mon}$ is cocomplete we can take their coequalizer in that category. This will give us a new homomorphism, whose source is $\mathbb{N}^{\ast 2n}$ and whose target is the quotient of $\mathbb{N}^{\ast 2n}$ by the relations $\mathrm{Ob}(\delta)(x) = \mathrm{Ob}(\zeta)(x)$. Remembering \cref{qdef}, and that $\mathbb{N}^{\ast 2n}$ is the free monoid on $2n$ generators, this quotient monoid will have the following presentation:
\begin{eq*}\begin{array}{ll}
			\text{Generators:} & z_1, \, ..., \, z_{2n} \\
			\text{Relations:} & z_i \otimes z_{n+i} = I, \\
			& z_{n+i} \otimes z_i = I
		\end{array}
\end{eq*}
This is just the same as
\begin{eq*}\begin{array}{ll}
			\text{Generators:} & z_1, \, ..., \, z_{2n} \\
			\text{Relations:} & z_{n+i} = z_i^*, \\
		\end{array}
\end{eq*}
which is the presentation of $\mathbb{Z}^{\ast n}$. 

But by \cref{Obadj}, $\mathrm{Ob}$ is a left adjoint and hence preserves all colimits. Thus the coequalizer of $\mathrm{Ob}(\delta)$ and $\mathrm{Ob}(\zeta)$ is just the underlying homomorphism $\mathrm{Ob}(q)$ of the coequalizer $q$ of $\delta, \zeta$. Therefore $\mathrm{Ob}(Q) = \mathbb{Z}^{\ast n}$, and $\mathrm{Ob}(q)$ is the quotient map $\mathbb{N}^{*2n} \to \mathbb{Z}^{*n}$ sending $z_i \mapsto z_i$ and $z_{n+i} \mapsto z_i^*$ for $1 \le i \le n$.
\end{proof}

An immediate corollary of \cref{Qobj} is that every object of the coequalizer algebra $Q$ is invertible. Thus $q: \mathbb{G}_{2n} \to Q$ is an object of the category $(\mathbb{G}_n \downarrow \mathrm{inv})$, and hence we can use the initiality of $\eta$ to determine the following result:

\begin{prop}\label{coeq} Let $i: \mathbb{G}_n \to \mathbb{G}_{2n}$ be the inclusion of $\mathrm{E}G$-algebras defined on generators by
\begin{eq*} i(z_i) \, = \, z_i \end{eq*}
Then $i \circ q$ is an initial object of $(\mathbb{G}_n \downarrow \mathrm{inv})$. In particular, this means that
\begin{eq*} Q \, \cong \, L\mathbb{G}_n \end{eq*}
\end{prop}
\begin{proof}
Let $\psi: \mathbb{G}_n \to X$ be an arbitrary object of $(\mathbb{G}_n \downarrow \mathrm{inv})$. The map $\psi^*: \mathbb{G}_n \to X$ which takes values
\begin{eq*} \psi^*(z_i) \, = \, \psi(z_i)^* \end{eq*}
is also an object of $(\mathbb{G}_n \downarrow \mathrm{inv})$, and using these two we can define a new map, $\psi + \psi^*$, using the universal property of the colimit:
\begin{eq*} \begin{tikzcd}
& \mathbb{G}_n + \mathbb{G}_n \ar[dd, dashed, "\psi + \psi^*"] & \\
\mathbb{G}_n \ar[ur, hookrightarrow, "i"] \ar[dr, "\psi"'] & & \mathbb{G}_n \ar[ul, hookrightarrow, "i'"'] \ar[dl, "\psi^*"] \\
& X & 
\end{tikzcd} \end{eq*}
But $\mathbb{G}_n$ is the free $\mathrm{E}G$-algebra on $n$ objects, and the free functor $F : \mathrm{Cat} \to \mathrm{E}G\mathrm{Alg}_S$ preserves colimits because it is a left adjoint, so clearly
\begin{eq*} \begin{array}{rll}
		\mathbb{G}_n + \mathbb{G}_n & = & F(\{ z_1, ..., z_n\}) + F(\{ z'_1, ..., z'_n\}) \\
		& = & F( \, \{ z_1, ..., z_n\} + \{ z'_1, ..., z'_n\} \, ) \\
		& = & F(\{ z_1, ..., z_{2n} \}) \\
		& = & \mathbb{G}_{2n} 
		\end{array}
\end{eq*}
This means that we can compose $\psi + \psi^*: \mathbb{G}_{2n} \to X$ with the maps $\delta, \zeta:  \mathbb{G}_{4n} \to  \mathbb{G}_{2n} $, though we need to be careful to specify which inclusions we really used in the definition of $\psi + \psi^*$. Suppose that the lefthand inclusion is $i$, the one given in the statement of the proposition, and the other is defined by the assignment $z_i \mapsto z_{i+n}$. Then for $1 \leq i \leq n$,
\begin{eq*}\begin{array}{rll}
		(\psi + \psi^*)\delta(z_i) & = & (\psi + \psi^*)(z_i) \\
		& = & (\psi + \psi^*)\zeta(z_i) \\
		& & \\
		(\psi + \psi^*)\delta(z_{n+i}) & = & (\psi + \psi^*)(z_{n+i})  \\
		& = & (\psi + \psi^*)\zeta(z_i) \\
		& & \\
		(\psi + \psi^*)\delta(z_{2n+i}) & = & (\psi + \psi^*)(z_i \otimes z_{n+i}) \\
		& = & \psi(z_i) \otimes \psi(z_i)^* \\
		& = & I \\
		& = & (\psi + \psi^*)(I) \\
		& = & (\psi + \psi^*)\zeta(z_{2n+i}) \\
		& & \\
		(\psi + \psi^*)\delta(z_{3n+i}) & = & (\psi + \psi^*)(z_{n+i} \otimes z_i) \\
		& = & \psi(z_i)^* \otimes \psi(z_i) \\
		& = & I \\
		& = & (\psi + \psi^*)(I) \\
		& = & (\psi + \psi^*)\zeta(z_{3n+i})
		\end{array}
\end{eq*}
That is, $(\psi + \psi^*) \circ \delta = (\psi + \psi^*) \circ \zeta$. However, we've already defined $q: \mathbb{G}_{2n} \to Q$ to be the coequalizer of $\delta$ and $\zeta$, the universal map such that its composites with them are equal. Therefore, there must exist a unique  $\mathrm{E}G$-algebra map $u: Q\to X$ making the righthand triangle below diagram commute:
\begin{eq*} \begin{tikzcd}
\mathbb{G}_n \ar[rr, "i"] \ar[ddrr, "\psi"'] & & \mathbb{G}_{2n} \ar[rr, "q"] \ar[dd, "\psi + \psi^*", near start] & & Q \ar[ddll, "u"] \\
& & & & \\ 
& & X & &
\end{tikzcd} \end{eq*}
The other triangle commutes by the definition of $\psi + \psi^*$, and so together the diagram tells us that for any object $\psi$ of $(\mathbb{G}_n \downarrow \mathrm{inv})$, there exists at least one morphism $u$ in $(\mathbb{G}_n \downarrow \mathrm{inv})$ going from $q \circ i$ to $\psi$. 

Next, let $v: Q \to X$ be an arbitrary morphism $q \circ i \to \psi$ in $(\mathbb{G}_n \downarrow \mathrm{inv})$. By definition, this means that
\begin{eq*}\begin{array}{rll}
			\psi & = & vqi \\
			\implies \quad \psi + \psi^* & = & vqi + (vqi)^* 
		\end{array}
\end{eq*}
Also, for $1 \leq i \leq n$ we have
\begin{eq*}\begin{array}{rcrllcccl}
			q(z_i) \otimes q(z_{n+i}) & = & q(z_{i-n} \otimes z_i) & = & q\delta(z_{2n+i}) & = & q\zeta(z_{2n+i}) & = & I \\
			q(z_{n+i}) \otimes q(z_i) & = & q(z_i \otimes z_{n+i}) & = & q\delta(z_{3n+i}) & = & q\zeta(z_{3n+i}) & = & I \\
			& \implies & q(z_{n+i}) & = & q(z_i)^* & & & &
		\end{array}
\end{eq*}
Therefore,
\begin{eq*}\begin{array}{rll}
			(\psi + \psi^*)(z_i) & = & \big( vqi + (vqi)^* \big)(z_i) \\
			& = & vqi(z_i) \\
			& = & vq(z_i) \\
			& & \\
			(\psi + \psi^*)(z_{n+i}) & = & \big( vqi + (vqi)^* \big)(z_{n+i}) \\
			& = & vqi(z_i)^* \\
			& = & v \big( q(z_i)^* \big) \\
			& = & vq(z_{n+i})
		\end{array}
\end{eq*}
or in other words $\psi + \psi^* = v \circ q$ for any morphism $v: q \circ i \to \psi$ in $(\mathbb{G}_n \downarrow \mathrm{inv})$. But this is the property that the map $u$ was supposed to satisfy uniquely, and thus it must be the only morphism $q \circ i \to \psi$ in $(\mathbb{G}_n \downarrow \mathrm{inv})$. Therefore $q \circ i$ is an initial object, and hence it is isomorphic in $(\mathbb{G}_n \downarrow \mathrm{inv})$ to any other initial object, such as $\eta$. It follows that the targets of these two maps, $Q$ and $L\mathbb{G}_n$ respectively, are isomorphic as $\mathrm{E}G$-algebras.
\end{proof}

It's worth noting that we have not given a method for actually taking coequalizers in $\mathrm{E}G\mathrm{Alg}_S$, and so \cref{coeq} doesn't immediately provide an explicit description of $L\mathbb{G}_n$. Nevertheless, we will be able to use this new perspective to fix the last few ways that our current description is incomplete.

\subsection{$L\mathbb{G}_n$ as a reflexive coequalizer}

Since left adjoint functor preserve colomits, \cref{Obadj,concompadj} both imply results about the partial surjectivity of this new map $q$. The former says that since $\mathrm{Ob}(q)$ is a coequalizer map of monoids, and hence that every object of $L\mathbb{G}_n$ is the image under $q$ of some object of $\mathbb{G}_{2n}$; the latter says a similar thing for connected components. From this one might guess that $q$ is will just turn out to be a surjective map of $\mathrm{E}G$-algebras, and indeed this is the case. Moreover, much as \cref{Zobj,Zconcomp} are analogues of \cref{Gnobj,Gnconcomp} respectively, the fact that $q$ is surjective on morphisms means that there is a result analagous to \cref{Gnmapsaction} as well. That is, since every morphism of $\mathbb{G}_{2n}$ is an action morphism, and since $\mathrm{E}G$-algebra maps always send action morphisms to action morphisms, if $q$ is surjective then every morphism of $L\mathbb{G}_n$ is also an action morphism. 

However, the proof of this is not so simple. Unfortunately, we can not go about proving that $q$ is surjective on morphisms by using the adjunction from \cref{Moradj}, since this will only tell us about the map $\mathrm{Mor}(q)^{gp, ab}$. Without this, it is not obvious how the morphisms of some algebras should be related to the morphisms of their coequalizer in this way. In particular, for it to make sense that $q$ is surjective, we would need the image of $q$ to be closed under composition, since it is supposed to be the whole algebra $L\mathbb{G}_n$. Again by \cref{Gnmapsaction}, this is equivalent to saying that the set of all action morphisms of $L\mathbb{G}_n$ would have to be closed under composition, a statement which is not at all obvious. In the case of $\mathbb{G}_{2n}$, we know that the action morphism are closed because maps $\alpha(g; \mathrm{id}_{x_1}, ..., \mathrm{id}_{x_m})$, $\alpha(g'; \mathrm{id}_{x'_1}, ..., \mathrm{id}_{x'_m})$, $x_i, x'_i \in \{ z_1, ..., z_{2n} \}$ are composable only if the target of the first is equal to the source of the second, and hence
\begin{eq*}\begin{array}{rrll}
		& x_{\pi(g^{-1})(1)} \otimes ... \otimes x_{\pi(g^{-1})(m)} & = & x'_1 \otimes ... \otimes x'_m, \quad \quad \, x_i, x'_i \in \{ z_1, ..., z_{2n} \} \\
		\implies & x_{\pi(g^{-1})(i)} & = & x'_i, \quad \quad \quad \quad \quad \quad \quad 1 \le i \le m \\
		\implies & \alpha(gg'; \mathrm{id}_{x_1}, ..., \mathrm{id}_{x_m}) & = & \alpha(g; \mathrm{id}_{x_1}, ..., \mathrm{id}_{x_m}) \circ \alpha(g'; \mathrm{id}_{x'_1}, ..., \mathrm{id}_{x'_m})
		\end{array}
\end{eq*}
However, in $L\mathbb{G}_n$ this line of reasoning does not work, because for example
\begin{eq*} z_1 \otimes z^*_1 \quad = \quad I \quad = \quad z_2 \otimes z^*_2 \end{eq*}
but this does not imply that $z_1 = z_2$. 

The key to soving this problem is noticing that the maps $\delta$ and $\zeta$ form a reflexive pair --- parallel maps which share a right-inverse.

\begin{lem} \label{sect} Let $\iota: \mathbb{G}_{2n} \to \mathbb{G}_{4n}$ be the inclusion defined on generators by $z_i \mapsto z_i$. Then $\iota$ is a right-inverse of both $\delta$ and $\zeta$. \end{lem} 
\begin{proof}
For $1 \le i \le 2n$,
\begin{eq*}\begin{array}{rcccl}
			\delta \iota(z_i) & = & \delta(z_i) & = & z_i \\
			\zeta \iota(z_i) & = & \zeta(z_i) & = & z_i \\
			& & & & \\
			\implies \quad \delta \circ \iota & = & \mathrm{id}_{\mathbb{G}_{2n}} & = & \zeta \circ \iota
		\end{array}
\end{eq*}
\end{proof}

In other words, $q$ is a reflexive coequalizer. Using this property of $q$, we will eventually be able to prove that the image of $q$ is closed under composition. First though, we need some intermediate results.

\begin{defn}\label{decompdef} Let $x$ be an object of $\mathbb{G}_m$ for some $m \in \mathbb{N}$, and hence an element of $\mathbb{N}^{\ast m}$, the free monoid on $m$ generators $\{ z_1, ..., z_m \}$. Then by the definition of the free product of monoids, $x$ will have a unique decomposition of the form
\begin{eq*} x \, = \, \bigotimes_{i=1}^{|x|} g(x, i) \end{eq*}
where the upper bound $|x|$ is the length of the element $x$ as in \cref{lengthdef}, and each $g(x, i)$ is a generator of one of the $m$ copies of $\mathbb{N}$, so $g(x, i) \in \{ z_1, ..., z_m \}$.
\end{defn}

Note that this decomposition shows that the length of $x$ is just the sum of all of the coordinates of its connected component, $[x] \in \mathbb{N}^m$:
\begin{lem}
\begin{eq*} |x| \, = \, \sum_{i = 1}^{m} [x]_i \end{eq*}
\end{lem}
\begin{proof}
For any generator $z_j$ its connected component has coordinates
\begin{eq*} [z_j]_i \, = \, \begin{cases}
					1 & \text{if} \quad i = j \\
					0 & \text{otherwise}
				\end{cases}
\end{eq*}
and hence
\begin{eq*} \begin{array}{rll}
		\sum_{i = 1}^{m} [x]_i & = & \sum_{i = 1}^{m} \left[ \, \bigotimes_{j=1}^{|x|} g(x, j) \, \right]_i \\
		& = & \sum_{i = 1}^{m} \sum_{j=1}^{|x|} [ \, g(x, j) \, ]_i \\
		& = & \sum_{j=1}^{|x|} \sum_{i = 1}^{m} [ \, g(x, j) \, ]_i \\
		& = & \sum_{j=1}^{|x|} 1 \\
		& = & |x|	
		\end{array}
\end{eq*}
\end{proof}

\begin{prop}\label{ama1} Let $w$ be an object of $\mathbb{G}_{2n}$. Then there exist objects $w_1, ..., w_k$ in $\mathbb{G}_{2n}$ and $u_1, ..., u_k$ in $\mathbb{G}_{4n}$, for some value of $k \in \mathbb{N}$, such that
\begin{eq*} w_1 \, = \, w, \quad \quad \zeta(u_{i-1}) \, = \, w_i \, = \, \delta(u_i), \quad \quad u_k \, = \, \iota(w_k) \end{eq*}
for $1 \le i \le k$, and for any object $u$ of $\mathbb{G}_{4n}$,
\begin{eq*} \delta(u) \, = \, w_k \quad \implies \quad u \, = \, u_k \end{eq*}
\end{prop}
\begin{proof}
From \cref{lengthdef,qdef}, we know that for any generator $z_i$ of $\mathbb{G}_{4n}$,
\begin{eq*}\begin{array}{rllll}
				 | \delta(z_i) |  & = & \left. \begin{cases}
								1 & \text{if} \quad 1 \le i \le 2n \\
								2 & \text{if} \quad 2n+1 \le i \le 4n
							\end{cases} \right \rbrace & \ge 1 \\
				& & \\
				| \zeta(z_i) |  & = & \left. \begin{cases}
								1 & \text{if} \quad 1 \le i \le 2n \\
								0 & \text{if} \quad 2n+1 \le i \le 4n
							\end{cases} \right \rbrace & \le 1 
		\end{array}
\end{eq*}
Also these lengths are additive across tensor products, since $| \, \_ \, |$ is a monoid homomorphism $\mathbb{G}_{2n} \to \mathbb{N}$. Thus for any object $u$ in $\mathbb{G}_{4n}$, we can conclude that
\begin{eq*}\begin{array}{rllllllll}
			| \delta(u) | & = & | \, \delta\big( \, \mathlarger{\bigotimes_{i=1}^{|u|} g(u, i)} \, \big) \, | & = & | \, \mathlarger{\bigotimes_{i=1}^{|u|} \delta \big( \, g(u, i) \, \big)} \, | & = & \mathlarger{\sum_{i=1}^{|u|} | \, \delta \big( \, g(u, i) \, \big) \, |} & \ge & |u| \\
			| \zeta(u) | & = & | \, \zeta\big( \, \mathlarger{\bigotimes_{i=1}^{|u|} g(u, i)} \, \big) \, | & = & | \, \mathlarger{\bigotimes_{i=1}^{|u|} \zeta \big( \, g(u, i) \, \big)} \, | & = & \mathlarger{\sum_{i=1}^{|u|} | \, \zeta \big( \, g(u, i) \, \big) \, |} & \le & |u|
		\end{array}
\end{eq*}
Also, since the only generators that have $| \delta(z_i) | = | \zeta(z_i) | = 1$ are those from the $\mathbb{G}_{2n}$ subalgebra associated with $\iota$, the inequality above becomes an equality if and only if $u$ is in the image of $\iota$. That is,
\begin{eq*} | \zeta(u) | \, = \, |u| \, = \, |\delta(u)|  \quad \iff \quad \exists \, v \in \mathbb{N}^{\ast 2n} \, : \, u \, = \, \iota(v) \end{eq*}

Next, consider the set
\begin{eq*} \delta^{-1}(w) \, = \, \{ \, u \in \mathrm{N}^{\ast 4n} : \delta(u) = w \, \} \end{eq*}
of all objects in $\mathbb{G}_{4n}$ which $\delta$ sends to $w$. This set is always nonempty, since by \cref{sect} $\iota$ is a right-inverse of $\delta$:
\begin{eq*} \delta \iota(w) \, = \, w \quad \implies \quad \iota(w) \in \delta^{-1}(w) \end{eq*}
Moreover, $\iota(w)$ is the only element of $\delta^{-1}(w)$ which can be expessed as $\iota(v)$ for some object $v$ in $\mathbb{G}_{2n}$, because
\begin{eq*} \delta \big( \, \iota(v) \, \big) \, = \, w \quad \implies v \, = \, w \end{eq*}

With all of this now in place, we can begin constructing the sequences $w_1, ..., w_k$ and $u_1, ..., u_k$. Start by setting $w_1 = w$ and $i=1$, then apply the following algorithm:
\begin{enumerate}
\item If $\delta^{-1}(w_i)$ is just the set $\{ \iota(w_i) \}$, choose $u_i = \iota(w_i)$, set $k$ to be the current value of $i$, and terminate.
\item Otherwise, choose $u_i$ to be any element of $\delta^{-1}(w_i)$ other than $\iota(w_i)$.
\item Set $w_{i+1} = \zeta(u_i)$.
\item Increase the value of $i$ by 1, then return to step 1.
\end{enumerate}
By design, none of the $u_i$ produced by this process can be expressed as $u_i = \iota(v)$ for some $v$ in $\mathbb{G}_{2n}$, with the possible exception of $u_k$ if the algorithm terminates. This is because $\iota(w_i)$ is the only element of $\delta^{-1}(w_i)$ that can be expressed that way, and the above process will terminate the first time it has to pick $u_i = \iota(w_i)$, at which point $i$ is set equal to $k$. Therefore, for any $i \neq k$, we will have the following strict inequalities:
\begin{eq*} |w_{i+1}| \, = \, | \zeta(u_i) | \, < \, |u_i| \, < \, |\delta(u_i)| \, = \, |w_i| \end{eq*}
That is, the $w_i$ generated by this algorithm form a sequence with strictly decreasing length. However, it is impossible to have a infinite sequence of strictly decreasing natural numbers, and hence we can be sure that this process will terminate at some finite $k$. 

But in order for the algorithm to have terminate, it must be the case that 
\begin{eq*} \delta^{-1}(w_k) \, = \, \{ \iota(w_k) \} \end{eq*}
and hence
\begin{eq*} \delta(u) \, = \, w_k \quad \implies \quad u \, = \, \iota(w_k) \, = \, u_k \end{eq*}
Thus the sequences $w_1, ..., w_k$ and $u_1, ..., u_k$ satisfy all of the conditions in the statement of the lemma.
\end{proof}

The intuition behind \cref{ama1} is that we are successively removing parts of the object $w$, without changing its image under $q$. The map $\delta$ sends $z_{2n+i} \mapsto z_i \otimes z_{n+i}$ and $z_{3n+i} \mapsto z_{n+1} \otimes z_i$ while $\zeta$ sends these all to $I$, and so for any $u$ in $\mathbb{G}_{4n}$ the object $\zeta(u)$ will look like $\delta(u)$ except missing some number of $z_i \otimes z_{n+i}$ or $z_{n+1} \otimes z_i$ substrings. But since $q$ sends $z_{n+i} \mapsto z_i^*$, these are exactly the sort of omissions which the coequaliser doesn't care about. If we repeat this process then it will eventually terminate at $u_k = \iota(w_k)$, so we really have a method for removing \emph{all} of the relevent substrings from objects of $\mathbb{G}_{2n}$. In other words, $w_k$ has the smallest possible length while still having $q(w_k) = q(w)$. In fact, we will show that it is the unique shortest object of $\mathbb{G}_{2n}$ with this property.

\begin{defn}\label{Zlengthdef} Let $x$ be an object of $L\mathbb{G}_m$ for some $m \in \mathbb{N}$, and hence an element of $\mathbb{Z}^{\ast m}$, the free monoid on $m$ invertible generators $\{ z_1, ..., z_m \}$. Then as in \cref{decompdef} we can use the definition of the free product of monoids to write $x$ uniquely in the form
\begin{eq*} x \, = \, \bigotimes_{i=1}^{|x|} g(x, i) \end{eq*}
This time each $g(x, i)$ is either a generator of one of the copies of $\mathbb{N}$ or the inverse of a generator, $g(x, i) \in \{ z_1, ..., z_m, z_1^*, ..., z_m^* \}$, and for any $1 \le i < |x|$ we will always have $g(x, i+1) \neq g(x, i)^*$. As before, we will call the upper bound $|x|$ of this tensor product the \emph{length} of the element $x$. 
\end{defn}

\begin{prop}\label{ama2} Let $w$, $w'$ be objects of $\mathbb{G}_{2n}$ such that $q(w) = q(w')$. If $w_1, ..., w_k$ and $u_1, ..., u_k$ are sequences generated from $w$ via \cref{ama1}, and likewise $w'_1, ..., w'_{k'}$ and $u'_1, ..., u'_{k'}$ from $w'$, then $w_k = w'_{k'}$ and $u_k = u'_{k'}$.
\end{prop}
\begin{proof}
Consider the decomposition of the object $w_k \in \mathbb{N}^{\ast 2n}$ as in \cref{decompdef}. Assume, for the sake of contradiction, that there exist $1 \le j < |w_k|$ and $1 \le m \le n$ such that
\begin{eq*} g(w_k, j) \, = \, z_m, \quad \quad g(w_k, j+1) \, = \, z_{n+m} \end{eq*}
Then we can use $j, m$ to contruct an element $u \in \mathbb{N}^{\ast 4n}$, defined by
\begin{eq*} |u| \, = \, |w| - 1, \quad \quad g(u, i) \, = \, \begin{cases}
									\iota \big( \, g(w_k, i) \, ) & \text{if} \quad 1 \le i < j \\
									z_{2n + m} & \text{if} \quad i = j \\
									\iota \big( \, g(w_k, i+1) \, ) & \text{if} \quad j < i \le |u|
								\end{cases}
\end{eq*}
This $u$ will then have the property that
\begin{eq*} \begin{array}{rll}
			\delta(u) & = & \delta \big( \, \mathlarger{\bigotimes_{i=1}^{|u|} g(u, i)} \, \big) \\
			& = & \mathlarger{\bigotimes_{i=1}^{|u|} \delta \big( \, g(u, i) \, \big)} \\
			& = & \mathlarger{\bigotimes_{i=1}^{j-1} \delta \iota \big( \, g(w_k, i) \, )} \otimes \delta(z_{2n + m}) \otimes \mathlarger{\bigotimes_{i=j+1}^{|u|} \delta \iota \big( \, g(w_k, i+1) \, \big)} \\
			& = & \mathlarger{\bigotimes_{i=1}^{j-1} g(w_k, i)} \otimes g(w_k, j) \otimes g(w_k, j+1) \otimes \mathlarger{\bigotimes_{i=j+2}^{|u|+1} g(w_k, i)} \\
			& = & w_k
		\end{array}
\end{eq*}
But this is impossible, since by \cref{ama1} $u_k$ is the only object of $\mathbb{G}_{4n}$ whose image under $\delta$ is $w_k$, and this $u$ we have constructed is manifestly not $w_k$. Thus we can conclude that there are no values of $j$ and $m$ for which
\begin{eq*} g(w_k, j) \, = \, z_m, \quad \quad g(w_k, j+1) \, = \, z_{n+m} \end{eq*}
An analogous line of reasoning --- using $z_{3n + m}$ rather than $z_{2n + m}$ in the definition of $u$ --- demonstrates that there are also no $j, m$ with
\begin{eq*} g(w_k, j) \, = \, z_{n+m}, \quad \quad g(w_k, j+1) \, = \, z_m \end{eq*}
As a result, for all $1 \le i < |w_k|$
\begin{eq*} q \big( \, g(w_k, i+1) \, \big) \, \neq \, q \big( \, g(w_k, i) \, \big)^* \end{eq*}
 and this combined with the fact that
\begin{eq*} \mathlarger{\bigotimes_{i=1}^{|w_k|} q \big( \, g(w_k, i) \, \big)} \, = \, q \big( \, \mathlarger{\bigotimes_{i=1}^{|w_k|} g(w_k, i)} \, \big) \, = \, q(w_k) \end{eq*}
shows that the unique decomposition of $q(w_k)$ as in \cref{Zlengthdef} is given by
\begin{eq*} |q(w_k)| \, = \, |w_k|, \quad \quad g\big( \, q(w_k), i \, \big) \, = \, q \big( \, g(w_k, i) \, \big) \end{eq*}

Next, let $s$ be a function defined by
\begin{eq*} \begin{array}{rllll}
			s & : & \mathbb{Z}^{\ast n} & \to & \mathbb{N}^{\ast 2n} \\
			& : & z_i & \mapsto & z_i \\
			& : & z_i^* & \mapsto & z_{n+i} \\
			& : & x & \mapsto & \bigotimes_{i=1}^{|x|} s \big( \, g(x, i) \, \big)
		\end{array}
\end{eq*}
Then for $1 \le i \le n$,
\begin{eq*} sq(z_i) \, = \, s(z_i) \, = \, z_i, \quad \quad sq(z_{n+i}) \, = \, s(z_i^*) \, = \, z_{n+i} \end{eq*}
and so it follows that
\begin{eq*} \begin{array}{rll}
			sq(w_k) & = & \mathlarger{\bigotimes_{i=1}^{|w_k|} s \Big( \, g\big( \, q(w_k), i \, \big) \, \Big)} \\
			& = & \mathlarger{\bigotimes_{i=1}^{|w_k|} sq \big( \, g(w_k, i) \, \big)} \\
			& = & \mathlarger{\bigotimes_{i=1}^{|w_k|} g(w_k, i)} \\
			& = & w_k
		\end{array}
\end{eq*}

Finally, notice that the exact same logic as we've used above will also work for $w'_{k'}$, so that $sq(w'_{k'}) = w'_{k'}$. Therefore,
\begin{eq*} \begin{array}{rll}
			w_k & = & sq(w_k) \\
			& = & sq\zeta(u_{k-1}) \\
			& = & sq\delta(u_{k-1}) \\
			& = & sq(w_{k-1}) \\
			& \vdots & \\
			& = & sq(w) \\
			& = & sq(w') \\
			& = & sq\delta(u'_1) \\
			& = & sq\zeta(u'_1) \\
			& = & sq(w'_2) \\
			& \vdots & \\
			& = & sq(w'_{k'}) \\
			& = & w'_{k'}			
		\end{array}
\end{eq*}
as required.
\end{proof}

Next, we need to take things one step further. We've already shown that we can find an object $w_k$ which is like $w$ but with all $z_i \otimes z_{n+i}$ and $z_{n+i} \otimes z_i$ substrings removed. In order to prove the closure of $\mathrm{im}(q)$, we will need to construct a morphism of $\mathbb{G}_{2n}$ which witnesses this removal --- that is, a map $h: w \to v \otimes w_k$ where $v$ is made up of all of the $z_i \otimes z_{n+i}$ that need to be taken out of $w$ to make $w_k$. Likewise, the fact that $q(w_k) = q(w)$ will be embodied by the fact that this $h$ becomes an identity morphism under $q$. This map $h$ will need to be built out of smaller pieces, using the sequence $u_1, ..., u_k$ and the following proposition:

\begin{prop}\label{ama3} Let $u$ be an object of $\mathbb{G}_{4n}$. Then there exists another object $x$ in $\mathbb{G}_{4n}$ and a morphism $h: \delta(u) \to \delta(x) \otimes \zeta(u)$ in $\mathbb{G}_{2n}$ such that $q(h) = \mathrm{id}_{q\delta(u)}$.
\end{prop}
\begin{proof}
First, consider the object $\iota \zeta(u) \in \mathbb{N}^{\ast 4n}$. Because $\zeta$ acts by sending the generator $z_{2n+i}$ to $I$ for each $1 \le i \le n$, and since $\iota$ is an inclusion, the connected component $[\iota \zeta(u)] \in \mathbb{N}^{4n}$ will have $i$th coordinate
\begin{eq*} 
[ \, \iota \zeta(u) \, ]_i \, = \, [ \, \zeta(u) \, ]_i \, = \, 
									\begin{cases}
										[u]_i & \text{if} \quad 1 \le i \le 2n \\
										0 & \text{if} \quad 2n+1 \le i \le 4n \\
									\end{cases}
\end{eq*}
Intuitively, $\iota \zeta(u)$ is what would be left if we were to remove all of the $z_{2n+i}$ generators from $u$. Conversely, the object $x \in \mathbb{G}_{4n}$ defined as
\begin{eq*} x \, = \, {z_{2n+1}}^{\otimes [u]_{2n+1}} \otimes ... \otimes {z_{4n}}^{\otimes [u]_{4n}} \end{eq*}
is like a reordered list of all of the $z_{2n+i}$ that we would have to remove from $u$ in order to make $\iota \zeta(u)$, and will have the property that
\begin{eq*} 
[x]_i \, = \, 
		\begin{cases}
			0 & \text{if} \quad 1 \le i \le 2n \\
			[u]_i  & \text{if} \quad 2n+1 \le i \le 4n \\
		\end{cases}
\end{eq*}
Thus the object $x \otimes \iota \zeta(u)$ is simply a reordered version of $u$, or more concretely
\begin{eq*} 
[ \, x \otimes \iota \zeta(u) \, ]_i \, = \, [x]_i + [ \, \iota \zeta(u) \, ]_i \, = \, \left.
												\begin{cases}
													0 + [u]_i & \text{if} \quad 1 \le i \le 2n \\
													[u]_i + 0 & \text{if} \quad 2n+1 \le i \le 4n \\
												\end{cases}
\right \rbrace \, = \, [u]_i
\end{eq*}
for all $i$, and hence $u$ and $x \otimes \iota \zeta(u)$ are part of the same connected component.

Knowing this, we can now choose an arbitrary morphism $f: u \to x \otimes \iota \zeta(u)$ from $\mathbb{G}_{4n}$. The map $\zeta(f)$ will then have source $\zeta(u)$ and target
\begin{eq*} \zeta \big( \, x \otimes  \iota \zeta(u) \, \big) \, = \, \zeta(x) \otimes \zeta \iota \zeta (u) \, = \, \zeta(u) \end{eq*}
because $\iota$ is a right inverse of $\zeta$, and $x$ was defined in such a way that $\zeta(x) = I$. It follows that $\iota \zeta(f)$ is a map $\iota \zeta(u) \to \iota \zeta(u)$, and therefore it is possible to form the composite
\begin{eq*} \begin{tikzcd}
u \ar[r, "f"] & x \otimes \iota \zeta(u) \ar[rrr, "\mathrm{id}_x \, \otimes \, \iota \zeta(f)^{-1}"] & & & x \otimes \iota \zeta(u)
\end{tikzcd} \end{eq*}
which we'll call $g$. Effectively, what $g$ does is to first apply the map $f$, and then `undo' its effect on the generators $z_1, ..., z_{2n}$, while leaving the $z_{2n+1}, ..., z_{3n}$ in $x$ untouched. It should not be surprising then that if we take the image of $g$ under $\zeta$, we get
\begin{eq*} \begin{array}{rll}
		\zeta(g) & = & \zeta \Big( \, \big( \, \mathrm{id}_x \otimes \iota \zeta(f)^{-1} \, ) \circ f \, \Big) \\
		& = & \big( \, \zeta( \mathrm{id}_x ) \otimes \zeta \iota \zeta(f)^{-1} \, \big) \circ \zeta(f) \\
		& = & \zeta(f)^{-1} \circ \zeta(f) \\
		& = & \mathrm{id}_{\zeta(u)}
		\end{array}
\end{eq*}

Finally, consider the morphism $\delta(g)$. This has source $\delta(u)$ and target
\begin{eq*} \delta \big( \, x \otimes \iota \zeta(u) \, \big) \, = \, \delta(x) \otimes \delta \iota \zeta (u) \, = \, \zeta(u) \end{eq*}
because $\iota$ is also the right inverse of $\delta$, and so we can choose $\delta(g)$ to be the map $h$ that we are looking for. If we do, then since $q$ is the coequalizer of $\delta$ and $\zeta$ we'll have
\begin{eq*} q(h) \, = \, q \delta(g) \, = \, q \zeta(g) \, = \, q(\mathrm{id}_{\zeta(u)}) \, = \, \mathrm{id}_{q \zeta(u)} \, = \, \mathrm{id}_{q \delta(u)} \end{eq*}
as required.
\end{proof}

Hopefully it is clear that nothing in the proof of \cref{ama2} really depended on the order in which we tensored together $\delta(x) \otimes \zeta(u)$. Consequently, the same basic proof will also work when this order is reversed:

\begin{cor}\label{ama4} Let $u$ be an object of $\mathbb{G}_{4n}$. Then there exists another object $x$ in $\mathbb{G}_{4n}$ and a morphism $h: \delta(u) \to \zeta(u) \otimes \delta(x)$ in $\mathbb{G}_{2n}$ such that $q(h) = \mathrm{id}_{q\delta(u)}$.
\end{cor}

Now we finally have enough results in place to show that the image of $q$ is closed under composition.

\afterpage{
\begin{figure}[ht]
\begin{eq*} \begin{tikzcd}
\quad \quad \quad \quad \quad \quad \quad \quad \quad \quad \, \, v \otimes \delta(x'_{k'-1}) \otimes ... \otimes \delta(x'_1) \ar[dd, "f \, \otimes \, \mathrm{id}"'] \\
\\
\quad \quad \quad \quad \quad \quad \quad \quad \quad \quad \, \, w \otimes \delta(x'_{k'-1}) \otimes ... \otimes \delta(x'_1) \ar[dd, "h_1 \, \otimes \, \mathrm{id}"'] \\
\\
\quad \quad \quad \quad \quad \quad \quad \delta(x_1) \otimes w_2 \otimes \delta(x'_{k'-1}) \otimes ... \otimes \delta(x'_1) \ar[dd,  "\mathrm{id} \, \otimes \, h_2 \, \otimes \, \mathrm{id}"'] \\
\\
\quad \quad \quad \delta(x_1) \otimes \delta(x_2) \otimes w_3 \otimes \delta(x'_{k'-1}) \otimes ... \otimes \delta(x'_1) \ar[dd,  "\mathrm{id} \, \otimes \, h_3 \, \otimes \, \mathrm{id}"'] \\
\\
\vdots \ar[dd, "\mathrm{id} \, \otimes \, h_{k-1} \, \otimes \, \mathrm{id}"'] \\
\\
\, \delta(x_1) \otimes ... \otimes \delta(x_{k-1}) \otimes w_k \otimes \delta(x'_{k'-1}) \otimes ... \otimes \delta(x'_1) \ar[dd, equals] \\
\\
\, \delta(x_1) \otimes ... \otimes \delta(x_{k-1}) \otimes w'_{k'} \otimes \delta(x'_{k'-1}) \otimes ... \otimes \delta(x'_1) \ar[dd, "\mathrm{id} \, \otimes \, h'_{k'-1} \, \otimes \, \mathrm{id}"] \\
\\
\vdots \ar[dd, "\mathrm{id} \, \otimes \, h'_1"] \\
\\
\delta(x_1) \otimes ... \otimes \delta(x_{k-1}) \otimes w' \quad \quad \quad \quad \quad \quad \quad \quad \quad \quad \ar[dd, "\mathrm{id} \, \otimes \, f'"] \\
\\
\delta(x_1) \otimes ... \otimes \delta(x_{k-1}) \otimes v' \quad \quad \quad \quad \quad \quad \quad \quad \quad \quad
\end{tikzcd} \end{eq*}
\caption{The composite $h$ from \cref{ama5} }
\label{composite}
\end{figure}
\clearpage
}

\begin{prop}\label{ama5} Let $f: v \to w$ and $f' : w' \to v'$ be two morphisms of $\mathbb{G}_{2n}$ such that $q(f)$ and $q(f')$ are composable. Then there exist objects $y, y'$ and a morphism $h: y \to y'$ in $\mathbb{G}_{2n}$ such that $q(h) = q(f') \circ q(f)$.
\end{prop}
\begin{proof}
To begin, we'll use \cref{ama1} to contruct from $w$ the sequences of objects $w_1, ..., w_k$ in $\mathbb{G}_{2n}$ and $u_1, ..., u_k$ in $\mathbb{G}_{4n}$, for some $k \in \mathbb{N}$. These have
\begin{eq*} w_1 \, = \, w, \quad \quad \zeta(u_{i-1}) \, = \, w_i \, = \, \delta(u_i), \quad \quad \iota(w_k) \, = \, u_k \end{eq*}
Then we can apply \cref{ama3} to each $u_i$ in turn, producing a new sequence of object $x_1, ..., x_{k-1}$ in $\mathbb{G}_{4n}$ and morphisms 
\begin{eq*} \begin{array}{rrrll}
			h_i & : & \delta(u_i) & \to & \delta(x_i) \otimes \zeta(u_i) \\
			& : & w_i & \to & \delta(x_i) \otimes w_{i+1}
		\end{array}
\end{eq*}
for each $1 \le i < k$, with the properties
\begin{eq*} \zeta(x_i) \, = \, I, \quad \quad \quad q(h_i) \, = \, \mathrm{id}_{q(w_i)} \end{eq*}
We can do a similar thing with $w'$, first using \cref{ama1} to get sequences $w'_1, ..., w'_{k'}$ and $u'_1, ..., u'_{k'}$ with the appropriate properties, but this time applying \cref{ama4} to find objects $x'_1, ..., x'_{k'-1}$ and morphisms $h'_i: w'_i \to w_{i+1} \otimes \delta(x_i)$. Moreover, since the morphisms $q(f): q(v) \to q(w)$ and $q(f') : q(w') \to q(v')$ arecomposable, it must be the case that $q(w)$ and $q(w')$ are equal.
By \cref{ama2}, this means that $w_k = w'_{k'}$

Putting all of this together, we can form the composite morphism shown in \cref{composite}. Here the subscripts of the identity maps have been suppressed for clarity. This morphism will be our choice of $h: y \to y'$. To complete the proof, notice that 
\begin{eq*} \begin{array}{rrrcccccccl}
			& & q\delta(x_i) & = & q \zeta(x_i) & = & I & = & q \zeta(x'_i) & = & q\delta(x'_i) \\
			\implies & & q( \mathrm{id}_{\delta(x_i)} ) & = & \mathrm{id}_{q\delta(x_i)} & = & \mathrm{id}_I & = & \mathrm{id}_{q\delta(x'_i)} & = & q( \mathrm{id}_{\delta(x'_i)} )
		\end{array}
\end{eq*}
for each value of $i$, and so when we take the image of the composite $h$ under $q$ all of the $\mathrm{id}_{\delta(x_i)}$ and $\mathrm{id}_{\delta(x'_i)}$ terms will cancel out, leaving just
\begin{eq*} \begin{array}{rll}
			q(h) & = & q(f') \circ q(h'_1) \circ ... \circ q(h'_{k'-1}) \circ q(h_{k-1}) \circ ... \circ q(h_1) \circ q(f) \\
			& = & q(f') \circ \mathrm{id}_{q(w)} \circ ... \circ \mathrm{id}_{q(w)} \circ \mathrm{id}_{q(w)} \circ ... \circ \mathrm{id}_{q(w)} \circ q(f) \\
			& = &  q(f') \circ q(f)
		\end{array}
\end{eq*}
as required.
\end{proof}

Finally, we can return to how we began this section. We wished to show that the map $q$ is surjective, and from this conclude that all morphisms of $L\mathbb{G}_n$ are action morphisms. However, it was not obvious that such a statement would even make sense, since the image of $q$ would consist entirely of morphisms of the form $\alpha_{L\mathbb{G}_n}(g; \mathrm{id}_{x_1}, ..., \mathrm{id}_{x_m})$, and these are not a priori closed under composition. But with \cref{ama5} we now know that the image of $q$ is indeed closed, and we can immediately use this to demonstrate surjectivity.

\begin{prop}\label{allmapsaction} The quotient map $q: \mathbb{G}_{2n} \to L\mathbb{G}_n$ is surjective. Therefore, every morphism in $L\mathbb{G}_n$ can be expressed as $\alpha_{L\mathbb{G}_n}(g; \mathrm{id}_{x_1}, ..., \mathrm{id}_{x_m})$ for some $g \in G(m)$ and $x_i \in \{z_1, ..., z_n, z^*_1, ..., z^*_n  \}$.
\end{prop}
\begin{proof}
Consider $q(\mathbb{G}_{2n})$, the subcategory of $L\mathbb{G}_n$ that contains every object $x$ for which there exists $x'$ in $\mathbb{G}_{2n}$ with $q(x') = x$, and every morphism $f$ for which there exists $f'$ in $\mathbb{G}_{2n}$ with $q(f') = f$. By \cref{ama5} the morphisms of $q(\mathbb{G}_{2n})$ are closed under composition, and so this is a well-defined category. Moreover, since $q$ is a map of $\mathrm{E}G$-algebras then  for any morphisms $f_1, ..., f_m$ of $q(\mathbb{G}_{2n})$ we'll have
\begin{eq*} \begin{array}{rll}
 			\alpha_{L\mathbb{G}_n}(g; f_1, ..., f_m) & = & \alpha_{L\mathbb{G}_n}\big( \, g \, ; \, q(f'_1), ..., q(f'_m) \, \big) \\
			& = & q \big( \, \alpha_{\mathbb{G}_{2n}}(g; f'_1, ..., f'_m) \, \big) \\
			& \in & q(\mathbb{G}_{2n}) 
		\end{array}
\end{eq*}
Thus $q(\mathbb{G}_{2n})$ is also a well-defined sub-$\mathrm{E}G$-algebra of $L\mathbb{G}_n$. 

Next, let $q': \mathbb{G}_{2n} \to q(\mathbb{G}_{2n})$ be the surjective map which acts on objects and morphism exactly as $q$. Also denote by $i$ the evident inclusion of algebras $q(\mathbb{G}_{2n}) \hookrightarrow L\mathbb{G}_n$, so that for instance $i \circ q' = q$.
\begin{eq*} \begin{tikzcd}
& \mathbb{G}_{3n} \ar[dd, bend right, "\delta"'] \ar[dd, bend left, "\zeta"] & \\
& & \\
& \mathbb{G}_{2n} \ar[ddl, "q'"'] \ar[dd, "q"] \ar[ddr, "p"] & \\ 
& & \\
q(\mathbb{G}_{2n}) \ar[r, hookrightarrow, "i"] & L\mathbb{G}_n \ar[r, "u"] & X
\end{tikzcd} \end{eq*}
Further, let $p: \mathbb{G}_{2n} \to X$ be any map of $\mathrm{E}G$-algebras with the property that $p \circ \delta = p \circ \zeta$. Since $q$ is the coequalizer of $\delta$ and $\zeta$, there will then exist a unique map $u:  L\mathbb{G}_n \to X$ such that $p = u \circ q$. This means that $p = u \circ i \circ q'$, and hence there is obviously at least one map, $u \circ i$, which lets us factors $p$ through $q'$. But for any other map $v': q(\mathbb{G}_{2n}) \to X$ that factors $p$ like this, we have
\begin{eq*} \begin{array}{rrll}
			& v' \circ q' & = & p \\
			& & = & v \circ i \circ q' \\
			\implies \quad & v' & = & v \circ i
		\end{array}
\end{eq*}
because $q'$ is surjective, and thus $u \circ i$ is the unique map with this property. That is, $q'$ is also a coequalizer of $\delta$ and $\zeta$. Since colimits are always unique up to isomorphism, this means that there exists an invertible map $v: q(\mathbb{G}_{2n}) \to L\mathbb{G}_n$ such that $q = v \circ q'$. But $q'$ is surjective, and so it immediately follows that $q$ is too. 

Finally, let $f$ be an arbitrary morphism in $L\mathbb{G}_n$. By surjectivity there exists at least one morphism $f'$ in $\mathbb{G}_{2n}$ such that $q(f') = f$, and from \cref{Gnmapsaction} we know that this $f'$ can be expressed uniquely as $\alpha(g; \mathrm{id}_{x'_1}, ..., \mathrm{id}_{x'_m})$ for some $g \in G(m)$ and $x'_i \in \{z_1, ..., z_{2n} \}$. Thus, because $q$ is a map of $\mathrm{E}G$-algebras, we will have
\begin{eq*} f \, = \, q(f') \, = \, q \big( \, \alpha_{\mathbb{G}_{2n}}(g; \mathrm{id}_{x'_1}, ..., \mathrm{id}_{x'_m}) \, \big)  \, = \, \alpha_{L\mathbb{G}_n}(g; \mathrm{id}_{q(x'_1)}, ..., \mathrm{id}_{q(x'_m)}) \end{eq*}
But by \cref{Qobj}, for each generator $x'_i \in \{z_1, ..., z_{2n} \}$ the object $q(x'_i)$ is either one of the generators $z_1, ..., z_n$ of $L\mathbb{G}_n$ or one of their inverse $z_1^*, ..., z_n^*$. Therefore, there is at least one collection of $x_i = q(x'_i)$ for which the statement of the proposition holds. 
\end{proof}

\cref{allmapsaction} formalises a certain intuition about how the functor $L$ should act on algebras, the idea that a `free' structure really shouldn't have any `superfluous' components, only whatever data is absolutely required for it to be well-defined. In the case of $L\mathbb{G}_n$, we have proven that the only morphisms contained in the free $\mathrm{E}G$-algebra on invertible objects are $\mathrm{E}G$-action morphisms. However, while this is very similar to what we have in the non-invertible case it should be stressed that \cref{allmapsaction} does \emph{not} prove that the morphisms of $L\mathbb{G}_n$ have \emph{unique} representations $\alpha(g; \mathrm{id}_{w_1}, ..., \mathrm{id}_{w_m})$, as morphisms of $\mathbb{G}_n$ do.

\subsection{The action of $L\mathbb{G}_n$}

\begin{prop} Propositions in previous section do not depend on $q$ having any property except $q \circ \delta = q \circ \zeta$.
\end{prop}

\begin{prop} The coequalizer of $\delta$ and $\zeta$ in $\mathrm{MonCat}$ is surjective
\end{prop}

\begin{prop} The coequalizer of $\delta$ and $\zeta$ in $\mathrm{MonCat}$ is $q$
\end{prop}

\begin{cor} The coequalizer of $\mathrm{Mor}(\delta)$ and $\mathrm{Mor}(\zeta)$ is $\mathrm{Mor}(q)$.
\end{cor}

\begin{prop} The action of $L\mathbb{G}_n$ is given by the following map:
\end{prop}

With this proposition proven, the results in this chapter now collectively describe how to construct free $\mathrm{E}G$-algebras on $n$ invertible objects. Since the argument is obviously arranged in a rather piecemeal fashion, it would be best to restate the general conclusion all in one place.

\begin{thm}\label{freeinvalg} Let $\mathbb{G}_n$ be the free $\mathrm{E}G$-algebra on $n$ objects. Then the free $\mathrm{E}G$-algebra on $n$ invertible objects, $L\mathbb{G}_n$, is the algebra described by
\end{thm}
\begin{proof}
\end{proof}

\subsection{Examples}

With \cref{freeinvalg} proven we can now finally achieve the primary goal of this chapter --- to describe the free braided monoidal category on $n$ invertible objects. In addition, this section will provide a few other simple applications of the theorem, in an effort to build up to the main result more gently. The definition of $L\mathbb{G}_n$ given in \ref{freeinvalg} is after all a little difficult to parse on first reading, because of the fairly abstract way it is presented, and hopefully the following concrete examples should allow the braided case to be properly understood.

\begin{prop} One object case
\end{prop}

\begin{prop} Symmetric case
\end{prop}

\begin{prop} Cactus group case
\end{prop}

.
.
.
.
.


\subsection{The free algebra on $n$ weakly invertible objects}

Up until now, we've been working under the convention that by `invertible' objects we mean stictly invertible --- $x \otimes x^* = I$. As an additional exercise, we can ask ourselves how all of this would change if we permitted our objects to be only weakly invertible, that is $x \otimes x^* \cong I$. The situation is actually quite elegant, in that the effect of weakening in our objects can be offset completely by the effect of also weakening our algebra homomorphisms, such that we won't need to calculate any new free algebras other than those given by \cref{freeinvalg}. Before proving this though, we first to need to set out some definitions.

\begin{defn} Given an $\mathrm{E}G$-algebra $X$, we denote by $X_{\mathrm{wkinv}}$ the category whose
\begin{itemize}
\item objects are tuples $(x, x^*, \eta, \epsilon)$, where $x$ and $x^*$ are objects of $X$ and $\eta: I \to x^* \otimes x$ and $\epsilon : x \otimes x^* \to I$ are morphisms such that the composites
\begin{eq*} \begin{tikzcd}
x \ar[r, "\mathrm{id} \otimes \eta"] & x \otimes x^* \otimes x \ar[r, "\epsilon \otimes \mathrm{id}"] & x &
x^* \ar[r, "\eta \otimes \mathrm{id}"] & x^* \otimes x \otimes x^* \ar[r, "\mathrm{id} \otimes \epsilon"] & x^* 
\end{tikzcd} \end{eq*}
are identity morphisms.
\item maps $(f, f^*): (x, x^*, \eta_x, \epsilon_x) \to (y, y^*, \eta_y, \epsilon_y)$ are pairs $f: x \to y$, $f^* : x^* \to y^*$ of morphisms such that the diagrams
\begin{eq*} \begin{tikzcd}
& I \ar[dl, "\eta_x"'] \ar[dr, "\eta_y"] & & x \otimes x^* \ar[rr, "f \otimes f^*"] \ar[dr, "\epsilon_x"'] & & y \otimes y^* \ar[dl, "\epsilon_y"] \\
x^* \otimes x \ar[rr, "f^* \otimes f"] & & y \otimes y^* & & I &
\end{tikzcd} \end{eq*}
commute.
\end{itemize}
\end{defn}

\begin{defn}\label{weakmonfunc} Let $(X, \alpha)$ and $(Y, \beta)$ be $\mathrm{E}G$-algebras. A \emph{weak $\mathrm{E}G$-algebra homorphism} between them is a weak monoidal functor $\psi: X \to Y$ such that all diagrams of the form
\begin{eq*} \begin{tikzcd}
\psi( x_1 \otimes ... \otimes x_m) \ar[r, "\sim"] \arrow{d}[']{\psi(\alpha(g; h_1, ... , h_m))} & \psi(x_1) \otimes ... \otimes \psi(x_m) \arrow{d}{\beta(g; \psi(h_1), ..., \psi(h_m))} \\
\psi( y_{\pi(g)^{-1}(1)} \otimes ... \otimes y_{\pi(g)^{-1}(m)}) \ar[r, "\sim"] & \psi(y_{\pi(g)^{-1}(1)}) \otimes ... \otimes \psi(y_{\pi(g)^{-1}(m)})
\end{tikzcd} \end{eq*}
commute.
\end{defn} 

\begin{defn} We denote by $\mathrm{E}G\mathrm{Alg}_W$ the 2-category of $\mathrm{E}G$-algebras, weak $\mathrm{E}G$-algebra homomorphisms, and weak monoidal transformations.\end{defn}

Now we can properly express what we mean by the free algebras on weakly invertible objects being the same as those in the strict case.

\begin{thm} The algebra $L\mathbb{G}_n$ is also the free $\mathrm{E}G$-algebra on $n$ weakly invertible objects. Specifically, for any other $\mathrm{E}G$-algebra $X$ there is an equivalence of categories
\begin{eq*} \mathrm{E}G\mathrm{Alg}_W(L\mathbb{G}_n, X) \simeq (X_{\mathrm{wkinv}})^n \end{eq*}
\end{thm}
\begin{proof}
We begin by defining a functor $F : \mathrm{E}G\mathrm{Alg}_W(L\mathbb{G}_n, X) \to (X_{\mathrm{wkinv}})^n$. On weak maps, $F$ acts as 
\begin{eq*} F( \, \psi: L\mathbb{G}_n \to X \, ) = \big\{ \, ( \, \psi(z_i), \, \psi(z_i^*), \, I \xrightarrow{\sim} \psi(I) \xrightarrow{\sim} \psi(z_i^*)\psi(z_i), \, \psi(z_i)\psi(z_i^*) \xrightarrow{\sim} \psi(I) \xrightarrow{\sim} I \, ) \, \big\}_{i \in \{z_1, ..., z_n\} } \end{eq*}
where the $z_i$ are the generators of $\mathbb{Z}^{*n}$ and the isomorphisms are those given by $\psi$ being a weak moniodal functor. On weak monoidal transformations, $F$ acts as
\begin{eq*} F( \, \theta : \psi \to \chi \, ) = \big\{ \, ( \, \theta_{z_i}, \, \theta_{z_i^*} \, ) \, \big\}_{i \in \{z_1, ..., z_n\} }\end{eq*}
This choice does satisfy the condition on morphisms of $(X_{\mathrm{wkinv}})^n$, since we can build the required commuting diagrams out of smaller ones given by $\theta$ being a weak monoidal transfomation:
\begin{eq*} \begin{tikzcd}
& I \ar[dl, "\sim"'] \ar[dr, "\sim"] & & \psi(z_i) \otimes \psi(z_i^*) \ar[rr, "\theta_{z_i} \otimes \theta_{z_i^*}"] \ar[d, "\sim"'] & & \chi(z_i) \otimes \chi(z_i^*) \ar[d, "\sim"] \\
\psi(I) \ar[d, "\sim"'] \ar[rr, "\theta_I"] & & \chi(I) \ar[d, "\sim"] & \psi(I) \ar[dr, "\sim"'] \ar[rr, "\theta_I"] & & \chi(I) \ar[dl, "\sim"] \\
\psi(z_i^*) \otimes \psi(z_i) \ar[rr, "\theta_{z_i^*} \otimes \theta_{z_i}"] & & \chi(z_i^*) \otimes \chi(z_i) & & I & 
\end{tikzcd} \end{eq*}

Now we need to check if $F$ is an equivalence of categories. First, let $\big\{ ( x_i, x_i^*, \eta_i, \epsilon_i ) \big\}_{i \in \{z_1, ..., z_n\} }$ be an arbitrary object of $(X_{\mathrm{wkinv}})^n$. We can construct a weak algebra map $\psi: L\mathbb{G}_n \to X$ from it as follows. Define
\begin{eq*} \psi(I) = I, \quad \psi(z_i) = x_i, \quad \psi(z_i^*) = x_i^* \end{eq*}
and choose the isomorphisms
\begin{eq*} \begin{array}{rllllll}
		\psi_I & : & I \to \psi(I) & = & \mathrm{id}_I & : & I \to I \\
		\psi_{z_i, z_i^*} & : & \psi(z_i) \otimes \psi(z_i^*) \to \psi(I) & = & \epsilon_i & : & x_i \otimes x_i^* \to I \\
		\psi_{z_i^*, z_i} & : & \psi(z_i^*) \otimes \psi(z_i) \to \psi(I) & = & \eta_i^{-1} & : & x_i^* \otimes x_i \to I
		\end{array} .
\end{eq*}
Then for any $w, w' \in \mathrm{Ob}(L\mathbb{G}_n)$ such that $d(w \otimes w') = d(w) \otimes d(w')$, where $d(-)$ is the minimal generator decomposition from \cref{mgd}, set 
\begin{eq*} \psi(w \otimes w') = \psi(w) \otimes \psi(w'), \quad \quad \psi_{w, w'} = \mathrm{id}_{\psi(w) \otimes \psi(w')} \end{eq*}
This is enough to determine the value of $\psi$ on all of the remaining objects, via successive decompositions. For the isomorphisms, first note that the ones we have already defined satisfy the associativity and unitality required of weak monoidal functors. Now consider some $w, w'$ with $d(w \otimes w') \neq d(w) \otimes d(w')$. The fact that they differ implies that tensoring $w$ with $w'$ causes some cancellation of inverses to occur where the end of one sequence meets the beginning of another. In particular, if we let $b$ be the last term in the minimal generator decomposition of $w$, and let $c = w'$, then we conclude that the length $d(b \otimes c)$ is smaller than the length of $d(c)$. Let $a$ be the product of the rest of $d(w)$, so that $a \otimes b = w$. Then we can use requirement for associativity,
\begin{eq*} \begin{tikzcd}
\psi(a) \otimes \psi(b) \otimes \psi(c) \ar[rr, "\mathrm{id} \otimes \psi_{b, c}"] \ar[d, "\psi_{a, b} \otimes \mathrm{id}"'] & & \psi(a) \otimes \psi(b \otimes c) \ar[d, "\psi_{a, b \otimes c}"] \\
\psi(a \otimes b) \otimes \psi(c) \ar[rr, "\psi_{a \otimes b, c}"] && \psi(a \otimes b \otimes c)
\end{tikzcd} \end{eq*}
to define $\psi_{w, w'} = \psi{a\otimes b, c}$ in terms of three other isomorphisms that each have strictly smaller decompositions. Repeating this process will therefore eventually yield a definition in terms of our previous isomorphisms.

By \cref{allmapsaction}, every morphism in $L\mathbb{G}_n$ can be written as $\alpha(g; \mathrm{id}_{w_1}, ..., \mathrm{id}_{w_m})$ for some $g \in G(m)$, $w_i \in \mathbb{Z}^{*n}$. The action of $\psi$ on morphisms is thus determined by the diagram in \cref{weakmonfunc}, that is
\begin{eq*} \psi(\alpha(g; w_1, ... w_m)) \, = \, \psi_{\mathbf{w}_{\pi(g)^{-1}}} \circ \beta(\, g \, ; \, \mathrm{id}_{\psi(w_1)}, \, ..., \, \mathrm{id}_{\psi(w_m)}\, ) \circ \psi_{\mathbf{w}}^{-1}\end{eq*} 
However, morphisms do not have a unique representation of this form, so we must check that whenever we have different representations of the same morphism
\begin{eq*} \alpha(g; \mathrm{id}_{w_1}, ..., \mathrm{id}_{w_m}) = \alpha(g'; \mathrm{id}_{w_1'}, ..., \mathrm{id}_{w_{m'}'}) \end{eq*}
their diagrams give the same image under $\psi$. There are two cases to consider here;
\begin{eq*} \alpha(g; \mathrm{id}_{w_1}, ..., \mathrm{id}_{w_m}) = \alpha( \, g \otimes e_k \, ; \, \mathrm{id}_{w_1}, \, ..., \, \mathrm{id}_{w_m}, \, \mathrm{id}_{v_1}, \, ..., \, \mathrm{id}_{v_k} \, ) \end{eq*}
when $v_1 \otimes ... \otimes v_k = 0$, which comes from the edges of the colimit diagram $D_n$ in \cref{colimthm}; and
\begin{eq*} \begin{array}{rll}
		\alpha(g; \mathrm{id}_{w_1}, ..., \mathrm{id}_{w_m}) & = & \alpha(\, h \, ; \, \mathrm{id}_{w_1'}, \, ..., \, \mathrm{id}_{w_{m'}} \, ) \\
		&& \circ \, \, \alpha(\, j \, ; \, \mathrm{id}_{w_1''}, \, ..., \, \mathrm{id}_{w_{m''}''} \, ) \\
		&& \circ \, \, \alpha(\, h^{-1} \, ; \, \mathrm{id}_{w_1'}, \, ..., \, \mathrm{id}_{w_{m'}'} \, ) \\
		&& \circ \, \, \alpha(\, j^{-1} \, ; \, \mathrm{id}_{w_1''}, \, ..., \, \mathrm{id}_{w_{m''}''} \, ) \\
		& = & \mathrm{id}_{w_1 \otimes ... \otimes w_m} 
		\end{array}
\end{eq*}
for $ \alpha(\, h \, ; \, \mathrm{id}_{w_1'}, \, ..., \, \mathrm{id}_{w_{m'}} \, ), \alpha(\, j \, ; \, \mathrm{id}_{w_1''}, \, ..., \, \mathrm{id}_{w_{m''}''} \, ) \in \mathbb{G}_n(w_1 \otimes ... \otimes w_m,  w_1 \otimes ... \otimes w_m)$, which comes from the abelianisation of the vertices of $D_n$. All other ways for a morphism to have different representations must be generated by successive examples of these cases, since otherwise they wouldn't be coequalised by the colimit in \cref{colimthm}. In the first case we just have
\begin{eq*} \begin{array}{rl}
		& \psi( \, \alpha( \, g \otimes e_k \, ; \, \mathrm{id}_{w_1}, \, ..., \, \mathrm{id}_{w_m}, \, \mathrm{id}_{v_1}, \, ..., \, \mathrm{id}_{v_k} \, ) \, ) \\
		= & \psi_{\mathbf{w}_{\pi(g)^{-1}}, \mathbf{v}} \circ \beta(\, g \otimes e_k \, ; \, \mathrm{id}_{\psi(w_1)}, \, ..., \, \mathrm{id}_{\psi(w_m)}, \, \mathrm{id}_{\psi(v_1)}, \, ..., \, \mathrm{id}_{\psi(v_k)} \, ) \circ \psi_{\mathbf{w}, \mathbf{v}}^{-1} \\
		= & \big( \psi_{\mathbf{w}_{\pi(g)^{-1}}} \otimes \psi_{\mathbf{v}} \big) \circ \big( \beta( g ; \mathrm{id}_{\psi(w_1)}, ..., \mathrm{id}_{\psi(w_m)}) \otimes \mathrm{id}_{\psi(\mathbf{v})} \big) \circ \big( \psi_{\mathbf{w}}^{-1} \otimes \psi_{\mathbf{v}}^{-1} \big) \\
		= & \big( \psi_{\mathbf{w}_{\pi(g)^{-1}}} \circ \beta( g ; \mathrm{id}_{\psi(w_1)}, ..., \mathrm{id}_{\psi(w_m)}) \circ \psi_{\mathbf{w}}^{-1} \big) \otimes \big( \psi_{\mathbf{v}} \circ \mathrm{id}_{\psi(\mathbf{v})} \circ \psi_{\mathbf{v}}^{-1} \big) \\
		= & \psi_{\mathbf{w}_{\pi(g)^{-1}}} \circ \beta( g ; \mathrm{id}_{\psi(w_1)}, ..., \mathrm{id}_{\psi(w_m)}) \circ \psi_{\mathbf{w}}^{-1} \\
		=& \psi( \, \alpha(g; \mathrm{id}_{w_1}, ..., \mathrm{id}_{w_m}) \, )
		\end{array}
\end{eq*}
as required. The second case is more subtle. We begin by expanding
\begin{eq*} \begin{array}{rl}
		& \psi( \, \alpha( \, g \, ; \, \mathrm{id}_{w_1}, \, ..., \, \mathrm{id}_{w_m} \, ) \\
		= & \psi( \, \alpha(\, h \, ; \, \mathrm{id}_{w_1'}, \, ..., \, \mathrm{id}_{w_{m'}} \, ) \, ) \\
		& \circ \, \, \psi( \, \alpha(\, j \, ; \, \mathrm{id}_{w_1''}, \, ..., \, \mathrm{id}_{w_{m''}''} \, ) \, ) \\
		& \circ \, \, \psi( \, \alpha(\, h^{-1} \, ; \, \mathrm{id}_{w_1'}, \, ..., \, \mathrm{id}_{w_{m'}'} \, ) \, ) \\
		&\circ \, \, \psi( \, \alpha(\, j^{-1} \, ; \, \mathrm{id}_{w_1''}, \, ..., \, \mathrm{id}_{w_{m''}''} \, ) \, ) \\
		= & \psi_{\mathbf{w'}} \circ \beta(\, h \, ; \, \mathrm{id}_{\psi(w_1')}, \, ..., \, \mathrm{id}_{\psi(w_{m'})} \, ) \circ \psi_{\mathbf{w'}}^{-1} \\
		& \circ \, \, \psi_{\mathbf{w''}} \circ\beta(\, j \, ; \, \mathrm{id}_{\psi(w_1'')}, \, ..., \, \mathrm{id}_{\psi(w_{m''}'')} \, ) \circ \psi_{\mathbf{w''}}^{-1} \\
		& \circ \, \, \psi_{\mathbf{w'}} \circ \beta(\, h^{-1} \, ; \, \mathrm{id}_{\psi(w_1')}, \, ..., \, \mathrm{id}_{\psi(w_{m'}')} \, ) \circ \psi_{\mathbf{w'}}^{-1}  \\
		&\circ \, \, \psi_{\mathbf{w''}} \circ \beta(\, j^{-1} \, ; \, \mathrm{id}_{\psi(w_1'')}, \, ..., \, \mathrm{id}_{\psi(w_{m''}'')} \, ) \circ \psi_{\mathbf{w''}}^{-1} \\
		\end{array}
\end{eq*}
Here the objects $w_i, w_i', w_i''$ are all in $\mathbb{G}_n \subseteq L\mathbb{G}_n$, and so we know their minimal generator decompositions are also in $\mathbb{G}_n$. It follows that $d(w_i \otimes w_j) = d(w_i) \otimes d(w_j)$ for all $i,j$, and hence by our definition of $\psi$ we have $\psi(w_i \otimes w_j) = \psi(w_i) \otimes \psi(w_j)$ and also $\psi_{\mathbf{w}_{\sigma}} = id$ for any permuation $\sigma$ --- and the same for $\mathbf{w'}$ and $\mathbf{w''}$. Also, note that since we are working in $\mathbb{G}_n(w_1 \otimes ... \otimes w_m,  w_1 \otimes ... \otimes w_m)$, all of the action morphisms in the above composite have the same source and target, $\psi(w_1 \otimes ...\otimes w_m)$. This object is weakly invertible, because each of the $w_i$ are invertible. However, the automorphisms of any weakly invertible object are isomorphic to the automorphisms of the unit object, as in the proof of \cref{zerotree}, and hence form an abelian group, by an Eckmann-Hilton argument like in the proof of \cref{colimthm}. Therefore we may permute these action morphisms freely, and so
\begin{eq*} \begin{array}{rl}
& \psi( \, \alpha( \, g \, ; \, \mathrm{id}_{w_1}, \, ..., \, \mathrm{id}_{w_m} \, ) \\
		= & \beta(\, h \, ; \, \mathrm{id}_{\psi(w_1')}, \, ..., \, \mathrm{id}_{\psi(w_{m'})} \, ) \\
		& \circ \, \, \beta(\, h^{-1} \, ; \, \mathrm{id}_{\psi(w_1')}, \, ..., \, \mathrm{id}_{\psi(w_{m'}')} \, )  \\
		& \circ \, \, \beta(\, j \, ; \, \mathrm{id}_{\psi(w_1'')}, \, ..., \, \mathrm{id}_{\psi(w_{m''}'')} \, ) \\
		& \circ \, \, \beta(\, j^{-1} \, ; \, \mathrm{id}_{\psi(w_1'')}, \, ..., \, \mathrm{id}_{\psi(w_{m''}'')} \, ) \\
		= & \mathrm{id}_{\psi(w_1) \otimes ... \otimes \psi(w_m)} \\
		= & \psi_{\mathbf{w}} \circ \beta(\, e_m \, ; \, \mathrm{id}_{\psi(w_1)}, \, ..., \, \mathrm{id}_{\psi(w_{m})} \, ) \circ \psi_{\mathbf{w}}^{-1}
		\end{array}
\end{eq*}
as required. 

With $\psi$ now fully defined, notice that
\begin{eq*} \begin{array}{rll}
		F(\psi) & = & \big\{ \, ( \, \psi(z_i), \, \psi(z_i^*), \, I \xrightarrow{\sim} \psi(I) \xrightarrow{\sim} \psi(z_i^*)\psi(z_i), \, \psi(z_i)\psi(z_i^*) \xrightarrow{\sim} \psi(I) \xrightarrow{\sim} I \, ) \, \big\}_{i \in \{z_1, ..., z_n\} } \\
		& = & \big\{ \, ( \, x_i, \, x_i^*, \, \eta_i, \, \epsilon_i \, ) \, \big\}_{i \in \{z_1, ..., z_n\} } \\
		\end{array}
\end{eq*}
which was our arbitrary object in $(X_{\mathrm{wkinv}})^n$. Therefore, $F$ is surjective on objects.

Next, choose an arbitrary monoidal transformation $\theta : \psi \to \chi$ from $\mathrm{E}G\mathrm{Alg}_W(L\mathbb{G}_n, X)$. By naturality, for any $w, w' \in \mathrm{Ob}(L\mathbb{G}_n)$ we have that
\begin{eq*} \begin{tikzcd}
\psi(w) \otimes \psi(w') \ar[r, "\sim"] \ar[d, "\theta_w \otimes \theta_{w'}"'] & \psi(w \otimes w') \ar[d, "\theta_{w \otimes w'}"] \\
\chi(w) \otimes \chi(w') \ar[r, "\sim"] & \chi(w \otimes w')
\end{tikzcd} \end{eq*}
or equivalently, $\theta_{w \otimes w'} = \chi_{w, w'} \circ (\theta_w \otimes \theta_{w'}) \circ \psi_{w, w'}^{-1}$. It follows from this that the components of $\theta$ are generated by the components on the generators of $\mathrm{Ob}(L\mathbb{G}_n)$, namely $\{ \, ( \, \theta_{z_i}, \, \theta_{z_i^*} \, ) \, \}_{i \in \{z_1, ..., z_n\} }$. But this is just $F(\theta)$, and thus any monoidal transformation $\theta$ is determined uniquely by its image under $F$, or in other words $F$ is faithful.

Finally, let $\psi, \chi$ be objects of $\mathrm{E}G\mathrm{Alg}_W(L\mathbb{G}_n, X)$, and choose an arbitrary map $\{ \, ( \, f_i, \, f^*_i \, ) \, \}_{i \in \{z_1, ..., z_n\} } : F(\psi) \to F(\chi)$ from $(X_{\mathrm{wkinv}})^n$. We can use this to construct a monoidal transformation $\theta : \psi \to \chi$ via the reverse of process we just used. Specifically, if we define
\begin{eq*} \theta_I = \chi_I \circ \psi_I^{-1}, \quad \quad \theta_{z_i} =  f_i, \quad \quad \theta_{z_i^*} = f_i^*\end{eq*}
then these will automatically form the naturality squares
\begin{eq*} \begin{tikzcd}
\psi(z_i) \otimes \psi(z_i^*) \ar[rr, "\psi_{z_i, z_i^*}"] \ar[dd, "f_i \otimes f_i^*"'] & & \psi(I) \ar[d, "\psi_I^{-1}"] & \psi(z_i^*) \otimes \psi(z_i) \ar[rr, "\psi_{z_i^*, z_i}"] \ar[dd, "f_i^* \otimes f_i"'] & & \psi(I) \ar[d, "\psi_I^{-1}"] \\
& & I \ar[d, "\chi_I"] & & & I \ar[d, "\chi_I"] \\
\chi(z_i) \otimes \chi(z_i^*) \ar[rr, "\chi_{z_i, z_i^*}"] & & \chi(I) & \chi(z_i^*) \otimes \chi(z_i) \ar[rr, "\chi_{z_i^*, z_i}"] & & \chi(I)
\end{tikzcd} \end{eq*}
since these are just the conditions for $\{ \, ( \, f_i, \, f^*_i \, ) \, \}_{i \in \{z_1, ..., z_n\} }$ to be a map $F(\psi) \to F(\chi)$ in $(X_{\mathrm{wkinv}})^n$. Repeatedly applying the naturality condition $\theta_{w \otimes w'} = \chi_{w, w'} \circ (\theta_w \otimes \theta_{w'}) \circ \psi_{w, w'}^{-1}$ will then generate all of the other components of $\theta$, in a way that clearly satisfies naturality. Thus we have a well-defined monoidal transformation $\theta : \psi \to \chi$, and applying $F$ to it gives
\begin{eq*} \begin{array}{rll}
		F(\theta) & = & \big\{ \, ( \, \theta_{z_i}, \, \theta_{z_i^*} \, ) \, \big\}_{i \in \{z_1, ..., z_n\} } \\
		& = & \big\{ \, ( \, f_i, \, f_i^* \, ) \, \big\}_{ i \in \{z_1, ..., z_n\} },
		\end{array}
\end{eq*}
our arbitrary map. Therefore $F$ is full and, putting this together with the previous results, is an equivalence of categories.
\end{proof}