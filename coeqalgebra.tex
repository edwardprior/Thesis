\chapter{Free invertible algebras as colimits}
\label{colimalgebra}

In the previous chapter, we made progress towards understanding the structure of $L\mathbb{G}_n$ by showing that the algebra was an initial object in a certain comma category. Specifically, we saw that the map $\eta: \mathbb{G}_n \to L\mathbb{G}_n$ is initial among all $\mathrm{E}G$-algebra maps $\mathbb{G}_n \to X_{\mathrm{inv}}$. This fact is the rigourous way of expressing a fairly obvious intuition about $L\mathbb{G}_n$ --- that we should expect the free algebra on $n$ invertible objects to be like the free algebra on $n$ objects, except that its objects are invertible.

However, this not the only way of thinking about $L\mathbb{G}_n$. Consider for a moment the free $\mathrm{E}G$-algebra on $2n$ objects, $\mathbb{G}_{2n}$. Intuitively, if we were to take this algebra and then enforce upon it the extra relations $z_{n+1} = z_1^*, ..., z_{2n} = z_n^*$, then we would be changing it from a structure with $2n$ independent generators into one with $n$ indepedent generators and their inverses. That is, there seems to be a natural way to think about $L\mathbb{G}_n$ as a quotient of the larger algebra $\mathbb{G}_{2n}$. In this chapter we will work towards making this idea precise, and then examine some of its consequences. Together with the information we have already gleaned from the initial object persepctive, this will then provide us with a complete description of the algebra $L\mathbb{G}_n$.

\section{$L\mathbb{G}_n$ as a cokernel in $\mathrm{E}G\mathrm{Alg}_S$} 

We'll begin with some definitions.

\begin{defn}\label{qdef} Let $\delta$ be the map of $\mathrm{E}G$-algebras defined on generators by
\begin{eq*} \begin{array}{rlrlll}
			\delta & : & \mathbb{G}_{2n} & \to & \mathbb{G}_{2n} \\
			& : & z_{i} & \mapsto & z_i \otimes z_{n+i} \\
			& : & z_{n+i} & \mapsto & z_{n+i} \otimes z_i			
		\end{array}
\end{eq*}
for $1 \le i \le n$. We will also denote by $q: \mathbb{G}_{2n} \to Q$ the cokernel this map.
\end{defn}

Note that the above definition does actually make sense. The given descriptions of $\delta$ is enough to specify it uniquely because $\mathbb{G}_{2n}$ is the free $\mathrm{E}G$-algebra on $2n$ objects, and hence algebra maps $\mathbb{G}_{2n} \to \mathbb{G}_{2n}$ are canonically isomorphic to functions $\{z_1, ..., z_{2n}\} \to \mathrm{ob}(\mathbb{G}_{2n})$. Also we can be sure that the map $q$ exists, because $\mathrm{E}G\mathrm{Alg}_S$ is a locally finitely presentable category and thus has all finite colimits.

The goal of this approach will be show that $Q$ is in fact that same algebra as $L\mathbb{G}_n$. In order to do this, it would help if we could easily compare $q: \mathbb{G}_{2n} \to Q$ to our initial object $\eta: \mathbb{G}_{2n} \to L\mathbb{G}_n$. In other words, we really want to show that $q$ is an object of $(\mathbb{G}_n \downarrow \mathrm{inv})$ --- that $Q$ has only invertible objects. This can be done using the adjunction we found in \cref{Obadj}.

\begin{prop}\label{Qobj} The object monoid of $Q$ is $\mathbb{Z}^{*n}$, and the restriction of $q$ to objects $\mathrm{Ob}(q): \mathrm{Ob}(\mathbb{G}_{2n}) \to \mathrm{Ob}(Q)$ is the monoid homomorphism defined on generators as
\begin{eq*} \begin{array}{rlrlll}
			\mathrm{Ob}(q) & : & \mathbb{N}^{\ast 2n} & \to & \mathbb{Z}^{\ast n} \\
			& : & z_i & \mapsto & z_i  \\
			& : & z_{n+i} & \mapsto & z_i^*		
		\end{array}
\end{eq*}
\end{prop}
\begin{proof}
Consider $\mathrm{Ob}(\delta)$, the restrictions on objects of the algebra maps $\delta: \mathbb{G}_{2n} \to \mathbb{G}_{2n}$. By \cref{Gnobj}, this is a monoid homomorphism $\mathbb{N}^{\ast 2n} \to \mathbb{N}^{\ast 2n}$, and since $\mathrm{Mon}$ is cocomplete it too must have a cokernel. This will be a new homomorphism whose source is $\mathbb{N}^{\ast 2n}$ and whose target is the quotient of $\mathbb{N}^{\ast 2n}$ by the relations $\mathrm{Ob}(\delta)(x) = I$. Remembering \cref{qdef}, and that $\mathbb{N}^{\ast 2n}$ is the free monoid on $2n$ generators, this quotient monoid will have the following presentation:
\begin{eq*}\begin{array}{ll}
			\text{Generators:} & z_1, \, ..., \, z_{2n} \\
			\text{Relations:} & z_i \otimes z_{n+i} = I, \\
			& z_{n+i} \otimes z_i = I
		\end{array}
\end{eq*}
This is just the same as
\begin{eq*}\begin{array}{ll}
			\text{Generators:} & z_1, \, ..., \, z_{2n} \\
			\text{Relations:} & z_{n+i} = z_i^*, \\
		\end{array}
\end{eq*}
which is the presentation of $\mathbb{Z}^{\ast n}$. 

But by \cref{Obadj}, $\mathrm{Ob}$ is a left adjoint and hence preserves all colimits. Thus the cokernel of $\mathrm{Ob}(\delta)$ is just the underlying homomorphism of the cokernel of $\delta$. Therefore $\mathrm{Ob}(Q) = \mathbb{Z}^{\ast n}$, and $\mathrm{Ob}(q)$ is the quotient map $\mathbb{N}^{\ast 2n} \to \mathbb{Z}^{\ast n}$ sending $z_i \mapsto z_i$ and $z_{n+i} \mapsto z_i^*$ for $1 \le i \le n$.
\end{proof}

An immediate corollary of \cref{Qobj} is that every object of the cokernel algebra $Q$ is invertible. Thus $q: \mathbb{G}_{2n} \to Q$ is an object of the category $(\mathbb{G}_n \downarrow \mathrm{inv})$, and hence we can use the initiality of $\eta$ to determine the following result:

\begin{prop}\label{coker} Let $i: \mathbb{G}_n \to \mathbb{G}_{2n}$ be the inclusion of $\mathrm{E}G$-algebras defined on generators by $i(z_i) = z_i$. Then $i \circ q$ is an initial object of $(\mathbb{G}_n \downarrow \mathrm{inv})$. In particular, this means that
\begin{eq*} Q \quad \cong \quad L\mathbb{G}_n \end{eq*}
\end{prop}
\begin{proof}
Let $\psi: \mathbb{G}_n \to X$ be an arbitrary object of $(\mathbb{G}_n \downarrow \mathrm{inv})$. Since $\mathbb{G}_n$ is the free $\mathrm{E}G$-algebra on $n$ objects, we can use it and $\psi$ to define a new map, $\psi^*: \mathbb{G}_n \to X$, which takes the values
\begin{eq*} \psi^*(z_i) \quad := \quad \psi(z_i)^* \end{eq*}
on generators. Using these two functors we can define a new map, $\psi + \psi^*$, via the universal property of the colimit:
\begin{eq*} \begin{tikzcd}
& \mathbb{G}_n + \mathbb{G}_n \ar[dd, dashed, "\psi + \psi^*"] & \\
\mathbb{G}_n \ar[ur, hookrightarrow, "i"] \ar[dr, "\psi"'] & & \mathbb{G}_n \ar[ul, hookrightarrow, "i'"'] \ar[dl, "\psi^*"] \\
& X & 
\end{tikzcd} \end{eq*}
But because $\mathbb{G}_n$ is the free algebra on $n$ objects, and the free functor $F : \mathrm{Cat} \to \mathrm{E}G\mathrm{Alg}_S$ is a left adjoint and thus preserves colimits, we must have
\begin{eq*} \begin{array}{rll}
		\mathbb{G}_n + \mathbb{G}_n & = & F(\{ z_1, ..., z_n\}) + F(\{ z'_1, ..., z'_n\}) \\
		& = & F( \, \{ z_1, ..., z_n\} + \{ z'_1, ..., z'_n\} \, ) \\
		& = & F(\{ z_1, ..., z_{2n} \}) \\
		& = & \mathbb{G}_{2n} 
		\end{array}
\end{eq*}
This means that we can compose $\psi + \psi^*: \mathbb{G}_{2n} \to X$ with the map $\delta: \mathbb{G}_{2n} \to  \mathbb{G}_{2n}$, though we need to be careful to specify exactly which inclusions we used in the definition of $\psi + \psi^*$. Suppose that the lefthand inclusion is $i$, the one given in the statement of the proposition, and the other is defined by the assignment $z_i \mapsto z_{i+n}$. Then for $1 \leq i \leq n$,
\begin{eq*} \begin{array}{rll}
			(\psi + \psi^*)\delta(z_i) & = & (\psi + \psi^*)(z_i \otimes z_{n+i}) \\
			& = & \psi(z_i) \otimes \psi(z_i)^* \\
			& = & I \\
			& & \\
			(\psi + \psi^*)\delta(z_{n+i}) & = & (\psi + \psi^*)(z_{n+i} \otimes z_i) \\
			& = & \psi(z_i)^* \otimes \psi(z_i) \\
			& = & I
		\end{array}
\end{eq*}
That is, $(\psi + \psi^*) \circ \delta = I$. But we've already defined $q: \mathbb{G}_{2n} \to Q$ to be the cokernel of $\delta$, the universal map with this property, and so there must exist a unique $\mathrm{E}G$-algebra map $u: Q \to X$ making the righthand triangle below diagram commute:
\begin{eq*} \begin{tikzcd}
\mathbb{G}_n \ar[rr, hookrightarrow, "i"] \ar[ddrr, "\psi"'] & & \mathbb{G}_{2n} \ar[rr, "q"] \ar[dd, "\psi + \psi^*", near start] & & Q \ar[ddll, "u"] \\
& & & & \\ 
& & X & &
\end{tikzcd} \end{eq*}
The other triangle commutes by the definition of $\psi + \psi^*$, and so together the diagram tells us that for any object $\psi$ of $(\mathbb{G}_n \downarrow \mathrm{inv})$, there exists at least one morphism $u$ in $(\mathbb{G}_n \downarrow \mathrm{inv})$ going from $q \circ i$ to $\psi$. 

Next, let $v: Q \to X$ be an arbitrary morphism $q \circ i \to \psi$ in $(\mathbb{G}_n \downarrow \mathrm{inv})$. By definition, this means that
\begin{eq*}\begin{array}{rll}
			\psi & = & vqi \\
			\implies \quad \psi + \psi^* & = & vqi + (vqi)^* 
		\end{array}
\end{eq*}
Also, for $1 \leq i \leq n$ we have
\begin{eq*}\begin{array}{rcrllcccl}
			q(z_i) \otimes q(z_{n+i}) & = & q(z_{i-n} \otimes z_i) & = & q\delta(z_i) & = &  I \\
			q(z_{n+i}) \otimes q(z_i) & = & q(z_i \otimes z_{n+i}) & = & q\delta(z_{n+i}) & = & I \\
			& \implies & q(z_{n+i}) & = & q(z_i)^* & & & &
		\end{array}
\end{eq*}
Therefore,
\begin{eq*}\begin{array}{rll}
			(\psi + \psi^*)(z_i) & = & \big( vqi + (vqi)^* \big)(z_i) \\
			& = & vqi(z_i) \\
			& = & vq(z_i) \\
		\end{array}
\end{eq*}
\begin{eq*} \begin{array}{rll}
			(\psi + \psi^*)(z_{n+i}) & = & \big( vqi + (vqi)^* \big)(z_{n+i}) \\
			& = & vqi(z_i)^* \\
			& = & v \big( q(z_i)^* \big) \\
			& = & vq(z_{n+i})
		\end{array}
\end{eq*}
or in other words $\psi + \psi^* = v \circ q$ for any morphism $v: q \circ i \to \psi$ in $(\mathbb{G}_n \downarrow \mathrm{inv})$. But this is the property that the map $u$ was supposed to satisfy uniquely, and thus it must be the only morphism $q \circ i \to \psi$ in $(\mathbb{G}_n \downarrow \mathrm{inv})$. Therefore $q \circ i$ is an initial object, and hence it is isomorphic in $(\mathbb{G}_n \downarrow \mathrm{inv})$ to any other initial object, such as $\eta$. It follows that the targets of these two maps, $Q$ and $L\mathbb{G}_n$ respectively, are isomorphic as $\mathrm{E}G$-algebras.
\end{proof}

It's worth noting that we have not given a method for actually taking cokernels in $\mathrm{E}G\mathrm{Alg}_S$, and so \cref{coker} doesn't immediately provide an explicit description for the whole of $L\mathbb{G}_n$. However, it does offer us another way to extract partial information, like what we were doing in \cref{initialalgebra}. Consider \cref{Qobj}; now that we know that $Q$ is actually $L\mathbb{G}_n$, the statement of this proposition is just the same as that of \cref{Zobj}. But the proof of the former uses the ability of cokernels to preserve left adjoint functors, rather than any of the initial algebra and group completion properties that appear in the latter.

Of course, by \cref{coker} the fact that $q$ is a cokernel is equivalent to it being initial, and so while they may not look it at first glance, these two approaches are secretly the same. Thus from now on whenever we are trying to determine some aspect of $L\mathbb{G}_n$, we will make sure to take a look at both methods, just in case there are some properties of our free algebra which are more readily apparent from one description than another.

\section{$L\mathbb{G}_n$ as a surjective coequaliser}

One immediate consequence our new cokernel perspective of $L\mathbb{G}_n$ is that, since left adjoint functor all preserve colimits, \cref{Obadj,concompadj} now both imply results about the partial surjectivity of this new map $q$. The former says that since $\mathrm{Ob}(q)$ is a cokernel map of monoids, and hence that every object of $L\mathbb{G}_n$ is the image under $q$ of some object of $\mathbb{G}_{2n}$; the latter says a similar thing for connected components. From this one might guess that $q$ is will just turn out to be a surjective map of $\mathrm{E}G$-algebras, and indeed this is the case. Moreover, much as \cref{Zobj} is an analogue of \cref{Gnobj}, the fact that $q$ is surjective on morphisms means that there will be a result analagous to \cref{Gnmapsaction} as well. That is, since every morphism of $\mathbb{G}_{2n}$ is an action morphism, and since $\mathrm{E}G$-algebra maps always send action morphisms to action morphisms, if $q$ is surjective then every morphism of $L\mathbb{G}_n$ is also an action morphism. 

Unfortunately, we can not go about proving that $q$ is surjective on morphisms by a similar adjunction technique, since this best we have is the one from \cref{Moradj} and it will only tell us about the map $\mathrm{Mor}(q)^{\mathrm{gp}, \mathrm{ab}}$. However, there is a general result about the coequalisers of $\mathrm{E}G$-algebras that we can prove to get us around this.

\begin{prop}\label{coeqsurj} Let $\phi, \phi' : X \to Y$ be a pair of parallel $\mathrm{E}G$-algebra maps, and $k: Y \to Z$ their coequalizer in $\mathrm{E}G\mathrm{Alg}_S$. If the monoid $\mathrm{Ob}(Z)$ is also a group, then the functor $k$ is surjective.
\end{prop}
\begin{proof}
We begin by mirroring the proof of \cref{Qobj}. We know that the functor $\mathrm{Ob} : \mathrm{E}G\mathrm{Alg}_S \to \mathrm{Mon}$ is a left adjoint, by \cref{Obadj}, and thus preserves all colimits. It follows that the monoid homomorphism $\mathrm{Ob}(k): \mathrm{Ob}(Y) \to \mathrm{Ob}(Z)$ is the coequaliser of the parallel pair $\mathrm{Ob}(\phi), \mathrm{Ob}(\phi') : \mathrm{Ob}(X) \to \mathrm{Ob}(Y)$ in $\mathrm{Mon}$, or in other words
\begin{eq*} \mathrm{Ob}(Z) \quad = \quad \bigquotient{\mathrm{Ob}(Y)}{\sim}\end{eq*}
where $\sim$ is the relation defined by
\begin{eq*}\mathrm{Ob}(\phi)(y) \sim \mathrm{Ob}(\phi')(y), \quad \quad \quad a \sim a', b \sim b' \implies ab \sim a'b' \end{eq*}
The map $\mathrm{Ob}(k): \mathrm{Ob}(Y) \to \mathrm{Ob}(Y)/\sim$ is then clearly surjective.

Next, let $f: v \to w$ and $f' : w' \to v'$ be any two morphisms of the algebra $Y$ for which $k(f)$ and $k(f')$ are composable in $Z$. Since these maps are composable we know that $k(w)$ and $k(w')$ must be the same object of $Z$, and since $Z$ is a group we know this object has an inverse $k(w)^* = k(w')^*$. So by the surjectivity of $k$ we can find another object $y$ of $Y$ for which $k(y) = k(w)^*$. Using this, define the morphism $h: x \to x'$ to be the tensor product $f' \otimes \mathrm{id}_y \otimes f$. Then
\begin{eq*} \begin{array}{rll}
		k(h) & = & k(f' \otimes \mathrm{id}_y \otimes f) \\
		& = & k(f') \otimes \mathrm{id}_{k(y)} \otimes k(f) \\
		& = & k(f') \otimes \mathrm{id}_{k(w)^*} \otimes k(f)
		\end{array}
\end{eq*}
But by \cref{tenscomp}, this is really just the composite $k(f') \circ k(f)$. Thus the set of morphisms of $Z$ which are images of morphisms of $Y$ is closed under composition. 

So now consider $k(Y)$, the subcategory of $Z$ that contains every object $x'$ for which there exists $x$ in $Y$ with $k(x) = x'$, and every morphism $f'$ for which there exists $f$ in $Y$ with $q(f) = f'$. We know that the morphisms of $k(Y)$ are closed under composition, and so this is indeed a well-defined category. Moreover, for any collection of morphisms $f'_1, ..., f'_m$ of $k(Y)$ we'll have
\begin{eq*} \begin{array}{rll}
 			\alpha_{Z}(g; f'_1, ..., f'_m) & = & \alpha_Z\big( \, g \, ; \, k(f_1), ..., k(f_m) \, \big) \\
			& = & k \big( \, \alpha_{Y}(g; f_1, ..., f_m) \, \big) \\
			& \in & k(Y) 
		\end{array}
\end{eq*}
for some $f_1, ..., f_m$, since $k$ is a map of $\mathrm{E}G$-algebras. Thus $k(Y)$ is also a well-defined sub-$\mathrm{E}G$-algebra of $Z$. There is also clearly a canonical map $k': Y \to k(Y)$, the unique surjective map of $\mathrm{E}G$-algebras with the property that $k'(x) = k(x)$ for any object $x$ and $k'(f) = k(f)$ for any morphism $f$. If we denote by $i$ the evident inclusion of algebras $i: k(Y) \hookrightarrow Z$, then these maps are related by the fact that $i \circ k' = k$.
\begin{eq*} \begin{tikzcd}
& & X \ar[dd, bend right, "\phi"'] \ar[dd, bend left, "\phi'"] & & \\
& & & & \\
& & Y \ar[ddll, "k'"'] \ar[dd, "k"] \ar[ddrr, "j"] & & \\ 
& & & & \\
k(Y) \ar[rr, hookrightarrow, "i"] & & Z \ar[rr, "u"] & & U
\end{tikzcd} \end{eq*}
Given all of this, let $j: Y \to U$ be any map of $\mathrm{E}G$-algebras with the property that $j \circ \phi = j \circ \phi'$. Since $h$ is the coequaliser of $\phi$ and $\phi'$, it follows that there exists a unique map $u:  Y \to U$ such that $j = u \circ k$. This means that $j = u \circ i \circ k'$, and hence there is obviously at least one map, $u \circ i$, which lets us factors $j$ through $k'$. But for any other map $v: k(Y) \to U$ that factors $j$ like this, we'll have
\begin{eq*} \begin{array}{rrll}
			& v \circ k' & = & j \\
			& & = & u \circ i \circ k' \\
			\implies \quad & v & = & u \circ i
		\end{array}
\end{eq*}
because $k'$ is surjective, and thus $u \circ i$ is the unique map with this property. That is, $k'$ is also a coequaliser of $\phi$ and $\phi'$. But colimits are always unique up to a unique isomorphism, and so there should be a unique invertible map $k(Y) \to Z$ factoring $k$ through $k'$. This is clearly just the inclusion $i$, and as a result $k(Y) = Z$ and $k' = k$. In other words, the map coequaliser mapm $k$ is surjective. 
\end{proof}

Because a cokernel of a morphism is just its coequaliser with the zero map, and since we know that the objects of $L\mathbb{G}_n$ form a group, we can immediately apply this result to the functor $q$.

\begin{cor}\label{qsurj} The cokernel map $q: \mathbb{G}_{2n} \to L\mathbb{G}_n$ is surjective.
\end{cor}

As we noted earlier, knowing that $q$ is surjective will now allow use to take \cref{Gnmapsaction}, a statement about the morphisms $\mathbb{G}_{2n}$, and extend the result onto $L\mathbb{G}_n$:

\begin{lem} \label{allmapsaction} Every morphism in $L\mathbb{G}_n$ can be expressed as $\alpha_{L\mathbb{G}_n}(g; \mathrm{id}_{x_1}, ..., \mathrm{id}_{x_m})$, for some $g \in G(m)$ and $x_i \in \{z_1, ..., z_n, z_1^*, ..., z_n^* \}$.
\end{lem}
\begin{proof}
Let $f$ be an arbitrary morphism in $L\mathbb{G}_n$. By surjectivity of $q$, there must exist at least one morphism $f'$ in $\mathbb{G}_{2n}$ such that $q(f') = f$, and from \cref{Gnmapsaction} we know that this $f'$ can be expressed uniquely as $\alpha(g; \mathrm{id}_{x'_1}, ..., \mathrm{id}_{x'_m})$ for some $g \in G(m)$ and $x'_i \in \{z_1, ..., z_{2n} \}$. Thus, because $q$ is a map of $\mathrm{E}G$-algebras, we will have
\begin{eq*}\begin{array}{rll}
			f & = & q(f') \\
			& = & q\big( \, \alpha_{\mathbb{G}_{2n}}( \, g \, ; \, \mathrm{id}_{x'_1}, ..., \mathrm{id}_{x'_m} \, ) \, \big) \\
			& = & \alpha_{L\mathbb{G}_n}( \, g \, ; \, \mathrm{id}_{q(x'_1)}, ..., \mathrm{id}_{q(x'_m)} \, ) 
		\end{array}
\end{eq*}
Therefore there is at least one collection of $x_i = q(x'_i)$ for which the statement of the proposition holds.
\end{proof}

\cref{allmapsaction} formalises a certain intuition about how the functor $L$ should act on algebras, the idea that a `free' structure really shouldn't have any `superfluous' components, only whatever data is absolutely required for it to be well-defined. In the case of $L\mathbb{G}_n$, we have proven that the only morphisms contained in the free $\mathrm{E}G$-algebra on invertible objects are $\mathrm{E}G$-action morphisms. However, while this is very similar to what we have in the non-invertible case it should be stressed that \cref{allmapsaction} does \emph{not} prove that the morphisms of $L\mathbb{G}_n$ have \emph{unique} representations $\alpha(g; \mathrm{id}_{w_1}, ..., \mathrm{id}_{w_m})$, as morphisms of $\mathbb{G}_n$ do.

\section{$L\mathbb{G}_n$ as a colimit in $\mathrm{MonCat}$}

Looking back at the proof of \cref{coeqsurj}, notice that we never needed to use the fact that $\phi$, $\phi'$ and $k$ were maps of $\mathrm{E}G$-algebras, only that they were monoidal functors. Because we had assumed from the beginning that we were working in $\mathrm{E}G\mathrm{Alg}_S$, we did at one point have to show that the category $k(Y)$ was an algebra, so that we could then use the universal property of $k$ in $\mathrm{E}G\mathrm{Alg}_S$, but if $k$ had just been a coequaliser in $\mathrm{MonCat}$ from the start then this part would not have been necessary. We also had to invoke \cref{Obadj} --- which says that $\mathrm{Ob}: \mathrm{E}G\mathrm{Alg}_S \to \mathrm{Mon}$ is a left adjoint --- so that we could exploit preservation of colimits. But since $\mathrm{Ob}$ clearly doesn't care about the morphisms of an algebra, it doesn't really matter whether we are applying it to an algebra in the first place. The actions of $X$, $Y$ and $Z$ just never came into play.

With that in mind, we can co-opt all of these previous proofs about $\mathrm{E}G$-algebra maps to prove the analagous statements about monoidal functors.

\begin{prop}\label{Obadjmon} Let the functors 
\begin{eq*} \mathrm{Ob} \, : \, \mathrm{MonCat} \to \mathrm{Mon}, \quad \quad \quad \mathrm{E} \, : \, \mathrm{Mon} \to \mathrm{MonCat} \end{eq*}
be defined exactly as those from \cref{Obdef,Edef}, except without the requirement that the monoidal categories be $\mathrm{E}G$-algebras. Then $\mathrm{E}$ is a right adjoint to the functor $\mathrm{Ob}$. 
\end{prop}
\begin{proof}
The same as the proof of \cref{Obadj}.
\end{proof}

\begin{prop} \label{coeqsurjmon} Let $\phi, \phi' : X \to Y$ be a pair of parallel monoidal functors, and $k: Y \to Z$ their coequalizer in $\mathrm{Mon}$. If the monoid $\mathrm{Ob}(Z)$ is also a group, then the functor $k$ is surjective.
\end{prop}
\begin{proof}
The same as the proof of \cref{coeqsurj}, but with \cref{Obadjmon} in place of \cref{Obadj}, and no reference to $k(Y)$ being a sub-$\mathrm{E}G$-algebra.
\end{proof}

Further, these new propositions prove a surjectivity statement just like \cref{qsurj}. 

\begin{defn}\label{Cdef} Let the monoidal functor $c: \mathbb{G}_{2n} \to C$ onto some monoidal category $C$ be the cokernel of the underlying monoidal functor of $\delta$ in $\mathrm{MonCat}$. This map definitely exists because $\mathrm{MonCat}$ is cocomplete, and like with $q$ we can show that its target has a group of objects.
\end{defn}

\begin{prop}\label{Cobj} The object monoid of $C$ is $\mathbb{Z}^{*n}$, and the restriction of $c$ to objects $\mathrm{Ob}(c): \mathrm{Ob}(\mathbb{G}_{2n}) \to \mathrm{Ob}(C)$ is the monoid homomorphism defined on generators as
\begin{eq*} \begin{array}{rlrlll}
			\mathrm{Ob}(c) & : & \mathbb{N}^{\ast 2n} & \to & \mathbb{Z}^{\ast n} \\
			& : & z_i & \mapsto & z_i  \\
			& : & z_{n+i} & \mapsto & z_i^*		
		\end{array}
\end{eq*}
\end{prop}
\begin{proof}
The same as the proof of \cref{Qobj}, but with $c: \mathbb{G}_{2n} \to C$ in place of $q: \mathbb{G}_{2n} \to Q$ and \cref{Obadjmon} in place of \cref{Obadj}.
\end{proof}

\cref{coeqsurjmon,Cobj} then immediately combine to give:

\begin{cor}\label{csurj} The cokernel map $c: \mathbb{G}_{2n} \to C$ is surjective.
\end{cor}

This statement is actually pretty unusual. In \cref{qsurj} it made sense that $q$ would be surjective, but that was because its source and target were special. $\mathbb{G}_{2n}$ is the free $\mathrm{E}G$-algebra on $2n$ objects, and $L\mathbb{G}_n$ is the free $\mathrm{E}G$-algebra on $n$ objects and their $n$ inverses, and so intuitively the map identifying those sets generators would tell us everything we need to know about the algebra structure of $L\mathbb{G}_n$. And since by freeness we expect algebra maps to be all there really is to $L\mathbb{G}_n$, it was a safe bet that $q$ was going to be surjective.

But none of that is true for $c$. The underlying monoidal category of $\mathbb{G}_{2n}$ is not anything special in $\mathrm{MonCat}$, and neither is $C$. So what is going on here? The answer is that category $C$ is \emph{almost} the algebra $L\mathbb{G}_n$, and likewise the functor $c$ is \emph{almost} the map $q$. To see this, consider the following niave method for assigning an $\mathrm{E}G$-action $\alpha^C$ to $C$:

\begin{eq*} \alpha^C( \, g \, ; \, c(f_1), ..., c(f_m) \, ) \quad := \quad c \big( \, \alpha^{\mathbb{G}_{2n}}( \, g \, ; \, f_1, ..., f_m \, ) \, \big) \end{eq*}

Any action on $C$ that made $c$ into a map of $\mathrm{E}G$-algebras would have to satify this condition, of course. But because $c$ is surjective, every collection of morphisms in $C$ can be written as $c(f_1), ..., c(f_m)$, and this forces $\alpha^C$ to take a unique value everywhere, assuming it is well-defined. Then, since the the cokernel of $\delta$ in $\mathrm{MonCat}$ would be an $\mathrm{E}G$-algebra map, we could conclude that it was also the cokernel of $\delta$ in $\mathrm{E}G\mathrm{Alg}_S$ too.

However, `assuming it is well-defined' is where the problems lie. In particular, since $c$ is not injective on objects we can find $w_1, ..., w_m$ and $w'_1, ..., w'_m$ in $\mathbb{G}_{2n}$ for which $c(w_i) = c(w'_i)$, and so $\alpha^C$ would only be well-defined if

\begin{eq*} c \big( \, \alpha^{\mathbb{G}_{2n}}( \, g \, ; \, \mathrm{id}_{w_1}, ..., \mathrm{id}_{w_m} \, ) \, \big) \quad = \quad c \big( \, \alpha^{\mathbb{G}_{2n}}( \, g \, ; \, \mathrm{id}_{w'_1}, ..., \mathrm{id}_{w'_m} \, ) \, \big) \end{eq*}

which we have no reason to believe is true. Fixing this issue is not to difficult though, it just means that we have to employ yet more colimits.
 
\begin{defn}\label{Pdef} Denote by $P$ following the pushout in $\mathrm{MonCat}$:
\begin{eq*} \begin{tikzcd}
\mathrm{E}G \times_{\mathbb{N}} F(\mathbb{G}_{2n}) \ar[ddd, "\mathrm{id} \times F(c)"'] \ar[rrr, "\alpha"] & & & \mathbb{G}_{2n} \ar[rrr, "c"] & & & C \ar[ddd, "i_C"] \\
& & & & & &\\
& & & & & & \\
\mathrm{E}G \times_{\mathbb{N}} F(C) \ar[rrrrrr, "i_{\mathrm{E}G \times F(C)}"] & & & & & & P \ar[uuulll, phantom, "\mathlarger{\mathlarger{\mathlarger{\mathlarger{\ulcorner}}}}", very near start]
\end{tikzcd} \end{eq*}
We'll also denote the composite $i_C \circ c$ by the name $p$.
\end{defn}

The form of this pushout has been chosen precisely so that the category $P$ and the functor $p$ will retain all of the important features that we liked about $C$ and $c$, whilst eventually allowing us to show that they belong in the category $\mathrm{E}G\mathrm{Alg}_S$. In particular:

\begin{lem}\label{psurj} The functor $p: \mathbb{G}_{2n} \to P$ is surjective.
\end{lem}
\begin{proof}
We know from \cref{csurj} that $c$ is surjective, and hence so is the map $\coprod_m (\mathrm{id} \times c^m): \coprod_m \mathrm{E}G(m) \times (\mathbb{G}_{2n})^m \to \coprod_m \mathrm{E}G(m) \times C^m$. But surjective functors are preserved by pullback, and so the map $i_C$ must also be surjective, from which it follows that $p =  i_C \circ c$ is too. 
\end{proof}

\begin{prop}\label{Pobj} The object monoid of $P$ is $\mathbb{Z}^{*n}$, and the restriction of $p: \mathbb{G}_{2n} \to P$ to objects is the same as the underlying monoid homomorphism of $c$.
\end{prop}
\begin{proof}
To start off with, take the pushout diagram in \cref{Pdef} and apply the functor $\mathrm{Ob}: \mathrm{MonCat} \to \mathrm{Mon}$ to the whole thing. Since this is a left adjoint functor, the resulting diagram,
\begin{eq*} \begin{tikzcd}
G \times_{\mathbb{N}} F(\mathbb{N}^{\ast 2n}) \ar[ddd, "\mathrm{id} \times \mathrm{Ob}(Fc)"'] \ar[rrr, "\mathrm{Ob}(\alpha)"] & & & \mathbb{N}^{\ast 2n} \ar[rrr, "\mathrm{Ob}(c)"] & & & \mathbb{Z}^{\ast n} \ar[ddd, "\mathrm{Ob}(i_C)"] \\
& & & & & &\\
& & & & & & \\
G \times F(\mathbb{Z}^{\ast n}) \ar[rrrrrr, "\mathrm{Ob}(i_{\mathrm{E}G \times F(C)})"] & & & & & & \mathrm{Ob}(P) \ar[uuulll, phantom, "\mathlarger{\mathlarger{\mathlarger{\mathlarger{\ulcorner}}}}", very near start]
\end{tikzcd} \end{eq*}
must be a pullback diagram of monoids. But recall that $\alpha(g; x_1, ..., x_m) = x_1 \otimes ... \otimes x_m$ for any objects $x_i$ and element $g \in G(m)$. This means that the homomorphism $\mathrm{Ob}(\alpha)$ is just the map turning concatenation in $F(\mathbb{N}^{\ast 2n})$ into multiplication and ignoring the $G$ component. It is also possible to form a similar concatenation-to-multiplication map $f: G \times F(\mathbb{Z}^{\ast n}) \to \mathbb{Z}^{\ast n}$, and since $\mathrm{Ob}(c)$ as a monoid homomorphism respects multiplication, the top-left triangle created by this new map below will commute.
\begin{eq*} \begin{tikzcd}
G \times_{\mathbb{N}} F(\mathbb{N}^{\ast 2n}) \ar[ddd, "\mathrm{id} \times \mathrm{Ob}(Fc)"'] \ar[rrr, "\mathrm{Ob}(\alpha)"] & & & \mathbb{N}^{\ast 2n} \ar[rrr, "\mathrm{Ob}(c)"] & & & \mathbb{Z}^{\ast n}  \ar[ddd, "\mathrm{Ob}(i_C)"] \\
& & & & & &\\
& & & & & & \\
G \times F(\mathbb{Z}^{\ast n}) \ar[rrrrrruuu, "f"] \ar[rrrrrr, "\mathrm{Ob}(i_{\mathrm{E}G \times F(C)})"] & & & & & &  \mathrm{Ob}(P) \ar[uuulll, phantom, "\mathlarger{\mathlarger{\mathlarger{\mathlarger{\ulcorner}}}}", very near start]
\end{tikzcd} \end{eq*}
Because $c$ is surjective, $\mathrm{id} \times \mathrm{Ob}(Fc)$ will be to, and this lets us conclude that the bottom-right triangle will also commute. In other words, the pushout map $\mathrm{Ob}(i_{\mathrm{E}G \times F(C)})$ is just the composite $\mathrm{Ob}(i_C) \circ f$. 

Indeed, the same reasoning tells us that for any monoid $M$ which is a cocone of the maps $\mathrm{id} \times \mathrm{Ob}(Fc)$ and $\mathrm{Ob}(c) \circ \mathrm{Ob}(\alpha)$, the canonical map $G \times F(\mathbb{Z}^{\ast n}) \to M$ will just be the composite of the map $\mathbb{Z}^{\ast n} \to M$ with $f$. If we denote by $u: \mathrm{Ob}(P) \to M$ the unique map we get from the universal property of $\mathrm{Ob}(P)$, the condition that $u$ factorises both of the maps into $M$ will then just depend on its ability to uniquely factorise the one corresponding to $\mathrm{Ob}(i_C)$. But if we choose $\mathrm{Ob}(i_C)$ to be the identity homomorphism, then this property holds automatically --- the unique value of $u$ would just be the same as the canonical $\mathbb{Z}^{\ast n} \to M$. Thus the monoid $\mathbb{Z}^{\ast n}$ equipped with the maps $\mathrm{id}_{\mathbb{Z}^{\ast n}}$ and $f$ is pushout of our diagram in $\mathrm{Mon}$, and hence
\begin{eq*} \begin{array}{rclcrcl}
			 \mathrm{Ob}(P) & \cong & \mathbb{Z}^{\ast n}, & \quad \quad \quad \quad \quad & \mathrm{Ob}(p) & = & \mathrm{Ob}(i_C) \circ \mathrm{Ob}(c) \\
			& & & & & = & \mathrm{id}_{\mathbb{Z}^{\ast n}} \circ \mathrm{Ob}(c) \\
			& & & & & = & \mathrm{Ob}(c)
		\end{array}
\end{eq*}
\end{proof}

With these results in hand, a simple calculation causes the algebra nature of $P$ to drop straight out of its definition.

\begin{prop}\label{Paction} There is a unique action $\alpha^P$ making the category $P$ into $\mathrm{E}G$-algebra and the functor $p: \mathbb{G}_{2n} \to P$ into a map of $\mathrm{E}G$-algebras.  
\end{prop}
\begin{proof}
We will try to affix an action to $P$ in the same way we thought about doing with the category $C$. In order for the functor $p : \mathbb{G}_{2n} \to P$ to be an $\mathrm{E}G$-algebra map with respect to some $\alpha^P$, it must satisfy
\begin{eq*} \alpha^P( \, g \, ; \, p(f_1), ..., p(f_m) \, ) \quad = \quad p \big( \, \alpha^{\mathbb{G}_{2n}}( \, g \, ; \, f_1, ..., f_m \, ) \, \big) \end{eq*}
for all morphisms $f_1, ..., f_m$ in $\mathbb{G}_{2n}$. But from \cref{psurj} we know that $p$ is surjective, and so this condition will actually suffice as a definition for $\alpha^P$, provided that we can prove it to be well-defined. To that end, let $f_i: v_i \to w_i$ and $f'_i : v'_i \to w'_i$, for $1 \le i \le m$, be any two sequences of morphisms in $\mathbb{G}_{2n}$ such that $p(f_i) = p(f'_i)$. Then by \cref{Pobj},
\begin{eq*} \begin{array}{rrll}
			& p(f_i : v_i \to w_i) & = & p(f'_i : v'_i \to w'_i) \\
			\implies & p(w_i) & = & p(w'_i) \\
			\implies & c(w_i) & = & c(w'_i)
		\end{array}
\end{eq*}
and hence
\begin{eq*} \begin{array}{rll}
			p \alpha^{\mathbb{G}_{2n}}( \, g \, ; \, f_1, ..., f_m \, ) & = & p \big( \, \alpha^{\mathbb{G}_{2n}}( \, g \, ; \, \mathrm{id}_{w_1}, ..., \mathrm{id}_{w_m} \, ) \circ ( f_1 \otimes ... \otimes f_m) \, \big) \\
			& = & p \alpha^{\mathbb{G}_{2n}}( \, g \, ; \, \mathrm{id}_{w_1}, ..., \mathrm{id}_{w_m} \, ) \circ \big( \, p(f_1) \otimes ... \otimes p(f_m) \, \big) \\
			& = & i_C c \alpha^{\mathbb{G}_{2n}}( \, g \, ; \, \mathrm{id}_{w_1}, ..., \mathrm{id}_{w_m} \, ) \circ \big( \, p(f_1) \otimes ... \otimes p(f_m) \, \big) \\
			& = & i_{\mathrm{E}G \times F(C)}( \, g \, ; \, \mathrm{id}_{c(w_1)}, ..., \mathrm{id}_{c(w_m)} \, ) \circ \big( \, p(f_1) \otimes ... \otimes p(f_m) \, \big) \\
			& = &  i_{\mathrm{E}G \times F(C)}( \, g \, ; \, \mathrm{id}_{c(w'_1)}, ..., \mathrm{id}_{c(w'_m)} \, ) \circ \big( \, p(f'_1) \otimes ... \otimes p(f'_m) \, \big) \\
			& = & p \alpha^{\mathbb{G}_{2n}}( \, g \, ; \, f'_1, ..., f'_m \, )		
		\end{array} 
\end{eq*}
Therefore the value of $\alpha^P( \, g \, ; \, p(f_1), ..., p(f_m) \, )$ does not depend on our particular choice of $f_i$, and so $\alpha^P$ is indeed well-defined. Thus the pushout category $P$ is an $\mathrm{E}G$-algebra, and $p$ an $\mathrm{E}G$-algebra map. 
\end{proof}

Now we're finally ready to address problem 1 from the end of the previous chapter: how can we deal with the fact that our adjunction $\mathrm{Mor}( \, \_ \, )^{\mathrm{gp}, \mathrm{ab}} \dashv C$ just involves monoidal categories and not full $\mathrm{E}G$-algebras? The answer is that this is all we really need, as despite us originally conceiving of $L\mathbb{G}_n$ as a colimit in $\mathrm{E}G\mathrm{Alg}_S$ it can equally be viewed as a slightly more complicated colimit in $\mathrm{MonCat}$.

\begin{prop} \label{p=q} The colimit functor $p: \mathbb{G}_{2n} \to P$ defined in \cref{Pdef} is isomorphic as an map of $\mathrm{E}G$-algebras to $q: \mathbb{G}_{2n} \to L\mathbb{G}_n$, the cokernel of $\delta$ in $\mathrm{E}G\mathrm{Alg}_S$.
\end{prop}
\begin{proof}
First, consider what we know of the functor $p$. By definition it is equal to the composite $i_C \circ c$, where $c: \mathbb{G}_{2n} \to C$ is the cokernel of the map $\delta: \mathbb{G}_{2n} \to \mathbb{G}_{2n}$ in $\mathrm{MonCat}$, and hence $p$ has the property
\begin{eq*} p \circ \delta \quad = \quad i_C \circ c \circ \delta \quad = \quad i_C \circ I \quad = \quad I \end{eq*}
Moreover, given what we saw in \cref{Paction} we know that $p$ is map of $\mathrm{E}G$-algebras with this property. But the cokernel map $q$ is universal amongst maps like these, and so it follows that there must exist a unique map of $\mathrm{E}G$-algebras $u: L\mathbb{G}_n \to P$ factoring $p$ through $q$.

Conversely, now consider the algebra map $q$. We know that it is a monoidal functor for which $q \circ \delta = I$, and that $c$ is the universal map in $\mathrm{MonCat}$ with this property. Thus there exists a unique monoidal functor $v : C \to L\mathbb{G}_n$ with $q = v \circ c$, or put another way, the two triangles in the diagram below both commute.
\begin{eq*} \begin{tikzcd}
\mathrm{E}G \times_{\mathbb{N}} F(\mathbb{G}_{2n}) \ar[ddd, "\mathrm{id} \times F(c)"'] \ar[rrr, "\alpha^{\mathbb{G}_{2n}}"] \ar[dddrrr, "\mathrm{id} \times F(q)"] & & & \mathbb{G}_{2n} \ar[rrr, "c"] \ar[dddrrr, "q"] & & & C \ar[ddd, "v"] \\
& & & & & &\\
& & & & & & \\
\mathrm{E}G \times_{\mathbb{N}} F(C) \ar[rrr, "\mathrm{id} \times F(v)"'] & & & \mathrm{E}G \times_{\mathbb{N}} F(L\mathbb{G}_n) \ar[rrr, "\alpha^{L\mathbb{G}_{n}}"'] & & & L\mathbb{G}_n
\end{tikzcd} \end{eq*}
The middle square here also commutes --- due to the fact that $q$ is a map of $\mathrm{E}G$-algebras --- and so we can conclude that the entire diagram does as well. But this just says that $L\mathbb{G}_n$ is a cocone of the same diagram we used to define the pushout category $P$, and hence that there exists some monoidal functor $u': P \to L\mathbb{G}_n$ factoring the diagram above through the one in \cref{Pdef}. In particular, we get that
\begin{eq*} u' \circ i_C \quad = \quad v \quad \quad \implies \quad \quad u' \circ p \quad = \quad u' \circ i_C \circ c \quad = \quad v \circ c \quad = \quad q \end{eq*}
Putting this fact together with $u \circ q = p$ that we saw earlier, and also the surjectivity of both $q$ and $p$ (from \cref{qsurj,psurj} respectively), we can conclude that the maps $u$ and $u'$ form a isomorphism of monoidal categories:
\begin{eq*} \begin{array}{rccclcrcl}
			u \circ u' \circ p & = & u \circ q & = & p & \quad \implies \quad  & u \circ u' & = & \mathrm{id}_P \\
			u' \circ u \circ q & = & u' \circ p & = & q & \quad \implies \quad & u' \circ u & = & \mathrm{id}_{ L\mathbb{G}_n}
		\end{array}
\end{eq*}
Furthermore, not only is $u$ an algebra map but $u'$ is too, as by surjectivity of $p$ we have 
\begin{eq*} \begin{array}{rll}
			u' \big( \, \alpha^P( \, g \, ; \, f_1, ..., f_m \, ) \, \big) & = & u' \big( \, \alpha^P( \, g \, ; \, p(f'_1), ..., p(f'_m) \, ) \, \big) \\
			& = & u' p \big( \, \alpha^{\mathbb{G}_{2n}}( \, g \, ; \, f'_1, ..., f'_m \, ) \, \big) \\
			& = & q \big( \, \alpha^{\mathbb{G}_{2n}}( \, g \, ; \, f'_1, ..., f'_m \, ) \, \big) \\
			& = & \alpha^{L\mathbb{G}_{n}}( \, g \, ; \, q(f'_1), ..., q(f'_m) \, ) \, \big) \\
			& = & \alpha^{L\mathbb{G}_{n}}( \, g \, ; \, u'p(f'_1), ..., u'p(f'_m) \, ) \, \big) \\
			& = & \alpha^{L\mathbb{G}_{n}}( \, g \, ; \, u'(f_1), ..., u(f_m) \, ) \, \big) \\
		\end{array}
\end{eq*}
Therefore $u$ and $u'$ are also an isomorphism of $\mathrm{E}G$-algebras. In other words $P \cong L\mathbb{G}_n$, and up to this isomorphism the maps $p: \mathbb{G}_{2n} \to P$ and $q: \mathbb{G}_{2n} \to L\mathbb{G}_n$ are the same.
\end{proof}

\section{The abelianised morphisms of $L\mathbb{G}_n$}

With our newfound ability to express the map $q: \mathbb{G}_{2n} \to L\mathbb{G}_n$ as a colimit of monoidal categories, we can now set about using the adjunction from \cref{Moradj} to calculate $\mathrm{Mor}(L\mathbb{G}_n)^{\mathrm{gp}, \mathrm{ab}}$. The most obvious way to do this is to mimic what we did in \cref{Qobj} --- apply the left adjoint functor to $q$ and then commute it with the colimit to get a formula in terms of the known monoid $\mathrm{Mor}(\mathbb{G}_{2n})$.

\begin{prop}\label{Zmor2} The abelianisation of the group completion of the morphism monoid of $L\mathbb{G}_n$ is 
\begin{eq*} \mathrm{Mor}(L\mathbb{G}_n)^{\mathrm{gp, ab}} \quad = \quad \mathrm{Mor}(C)^{\mathrm{gp, ab}} \quad = \quad \bigquotient{{\mathrm{Mor}(\mathbb{G}_{2n})}^{\mathrm{gp, ab}}}{\mathrm{im}\big( \, {\mathrm{Mor}(\delta)}^{\mathrm{gp, ab}} \, \big)} \end{eq*}
\end{prop}
\begin{proof}
From \cref{Moradj}, we know that $\mathrm{Mor}(\, \_ \,)^{\mathrm{gp, ab}}: \mathrm{MonCat} \to \mathrm{Ab}$ is a left adjoint functor. This means that it preserves all colimits in $\mathrm{MonCat}$, including the cokernel category $C$ and the pushout category $P$, which from \cref{p=q} we now know to be $L\mathbb{G}_n$. 

For the first of these, we have
\begin{eq*} \mathrm{coker}\big( \, \mathrm{Mor}(\delta)^{\mathrm{gp, ab}} \, \big) \quad = \quad \mathrm{Mor}\big( \, \mathrm{coker}(\delta) \, \big)^{\mathrm{gp, ab}} \quad = \quad \mathrm{Mor}(c)^{\mathrm{gp, ab}} \end{eq*}
or in other words, the following is a cokernel diagram in the category of abelian groups:
\begin{eq*} \begin{tikzcd}
\mathrm{Mor}(\mathbb{G}_{2n})^{\mathrm{gp, ab}} \ar[rrr, "\mathrm{Mor}(\delta)^{\mathrm{gp, ab}}"] & & &
\mathrm{Mor}(\mathbb{G}_{2n})^{\mathrm{gp, ab}} \ar[rrr, "\mathrm{Mor}(q)^{\mathrm{gp, ab}}"] & & &
\mathrm{Mor}(C)^{\mathrm{gp, ab}}
\end{tikzcd} \end{eq*} 
But the cokernel of an abelian group homomorphism is just the quotient of its target group by its image. Hence in this case we have
\begin{eq*} \mathrm{Mor}(C)^{\mathrm{gp, ab}} \quad = \quad \bigquotient{{\mathrm{Mor}(\mathbb{G}_{2n})}^{\mathrm{gp, ab}}}{\mathrm{im}\big( \, {\mathrm{Mor}(\delta)}^{\mathrm{gp, ab}} \, \big)} \end{eq*}
Likewise, using the definition of $P$ we can construct the pushout diagram of abelian groups below:
\begin{eq*} \begin{tikzcd}
\mathrm{Mor}\big( \, \mathrm{E}G \times_{\mathbb{N}} F(\mathbb{G}_{2n}) \, \big)^{\mathrm{gp, ab}} \ar[ddd, "\mathrm{Mor}\big( \, \mathrm{id} \times F(c) \, \big)^{\mathrm{gp, ab}}"] \ar[rr, "\mathrm{Mor}(\alpha)^{\mathrm{gp, ab}}"] & & \mathrm{Mor}(\mathbb{G}_{2n})^{\mathrm{gp, ab}} \ar[rr, "\mathrm{Mor}(c)^{\mathrm{gp, ab}}"] & & \mathrm{Mor}(C)^{\mathrm{gp, ab}} \ar[ddd, "\mathrm{Mor}(i_C)^{\mathrm{gp, ab}}"'] \\
& & & & \\
& & & &  \\
\mathrm{Mor}\big( \, \mathrm{E}G \times_{\mathbb{N}} F(C) \, \big)^{\mathrm{gp, ab}} \ar[rrrr, "\mathrm{Mor}(i_{\mathrm{E}G \times F(C)})^{\mathrm{gp, ab}}"] & & & & \mathrm{Mor}(L\mathbb{G}_n)^{\mathrm{gp, ab}} \ar[uuull, phantom, "\mathlarger{\mathlarger{\mathlarger{\mathlarger{\ulcorner}}}}", very near start]
\end{tikzcd} \end{eq*}

Now, for two arbitrary abelian group homomorphisms $h:A \to B$, $h':A \to B'$, their pushout is the quotient of $B \times B'$ by a subgroup $H$ which contains all elements of the form $(h_1(a), h_2(a)^{-1})$. Moreover, if the map $h'$ is surjective then the image in $(B \times B') / H$ of any element $(b, b') \in B \times B'$ is the same as one whose second coordinate is the identity, since
\begin{eq*} \begin{array}{rll}
			[ \, (b, b') \, ] & = & \big[ \, \big(\, b, h'(a) \, \big) \, \big] \\[\medskipamount]
			& = & \big[ \, \big( \, b, h'(a) \, \big) \cdot \big( \, h(a), h'(a)^{-1} \, \big) \, \big] \\[\medskipamount]
 			& = & \big[ \, \big( \, b \cdot h(a), e \, \big) \, \big]
		\end{array}
\end{eq*}
and elements of this form are equivalent if and only if their first coordinates differ by an element of $h(\mathrm{ker}(h'))$:
\begin{eq*} \begin{array}{rclcrrrll}
			[ \, (b_1, e) \, ] & = & [ \, (b_2, e) \, ] & \quad \implies \quad & \exists \, a \in A & : & (b_1, e) & = & (b_2, e) \big( \, h(a), h'(a)^{-1} \, \big) \\
			& & & \quad \implies \quad & \exists \, a \in A & : & (b_1, e) & = & \big( \, b_2 \cdot h(a), h'(a)^{-1} \, \big) \\
			& & & \quad \implies \quad & \exists \, a \in \mathrm{ker}(h') & : & b_1 & = & b_2 \cdot h(a) \\
		\end{array}
\end{eq*}
Therefore the pushout $(B \times B') / H$ is really just the same as the quotient $B / h(\mathrm{ker}(h'))$.

Returning to our specific homomorphisms, we know from \cref{csurj} that $c$ is surjective, and hence so is $\mathrm{id} \times F(c)$. In particular it is surjective on morphisms, and so it follows that the homomorphism $\mathrm{Mor}(\mathrm{id}_{\mathrm{E}G} \times F(c))^{\mathrm{gp, ab}}$ will also be surjective. This means that for our pushout we'll have
\begin{eq*} \mathrm{Mor}(L\mathbb{G}_n)^{\mathrm{gp, ab}} \quad = \quad \bigquotient{\mathrm{Mor}(C)^{\mathrm{gp, ab}}}{\mathrm{Mor}(c \circ \alpha)^{\mathrm{gp, ab}}\bigg( \, \mathrm{ker}\Big( \, \mathrm{Mor}\big( \, \mathrm{id}_{\mathrm{E}G} \times F(c) \, \big)^{\mathrm{gp, ab}} \, \Big) \, \bigg)} \end{eq*}
Futhermore,
\begin{eq*} \begin{array}{rll}
			\mathrm{ker}\Big( \, \mathrm{Mor}\big( \, \mathrm{id}_{\mathrm{E}G} \times F(c) \, \big)^{\mathrm{gp, ab}} \, \Big) & = & \mathrm{ker}\Big( \, \mathrm{Mor}(\mathrm{id}_{\mathrm{E}G})^{\mathrm{gp, ab}} \times \mathrm{Mor}\big( \, F(c) \, \big)^{\mathrm{gp, ab}} \, \Big) \\
			 & = & \{e\} \times \mathrm{ker}\Big( \, F\big( \, \mathrm{Mor}(c)^{\mathrm{gp, ab}} \, \big) \, \Big) \\
			 & = & \{e\} \times F\Big( \, \mathrm{ker}\big( \, \mathrm{Mor}(c)^{\mathrm{gp, ab}} \, \big) \, \Big) \\
			 & = & \{e\} \times F\Big( \, \mathrm{ker} \, \mathrm{coker}\big( \, \mathrm{Mor}(\delta)^{\mathrm{gp, ab}} \, \big) \, \Big) \\
			 & = & \{e\} \times F\Big( \, \mathrm{im}\big( \, \mathrm{Mor}(\delta)^{\mathrm{gp, ab}} \, \big) \, \Big)
		\end{array}
\end{eq*}
where this $e$ is the identity element of the group
\begin{eq*} \mathrm{Mor}(\mathrm{E}G)^{\mathrm{gp, ab}} \quad = \quad G^{\mathrm{gp, ab}} \end{eq*}
Recalling \cref{}, we see that elements of $\mathrm{ker}(\mathrm{Mor}(\mathrm{id}_{\mathrm{E}G} \times F(c))^{\mathrm{gp, ab}})$ can therefore all be expressed uniquely in the form $[ ( e_m ; \delta(f_1), ..., \delta(f_m))]$, for some $m \in \mathbb{N}$ and $f_1, ..., f_m \in \mathrm{Mor}(\mathbb{G}_{2n})$. But then
\begin{eq*} \begin{array}{rll}
			\mathrm{Mor}(c \circ \alpha)^{\mathrm{gp, ab}}\big[ \, \big( \, e_m \, ; \, \delta(f_1), ..., \delta(f_m) \, \big) \, \big] & = & \Big[ \, c \Big( \, \alpha\big( \, e_m \, ; \, \delta(f_1), ..., \delta(f_m) \, \big) \, \Big) \, \Big] \\[\bigskipamount]
			& = & \big[ \, c \delta \big( \, \alpha( \, e_m \, ; \, f_1, ..., f_m \, ) \, \big) \, \big] \\[\medskipamount]
			& = & [ \mathrm{id}_I ]
		\end{array}
\end{eq*}
and so we have
\begin{eq*} \mathrm{Mor}(c \circ \alpha)^{\mathrm{gp, ab}}\bigg( \, \mathrm{ker}\Big( \, \mathrm{Mor}\big( \, \mathrm{id}_{\mathrm{E}G} \times F(c) \, \big)^{\mathrm{gp, ab}} \, \Big) \, \bigg) \quad = \quad \{ \mathrm{id}_{[I]} \} \end{eq*}
\begin{eq*} \begin{array}{rll}
		\implies \quad \mathrm{Mor}(L\mathbb{G}_n)^{\mathrm{gp, ab}} & = & \bigquotient{\mathrm{Mor}(C)^{\mathrm{gp, ab}}}{\{ \mathrm{id}_{[I]} \}} \\[\bigskipamount] 
		& = & \mathrm{Mor}(C)^{\mathrm{gp, ab}} \\[\bigskipamount]
		& = & \bigquotient{{\mathrm{Mor}(\mathbb{G}_{2n})}^{\mathrm{gp, ab}}}{\mathrm{im}\big( \, {\mathrm{Mor}(\delta)}^{\mathrm{gp, ab}} \, \big)}
		\end{array}
\end{eq*}
as required.
\end{proof} 

\cref{Zmor2} has at last given us a complete, usable description of the abelian group $\mathrm{Mor}(L\mathbb{G}_n)^{\mathrm{gp, ab}}$. However, its worth noting that the resulting quotient group is slightly odd, almost inefficient in some sense. The numerator involves the free $\mathrm{E}G$-algebra on $2n$ objects, twice as many as we're really interested in, and then we essentially fix this by having the denominator be the image of $\delta$, a map defined by sending each generator to a \emph{pair} of generators. We might ask if there is some way for these two doublings to cancel out, yielding a simpler description of $\mathrm{Mor}(L\mathbb{G}_n)^{\mathrm{gp, ab}}$.

But of course there is, we've seen it already. We know that the algebra map $q$ isn't just a cokernel, but also an initial object in the comma category $(\mathbb{G}_n \downarrow \mathrm{inv})$ --- indeed, we saw in \cref{coeq} that these are really equivalent properties of $q$. Now that we have this way of expanding the view of $q$ from being cokernel in $\mathrm{E}G\mathrm{Alg}_S$ to a larger colimit in the context of $\mathrm{MonCat}$, we can pass this back through the equivalence, and produce a corresponding extension of $q$'s initial object description.

We need to be careful here though. Up until now, we've been rather loose about distinguishing algebras from their underlying monoidal categories. In what follows it will help to be more precise.

\begin{defn} Denote by $U: \mathrm{E}G\mathrm{Alg}_S \to \mathrm{MonCat}$ the functor sending each $\mathrm{E}G$ to its underlying monoidal category, and each $\mathrm{E}G$-algebra map to its underlying monoidal functor. \end{defn}

\begin{prop} Let $U(i): U(\mathbb{G}_n) \to U(\mathbb{G}_{2n})$ be the obvious inclusion of monoidal categories, which is also the underlying monoidal functor of the obvious inclusion of algebras defined on generators by $i(z_i) = z_i$. Then $U(i) \circ c$ is an initial object of the comma category $(U(\mathbb{G}_n) \downarrow U(\mathrm{inv}))$.
\end{prop}
\begin{proof}
This proposition is clearly a direct parallel of \cref{coker}, and it proceeds as such.

Let $\psi: U(\mathbb{G}_n) \to X$ be an arbitrary object of $(U(\mathbb{G}_n) \downarrow U(\mathrm{inv}))$. From this, define a new functor $\psi_2: U(\mathbb{G}_{2n}) \to X$ by
\begin{eq*} \begin{array}{rrrll}
			\psi_2 & : & U(\mathbb{G}_{2n} & \to & X \\
			& : & z_i & \mapsto & \psi(z_i) \\ 
			& : & z_{n+i} & \mapsto & \psi(z_i)^* \\ 
			& : & \alpha( \, g \, ; \, \mathrm{id}_{x_1}, ..., \mathrm{id}_{x_m} \, ) & \mapsto & \alpha( \, g \, ; \, \mathrm{id}_{x_1}, ..., \mathrm{id}_{x_m} \, )
		\end{array}
\end{eq*}
for $1 \le i \le n$. This description is sufficient to defin $\psi_2$, since the $z_i$ generate the objects of $U(\mathbb{G}_{2n})$ under tensor product, and by \cref{Gnmapsaction} every morphism in $U(\mathbb{G}_{2n})$ can be expressed uniquely as This means that we can compose $\psi + \psi^*: \mathbb{G}_{2n} \to X$ with the maps $\delta: \mathbb{G}_{2n} \to  \mathbb{G}_{2n}$, though we need to be careful to specify exactly which inclusions we used in the definition of $\psi + \psi^*$. Suppose that the lefthand inclusion is $i$, the one given in the statement of the proposition, and the other is defined by the assignment $z_i \mapsto z_{i+n}$. Then for $1 \leq i \leq n$,
\begin{eq*} \begin{array}{rll}
			(\psi + \psi^*)\delta(z_i) & = & (\psi + \psi^*)(z_i \otimes z_{n+i}) \\
			& = & \psi(z_i) \otimes \psi(z_i)^* \\
			& = & I \\ 
			& & \\
			(\psi + \psi^*)\delta(z_{n+i}) & = & (\psi + \psi^*)(z_{n+i} \otimes z_i) \\
			& = & \psi(z_i)^* \otimes \psi(z_i) \\
			& = & I
		\end{array}
\end{eq*}
That is, $(\psi + \psi^*) \circ \delta = I$. But we've already defined $q: \mathbb{G}_{2n} \to Q$ to be the cokernel of $\delta$, the universal map with this property, and so there must exist a unique $\mathrm{E}G$-algebra map $u: Q \to X$ making the righthand triangle below diagram commute:
\begin{eq*} \begin{tikzcd}
\mathbb{G}_n \ar[rr, hookrightarrow, "i"] \ar[ddrr, "\psi"'] & & \mathbb{G}_{2n} \ar[rr, "q"] \ar[dd, "\psi + \psi^*", near start] & & Q \ar[ddll, "u"] \\
& & & & \\ 
& & X & &
\end{tikzcd} \end{eq*}
The other triangle commutes by the definition of $\psi + \psi^*$, and so together the diagram tells us that for any object $\psi$ of $(\mathbb{G}_n \downarrow \mathrm{inv})$, there exists at least one morphism $u$ in $(\mathbb{G}_n \downarrow \mathrm{inv})$ going from $q \circ i$ to $\psi$. 

Next, let $v: Q \to X$ be an arbitrary morphism $q \circ i \to \psi$ in $(\mathbb{G}_n \downarrow \mathrm{inv})$. By definition, this means that
\begin{eq*}\begin{array}{rll}
			\psi & = & vqi \\
			\implies \quad \psi + \psi^* & = & vqi + (vqi)^* 
		\end{array}
\end{eq*}
Also, for $1 \leq i \leq n$ we have
\begin{eq*}\begin{array}{rcrllcccl}
			q(z_i) \otimes q(z_{n+i}) & = & q(z_{i-n} \otimes z_i) & = & q\delta(z_i) & = &  I \\
			q(z_{n+i}) \otimes q(z_i) & = & q(z_i \otimes z_{n+i}) & = & q\delta(z_{n+i}) & = & I \\
			& \implies & q(z_{n+i}) & = & q(z_i)^* & & & &
		\end{array}
\end{eq*}
Therefore,
\begin{eq*}\begin{array}{rll}
			(\psi + \psi^*)(z_i) & = & \big( vqi + (vqi)^* \big)(z_i) \\
			& = & vqi(z_i) \\
			& = & vq(z_i) \\
		\end{array}
\end{eq*}
\begin{eq*} \begin{array}{rll}
			(\psi + \psi^*)(z_{n+i}) & = & \big( vqi + (vqi)^* \big)(z_{n+i}) \\
			& = & vqi(z_i)^* \\
			& = & v \big( q(z_i)^* \big) \\
			& = & vq(z_{n+i})
		\end{array}
\end{eq*}
or in other words $\psi + \psi^* = v \circ q$ for any morphism $v: q \circ i \to \psi$ in $(\mathbb{G}_n \downarrow \mathrm{inv})$. But this is the property that the map $u$ was supposed to satisfy uniquely, and thus it must be the only morphism $q \circ i \to \psi$ in $(\mathbb{G}_n \downarrow \mathrm{inv})$. Therefore $q \circ i$ is an initial object, and hence it is isomorphic in $(\mathbb{G}_n \downarrow \mathrm{inv})$ to any other initial object, such as $\eta$. It follows that the targets of these two maps, $Q$ and $L\mathbb{G}_n$ respectively, are isomorphic as $\mathrm{E}G$-algebras.
\end{proof}











