\section{Free invertible algebras as initial objects}
\label{initialalgebra}


\subsection{The free algebra on $n$ invertible objects}

We saw in \cref{freealg} that the existence of a free $\mathrm{E}G$-algebra on $n$ objects can be proven by taking the left adjoint of a 2-functor which forgets about the algebra structure. Now we want to extend this idea into the realm of algebras on invertible objects. For the analogous approach, we will need to find a new 2-functor that lets us forget about non-invertible objects, and then hopefully we can find its left adjoint too, and use it to freely add inverses to $\mathbb{G}_n$. First though, we need to make this concept of `forgetting non-invertible objects' a little more precise.

\begin{defn} Given an $\mathrm{E}G$-algebra $X$, we denote by $X_{\mathrm{inv}}$ the sub-$\mathrm{E}G$-algebra containing all invertible objects in $X$ and the isomorphisms between them. \end{defn}

Note that this is indeed a well-defined $\mathrm{E}G$-algebra. If $x_1, ..., x_m$ are invertible objects with inverses $x_1^*, ..., x_m^*$, then $\alpha(g; x_1, ..., x_m)$ is an invertible object with inverse $\alpha(g; x_m^*, ..., x_1^*)$, since 
\begin{eq*} \begin{array}{ll}
		& \alpha(g; x_1, ..., x_m) \otimes \alpha(g; x_m^*, ..., x_1^*) \\
		= & \big( x_{\pi(g)^{-1}(1)} \otimes ... \otimes x_{\pi(g)^{-1}(m)} \big) \otimes \big( x_{\pi(g)^{-1}(m)}^* \otimes ... \otimes x_{\pi(g)^{-1}(1)}^* \big) \\
		= & I \\
		& \\
		& \alpha(g; x_m^*, ..., x_1^*) \otimes \alpha(g; x_1, ..., x_m) \\
		= & \big( x_{\pi(g)^{-1}(m)}^* \otimes ... \otimes x_{\pi(g)^{-1}(1)}^* \big) \otimes \big( x_{\pi(g)^{-1}(1)} \otimes ... \otimes x_{\pi(g)^{-1}(m)} \big) \\
		= & I
		\end{array}
\end{eq*}
Likewise, if $f_1, ..., f_m$ are isomorphisms from invertible objects $x_1, ..., x_m$ to invertible objects $y_1, ..., y_m$, then $\alpha(g; f_1, ..., f_m)$ is a map from the invertible object $\alpha(g; x_1, ..., x_m)$ to the invertible object $\alpha(g; y_1, ..., y_m)$, and it has an inverse $\alpha(g^{-1}; f_{\pi(g)(1)}^{-1}, ..., f_{\pi(g)(m)}^{-1})$, since
\begin{eq*} \begin{array}{ll}
		& \alpha\big( \, g^{-1} \, ; \, f_{\pi(g)(1)}^{-1}, \, ..., \, f_{\pi(g)(m)}^{-1} \, \big) \circ \alpha( \, g \, ; \, f_1, ..., f_m \,) \\
		= & \alpha\big( \, g^{-1}g \, ; \, f_1^{-1} f_1, \, ..., \, f_m^{-1} f_m \, \big) \\
		= & \mathrm{id}_{x_1 \otimes ... \otimes x_m} \\
		& \\
		& \alpha( \, g \, ; \, f_1, ..., f_m \,) \circ \alpha\big( \, g^{-1} \, ; \, f_{\pi(g)(1)}^{-1}, \, ..., \, f_{\pi(g)(m)}^{-1} \, \big) \\
		= & \alpha\big( \, gg^{-1} \, ; \, f_{\pi(g)(1)} f_{\pi(g)(1)}^{-1}, \, ..., \, f_{\pi(g)(m)} f_{\pi(g)(m)}^{-1} \, \big) \\
		= & \mathrm{id}_{y_{\pi(g)(1)} \otimes ... \otimes y_{\pi(g)(m)}}
		\end{array}
\end{eq*}
Clearly then, $X_{\mathrm{inv}}$ is the correct algebra for our new forgetful 2-functor to send $X$ to. Knowing this, we can contruct the rest of the functor fairly easily.

\begin{prop} \label{invprop} The assignment $X \mapsto X_{\mathrm{inv}}$ can be extended to a 2-functor $(\_)_{\mathrm{inv}}: \mathrm{E}G\mathrm{Alg}_S \to \mathrm{E}G\mathrm{Alg}_S$.
\end{prop}
\begin{proof}
Let $F: X \to Y$ be a (strict) map of $\mathrm{E}G$-algebras. If $x$ is an invertible object in $X$ with inverse $x^*$, then $F(x)$ is an invertible object in $Y$ with inverse $F(x^*)$, by
\begin{eq*} F(x) \otimes F(x^*) = F(x \otimes x^*) = F(I) = I \end{eq*}
\begin{eq*} F(x^*) \otimes F(x) = F(x^* \otimes x) = F(I) = I \end{eq*}
Since $F$ sends invertible objects to invertible objects, it will also send isomorphisms of invertible objects to isomorphisms of invertible objects. In other words, the map $F: X \to Y$ can be restricted to a map $F_{\mathrm{inv}} : X_{\mathrm{inv}} \to Y_{\mathrm{inv}}$. Moreover, we have that
\begin{eq*} (F \circ G)_{\mathrm{inv}}(x) = F \circ G(x) = F_{\mathrm{inv}} \circ G_{\mathrm{inv}}(x) \end{eq*}
\begin{eq*} (F \circ G)_{\mathrm{inv}}(f) = F \circ G(f) = F_{\mathrm{inv}} \circ G_{\mathrm{inv}}(f) \end{eq*}
and so the assignment $F \mapsto F_{\mathrm{inv}}$ is clearly functorial. Next, let $\theta : F \Rightarrow G$ be an $\mathrm{E}G$-monoidal natural transformation. Choose an invertible object $x$ from $X$, and consider the component map of its inverse, $\theta_{x^*} : F(x^*) \to G(x^*)$. Since $\theta$ is monoidal, we have $\theta_{x^*} \otimes \theta_x = \theta_I = I$ and $\theta_x \otimes \theta_{x^*} = I$, or in other words that $\theta_{x^*}$ is the monoidal inverse of $\theta_x$. We can use this fact to construct a compositional inverse as well, namely $\mathrm{id}_{F(x)} \otimes \theta_{x^*} \otimes \mathrm{id}_{G(x)}$, which can be seen as follows:
\begin{eq*}  \begin{array}{rll}
		\big( \mathrm{id}_{F(x)} \otimes \theta_{x^*} \otimes \mathrm{id}_{G(x)} \big)  \circ \theta_x & = & \theta_x \otimes \theta_{x^*} \otimes \mathrm{id}_{G(x)} \\
		& = &  \mathrm{id}_{G(x)} \\
		&& \\
		\theta_x \circ  \big( \mathrm{id}_{F(x)} \otimes \theta_{x^*} \otimes \mathrm{id}_{G(x)} \big) & = & \mathrm{id}_{F(x)} \otimes \theta_{x^*} \otimes \theta_x \\
		& = &  \mathrm{id}_{F(x)} \\
		\end{array} 
\end{eq*}
Therefore, we see that all the components of our transformation on invertible objects are isomorphisms, and hence we can define a new transformation $\theta_{\mathrm{inv}}: F_{\mathrm{inv}} \Rightarrow G_{\mathrm{inv}}$ whose components are just $(\theta_{\mathrm{inv}})_x = \theta_x$. The assignment $\theta \mapsto \theta_{\mathrm{inv}}$ is also clearly functorial, and thus we have a complete 2-functor $(\_)_{\mathrm{inv}}: \mathrm{E}G\mathrm{Alg}_S \to \mathrm{E}G\mathrm{Alg}_S$.
\end{proof}

\begin{prop} The 2-functor $(\_)_{\mathrm{inv}}: \mathrm{E}G\mathrm{Alg}_S \to \mathrm{E}G\mathrm{Alg}_S$ has a left adjoint, $L: \mathrm{E}G\mathrm{Alg}_S \to \mathrm{E}G\mathrm{Alg}_S$.
\end{prop}
\begin{proof} To begin, consider the 2-monad $\mathrm{E}G(\_)$. This is a finitary monad, that is it preserves all filtered colimits, and it is a 2-monad over $\mathrm{Cat}$, which is locally finitely presentable. It follows from this that $\mathrm{E}G\mathrm{Alg}_S$ is itself locally finitely presentable. Thus if we want to prove $(\_)_{\mathrm{inv}}$ has a left adjoint, we can use the Adjoint Functor Theorem for locally finitely presentable categories, which amounts to showing that $(\_)_{\mathrm{inv}}$ preserves both limits and filtered colimits.
\begin{itemize}
\item Given an indexed collection of $\mathrm{E}G$-algebras $X_i$, the $\mathrm{E}G$-action of their product $\prod X_i$ is defined componentwise. In particular, this means that the tensor product of two objects in $\prod X_i$ is just the collection of the tensor products of their components in each of the $X_i$. An invertible object in $\prod X_i$ is thus simply a family of invertible objects from the $X_i$ --- in other words, $(\prod X_i)_{\mathrm{inv}} = \prod (X_i)_{\mathrm{inv}}$.
\item Given maps of $\mathrm{E}G$-algebras $F: X \to Z$, $G : Y \to Z$, the $\mathrm{E}G$-action of their pullback $X \times_Z Y$ is also defined componentwise. It follows that an invertible object in $X \times_Z Y$ is just a pair of invertible objects $(x, y)$ from $X$ and $Y$, such that $F(x) = G(y)$. But this is the same as asking for a pair of objects $(x, y)$ from $X_{\mathrm{inv}}$ and $Y_{\mathrm{inv}}$ such that $F_{\mathrm{inv}}(x) = G_{\mathrm{inv}}(y)$, and hence $(X \times_Z Y)_{\mathrm{inv}} = X_{\mathrm{inv}} \times_{Z_{\mathrm{inv}}} Y_{\mathrm{inv}}$.
\item Given a filtered diagram $D$ of $\mathrm{E}G$-algebras, the $\mathrm{E}G$-action of their colimit $\mathrm{colim}(D_n)$ is defined in the following way: use filteredness to find an algebra which contains (representatives of the classes of) all the things you want to act on, then apply the action of that algebra. In the case of tensor products this means that $[x]\otimes[y] = [x \otimes y]$, and thus an invertible object in $\mathrm{colim}(D_n)$ is just (the class of) an invertible object in one of the algebras of $D$. In other words, $\mathrm{colim}(D_n)_{\mathrm{inv}} = \mathrm{colim}(D_{\mathrm{inv}})$.
\end{itemize}
Preservation of products and pullbacks gives preservation of limits, and preservation of limits and filtered colimits gives the result.
\end{proof}

With this new 2-functor $L: \mathrm{E}G\mathrm{Alg}_S \to \mathrm{E}G\mathrm{Alg}_S$, we now have the ability to `freely add inverses to objects' in any $\mathrm{E}G$-algebra we want. The algebra $L\mathbb{G}_n$ is then a clear candidate for our free algebra on $n$ invertible objects, and indeed the proof of this is very simple.

\begin{thm} There exists a free $\mathrm{E}G$-algebra on $n$ invertible objects. Specifically, the algebra $L\mathbb{G}_n$ is such that for any other $\mathrm{E}G$-algebra $X$, we have an isomorphism of categories
\begin{eq*} \mathrm{E}G\mathrm{Alg}_S(L\mathbb{G}_n, X) \cong (X_{\mathrm{inv}})^n \end{eq*}
\end{thm}
\begin{proof}
Using the adjunction from the previous Proposition along with the one from \cref{freealg}, we see that
\begin{eq*}\begin{array}{rll}
		 U(X_{\mathrm{inv}})^n & = & \mathrm{Cat}(\{z_1, ..., z_n\}, U(X_{\mathrm{inv}}) ) \\
		& \cong & \mathrm{E}G\mathrm{Alg}_S( F(\{z_1, ..., z_n\}), X_{\mathrm{inv}}) \\
		& \cong & \mathrm{E}G\mathrm{Alg}_S( LF(\{z_1, ..., z_n\}), X)
\end{array}
 \end{eq*}
As before, $X_{\mathrm{inv}}$ and $U(X_{\mathrm{inv}})$ are obviously isomorphic as categories, and so \( LF(\{z_1, ..., z_n\}) = L\mathbb{G}_n \) satisfies the requirements for the free algebra on $n$ invertible objects.
\end{proof}

\subsection{$L\mathbb{G}_n$ as an initial algebra}

We have now proven that a free $\mathrm{E}G$-algebra on $n$ invertible objects does indeed exist. But this fact on its own is not very helpful. To be able to actually use the free algebra $L\mathbb{G}_n$, we need to know how to contruct it explicitly, in terms of its objects and morphisms. We could do this by finding a detailed characterisation of the 2-functor $L$, and then applying this to our explicit description of $\mathbb{G}_n$ from \cref{Gndef}. However, this would probably be much more effort than is required, since it would involve determining the behaviour of $L$ in many situtations we aren't interested in, and we also wouldn't be leveraging $\mathbb{G}_n$'s status as a free algebra to make the calculations any easier. We will try a different strategy instead. We begin by noticing a special property of the functor $L$.

\begin{prop} \label{linveql} For any $\mathrm{E}G$-algebra $X$, we have $L(X)_{\mathrm{inv}} = L(X)$.
\end{prop}
\begin{proof}
From the definition of adjunctions, the isomorphisms
\begin{eq*}\mathrm{E}G\mathrm{Alg}_S(LX , Y) \cong \mathrm{E}G\mathrm{Alg}_S(X, Y_{\mathrm{inv}}) \end{eq*}
are subject to certain naturality conditions. Specifically, given $F: X' \to X$ and $G: Y \to Y'$ we get a commutative diagram
\begin{eq*} \begin{tikzcd}
\mathrm{E}G\mathrm{Alg}_S(LX , Y) \ar[dd, "G \circ \_ \circ LF"'] \ar[r, "\sim"] & \mathrm{E}G\mathrm{Alg}_S(X, Y_{\mathrm{inv}}) \ar[dd, "G_{\mathrm{inv}} \circ \_ \circ F"] \\
& \\
\mathrm{E}G\mathrm{Alg}_S(LX' , Y') \ar[r, "\sim"] & \mathrm{E}G\mathrm{Alg}_S(X', Y'_{\mathrm{inv}})
\end{tikzcd} \end{eq*}
Consider the case where $F$ is the identity map $\mathrm{id}_X : X \to X$ and $G$ is the inclusion $j: L(X)_{\mathrm{inv}} \to L(X)$. Note that because $j$ is an inclusion, the restriction $j_{\mathrm{inv}}: (L(X)_{\mathrm{inv}})_{\mathrm{inv}} \to L(X)_{\mathrm{inv}}$ is also an inclusion, but since $((\_)_{\mathrm{inv}})_{\mathrm{inv}} = (\_)_{\mathrm{inv}}$, we have that $j_{\mathrm{inv}} = \mathrm{id}$. It follows that
\begin{eq*} \begin{tikzcd}
\mathrm{E}G\mathrm{Alg}_S(LX , LX_{\mathrm{inv}}) \ar[dd, "j \circ \_"'] \ar[r, "\sim"] & \mathrm{E}G\mathrm{Alg}_S(X, LX_{\mathrm{inv}}) \ar[dd, equal] \\
& \\
\mathrm{E}G\mathrm{Alg}_S(LX , LX) \ar[r, "\sim"] & \mathrm{E}G\mathrm{Alg}_S(X, LX_{\mathrm{inv}})
\end{tikzcd} \end{eq*}
Therefore, for any map $f: LX \to LX$ there exists a unique $g: LX \to LX_{\mathrm{inv}}$ such that $j \circ g =f$. But this means that for any such $f$, we must have $\mathrm{im}(f) \subseteq L(X)_{\mathrm{inv}}$, and so in particular $L(X) = \mathrm{im}(\mathrm{id}_{LX}) \subseteq L(X)_{\mathrm{inv}}$. Since $L(X)_{\mathrm{inv}} \subseteq L(X)$ by definition, we obtain the result.
\end{proof}

This result is not especially surprising. Intuitively, it just says that when you freely add inverses to an algebra, every object ends up with an inverse. But the upshot of this is that we now have another way of thinking about $L(X)$: as the target object of the unit of our adjunction, $\eta_X: X \to L(X)_{\mathrm{inv}}$. This means that we don't really need to know the entirety of $L$ in order to determine the free algebra $L\mathbb{G}_n$, just its unit. To find this unit directly, we can turn to the following fact about adjunctions, for which a proof can be found in Lemma 2.3.5 of Leinster's \textit{Basic Category Theory} \cite{bct}.

\begin{prop}\label{initial} Let $F \dashv G: A \to B$ be an adjunction with unit $\eta$. For any object $a$ in $A$, let $(a \downarrow G)$ denote the comma category whose objects are pairs $(b, f)$ consisting of an object $b$ from $B$ and a morphism $f: a \to G(b)$ from $A$, and whose morphisms $h: (b, f) \to (b', f')$ are morphisms $f: b \to b'$ from $B$ such that $G(f) \circ f = f'$. Then the pair $\big(F(a), \eta_a: a \to GF(a) \big)$ is an initial object of $(a \downarrow G)$.
\end{prop}

\begin{cor} If $\phi: \mathbb{G}_n \to Z$ is an initial object of $(\mathbb{G}_n \downarrow \mathrm{inv})$, then 
\begin{eq*} Z \, \cong \, (L\mathbb{G}_n)_{\mathrm{inv}} \, = \, L\mathbb{G}_n \end{eq*}
\end{cor}

Being able to view $L\mathbb{G}_n$ as the initial object in the comma category $(\mathbb{G}_n \downarrow \mathrm{inv})$ will prove immensely useful in the coming sections. This is because it lets us think about the properties of $L\mathbb{G}_n$ in terms of maps $\psi: \mathbb{G}_n \to X_{\mathrm{inv}}$, and this is exactly the context where we can exploit $\mathbb{G}_n$'s status as a free algebra. 

Before moving on, we'll make a small change in notation. From now on, rather than writing objects in $(\mathbb{G}_n \downarrow \mathrm{inv})$ as maps $\psi: \mathbb{G}_n \to Y_{\mathrm{inv}}$, we will instead just let $X = Y_{\mathrm{inv}}$ and speak of maps $\psi: \mathbb{G}_n \to X$. This is purely to prevent the notation from becoming cluttered, and shouldn't be a problem so long as we always remember that the targets of these maps only ever contain invertible objects and morphisms.

\subsection{The objects of $L\mathbb{G}_n$}

So now we know that $L\mathbb{G}_n$ is an initial object in the category $(\mathbb{G}_n \downarrow \mathrm{inv})$. But what does this actually tell us? After all, we do not currently have a method for finding initial objects in an arbitrary collection of $\mathrm{E}G$-algebra maps. Because of this, we'll have to approach the problem step-by-step, using the initiality of $\eta$ to extract different pieces of information about the algebra $L\mathbb{G}_n$ as we go. We'll begin by tring to find its objects.

\begin{defn} Denote by $\mathrm{Ob}: \mathrm{E}G\mathrm{Alg}_S \to \mathrm{Mon}$ be the functor that sends $\mathrm{E}G$-algebras $X$ to their monoid of objects $\mathrm{Ob}(X)$, and algebra maps $F: X \to Y$ to their underlying monoid homomorphism $\mathrm{Ob}(F): \mathrm{Ob}(X) \to \mathrm{Ob}(Y)$. \end{defn}

In order to find $\mathrm{Ob}(L\mathbb{G}_n)$, we'll need to make use of an important result about the nature of $\mathrm{Ob}$.

\begin{defn} Recall that given a monoid $M$, the monoidal category $\mathrm{E}M$ is the one whose monoid of objects is $M$ and which has a unique isomorphism between any two objects. We can view $\mathrm{E}M$ as not just a category but an $\mathrm{E}G$-algebra, by letting the action on morphisms take the only possible values it can, given the required source and target. Similarly, for any monoid homomorphisms $h: M \to M'$ we can define a map of $\mathrm{E}G$-algebras
\begin{eq*} \begin{array}{rlrll}
		\mathrm{E}h & : & \mathrm{E}M & \to & \mathrm{E}M' \\
		& : & m & \mapsto & h(m) \\
		& : & m \to m' & \mapsto & h(m) \to h(m')
		\end{array}
\end{eq*}
This definition of $\mathrm{E}h$ respects composition and identities, and so together with $\mathrm{E}M$ it describes a functor $\mathrm{E}: \mathrm{Mon} \to \mathrm{E}G\mathrm{Alg}_S$.
 \end{defn}

\begin{prop}\label{Obadj} $E$ is a right adjoint to the functor $\mathrm{Ob}$. 
\end{prop}
\begin{proof}
For any $\mathrm{E}G$-algebra $X$, a map $F: X \to \mathrm{E}M$ is determined entirely by its restriction to objects, the monoid homomorphism $\mathrm{Ob}(F) : \mathrm{Ob}(X) \to M$. This is because functorality of $F$ ensures that any map $x \to x'$ in $X$ must be sent to a map $F(x) \to F(x')$ in $\mathrm{E}M$, and by the definition of $\mathrm{E}$ there is always exactly one of these to choose from. In other words, we have an isomorphism between the homsets
\begin{eq*} \mathrm{E}G\mathrm{Alg}_S( \, X, \, \mathrm{E}M \, ) \quad \cong \quad \mathrm{Mon}( \, \mathrm{Ob}(X), \, M \, ) \end{eq*}
Additionally, this isomorphism is natural in both coordinates. That is, for any $G: X \to X'$ in $\mathrm{E}G\mathrm{Alg}_S$ and $h : M \to M'$ in $\mathrm{Mon}$, the diagram
\begin{eq*} \begin{tikzcd}
\mathrm{E}G\mathrm{Alg}_S(X, \mathrm{E}M) \ar[dd, "\mathrm{E}h \circ \_ \circ G"'] \ar[r, "\sim"] & \mathrm{Mon}(\mathrm{Ob}(X), M) \ar[dd, "h \circ \_ \circ \mathrm{Ob}(G)"] \\
& \\
\mathrm{E}G\mathrm{Alg}_S(X', \mathrm{E}M') \ar[r, "\sim"] & \mathrm{Mon}(\mathrm{Ob}(X'), M')
\end{tikzcd} \end{eq*}
commutes, because
\begin{eq*} \mathrm{Ob}( \, \mathrm{E}h \circ F \circ G \, ) \quad = \quad \mathrm{Ob}(Eh) \circ \mathrm{Ob}(F) \circ \mathrm{Ob}(G) \quad = \quad h \circ \mathrm{Ob}(F) \circ \mathrm{Ob}(G) \end{eq*}
Therefore, $\mathrm{Ob} \dashv \mathrm{E}$.
\end{proof}

What \cref{Obadj} is essentially saying is that the functor $\mathrm{Ob}$ provides a way for us to move back and forth between the categories $\mathrm{E}G\mathrm{Alg}_S$ and $\mathrm{Mon}$. By applying this reasoning to the universal property of the initial object $\eta$, we can then determine the value of $\mathrm{Ob}(L\mathbb{G}_n)$ in terms of a new universal property of $\mathrm{Ob}(\eta)$ in the category $\mathrm{Mon}$. In particular, the algebras in $(\mathbb{G}_n \downarrow \mathrm{inv})$ are those whose objects are all invertible, and so the induced property of $\mathrm{Ob}(\eta)$ will end up saying something about the relationship between $\mathrm{Ob}(\mathbb{G}_n)$ and groups --- those monoids whose elements are all invertible.

\begin{defn} Let $M$ be a monoid, $M^{\mathrm{gp}}$ a group, and $i: M \to M^{\mathrm{gp}}$ a monoid homomorphism between them. Then we say that $M^{\mathrm{gp}}$ is the \emph{group completion} of $M$ if for any other group $H$ and homomorphism $h: M \to H$, there exists a unique homomorphism $u: M^{\mathrm{gp}} \to H$ such that $u \circ i = h$.
\end{defn}

In practice, $M^{\mathrm{gp}}$ is the group whose group presentation is the same as the monoid presentation of $M$. That is, if $M$ is the quotient of the free monoid on generators $\mathcal{G}$ by the relations $\mathcal{R}$, then $M^{\mathrm{gp}}$ is the quotient of the free \emph{group} on generators $\mathcal{G}$ by relations $\mathcal{R}$.

\begin{prop}\label{Zobj} The object monoid of $L\mathbb{G}_n$ is $\mathbb{Z}^{*n}$, and the restriction of $\eta$ to objects $\mathrm{Ob}(\eta)$ is the obvious inclusion $\mathbb{N}^{*n} \hookrightarrow \mathbb{Z}^{*n}$.
\end{prop}
\begin{proof}
Let $H$ be a group, and $h: \mathrm{Ob}(\mathbb{G}_n) \to H$ a monoid homomorphism. By \cref{Obadj} we have an isomorphism of homsets
\begin{eq*} \mathrm{E}G\mathrm{Alg}_S( \, \mathbb{G}_n, \, \mathrm{E}H \, ) \quad \cong \quad \mathrm{Mon}( \, \mathrm{Ob}(\mathbb{G}_n), \, H \, ) \end{eq*}
Denote by $h': \mathbb{G}_n \to \mathrm{E}H$ the map of $\mathrm{E}G$-algebras corresponding to $h$ under this isomorphism. Since $H$ is a group, every object in $\mathrm{E}H$ is invertible, and so $h'$ is an object of $(\mathbb{G}_n \downarrow \mathrm{inv})$. Thus, by initiality of $\eta$, there must exist a unique map $u: L\mathbb{G}_n \to \mathrm{E}G$ making the lefthand traingle below commute:
\begin{eq*} \begin{tikzcd}
\mathbb{G}_n \ar[dd, "\eta"'] \ar[ddrr, "h'"] & & & & \mathrm{Ob}(\mathbb{G}_n) \ar[dd, "\mathrm{Ob}(\eta)"'] \ar[ddrr, "h"] & & \\
& & & & & & \\
L\mathbb{G}_n \ar[rr, "u"'] & & \mathrm{E}H & & \mathrm{Ob}(L\mathbb{G}_n) \ar[rr, "\mathrm{Ob}(u)"'] & & H
\end{tikzcd} \end{eq*}
It follows that the righthand triangle --- which is the image of the first under $\mathrm{Ob}$ --- also commutes. Hence for any group $H$ and homomorphism $h: \mathrm{Ob}(\mathbb{G}_n) \to H$, there is at least one map which factors $h$ through $\mathrm{Ob}(\eta)$.

But now let $v: \mathrm{Ob}(L\mathbb{G}_n) \to H$ be any homomorphism such that $v \circ \mathrm{Ob}(\eta) = h$. If $v': L\mathbb{G}_n \to \mathrm{E}H$ is the image of $v$ under the adjunction isomorphism, then by naturality $v' \circ \eta = h'$, a property that was supposed to be unique to $u$. Thus $v = \mathrm{Ob}(u)$, and so there is actually only one possible map which factors $h$ through $\mathrm{Ob}(\eta)$. 

Therefore every homomorphism from $\mathrm{Ob}(\mathbb{G}_n)$ onto a group factors uniquely through the $\mathrm{Ob}(L\mathbb{G}_n)$, or in other words $\mathrm{Ob}(L\mathbb{G}_n)$ is the group completion of the monoid $\mathrm{Ob}(\mathbb{G}_n)$. Since by \cref{Gnobj} the object monoid of $\mathbb{G}_n$ is $\mathbb{N}^{\ast n}$, the free monoid on $n$ generators, we can conclude that
\begin{eq*} \mathrm{Ob}(L\mathbb{G}_n) \, = \, \mathrm{Ob}(\mathbb{G}_n)^{\mathrm{gp}} \, = \, (\mathbb{N}^{\ast n})^{\mathrm{gp}} \, = \, \mathbb{Z}^{\ast n} \end{eq*}
the free group on $n$ generators. Moreover, the map $\mathrm{Ob}(\eta)$ is then the inclusion of $\mathrm{Ob}(\mathbb{G}_n)$ into its completion, which is just $\mathbb{N}^{*n} \hookrightarrow \mathbb{Z}^{*n}$.
\end{proof}

This result makes concrete the sense in which the functor $L$ represents `freely adding inverses' to objects. Extending this same logic to connected components as well, it seems reasonable to expect that $\pi_0(L\mathbb{G}_n)$ should be $\mathbb{Z}^n$, the group completion of $\pi_0(\mathbb{G}_n) = \mathbb{N}^n$. This is indeed the case, and the proof proceeds in a way completely analagous to \cref{Zobj}. First, we show that the process of taking connected components is part of an adjunction.

\begin{defn} Denote by $\pi_0: \mathrm{E}G\mathrm{Alg}_S \to \mathrm{CMon}$ be the functor that sends each algebra $X$ to its commutaive monoid of connected components, $\pi_0(X)$, and sends each map of algebras $F: X \to Y$ to its restriction to connected components $\pi_0(F): \pi_0(X) \to \pi_0(Y)$.\end{defn}

\begin{defn} There exists an inclusion of 2-categories $\mathrm{Set} \hookrightarrow \mathrm{Cat}$ which allows us to view any set $S$ as a discrete category, one whose objects are just the elements of $S$ and whose morphisms are all identities. If the given set also happens to be a monoid then there is an obvious way to see this discete category as a monoidal category, and so we have a similar inclusion $\mathrm{Mon} \hookrightarrow \mathrm{MonCat}$. Finally, if a given monoid happens to be commutative then there is a unique way to assign an $\mathrm{E}G$-action to its discete category. This works because for any elements $c_i$ of a commutive monoid $C$, the morphism $\alpha(g; \mathrm{id}_{c_1}, ..., \mathrm{id}_{c_m})$ must have source and target $c_1 \otimes ... \otimes c_m = c_{\pi(g^{-1})(1)} \otimes ... \otimes c_{\pi(g^{-1})(m)}$, and therefore it can only be $\mathrm{id}_{c_1 \otimes ... \otimes c_m}$. This choice of action yields one last inclusion $\mathrm{CMon} \hookrightarrow \mathrm{E}G\mathrm{Alg}_S$, which we shall call $\mathrm{D}$. \end{defn}

\begin{prop}\label{concompadj} $D$ is a right adjoint to the functor $\pi_0$. 
\end{prop}
\begin{proof}
Consider a map $F: X \to \mathrm{D}C$ from some $\mathrm{E}G$-algebra $X$ to the discrete $\mathrm{E}G$-algebra on a commutative monoid $C$. For any $f: x \to x'$ in $X$, the morphism $F(f)$ must be an identity map in $\mathrm{D}C$, since these are the only morphisms that $\mathrm{D}C$ has. It follows that $x$ and $x'$ being in the same connected component implies $F(x) = F(x')$, and so $F$ is determined entirely by its restriction to connected components, the monoid homomorphism $\pi_0(F) : \pi_0(X) \to C$. In other words, we have an isomorphism between the homsets
\begin{eq*} \mathrm{E}G\mathrm{Alg}_S( \, X, \mathrm{D}C \, ) \quad \cong \quad \mathrm{CMon}( \, \pi_0(X), C \, ) \end{eq*}
This isomorphism is natural in both coordinates, since for any $G: X \to X'$ in $\mathrm{E}G\mathrm{Alg}_S$ and $h : C \to C'$ in $\mathrm{CMon}$, 
\begin{eq*} \pi_0( \, \mathrm{D}h \circ F \circ G \, ) \quad = \quad \pi_0(\mathrm{D}h) \circ \pi_0(F) \circ \pi_0(G) \quad = \quad h \circ \pi_0(F) \circ \pi_0(G) \end{eq*}
and so the diagram
\begin{eq*} \begin{tikzcd}
\mathrm{E}G\mathrm{Alg}_S(X, \mathrm{D}C) \ar[dd, "\mathrm{D}h \circ \_ \circ G"'] \ar[r, "\sim"] & \mathrm{CMon}(\pi_0(X), C) \ar[dd, "h \circ \_ \circ \pi_0(G)"] \\
& \\
\mathrm{E}G\mathrm{Alg}_S(X', \mathrm{D}C') \ar[r, "\sim"] & \mathrm{CMon}(\pi_0(X'), C') 
\end{tikzcd} \end{eq*}
commutes. Therefore, $\pi_0 \dashv \mathrm{D}$.
\end{proof}

Now we can utilise \cref{concompadj} to draw out a universal property of $\pi_0(L\mathbb{G}_n)$, just as we did with \cref{Obadj} and $\mathrm{Ob}(L\mathbb{G}_n)$. This time, since we are dealing with commutative monoids, the requirement that everything be invertible will lead us to consider abelian groups.

\begin{prop}\label{Zconcomp} The connected components of $L\mathbb{G}_n$ are $\mathbb{Z}^n$, with the restriction of $\eta$ to components $\pi_0(\eta)$ being the obvious inclusion $\mathbb{N}^n \hookrightarrow \mathbb{Z}^n$, and the assignment of objects to their component $[ \, \_ \, ]: \mathrm{Ob}(L\mathbb{G}_n) \to \pi_0(L\mathbb{G}_n)$ being the quotient map of abelianisation $\mathbb{Z}^{\ast n} \to \mathbb{Z}^n$.
\end{prop}
\begin{proof}
Let $A$ be an abelian group and $h: \pi_0(\mathbb{G}_n) \to A$ a monoid homomorphism. By \cref{concompadj} there is a homset isomorphism
\begin{eq*} \mathrm{E}G\mathrm{Alg}_S( \, \mathbb{G}_n, \, \mathrm{D}A \, ) \quad \cong \quad \mathrm{Mon}( \, \pi_0(\mathbb{G}_n), \, A \, ) \end{eq*}
and thus some $\mathrm{E}G$-algebra map $h': \mathbb{G}_n \to \mathrm{D}A$ corresponding to $h$. As every object of $\mathrm{D}A$ is invertible, $h'$ is in $(\mathbb{G}_n \downarrow \mathrm{inv})$, and hence there exists a unique map $u: L\mathbb{G}_n \to \mathrm{D}A$ factoring $h'$ through the initial object $\eta$:
\begin{eq*} \begin{tikzcd}
\mathbb{G}_n \ar[dd, "\eta"'] \ar[ddrr, "h'"] & & & & \pi_0(\mathbb{G}_n) \ar[dd, "\pi_0(\eta)"'] \ar[ddrr, "h"] & & \\
& & & & & & \\
L\mathbb{G}_n \ar[rr, "u"'] & & \mathrm{D}A & & \pi_0(L\mathbb{G}_n) \ar[rr, "\pi_0(u)"'] & & A
\end{tikzcd} \end{eq*}
Applying the functor $\pi_0$ everywhere, we see that $\pi_0(u)$ must also factor $h$ through the homomorphism $\pi_0(\eta)$. Moreover, $\pi_0(u)$ is the only map with this property, since for any other map $v: \pi_0(L\mathbb{G}_n) \to A$ with $v \circ \pi_0(\eta) = h$, its image under the adjunction isomorphism $v': L\mathbb{G}_n \to \mathrm{D}A$ would have $v' \circ \eta = h'$ by naturality, which would mean that it was actually $u$. Therefore, any homomorphism $\mathrm{Ob}(\mathbb{G}_n) \to A$ will factor uniquely through $\mathrm{Ob}(L\mathbb{G}_n)$, so long as $A$ is an abelian group. 

Now assume that $h: \mathrm{Ob}(\mathbb{G}_n) \to H$ is more general sort of homomorphism, where $H$ is a group but not neccessarily an abelian one. Since $\pi_0(L\mathbb{G}_n)$ is a commutative monoid, it follows that its image $\langle \mathrm{im}(h) \rangle$ will be commutative too:
\begin{eq*} h(x)h(y) \, = \, h(xy) \, = \, h(yx) \, = \, h(y)h(x) \end{eq*}
But then $\langle \mathrm{im}(h) \rangle$ is a commutative submonoid of the group $H$, and so it is really an abelian group. If we denote by $h_{\mathrm{im}}: \mathrm{Ob}(\mathbb{G}_n) \to \langle \mathrm{im}(h) \rangle$ the restriction of $h$ to it image, then from what we have already shown we can conclude that there will be a unique homomorphism $v: \mathrm{Ob}(L\mathbb{G}_n) \to \langle \mathrm{im}(h) \rangle$ with the property $v \circ \pi_0(\eta) = h_{\mathrm{im}}$. Composing this $v$ with the inclusion $i: \langle \mathrm{im}(h) \rangle \hookrightarrow A$, we see that
\begin{eq*} i \circ v \circ \pi_0(\eta) \, = \, i \circ h_{\mathrm{im}} \, = \, h \end{eq*}
and $i \circ v$ must be the only map for which this is true, for restricting this equation back on $\langle \mathrm{im}(h) \rangle$ yields the unique property of $v$ again. Thus $\pi_0(L\mathbb{G}_n)$ will actually factor any homomorphism $\mathrm{Ob}(\mathbb{G}_n) \to H$ in a unique way, and hence by \cref{Gnconcomp} it is the group completion
\begin{eq*} \pi_0(L\mathbb{G}_n) \, = \, \pi_0(\mathbb{G}_n) ^{\mathrm{gp}} \, = \, (\mathbb{N}^n)^{\mathrm{gp}} \, = \, \mathbb{Z}^n \end{eq*}
The map $\mathrm{Ob}(\eta)$ is then the inclusion of $\mathrm{Ob}(\mathbb{G}_n)$ into its completion, $\mathbb{N}^{*n} \hookrightarrow \mathbb{Z}^{*n}$.

Something else we previously saw in \cref{Gnconcomp} was that that the map $[ \, \_ \, ] : \mathrm{Ob}(\mathbb{G}_n) \to \pi_0(\mathbb{G}_n)$ sending objects of $\mathbb{G}_n$ to their connected component is the quotient map of abelianisation, $\mathbb{N}^{\ast n} \to (\mathbb{N}^{\ast n})^{\mathrm{ab}} = \mathbb{N}^n$. If we also use $[ \, \_ \, ]$ to denote the map sending objects of $L\mathbb{G}_n$ to their components, then functoriality of $\eta$ tells us that 
\begin{eq*} \begin{tikzcd}
\mathbb{N}^{\ast n} \ar[dd, "\lbrack \, \_ \, \rbrack"'] \ar[rr, "\mathrm{Ob}(\eta)"] & & \mathbb{Z}^{\ast n} \ar[dd, "\lbrack \, \_ \, \rbrack"] \\
& & \\
\mathbb{N}^n \ar[rr, "\pi_0(\eta)"] & & \mathbb{Z}^n
\end{tikzcd} \end{eq*}
commutes. Using the values of $[ \, \_ \, ]$ from \cref{Gnconcomp}, $\mathrm{Ob}(\eta)$ from \cref{Zobj}, and $\pi_0(\eta)$ from the earlier parts of this proposition, it follows that for any generator $z_i$ of $\mathbb{Z}^{\ast n}$, 
\begin{eq*} [z_i] \, = \, [\mathrm{Ob}(\eta)(z_i)] \, = \, \pi_0(\eta)([z_i]) \, = \, \pi_0(\eta)(z_i) \, = \, z_i \end{eq*}
But this description of $[ \, \_ \, ]: \mathrm{Ob}(L\mathbb{G}_n) \to \pi_0(L\mathbb{G}_n)$ on generators is then just the definition of the quotient map of abelianisation $\mathbb{Z}^{\ast n} \to (\mathbb{Z}^{\ast n})^{\mathrm{ab}}$, as required
\end{proof}
 
\subsection{The morphisms of $L\mathbb{G}_n$}

Now that we understand the objects of the algebra $L\mathbb{G}_n$, the next most obvious thing to look for are its morphisms, $\mathrm{Mor}(L\mathbb{G}_n)$. It would be nice to construct this collection in the same way we constructed $\mathrm{Ob}(L\mathbb{G}_n)$ and $\pi_0(L\mathbb{G}_n)$, by applying the left adjoint of some adjunction to the initial map $\eta$. Before we can do this however, we need to ask ourselves a question. What sort of mathematical object is $\mathrm{Mor}(L\mathbb{G}_n)$, exactly?

Given a pair of morphisms $f: x \to y, f': x' \to y'$ in an $\mathrm{E}G$-algebra $X$, there are two basic binary operations we can perform. First, we can take their tensor product $f \otimes f'$, and this together with the unit map $\mathrm{id}_{I}$ imbues $\mathrm{Mor}(X)$ with the structure of a monoid. Second, if we have $y = x'$ then we can form the composite morphism $f' \circ f$. However, consider the following fact: if $y$ is an invertible object, then
\begin{eq*}\begin{array}{rll}
			f' \circ f & = & (f' \otimes \mathrm{id}_I) \circ (\mathrm{id}_I \otimes f) \\
			& = & (f' \otimes \mathrm{id}_{y*} \otimes \mathrm{id}_y) \circ (\mathrm{id}_y \otimes \mathrm{id}_{y*} \otimes f) \\
			& = & (f' \circ \mathrm{id}_y) \otimes (\mathrm{id}_{y*} \circ \mathrm{id}_{y*}) \otimes (\mathrm{id}_y \circ f) \\
			& = & f' \otimes \mathrm{id}_{y*} \otimes f 
		\end{array}
\end{eq*}
In other words, composition along invertible objects in $X$ is determined completely by the monoidal structure of $X$. In the case of $L\mathbb{G}_n$, where every object is invertible, this means that if we understand $\mathrm{Mor}(L\mathbb{G}_n)$ as a monoid then we will be able to recover the operation $\circ$ in its entirety. For that reason, we will choose to ignore composition of elements of $\mathrm{Mor}(X)$ for the time being, and focus on its status as a monoid.

Now we try to proceed as we did before, by showing that $\mathrm{Mor}(X)$ is part of an adjunction.

\begin{defn} Let $\mathrm{Mor} : \mathrm{MonCat} \to \mathrm{Mon}$ be the functor which sends algebras $X$ to their monoid of morphisms $\mathrm{Mor}(X)$, and sends algebra maps $F: X \to Y$ to the monoid homomorphism
\begin{eq*} \begin{array}{rlrll}
			\mathrm{Mor}(F) & : & \mathrm{Mor}(X) & \to & \mathrm{Mor}(Y) \\
			& : & f: x \to x' & \mapsto & F(f) : F(x) \to F(x') \\
		\end{array}
\end{eq*}
\end{defn}

\begin{defn} For a given abelian group $A$, let $C(A)$ represent the monoidal category defines as follows:
\begin{itemize}
\item The objects of $C(A)$ are the monoid $A$, with the monoid multiplication as the tensor product and the identity element $e$ as the monoidal unit.
\item For any two objects $a, a' \in A$, the homset $C(A)(a, a')$ is isomorphic to the underlying set of $A$.
\item From the above, the morphisms of $C(A)$ will clearly be
\begin{eq*} \mathrm{Mor}\big( \, C(A) \, \big) \, = \, A \times A^2 \, = \, A^3 \end{eq*}
when viewed as a set, but this equality also holds as monoids, so that the tensor product is defined componentwise using the monoid multiplication of $A$.
\item For any two composable morphisms $(a, b, b')$, $(a', b', b'')$ of $C(A)$, their composite is the morphism
\begin{eq*} (a', b', b'') \circ (a, b, b') \, = \, \big( \, a(b')^*a', \, b, \, b'' \, \big) \end{eq*}
\end{itemize}
Likewise, for any group homomorphism $h: A \to A'$ between abelian groups, denote by $C(h) : C(A) \to C(A')$ the obvious monoidal functor which acts like $h$ on objects and $h^3$ on morphisms. This defines a functor $C: \mathrm{Ab} \to \mathrm{MonCat}$ from the category of abelian groups onto the category of monoidal categories.
\end{defn}

Intuitively, $C(A)$ is the the monoidal category that we can build out of $A$ by using the trick we discussed before for extracting composition from the tensor product, $f' \circ f = f' \otimes \mathrm{id}_{y*} \otimes f$. This is why we had to choose $A$ to be a group, as this can only work when all of the objects of $C(A)$ are invertible. Notice also that commutativity is required in order for $C(A)$ to be a well-defined monoidal category, since we need its operations to obey an interchange law, and thus
\begin{eq*} \begin{array}{rll}
			(aa', e, e) & = & (a, e, e) \otimes (a', e, e) \\
			& = & \big( \, \mathrm{id}_e \circ (a, e, e) \, \big) \otimes \big( \, (a', e, e) \circ \mathrm{id}_e \, \big) \\
			& = & \big( \, (a', e, e) \otimes \mathrm{id}_e \, \big) \circ \big( \, \mathrm{id}_e \otimes (a, e, e) \, \big) \\
			& = & (a', e, e) \circ (a, e, e) \\
			& = & (a'a, e, e) 
		\end{array}
\end{eq*}
This is the classic Eckmann-Hilton argument.

\begin{prop}\label{Moradj} The functor $C$ is a right adjoint to the functor $\mathrm{Mor}( \, \_ \, )^{\mathrm{gp}, \mathrm{ab}} : \mathrm{MonCat} \to \mathrm{Ab}$.
\end{prop} 
\begin{proof}
Let $X$ be a monoidal category, $A$ an abelian group, and $F: X \to C(A)$ a monoidal functor. For any morphism $f: x \to y$ in $X$, by functoriality $F$ will send it onto some morphism $F(f)$ in the homset $C(A)\big( \, F(x), F(y) \, \big)$. However, every homset of $C(A)$ is isomorphic to a copy of $A$, and so clearly there is a way to extract from $F$ a map $\mathrm{Mor}(X) \to A$. Specifically, if we define the map $\epsilon_A$ to be the projection
\begin{eq*} \begin{array}{rlrll}
			\epsilon_A & : & \mathrm{Mor}\big( \, C(A) \, \big) & \to & A \\
			& : & A^3 & \to & A \\
			& : & (a, b, b') & \mapsto & a
		\end{array}
\end{eq*}
then we can use the functor $\mathrm{Mor}$ and the map $\epsilon_A$ to form the following composite map:
\begin{eq*} \begin{tikzcd}
\mathrm{Mor}(X) \ar[rr, "\mathrm{Mor}(F)"] & & \mathrm{Mor}\big( \, C(A) \, \big) \ar[rr, "\epsilon_A"] & & A
\end{tikzcd} \end{eq*}
Then, since $A$ is not just a monoid but an abelian group, we can factor the homomorphism $\mathrm{Mor}(F)$ through the group completion of $\mathrm{Mor}(X)$, and then through the abelianisation of that group, at last yielding a map 
\begin{eq*} F' \, := \, \epsilon_A \circ \big( \, \mathrm{Mor}(F)^{\mathrm{gp}} \, \big)^{\mathrm{ab}} : \big( \, \mathrm{Mor}(X)^{\mathrm{gp}} \, \big)^{\mathrm{ab}} \to A \end{eq*}
This $\epsilon$ will be the counit of our adjunction, with the assignment $F \mapsto F'$ being one direction of the eventual homset adjunction.

Conversely, let $\eta_X$ be the monoidal functor defined by
\begin{eq*} \begin{array}{rlrll}
			\eta_X & : & X & \to & C\big( \, \mathrm{Mor}(X)^{\mathrm{gp, ab}} \, \big) \\
			& : & x & \mapsto & [\mathrm{id}_x] \\
			& : & f: x \to y & \mapsto & ( \, [f], [\mathrm{id}_x], [\mathrm{id}_y] \, )
		\end{array}
\end{eq*}
where $[ \, \_ \, ]$ is the quotient of abelianisation $\mathrm{Mor}(X) \to \mathrm{Mor}(X)^{\mathrm{gp, ab}}$. Then any homomorphism $h: \mathrm{Mor}(X)^{\mathrm{gp, ab}} \to A$ can be used to construct a monoidal functor $h' : X \to C(A)$ as follows:
\begin{eq*} \begin{tikzcd}
X \ar[rr, "\eta_X"] & & C\big( \, \mathrm{Mor}(X) \, \big) \ar[rr, "C( \, \lbrack \, \_ \, \rbrack \, )"] & & C\big( \, \mathrm{Mor}(X)^{\mathrm{gp, ab}} \, \big) \ar[rr, "C(h)"] & & C(A)
\end{tikzcd} \end{eq*}
Moreover, if we compare this $\eta$ with $\epsilon$ then we see that the composites
\begin{eq*} \begin{tikzcd}
\mathrm{Mor}(X)^{\mathrm{gp, ab}} \ar[d,"\mathrm{Mor}(\eta_X)^{\mathrm{gp, ab}}"] & & C(A) \ar[d, "\eta_{C(A)}"] \\
\mathrm{Mor}\Big( \, C\big( \, \mathrm{Mor}(X)^{\mathrm{gp, ab}} \, \big) \, \Big)^{\mathrm{gp, ab}} \ar[d, equals] & & C\Big( \, \mathrm{Mor}\big( \, C(A) \, \big)^{\mathrm{gp, ab}} \, \Big) \ar[d, equals] \\
\mathrm{Mor}\Big( \, C\big( \, \mathrm{Mor}(X)^{\mathrm{gp, ab}} \, \big) \, \Big) \ar[d, "\epsilon_{\mathrm{Mor}(X)^{\mathrm{gp, ab}}}"] & & C\Big( \, \mathrm{Mor}\big( \, C(A) \, \big)\, \Big) \ar[d, "C(\epsilon_A)"] \\
\mathrm{Mor}(X)^{\mathrm{gp, ab}} & & C(A) \\
\end{tikzcd} \end{eq*}
must be the respective identity maps:
\begin{eq*} \begin{array}{rll} 
			\epsilon_{\mathrm{Mor}(X)^{\mathrm{gp, ab}}} \circ \mathrm{Mor}(\eta_X)^{\mathrm{gp, ab}}\big( \, \lbrack f: x \to y \rbrack \, \big) & = & \epsilon_{\mathrm{Mor}(X)^{\mathrm{gp, ab}}}\big( \, \lbrack f \rbrack, \lbrack \mathrm{id}_x \rbrack, \lbrack \mathrm{id}_y \rbrack \, \big) \\
			& = & \lbrack f \rbrack \\
			& & \\
			C(\epsilon_A) \circ \epsilon_{\mathrm{Mor}(X)^{\mathrm{gp, ab}}}\big( \, a \, \big) & = & C(\epsilon_A)\big( \, a, a, a \, \big) \\
			& = & a \\
			C(\epsilon_A) \circ \epsilon_{\mathrm{Mor}(X)^{\mathrm{gp, ab}}}\big( \, a, b, b' \, \big) & = & C(\epsilon_A)\big( \, (a,b,b'), \, (b,b,b), \, (b',b',b') \, \big) \\
			& = & (a, b, b') \\
		\end{array}
\end{eq*}
In other words, $\eta_X$ and $\epsilon_A$ really are the unit and counit of an adjunction $\mathrm{Mor}(\, \_ \,)^{\mathrm{gp, ab}} \dashv C$, whose isomorphism of homsets
\begin{eq*} \mathrm{MonCat}( \, X, C(A) \, ) \quad \cong \quad \mathrm{Ab}( \, \mathrm{Mor}(X)^{\mathrm{gp, ab}}, A \, ) \end{eq*}
is given by the assigments $F \mapsto F'$ and $h \mapsto h'$.
\end{proof}

\cref{Moradj} is very similar to \cref{Obadj,concompadj}, but it falls short in a few very important ways. First, if we want to use an adjunction to find a relationship between the morphisms of $\mathbb{G}_n$ and $L\mathbb{G}_n$, like what we did in \cref{Zobj,Zconcomp}, then what we need is an adjunction involving $\mathrm{E}G\mathrm{Alg}_{S}$, not $\mathrm{MonCat}$. This is because $\eta$ can only be used to factor algebra maps $\mathbb{G}_n \to X_{\mathrm{inv}}$ through $L\mathbb{G}_n$, and not arbitrary monoidal functors. Likewise, we would rather have the other side of the adjunction be $\mathrm{Mon}$ instead of $\mathrm{Ab}$ so that we could work with the functor $\mathrm{Mor}$ directly, and not its group completed, abelianised version.

Unfortunately this adjunction seems to be the best we can do. We already saw that we need $A$ to be an abelian group for $C(A)$ to have composition and interchange, and given an arbitrary abelian group that we want to be the morphisms of an algebra, there is no general method for assigning it an $\mathrm{E}G$-action. As this is what we are stuck with, we will not be able to use the $\eta$ method to extract the information we need about the morphisms of $L\mathbb{G}_n$, and so we must try a less straightforward approach.

\subsection{Sources and targets in $L\mathbb{G}_n$}  

The goal of these next couple of sections will be to show that we can reconstruct the all of morphisms of $L\mathbb{G}_n$ from just the abelian group $\mathrm{Mor}(L\mathbb{G}_n)^{\mathrm{gp, ab}}$, and therefore that we can actually use the adjunction from \cref{Moadj} to help find a description of $L\mathbb{G}_n$. The way we will do this is by splitting $\mathrm{Mor}(L\mathbb{G}_n)$ up as the product of two other monoids. The first of these will encode all of the possible combinations of source and target data for morphisms in $L\mathbb{G}_n$, while the second will just be the endomorphisms of the unit object, $L\mathbb{G}_n(I, I)$. In other words, we will see that the monoid $\mathrm{Mor}(L\mathbb{G}_n)$ can be broken down into a context where source and target are the only thing that matters, and another where they are irrelevant. Once we have done this, we can then use the fact that $L\mathbb{G}_n(I, I)$ is always an abelian group to rewrite $\mathrm{Mor}(L\mathbb{G}_n)$ in terms of $\mathrm{Mor}(L\mathbb{G}_n)^{\mathrm{gp, ab}}$.

To get things started, we will spend this section considering the source and target information of morphisms in $L\mathbb{G}_n$.

\begin{defn} For any $\mathrm{E}G$-algebra $X$, denote by $s: \mathrm{Mor}(X) \to \mathrm{Ob}(X)$ and $t: \mathrm{Mor}(X) \to \mathrm{Ob}X)$ the monoid homomorphisms which send each morphism of $X$ to its source and target, respectively. That is,
\begin{eq*} s( \, f: x \to y) \, = \, x, \quad \quad t( \, f: x \to y) \, = \, y \end{eq*}
\end{defn}

If we use the universal property of products, we can combine these two homomorphisms into a single map, $s \times t: \mathrm{Mor}(X) \to \mathrm{Ob}(X) \times \mathrm{Ob}(X)$. The monoid we are interested in finding is the image $L\mathbb{G}_n$ under its instance of this map.

\begin{lem}\label{stmon} Let $X$ be an $\mathrm{E}G$-algebra, and $s \times t: \mathrm{Mor}(X) \to \mathrm{Ob}(X)^2$ the map built from $s$ and $t$ using the universal property of products. Then the image of this map is
\begin{eq*} (s \times t)(X) \, = \, \mathrm{Ob}(X) \times_{\pi_0(X)} \mathrm{Ob}(X) \end{eq*}
where this pullback is taken over the canonical maps sending objects of $X$ to their connected components:
\begin{eq*} \begin{tikzcd}
\mathrm{Ob}(X) \times_{\pi_0(X)} \mathrm{Ob}(X) \ar[dd, shift left=12] \ar[rr] \ar[ddrr, phantom, "\lrcorner", near start, shift left=4] & & \mathrm{Ob}(X) \ar[dd, "\lbrack \, \_ \, \rbrack"] & \\
& & & \\
\quad \quad \quad \quad \quad \quad \mathrm{Ob}(X) \ar[rr, "\lbrack \, \_ \, \rbrack"] & & \pi_0(X)
\end{tikzcd} \end{eq*}
\end{lem}
\begin{proof}
By definition, there exists a morphism $f: x \to y$ between objects $x, y$ of $X$ if and only if they are in the same connected component, $[x] = [y]$. Thus
\begin{eq*} \begin{array}{rll}
		(x, y) \, \in \, (s \times t)(X) & \iff & \exists \, f \, : \quad s(f) \, = \, x, \quad t(f) \, = \, y \\
		& \iff & [x] = [y] \\
		& \iff & (x, y) \, \in \, \mathrm{Ob}(X) \times_{\pi_0(X)} \mathrm{Ob}(X)
		\end{array}
\end{eq*}
as required.
\end{proof}

Recalling \cref{Gnobj,Gnconcomp,Zobj,Zconcomp}, we can immediately conclude the following:

\begin{cor} \label{stpullback}
\begin{eq*} \begin{array}{rll} 
		(s \times t)(\mathbb{G}_n) & = & \mathbb{N}^{\ast n} \times_{\mathbb{N}^n} \mathbb{N}^{\ast n} \\
		(s \times t)(L\mathbb{G}_n) & = & \mathbb{Z}^{\ast n} \times_{\mathbb{Z}^n} \mathbb{Z}^{\ast n} 
		\end{array}
\end{eq*}
where these pullbacks are taken over the quotients of abelianisation for $(\mathbb{N}^{\ast n})^{\mathrm{ab}} = \mathbb{N}^n$ and $(\mathbb{Z}^{\ast n})^{\mathrm{ab}} = \mathbb{Z}^n$ respectively.
\end{cor}

Next, we want to show that this $(s \times t)(L\mathbb{G}_n)$ we have described is in fact a submonoid of $\mathrm{Mor}(L\mathbb{G}_n)$. This is a little tricky though, since we don't currently know what the morphisms of $L\mathbb{G}_n$ even are. We will sidestep this problem by first proving the analogous statement for all $\mathbb{G}_n$, and then recovering the $L\mathbb{G}_n$ version from it later.

To that end we need to find a description of the monoid $\mathrm{Mor}(\mathbb{G}_n)$. \cref{Gndef} already tells us everything we need to know about these morphisms, but it will be helpful for us to give a nice compact description.

\begin{defn}\label{lengthdef} Let $S$ be a set and $F(S)$ the free monoid on $S$, the monoid whose elements are strings of elements of $S$ and whose binary operation is concatenation. Then we will denote by
\begin{eq*} | \, \_ \, | : F(S) \to \mathbb{N} \end{eq*}
the monoid homomorphism defined by sending each element of $S \subseteq F(S)$ to 1, and therefore also each concatenation of $n$ elements of $S$ to the natural number $n$. We will call $|x|$ the \emph{length} of $x \in F(S)$.
\end{defn}

\begin{defn} Let $G$ be an action operad. Then we will also the notation $G$ to denote the \emph{underlying monoid} of this action operad. This is the natural way to consider $G$ as a monoid, with its element set being all of its elements together, $\bigsqcup_m G(m)$, and with tensor product as its binary operation, $g \otimes h = \mu(e_2; g, h)$.

Also, note that this monoid comes equipped with a homomorphism $| \, \_ \, | : G \to \mathbb{N}$, sending each $g \in G$ to the natural number $m$ if and only if $g$ is an element of the group $G(m)$. Again, we'll call this number $|g|$ the \emph{length} of $g$.
\end{defn}

\begin{lem} \label{Gnmor} The monoid of morphisms of the algebra $\mathbb{G}_n$ is
\begin{eq*} \mathrm{Mor}(\mathbb{G}_n) \, \cong \, G \times_{\mathbb{N}} \mathbb{N}^{\ast n} \end{eq*}
where this pullback is taken over the respective length homomorphisms,
\begin{eq*} \begin{tikzcd}
G \times_{\mathbb{N}} \mathbb{N}^{\ast n} \ar[dd, shift left=4] \ar[rr] \ar[ddrr, phantom, "\lrcorner", very near start, shift left] & & \mathbb{N}^{\ast n} \ar[dd, "| \, \_ \, |"] & \\
& & & \\
\quad \quad G \ar[rr, "| \, \_ \, |"] & & \mathbb{N} &
\end{tikzcd} \end{eq*}
using the fact that $\mathbb{N}^{\ast n}$ is the free monoid $F\big( \, \{z_1, ..., z_n\} \, \big)$.
\end{lem}
\begin{proof}
An element of $G \times_{\mathbb{N}} F\big( \, \{z_1, ..., z_n\} \, \big)$ is just an element $g \in G(m)$ for some $m$, together with an $m$-tuple of objects $(x_1, ..., x_m)$ from the set of generators $\{z_1, ..., z_n\}$. Thus the action on $\mathbb{G}_n$ defines an obvious function 
\begin{eq*} \begin{array}{rlrll}
			\alpha & : & G \times_{\mathbb{N}} F\big( \, \{z_1, ..., z_n\} \, \big) & \to & \mathrm{Mor}(\mathbb{G}_n) \\
			& : & (g;x_1, ..., x_m) & \mapsto & \alpha(g; \mathrm{id}_{x_1}, ..., \mathrm{id}_{x_m})
		\end{array}
\end{eq*}
But by \cref{Gnmapsaction}, each element of $\mathrm{Mor}(\mathbb{G}_n)$ can be expressed in the form $\alpha(g; \mathrm{id}_{x_1}, ..., \mathrm{id}_{x_m})$ for a unique collection $(g;x_1, ..., x_m)$, and so this function $\alpha$ is actually a bijection of sets. Furthermore, this function preserves tensor product, since
\begin{eq*} \begin{array}{rll}
			\alpha\big( \, (g;f_1, ..., f_m) \otimes (g';f'_1, ..., f'_m) \, \big) & = & \alpha( \, g \otimes g' \, ; \, f_1, ..., f_m, f'_1, ..., f'_m \, ) \\
			& = & \alpha( \, g \, ; \, f_1, ..., f_m \, ) \otimes \alpha( \, g' \, ; \, f'_1, ..., f'_m \, )
		\end{array}
\end{eq*}
and hence it is a monoid isomorphism, as required.
\end{proof}

Now, we want to know if $(s \times t)(\mathbb{G}_n)$ can be seen as a submonoid of $\mathrm{Mor}(\mathbb{G}_n)$, which we now see is the same as asking if we can find an injective homomorphism $\mathbb{N}^{\ast n} \times_{\mathbb{N}^n} \mathbb{N}^{\ast n} \to G \times_{\mathbb{N}} \mathbb{N}^{\ast n}$. Creating a \emph{function} like this would not be especially hard. For any pair $(w, w') \in \mathbb{N}^{\ast n} \times_{\mathbb{N}^n} \mathbb{N}^{\ast n}$, the image of $w$ and $w'$ in the abelian group $\mathbb{N}^n$ is the same, which is to say that the words $w, w' \in \mathbb{N}^{\ast n}$ are permuations of each other. Since the underlying permutation maps $\pi : G(m) \to \mathrm{S}_m$ in the definition of the action operad $G$ are all surjective, we can therefore always find an element of $g \in G(|w|)$ for which $\pi(g)(w) = w'$. Thus in order to make an injective function $\mathbb{N}^{\ast n} \times_{\mathbb{N}^n} \mathbb{N}^{\ast n} \to G \times_{\mathbb{N}} \mathbb{N}^{\ast n}$, all we need to do is make a choice $g_{(w, w')}$ like this for each $(w, w')$, and then set
\begin{eq*} \begin{array}{rll}
			\mathbb{N}^{\ast n} \times_{\mathbb{N}^n} \mathbb{N}^{\ast n} & \to & G \times_{\mathbb{N}} \mathbb{N}^{\ast n} \\
			(w, w') & \mapsto & ( \, g_{(w, w')}, w \, )
		\end{array}
\end{eq*}
Injectivity follows from
\begin{eq*} \begin{array}{rclcrcl}
		& & & & g_{(w, w')} & = & g_{(v, v')} \\
		( \, g_{(w, w')}, w \, ) & = & ( \, g_{(v, v')}, v \, ) & \implies & w & = & v \\
		& & & & w' & = & \pi(g_{(w, w')})(w) \\
		& & & & & = & \pi(g_{(v, v')})(v) \\
		& & & & & = & v'
		\end{array}
\end{eq*}
So how do we know if we can choose these $g_{(w, w')}$ in such a way that the resulting function is also a monoid homomorphism? If we could find a presentation of $\mathbb{N}^{\ast n} \times_{\mathbb{N}^n} \mathbb{N}^{\ast n}$ in terms of generators and relations then this would help a little, since we would only need to pick a $g_{(z, z')}$ for each generator $(z, z')$, and then define all other $g$ by way of products.
\begin{eq*} g_{(vw, v'w')} \, = \, g_{(v, v')} g_{(w, w')} \end{eq*}
But we would still need to know if our choice of $g_{(z, z')}$ obeyed the necessary relations on the generators of $\mathbb{N}^{\ast n} \times_{\mathbb{N}^n} \mathbb{N}^{\ast n}$. Luckily for us though, this turns out to be no problem at all. 

\begin{prop}\label{freemon} $\mathbb{N}^{\ast n} \times_{\mathbb{N}^n} \mathbb{N}^{\ast n}$ is a free monoid.
\end{prop}
\begin{proof}
Given an element $(w, w')$ of the monoid $\mathbb{N}^{\ast n} \times_{\mathbb{N}^n} \mathbb{N}^{\ast n}$, let $d(w, w')$ be the following set:
\begin{eq*} d(w, w') \, = \, \left\{ \begin{array}{rlrll}
							& & (w, w') & = & (u, u') \otimes (v, v'), \\
							(u, u'), (v, v') \in \mathbb{N}^{\ast n} \times_{\mathbb{N}^n} \mathbb{N}^{\ast n} & : & (u, u') & \neq & (I, I), \\
							& & (v, v') & \neq & (I,I)
					\end{array} \right\} 
\end{eq*}
We can use these sets to recursively define a decomposition of any element $(w, w')$ as a product of other elements of $\mathbb{N}^{\ast n} \times_{\mathbb{N}^n} \mathbb{N}^{\ast n}$. Specifically, if $d(w, w')$ is empty then we say that the decomposition of $(w, w')$ is just $(w, w')$ itself, and otherwise we choose any $\big( \, (u, u'), (v, v') \, \big) \in d(w, w')$ and say that the decomposition of $(w, w')$ is the concatenation of the decomposition of $(u, u')$ with the decomposition of $(v, v')$. Note that this process definitely terminates, since $|u|$ and $|v|$ are always strictly smaller that $|w|$, and any strictly decreasing sequence of natural numbers is finite.

Of course, we need to check that this decomposition of $(w, w')$ is well-defined, which amounts to checking that the choice of $(u, u'), (v, v')$ we make at each stage won't change the eventual output. To that end, suppose for the sake of contradiction that $(u_1, u'_1), ..., (u_m, u'_m)$ and $(v_1, v'_1), ..., (v_m', v'_{m'})$ are distinct decompositions of $(w, w')$ we could arrive at using the above process. Notice that we can assume without loss of generality that $|u_1| < |v_1|$. If instead $|u_1| > |w_1|$, we can just swap the labels of the sequences, and if $|u_1| = |v_1|$ then we can just discard those elements and  instead consider the decompositions $(u_2, u'_2), ..., (u_m, u'_m)$ and $(v_2, v'_2), ..., (v_m', v'_{m'})$ of $(u_1, u'_1) \otimes ... \otimes (u_m, u'_m) = (v_1, v'_1) \otimes ... \otimes (v_m', v'_{m'})$. Since $(u_1, u'_1), ..., (u_m, u'_m)$ and $(v_1, v'_1), ..., (v_m', v'_{m'})$ were distinct decompositions of $(w, w')$, in this way we will eventually reach some subsequences whose first elements are different; once we have, we can relabel them so that $|u_1| < |v_1|$. 

Then by definition,
\begin{eq*} u_1 \otimes \big( \, \bigotimes_{i=2}^m u_i \, ) \, = \, w \, = \, v_1 \otimes \big( \, \bigotimes_{i=2}^{m'} v_i \, )\end{eq*}
But $w, u_1, v_1, \bigotimes_{i=2}^m u_i, \bigotimes_{i=2}^{m'} v_i$ are all elements of $\mathbb{N}^{\ast n}$, which is a free monoid, and so they each have a unique decomposition as products of the generators $\{ z_1, ..., z_n \}$, and these all respect tensor products. Therefore, since $|u_1| < |v_1|$, there must exist some element $a$ of $\mathbb{N}^{\ast n}$ such that
\begin{eq*} w \, = \, u_1 \otimes a \otimes \big( \, \bigotimes_{i=2}^{m'} v_i \, )  \quad \implies \quad v_1 \, = \, u_1 \otimes a \end{eq*}
Since
\begin{eq*} |u'_1| \, = \, |u_1| \, < \, |v_1| \, = \, |v'_1| \end{eq*}
we can also use exactly the same reasoning to find an $a'$ in $\mathbb{N}^{\ast n}$ with $v'_1 = u'_1 \otimes a'$, and hence $(v_1, v'_1) = (u_1, u'_1) \otimes (a, a')$. Moreover, this $(a, a')$ is an element of $\mathbb{N}^{\ast n} \times_{\mathbb{N}^n} \mathbb{N}^{\ast n}$, because
\begin{eq*}\begin{array}{rrcccl}
			& v_1 & = & u_1 \otimes a & & \\
			\implies \quad & [v_1] & = & [u_1 \otimes a] & = & [u_1] + [a] \\
			& & & & & \\
			& v'_1 & = & u'_1 \otimes a' & & \\
			\implies \quad & [v'_1] & = & [u'_1 \otimes a'] & = & [u'_1] + [a'] \\
			& & & & & \\
			\implies \quad & [a] & = & [v_1] - [u_1] & & \\
			& & & [v'_1] - [u'_1] & = & [a']
		\end{array}
\end{eq*}
In other words, we have shown that the pair $\big( \, (u_1, u'_1) (a, a') \, \big)$ is an element of $d(v_1, v'_1)$. But by assumption $(v_1, v'_1), ..., (v_m', v'_{m'})$ was a decomposition of $(w, w')$, and hence the $d(v_i, v'_i)$ were supposed to be empty for each $i$, since that is when the decomposition finding process terminates. This is a contradiction, and hence our assumption that $(u_1, u'_1), ..., (u_m, u'_m)$ and $(v_1, v'_1), ..., (v_m', v'_{m'})$ were distinct decompositions of $(w, w')$ is false. Therefore, each $(w, w')$ in $\mathbb{N}^{\ast n} \times_{\mathbb{N}^n} \mathbb{N}^{\ast n}$ has a unique decomposition in terms of elements $(v_i, v'_i)$ for which $d(v_i, v'_i)$ is empty, and so $\mathbb{N}^{\ast n} \times_{\mathbb{N}^n} \mathbb{N}^{\ast n}$ is the free monoid whose generators are all such elements.
\end{proof}

It follows immediately from this that our contruction of an injective function $\mathbb{N}^{\ast n} \times_{\mathbb{N}^n} \mathbb{N}^{\ast n} \to G \times_{\mathbb{N}} \mathbb{N}^{\ast n}$ can be extended to be an inclusion of monoids.

\begin{prop} \label{stGnsub} $(s \times t)(\mathbb{G}_n)$ is (isomorphic to) a submonoid of $\mathrm{Mor}(\mathbb{G}_n)$
\end{prop}
\begin{proof}
For each generator $(z, z')$ of $\mathbb{N}^{\ast n} \times_{\mathbb{N}^n} \mathbb{N}^{\ast n}$, choose an element of $g_{(z, z')} \in G(|z|)$ with the property that $\pi(g_{(z, z')})(z) = z'$. This is always possible, since $(z, z') \in \mathbb{N}^{\ast n} \times_{\mathbb{N}^n} \mathbb{N}^{\ast n}$ implies that the words $z, z' \in \mathbb{N}^{\ast n}$ are permuations of each other, and the maps $\pi : G(m) \to \mathrm{S}_m$ are always surjective. Then we can define the homomorphism $i$ to be
\begin{eq*} \begin{array}{rlrll}
			i & : & \mathbb{N}^{\ast n} \times_{\mathbb{N}^n} \mathbb{N}^{\ast n} & \to & G \times_{\mathbb{N}} \mathbb{N}^{\ast n} \\
			& : & (z, z') & \mapsto & ( \, g_{(z, z')}, z \, )
		\end{array}
\end{eq*}
on generators. Since  by \cref{freemon} $\mathbb{N}^{\ast n} \times_{\mathbb{N}^n} \mathbb{N}^{\ast n}$ is free, this $i$ extends to a well-defined monoid homomorphism, as long as we choose $g_{(0, 0)} = e_0$ so that it preserves the identity. Moreover, for any two generators $(z_1, z'_1), (z_2, z'_2)$, we have
\begin{eq*} \begin{array}{rclcrcl}
		& & & & g_{(z_1, z'_1)} & = & g_{(z_2, z'_2)} \\
		( \, g_{(z_1, z'_1)}, z_1 \, ) & = & ( \, g_{(z_2, z'_2)}, z_2 \, ) & \implies & z_1 & = & z_2 \\
		& & & & z'_1 & = & \pi(g_{(z_1, z'_1)})(z_1) \\
		& & & & & = & \pi(g_{(z_2, z'_2)})(z_2) \\
		& & & & & = & z'_2
		\end{array}
\end{eq*}
and thus $i$ is injective. Therefore the image of $i$ is a submonoid of $G \times_{\mathbb{N}} \mathbb{N}^{\ast n}$ which is isomorphic to $\mathbb{N}^{\ast n} \times_{\mathbb{N}^n} \mathbb{N}^{\ast n}$, which we can see as a submonoid of $\mathrm{Mor}(\mathbb{G}_n)$ isomorphic to $(s \times t)(\mathbb{G}_n)$, as required.
\end{proof}

So, now we know that $(s \times t)(\mathbb{G}_n) \subseteq \mathrm{Mor}(\mathbb{G}_n)$. But what we are really interested in is whether $(s \times t)(\mathbb{G}_n) \subseteq \mathrm{Mor}(\mathbb{G}_n)$, and it is not really clear how we can recover this result from the former. In order to do this, we will need to employ the following result:

\begin{prop*} There exists a surjective map of $\mathrm{E}G$-algebras $q: \mathbb{G}_{2n} \to L\mathbb{G}_n$ \end{prop*} 

The proof of this fact is a long process that will require many subproofs, and so to allow us to remain focused on our current task we will gloss over the details for the time being. Instead, the question of the existence and surjectivity of $q$ will be covered at length in \cref{coeqalgebra}. 

Now, consider again \cref{stGnsub}, in particular the case $(s \times t)(\mathbb{G}_{2n}) \subseteq \mathrm{Mor}(\mathbb{G}_{2n})$. From this we can use the map $q$ to immediately conclude that $(s \times t)(L\mathbb{G}_n) \subseteq \mathrm{Mor}(L\mathbb{G}_n)$ as well, since by surjectivity of $q$ this statement is just equivalent to saying $q\big( \, (s \times t)(\mathbb{G}_{2n}) \, \big) \subseteq q\big( \, \mathrm{Mor}(\mathbb{G}_{2n}) \, \big)$. More precisely, we have the following:

\begin{cor} \label{stZsub} $(s \times t)(L\mathbb{G}_n)$ is (isomorphic to) a submonoid of $\mathrm{Mor}(L\mathbb{G}_n)$
\end{cor}
\begin{proof}
Let $i: (s \times t)(\mathbb{G}_{2n}) \hookrightarrow \mathrm{Mor}(\mathbb{G}_{2n})$ be an inclusion which allows us to view $(s \times t)(\mathbb{G}_{2n})$ as a submonoid of $\mathrm{Mor}(\mathbb{G}_{2n})$, as in \cref{stGnsub}. Also, let $\mathrm{Mor}(q): \mathrm{Mor}(\mathbb{G}_{2n}) \to \mathrm{Mor}(L\mathbb{G}_n)$ be the restriction of the monoidal functor $q$ on morphisms. Then the image of the composite of these two homomorphisms,
\begin{eq*} \mathrm{im}\big( \, \mathrm{Mor}(q) \circ i \, \big) \, = \, q\big( \, \mathrm{im}(i) \, \big) \, \cong \, q\big( \, (s \times t)(\mathbb{G}_{2n}) \, \big)\end{eq*}
is clearly a submonoid of $\mathrm{Mor}(L\mathbb{G}_n)$. 

But by supposition $q$ is a surjective functor. This means that there can exist a map $w \to v$ in $L\mathbb{G}_n$ if and only if there exists at least one map $w' \to v'$ in $\mathbb{G}_{2n}$, for some $w', v'$ which have $q(w') = w$ an $q(v') = v$. In other words,
\begin{eq*} q\big( \, (s \times t)(\mathbb{G}_{2n}) \, \big) \, = \, (s \times t)(L\mathbb{G}_n) \end{eq*}
and therefore the monoid $\mathrm{im}\big( \, \mathrm{Mor}(q) \circ i \, \big)$ that we saw above is really a submonoid of $\mathrm{Mor}(L\mathbb{G}_n)$ isomorphic to $(s \times t)(L\mathbb{G}_n)$, as required.
\end{proof} 

\subsection{Unit endomorphisms of $L\mathbb{G}_n$}

To help understand $\mathrm{Mor}(L\mathbb{G}_n)$, we intended to break it down into two pieces. The first was the source/target data $(s \times t)(L\mathbb{G}_n)$ which we explored in the previous section, and the second is the monoid of unit endomorphisms $L\mathbb{G}_n(I,I)$.

\begin{lem} \label{endcom} $L\mathbb{G}_n(I,I)$ is a commutative monoid.
\end{lem}
\begin{proof}
Let $f, f'$ be arbitrary elements of the monoid $L\mathbb{G}_n(I,I)$. Since both of these are morphisms in the monoidal category $L\mathbb{G}_n$, we can use the law of interchange to show that
\begin{eq*} \begin{array}{rll}
			f \otimes f' & = & (f \circ \mathrm{id}_I) \otimes (\mathrm{id}_I \circ f') \\
			& = & (f \otimes \mathrm{id}_I) \circ (\mathrm{id}_I \otimes f') \\
			& = & f \circ f' \\
			& = & (\mathrm{id}_I \otimes f) \circ (f' \otimes \mathrm{id}_I) \\
			& = & (f' \circ \mathrm{id}_I) \otimes (\mathrm{id}_I \circ f) \\
			& = & f' \otimes f
		\end{array}
\end{eq*}
This is another instance of the Eckmann-Hilton argument.
\end{proof}

We can actually conclude even more from the above proof. $f \otimes f' = f \circ f'$.  Recall our assumption from the previous section, which will be proven in the following chapter: that there exists a surjective $\mathrm{E}G$-algebra map $q: \mathbb{G}_{2n} \to L\mathbb{G}_n$. This map lets us extend the result from \cref{Gnmapsaction} onto $L\mathbb{G}_n$.

\begin{lem} \label{allmapsaction} Every morphism in $L\mathbb{G}_n$ can be expressed as $\alpha_{L\mathbb{G}_n}(g; \mathrm{id}_{x_1}, ..., \mathrm{id}_{x_m})$, for some $g \in G(m)$ and $x_i \in \mathbb{Z}^{\ast n}$.
\end{lem}
\begin{proof}
Let $f$ be an arbitrary morphism in $L\mathbb{G}_n$. By surjectivity of $q: \mathbb{G}_{2n} \to L\mathbb{G}_n$, there exists at least one morphism $f'$ in $\mathbb{G}_{2n}$ such that $q(f') = f$, and from \cref{Gnmapsaction} we know that this $f'$ can be expressed uniquely as $\alpha(g; \mathrm{id}_{x_1}, ..., \mathrm{id}_{x_m})$ for some $g \in G(m)$ and $x'_i \in \{z_1, ..., z_{2n} \}$. Thus, because $q$ is a map of $\mathrm{E}G$-algebras, we will have
\begin{eq*} f \, = \, q(f') \, = \, q \big( \, \alpha_{\mathbb{G}_{2n}}(g; \mathrm{id}_{x_1}, ..., \mathrm{id}_{x_m}) \, \big)  \, = \, \alpha_{L\mathbb{G}_n}(g; \mathrm{id}_{q(x'_1)}, ..., \mathrm{id}_{q(x'_m)}) \end{eq*}
Therefore, there is at least one collection of $x_i = q(x'_i)$ for which the statement of the proposition holds.
\end{proof}

It is worth noting that, unlike with $\mathbb{G}_n$, this ability to express a morphism from $L\mathbb{G}_n$ as an action morphism is \emph{not} necessarily unique. Regardless, \cref{allmapsaction} still tells us a very important fact about the morphisms of $L\mathbb{G}_n$, namely:

\begin{cor} \label{compinv} Every morphism $f: w \to v$ in $L\mathbb{G}_n$ has an inverse under composition, $f^{-1}: v \to w$.
\end{cor}
\begin{proof}
From \cref{allmapsaction} we know that every morphism $f$ in $L\mathbb{G}_n$ is of the form $\alpha(g; \mathrm{id}_{x_1}, ..., \mathrm{id}_{x_m})$, for some $g \in G(m)$ and $x_i \in \mathbb{Z}^{\ast n}$. It follows immediately that
\begin{eq*} \begin{array}{rl}
			& \alpha( \, g \, ; \, \mathrm{id}_{x_1}, ..., \mathrm{id}_{x_m} \, ) \circ \alpha( \, g^{-1} \, ; \, \mathrm{id}_{x_{\pi(g^{-1})(1)}}, ..., \mathrm{id}_{x_{\pi(g^{-1})(m)}} \, ) \\
			= & \alpha( \, gg^{-1} \, ; \, \mathrm{id}_{x_{\pi(g^{-1})(1)}}, ..., \mathrm{id}_{x_{\pi(g^{-1})(m)}} \, ) \\
			= & \alpha( \, e_m \, ; \, \mathrm{id}_{x_{\pi(g^{-1})(1)}}, ..., \mathrm{id}_{x_{\pi(g^{-1})(m)}} \, ) \\
			= & \mathrm{id}_{x_{\pi(g^{-1})(1)} \otimes ... \otimes x_{\pi(g^{-1})(m)}} \\
			& \\
			& \alpha( \, g^{-1} \, ; \, \mathrm{id}_{x_{\pi(g^{-1})(1)}}, ..., \mathrm{id}_{x_{\pi(g^{-1})(m)}} \, ) \circ \alpha( \, g \, ; \, \mathrm{id}_{x_1}, ..., \mathrm{id}_{x_m} \, ) \\
			= & \alpha( \, g^{-1}g \, ; \, \mathrm{id}_{x_1}, ..., \mathrm{id}_{x_m} \, ) \\
			= & \alpha( \, e_m \, ; \, \mathrm{id}_{x_1}, ..., \mathrm{id}_{x_m} \, ) \\
			= & \mathrm{id}_{x_1 \otimes ... \otimes x_m}
		\end{array}
\end{eq*}
In other words, $f^{-1} : = \alpha(g^{-1}; \mathrm{id}_{x_{\pi(g^{-1})(1)}}, ..., \mathrm{id}_{x_{\pi(g^{-1})(m)}})$ is the inverse of $f$.
\end{proof}

This result will now allow us to increase the results of our Eckmann-Hilton argument.

\begin{lem} \label{endab} $L\mathbb{G}_n(I,I)$ is an abelian group
\end{lem}
\begin{proof}
As we saw in the proof of \cref{endcom}, the operations of tensor product and composition coincide for endomorphisms of the unit object of $L\mathbb{G}_n$. In particular this means that if some morphism $f: I \to I$ has a compositional inverse $f^{-1}$, then it will also be an invertible element of the monoid $\big( \, L\mathbb{G}_n(I,I), \otimes \, \big)$. Thus by \cref{compinv}, every element of the commutative monoid $L\mathbb{G}_n(I,I)$ is invertible, or in other words $L\mathbb{G}_n(I,I)$ is an abelian group.
\end{proof}

In fact, with a slightly broader argument we can extend this result to every morphism of $L\mathbb{G}_n$.

\begin{prop} \label{tensinv} Every morphism $f: w \to v$ in $L\mathbb{G}_n$ has an inverse under tensor product, $f^*: w^* \to v^*$. That is, the monoid $\mathrm{Mor}(L\mathbb{G}_n)$ is actually a group.
\end{prop}
\begin{proof}
For any $f: w \to v$ in $L\mathbb{G}_n$, consider the map $\mathrm{id}_{w^*} \otimes f^{-1} \otimes \mathrm{id}_{v^*}$, where $f^{-1}$ is the compositional inverse of $f$ as in \cref{compinv}. This morphism has source $w^* \otimes v \otimes v^* = w^*$ and target $w^* \otimes w \otimes v^* = v^*$, which allows us to apply the law of interchange to get
\begin{eq*} \begin{array}{rll}
			f \otimes (\mathrm{id}_{w^*} \otimes f^{-1} \otimes \mathrm{id}_{v^*}) & = & \big( \, f \circ \mathrm{id}_w \, \big) \otimes \big( \, \mathrm{id}_{v^*} \circ  (\mathrm{id}_{w^*} \otimes f^{-1} \otimes \mathrm{id}_{v^*}) \, \big) \\
			& = & \big( \, f \otimes \mathrm{id}_{v^*} \, \big) \circ \big( \, \mathrm{id}_w \otimes (\mathrm{id}_{w^*} \otimes f^{-1} \otimes \mathrm{id}_{v^*}) \, \big) \\
			& = & ( f \otimes \mathrm{id}_{v^*} ) \circ ( f^{-1} \otimes \mathrm{id}_{v^*}) \\
			& = & (f \circ f^{-1}) \otimes (\mathrm{id}_{v^*} \circ \mathrm{id}_{v^*}) \\
			& = & \mathrm{id}_v \otimes \mathrm{id}_{v^*} \\
			& = & \mathrm{id}_I
		\end{array}
\end{eq*}
and likewise
\begin{eq*} \begin{array}{rll}
			(\mathrm{id}_{w^*} \otimes f^{-1} \otimes \mathrm{id}_{v^*}) \otimes f & = & \big( \, (\mathrm{id}_{w^*} \otimes f^{-1} \otimes \mathrm{id}_{v^*}) \circ \mathrm{id}_{w^*} \, \big) \otimes \big( \, \mathrm{id}_v \circ f \, \big) \\
			& = & \big( \, (\mathrm{id}_{w^*} \otimes f^{-1} \otimes \mathrm{id}_{v^*}) \otimes \mathrm{id}_v \, \big) \circ \big( \, \mathrm{id}_{w^*} \otimes f \, \big) \\
			& = & (\mathrm{id}_{w^*} \otimes f^{-1}) \circ (\mathrm{id}_{w^*} \otimes f) \\
			& = & (\mathrm{id}_{w^*} \circ \mathrm{id}_{w^*}) \otimes (f^{-1} \circ f)\\
			& = & \mathrm{id}_{w^*} \otimes \mathrm{id}_w \\
			& = & \mathrm{id}_I
		\end{array}
\end{eq*}
In other words, $f^* := \mathrm{id}_{w^*} \otimes f^{-1} \otimes \mathrm{id}_{v^*}$ is the inverse of $f$ in the monoid $\mathrm{Mor}(L\mathbb{G}_n)$, as required.
\end{proof}

\begin{prop} \label{morprod}
\begin{eq*} \mathrm{Mor}(L\mathbb{G}_n) \quad = \quad (s \times t)(L\mathbb{G}_n) \times L\mathbb{G}_n(I,I) \end{eq*}
\end{prop}
\begin{proof}
The first step in this proof will be to show that $L\mathbb{G}_n(I,I)$ is a normal subgroup of $\mathrm{Mor}(L\mathbb{G}_n)$. From the definition of $L\mathbb{G}_n(I,I)$ we already know that it is a submonoid of $\mathrm{Mor}(L\mathbb{G}_n)$, and \cref{endab,tensinv} tell us that it is also a subgroup. For normality, recall from \cref{spacial} that all $\mathrm{E}G$-algebras are spacial, and so in particular $L\mathbb{G}_n$ is. This means that for any $h \in L\mathbb{G}_n(I,I)$ and $w \in \mathrm{Ob}(L\mathbb{G}_n)$ we will always have $h \otimes \mathrm{id}_w = \mathrm{id}_w \otimes h$, and thus for any $f:w \to v$ in $\mathrm{Mor}(L\mathbb{G}_n)$ we get
\begin{eq*} \begin{array}{rll}
		h \otimes f & = & (\mathrm{id}_I \circ h) \otimes (f \circ \mathrm{id}_w) \\
		& = & (\mathrm{id}_I \otimes f) \circ (h \otimes \mathrm{id}_w) \\
		& = & (f \otimes \mathrm{id}_I) \circ (\mathrm{id}_w \otimes h) \\
		& = & (f \circ \mathrm{id}_w) \otimes (\mathrm{id}_I \circ h) \\
		& = & f \otimes h
		\end{array}
\end{eq*}
That is, $L\mathbb{G}_n(I,I)$ is a subgroup of the centre of $\mathrm{Mor}(L\mathbb{G}_n)$, and so because
\begin{eq*} f \otimes h \otimes f^* \, = \, h \otimes f \otimes f^* \, = \, h \, \in L\mathbb{G}_n(I,I) \end{eq*}
it follows that $L\mathbb{G}_n(I,I)$ is a normal subgroup of $\mathrm{Mor}(L\mathbb{G}_n)$.

Next, because we know that $L\mathbb{G}_n(I,I)$ is normal we can consider the quotient group
\begin{eq*} \begin{tikzcd}
L\mathbb{G}_n(I,I) \ar[r, hookrightarrow] & \mathrm{Mor}(L\mathbb{G}_n) \ar[r] & \bigquotient{\mathrm{Mor}(L\mathbb{G}_n)}{L\mathbb{G}_n(I,I)}
\end{tikzcd} \end{eq*}
Whenever they exist, quotient groups are an example of a cokernel in the category of groups and group homomorphisms. This means that the quotient map $\mathrm{Mor}(L\mathbb{G}_n) \to \mathrm{Mor}(L\mathbb{G}_n) / L\mathbb{G}_n(I,I)$ will factor any homomorphism whose composite with the inclusion $L\mathbb{G}_n(I,I) \to \mathrm{Mor}(L\mathbb{G}_n)$ is the zero map. But our source/target map $s \times t : \mathrm{Mor}(L\mathbb{G}_n) \to (s \times t)(L\mathbb{G}_n)$ is one such homomorphism, since for any $h: I \to I$ clearly $(s \times t)(h) = (I, I)$, which is the identity element in $(s \times t)(L\mathbb{G}_n)$. Therefore there must exist a unique homomorphism $u$ making the triangle below commute:
\begin{eq*} \begin{tikzcd}
\mathrm{Mor}(L\mathbb{G}_n) \ar[dd] \ar[ddrr, "s \times t"] & & \\
& & \\
\bigquotient{\mathrm{Mor}(L\mathbb{G}_n)}{L\mathbb{G}_n(I,I)} \ar[rr, "u"] & & (s \times t)(L\mathbb{G}_n)
\end{tikzcd} \end{eq*}
This map $u$ will be surjective, because $s \times t$ is, but in fact it will also be injective. This is because if two morphisms $f, f'$ in $L\mathbb{G}_n$ are not parallel, then there can be no $h \in L\mathbb{G}_n(I,I)$ for which $f \otimes h = f'$ --- the sources and targets just wouldn't match up --- and so $f$ and $f'$ must be in different equivalence classes in $\mathrm{Mor}(L\mathbb{G}_n)/L\mathbb{G}_n(I,I)$. More precisely,
\begin{eq*} \begin{array}{rclcrcl}
		(s \times t)(f') & \neq & (s \times t)(f) \\
		& = & (s \times t)(f) \otimes (I, I) \\
		& = & (s \times t)(f) \otimes (s \times t)(h) \\
		& = & (s \times t)(f \otimes h) \\
		& & \\
		\implies \quad f \otimes h & \neq & f' \\
		\implies \quad [f] & \neq & [f'] \\
		\end{array}
\end{eq*}
Thus $u$ is bijective, or in other words
\begin{eq*} \bigquotient{\mathrm{Mor}(L\mathbb{G}_n)}{L\mathbb{G}_n(I,I)} \quad \cong \quad (s \times t)(L\mathbb{G}_n) \end{eq*}

Finally, by \cref{stZsub} $(s \times t)(L\mathbb{G}_n)$ is also a submonoid (hence subgroup) of $\mathrm{Mor}(L\mathbb{G}_n)$. Combined with the identity above, we see that what have here is a split exact sequence of groups
\begin{eq*} \begin{tikzcd}
L\mathbb{G}_n(I,I) \ar[r] & \mathrm{Mor}(L\mathbb{G}_n) \ar[r] & (s \times t)(L\mathbb{G}_n)
\end{tikzcd} \end{eq*}
That is, $\mathrm{Mor}(L\mathbb{G}_n)$ is a split group extension of $(s \times t)(L\mathbb{G}_n)$ by $L\mathbb{G}_n(I,I)$, or equivalently $\mathrm{Mor}(L\mathbb{G}_n)$ is a semi direct product $L\mathbb{G}_n(I,I) \rtimes (s \times t)(L\mathbb{G}_n)$. Moreover, we saw earlier that $L\mathbb{G}_n(I,I)$ is a subgroup of the centre of $\mathrm{Mor}(L\mathbb{G}_n)$, and so it follows that $\mathrm{Mor}(L\mathbb{G}_n)$ is also a central extension of $(s \times t)(L\mathbb{G}_n)$. However, the only extensions which are both central and split are the trivial extensions, and therefore $\mathrm{Mor}(L\mathbb{G}_n)$ is really just the direct product $L\mathbb{G}_n(I,I) \times (s \times t)(L\mathbb{G}_n)$, as required.
\end{proof} 

\begin{prop}\label{Zmor1} The endomorphisms of the unit object of $L\mathbb{G}_n$ are
\begin{eq*} L\mathbb{G}_n(I, I) \quad = \quad \bigquotient{{\mathrm{Mor}(L\mathbb{G}_n)}^{\mathrm{ab}}}{\mathbb{Z}^n} \end{eq*}
and therefore
\begin{eq*} \mathrm{Mor}(L\mathbb{G}_n) \quad = \quad \mathbb{Z}^{\ast n} \times_{\mathbb{Z}^n} \mathbb{Z}^{\ast n} \, \times \, \bigquotient{{\mathrm{Mor}(L\mathbb{G}_n)}^{\mathrm{gp, ab}}}{\mathbb{Z}^n} \end{eq*}
\end{prop}
\begin{proof}
From \cref{morprod}, we know that
\begin{eq*} \mathrm{Mor}(L\mathbb{G}_n) \quad = \quad (s \times t)(L\mathbb{G}_n) \times L\mathbb{G}_n(I, I) \end{eq*}
Abelianising both sides of this equation, we get
\begin{eq*} \begin{array}{rll}
			{\mathrm{Mor}(L\mathbb{G}_n)}^{\mathrm{ab}} & = & \big( \, (s \times t)(L\mathbb{G}_n) \times L\mathbb{G}_n(I, I) \, \big)^{\mathrm{ab}} \\
			& = & {(s \times t)(L\mathbb{G}_n)}^{\mathrm{ab}} \times {L\mathbb{G}_n(I, I)}^{\mathrm{ab}} \\
			& = & {(s \times t)(L\mathbb{G}_n)}^{\mathrm{ab}} \times L\mathbb{G}_n(I, I) \\
		\end{array}
\end{eq*} 
since $L\mathbb{G}_n(I, I)$ is already abelian. Now, there is an obvious inclusion ${(s \times t)(L\mathbb{G}_n)}^{\mathrm{ab}} \hookrightarrow (s \times t)(L\mathbb{G}_n)^{\mathrm{ab}} \times L\mathbb{G}_n(I, I)$, and since everything here is abelain, all subgroups are normal subgroups. Thus we can take the quotient of the above equation by this map, to obtain 
\begin{eq*} L\mathbb{G}_n(I, I) \quad = \quad \bigquotient{{\mathrm{Mor}(L\mathbb{G}_n)}^{\mathrm{ab}}}{{(s \times t)(L\mathbb{G}_n)}^{\mathrm{ab}}} \end{eq*}
We can also now substitute this expression back into our original equation, which yields
\begin{eq*} \mathrm{Mor}(L\mathbb{G}_n) \quad = \quad (s \times t)(L\mathbb{G}_n) \times \bigquotient{{\mathrm{Mor}(L\mathbb{G}_n)}^{\mathrm{ab}}}{{(s \times t)(L\mathbb{G}_n)}^{\mathrm{ab}}} \end{eq*}
But from \cref{stpullback} we already know that the value of $(s \times t)(L\mathbb{G}_n)$ is $\mathbb{Z}^{\ast n} \times_{\mathbb{Z}^n} \mathbb{Z}^{\ast n}$. Moreover, the homomorphisms that this pullback is taken over are both the quotient map of abelianisation $\mathbb{Z}^{\ast n} \to \mathbb{Z}^n$, and as a result,
\begin{eq*} (\mathbb{Z}^{\ast n} \times_{\mathbb{Z}^n} \mathbb{Z}^{\ast n})^{\mathrm{ab}} \quad = \quad \mathbb{Z}^n \end{eq*}
Putting this all together, we get the two equations in the statement of the proposition.
\end{proof}

Note that its not entirely clear here exactly which $\mathbb{Z}^n$ subgroup of $\mathrm{Mor}(L\mathbb{G}_n)^{\mathrm{gp, ab}}$ is being referenced in the statement of \cref{Zmor1}. This is because the existence of such a quotient relied on our assumption that the algebra map $q: \mathbb{G}_{2n} \to L\mathbb{G}_n$ exists, and so we will not be able to actually perform this quotient until we understand where $q$ comes from.
