\documentclass{amsart} % default font size is 10pt
\setcounter{tocdepth}{3}
\setcounter{secnumdepth}{3}
\usepackage{latexsym}
\usepackage{amsfonts}
\usepackage{amssymb}
\usepackage{amsmath}
\usepackage{amscd}
\usepackage{graphicx}
\usepackage{eucal}
\usepackage{amsthm}
\usepackage[all]{xy}
\usepackage{tikz-cd}
\usepackage{pdfsync}
\usepackage{xspace}
\usepackage[unicode=true, pdfusetitle,
 bookmarks=true,bookmarksnumbered=false,
 breaklinks=false,
 backref=false,
 colorlinks=true,
 %linkcolor=blue!70!black,
 citecolor=black,
 urlcolor=blue!78!red,
 final
]{hyperref}
\usepackage[capitalise]{cleveref}
\newcommand{\fref}{\cref}
\newcommand{\Fref}{\Cref}
\newcommand{\prettyref}{\cref}
\newcommand{\newrefformat}[2]{}

\usepackage[color=orange!80,bordercolor=black,textwidth=3cm,textsize=small,colorinlistoftodos]{todonotes}
\makeatletter \providecommand\@dotsep{5}
\makeatother
\newcommand{\amsartlistoftodos}{\makeatother \listoftodos\relax}

%% E.P. notes
\newcommand{\epnote}[1]{\todo[color=blue!40,linecolor=blue!40!black,size=\tiny]{#1}}
\newcommand{\epmpar}[1]{\todo[noline,color=blue!40,linecolor=blue!40!black,size=\tiny]{#1}}
\newcommand{\epnoteil}[1]{\todo[inline,color=blue!40,linecolor=blue!40!black,size=\normalsize]{#1}}

%% N.G. notes
\newcommand{\ngnote}[1]{\todo[color=red!40,linecolor=red!40!black,size=\tiny]{#1}}
\newcommand{\ngmpar}[1]{\todo[noline,color=red!40,linecolor=red!40!black,size=\tiny]{#1}}
\newcommand{\ngnoteil}[1]{\todo[inline,color=red!40,linecolor=red!40!black,size=\normalsize]{#1}}

% Cleveref definitions
\crefname{prop}{Proposition}{Propositions}
\crefname{thm}{Theorem}{Theorems}
\crefname{defn}{Definition}{Definitions}
\crefname{notn}{Notation}{Notations}
\crefname{construction}{Construction}{Constructions}
\crefname{lem}{Lemma}{Lemmas}
\crefname{rem}{Remark}{Remarks}
\crefname{cor}{Corollary}{Corollaries}
\crefname{scholium}{Scholium}{Scholia}
\crefname{figure}{Figure}{Figures}
\crefname{equation}{Display}{Displays}
\crefname{eq}{Display}{Displays}
\crefname{eqn}{Display}{Displays}

\newenvironment{eq}{\begin{equation}}{\end{equation}}
\newenvironment{eqn}{\begin{equation}}{\end{equation}}
\newenvironment{eq*}{\begin{equation*}}{\end{equation*}}
\newenvironment{eqn*}{\begin{equation*}}{\end{equation*}}

%\usepackage{natbib}
\newcommand{\bs}{\boldsymbol}
\newcommand{\mb}{\mathbf}
\renewcommand{\dot}{\centerdot}
\newcommand{\R}{\mathbb{R}}
\newcommand{\m}[1]{\mathcal{#1}}
\renewcommand{\SS}{\mathcal{S}}
\newcommand{\colim}{\textrm{colim }}
\newcommand{\f}[1]{\ensuremath{\mathcal{#1}}\xspace}
\newcommand{\g}[1]{\ensuremath{\mathbb{#1}}\xspace}
\newcommand{\Cat}{\ensuremath{\textnormal{Cat}}\xspace}
\newcommand{\cd}[2][]{\vcenter{\hbox{\xymatrix#1{#2}}}}
\newcommand{\Set}{\ensuremath{\textnormal{Set}}\xspace}
\newcommand{\twocat}{\ensuremath{\textnormal{2-Cat}}\xspace}
\newcommand{\Icon}{\ensuremath{\textnormal{Icon}}\xspace}
%\newcommand{\to}{\rightarrow}

%\pdfshift

\begin{document}
\numberwithin{equation}{section}% [if desired]
\newtheorem{thm}[equation]{Theorem}
\newtheorem{prop}[equation]{Proposition}
\newtheorem{lem}[equation]{Lemma}
\newtheorem{cor}[equation]{Corollary}

\newtheoremstyle{example}{\topsep}{\topsep}%
     {}%         Body font
     {}%         Indent amount (empty = no indent, \parindent = para indent)
     {\bfseries}% Thm head font
     {.}%        Punctuation after thm head
     {2pt}%     Space after thm head (\newline = linebreak)
     {\thmname{#1}\thmnumber{ #2}\thmnote{ #3}}%         Thm head spec

   \theoremstyle{example}
   \newtheorem{nota}[equation]{Notation}
   \newtheorem{example}[equation]{Example}
   \newtheorem{defi}[equation]{Definition}
   \newtheorem{rem}[equation]{Remark}
	\newtheorem{comment}[equation]{Comment}

\tableofcontents

\begin{defi} A strict monoidal category $X$ is said to be spacial if, for any object $x \in \mathrm{Ob}(X)$ and any endomorphism of the unit object $f: I \to I$, 
\begin{eq*} f \otimes \mathrm{id}_x = \mathrm{id}_x \otimes f \end{eq*}
\end{defi}

The motivation for the name `spacial' comes from the context of string diagrams \cite{graphicalmon}. In a string diagram, the act of tensoring two strings together is represented by placing those strings side by side. Since the defining feature of the unit object is that tensoring it with other objects should have no effect, the unit object is therefore represented diagrammatically by the absense of a string. An endomorphism of the unit thus appears as an entity with no input or output strings, detached from the rest of the diagram. In a real-world version of these diagrams, made out of physical strings arranged in real space, we could use this detachedness to grab these endomorphisms and slide them over or under any strings we please, without affecting anything else in the diagram. This ability is embodied algebraically by the equation above, and hence categories which obey it are called `spacial'.

\begin{lem}\label{spacelem} All $\mathrm{E}G$-algebras are spacial. \end{lem}
\begin{proof}
Let $X$ be an $\mathrm{E}G$-algebra, and fix $x \in \mathrm{Ob}(X)$ and $f: I \to I$. By the surjectivity of $\pi : G(2) \to S_2$ we know that the set $\pi^{-1}( \, (1 \, 2) \, )$ is non-empty. Examining the properties of such elements, we see that for all $g \in \pi^{-1}( \, (1 \, 2) \, )$ we have
\begin{eq*}\begin{array}{rll}
		\alpha( \, g \, ; \, \mathrm{id}_x, \, \mathrm{id}_I \, ) \circ \alpha( \, e_2 \, ; \, \mathrm{id}_x, \, f \, ) & = & \alpha( \, g \, ; \, \mathrm{id}_x, \, f \, ) \\
		& = & \alpha( \, e_2 \, ; \, f, \, \mathrm{id}_I \, ) \circ \alpha( \, g \, ; \, \mathrm{id}_x, \, \mathrm{id}_I \, ) \\
		\end{array}.
\end{eq*}
Thus in order to obtain the result we're after, it suffices to find a particular $g \in \pi^{-1}( \, (1 \, 2) \, )$ for which
\begin{eq*}\alpha( \, g \, ; \, \mathrm{id}_x, \, \mathrm{id}_I \, ) = \mathrm{id}_x .\end{eq*}
Since
\begin{eq*}\begin{array}{rll}
		\alpha( \, g \, ; \, \mathrm{id}_x, \, \mathrm{id}_I \, ) & = & \alpha( \, g \, ; \, \mathrm{id}_x, \, \alpha( e_0; - ) \, ) \\
		& = & \alpha( \, \mu(g; e_1, e_0) \, ; \, \mathrm{id}_x \, )
		\end{array}
\end{eq*}
it would further suffice to find a $g \in \pi^{-1}( \, (1 \, 2) \, )$ for which
\begin{eq*} \mu(g; e_1, e_0) = e_1 .\end{eq*}
To this end, choose an arbtrary element $h \in \pi^{-1}( \, (1 \, 2) \, )$. This $h$ probably won't obey the above equation, but we must have
\begin{eq*} \mu( \, h \ ; \, e_1, \, e_0 \, ) = k \end{eq*}
for some some $k \in G(1)$, and using this we can construct a new element of $G(2)$ which does have all of the properties we require --- namely $h \cdot \mu(e_2; k^{-1}, e_1)$. To see that this is the correct choice, first note that $\pi(k) = e_1$, since this is the only element of $S_1$. Following from that, we have 
\begin{eq*}\begin{array}{rll}
		\pi \big( \, \mu(e_2; k^{-1}, e_1) \, \big) & = & \mu \big( \, \pi(e_2) \ ; \, \pi(k^{-1}), \, \pi(e_1) \, \big) \\
		& = & \mu \big( \, e_2  \ ; \, e_1, \, e_1 \, \big) \\
		& = & e_2
		\end{array}
\end{eq*}
and hence
\begin{eq*}\begin{array}{rll}
		\pi \big( h \cdot \mu(e_2; k^{-1}, e_1) \big) \\
		& = & \pi(h) \cdot \pi \big(\mu(e_2; k^{-1}, e_1) \big) \\
		& = & (1 \, 2) \cdot e_2 \\
		& = & (1 \, 2)
		.\end{array}
\end{eq*}
So $h \cdot \mu(e_2; k^{-1}, e_1)$ is in $\pi^{-1}( \, (1 \, 2) \, )$, and furthermore
\begin{eq*}\begin{array}{rll}
		\mu \big( \, h \cdot \mu(e_2; k^{-1}, e_1) \ ; \, e_1, \, e_0 \, \big) & = & \mu( \, h \ ; \, e_1, \, e_0 \, ) \cdot \mu \big( \, \mu(e_2; k^{-1}, e_1) \ ; \, e_1, \, e_0 \, \big) \\
		& = & \mu( \, h \ ; \, e_1, \, e_0 \, ) \cdot \mu \big( \, e_2 \ ; \, \mu(k^{-1}; e_1), \, \mu(e_1; e_0) \, \big) \\
		& = & \mu( \, h \ ; \, e_1, \, e_0 \, ) \cdot \mu( \, e_2 \ ; \, k^{-1}, e_0 \, ) \\
		& = & k \cdot k^{-1} \\
		& = & e_1
		\end{array}.
\end{eq*}
Therefore, $h \cdot \mu(e_2; k^{-1}, e_1)$ is exactly the $g$ we were looking for, and so we can finally conclude that
\begin{eq*} \alpha( \, e_2 \, ; \, \mathrm{id}_x, \, f \, ) = \alpha( \, e_2 \, ; \, f, \, \mathrm{id}_I \, ) .\end{eq*}
\end{proof}

\section{Free algebras of action operads}

Our ultimate goal for this chapter is to understand the free braided monoidal category on an finite number of invertible objects. Thus, now that we have a firm grasp on action operads and their algebras, we should begin to think about the various free constructions they can form. 

\subsection{The free algebra on $n$ objects} 

We begin with the simplest case, which we will use extensively when calculating the invertible case later on. In the paper \cite{operadborel}, Gurski establishes how to contruct certain free action operad algebras through the use of the Borel construction. What follows in this section is a quick summary of the results which will be useful for our purposes. For a more detailed treatment please refer to \cite{operadborel}.

\begin{prop}\label{freealg} There exists a free $\mathrm{E}G$-algebra on $n$ objects. That is, there is an $\mathrm{E}G$-algebra $Y$ such that for any other $\mathrm{E}G$-algebra $X$, we have an isomorphism of categories
\begin{eq*} \mathrm{E}G\mathrm{Alg}_S(Y, X) \cong X^n .\end{eq*}
\end{prop}
\begin{proof}
There is an obvious forgetful 2-functor $U: \mathrm{E}G\mathrm{Alg}_S \to \mathrm{Cat}$ sending $\mathrm{E}G$-algebras to their underlying categories. $U$ has a left adjoint, which we call the free 2-functor $F : \mathrm{Cat} \to \mathrm{E}G\mathrm{Alg}_S$joint to it. It follows immediately that
\begin{eq*}\begin{array}{rll}
		U(X)^n & = & \mathrm{Cat}(\{1, ..., n\}, U(X) ) \\
		& \cong & \mathrm{E}G\mathrm{Alg}_S( F(\{1, ..., n\}), X) 
		\end{array}.
\end{eq*}
Since $X$ and $U(X)$ are obviously isomorphic as categories, this shows that $F(\{1, ..., n\})$ is the free algebra on $n$ objects as required. 
\end{proof}

\begin{defi}\label{Gndef} Let $\mathbb{G}_n$ denote the $\mathrm{E}G$-algebra
\begin{eq*} \mathbb{G}_n = \coprod_{m \geq 0} \mathrm{E}G(m) \times_{G(m)} \{ 1,...,n \}^m .\end{eq*}
\end{defi}

Objects in this algebra are equivalence classes of tuples $(g; x_1, ..., x_m)$, $g \in G(m)$, $x_i \in \{1, ..., n\}$ under the relation
\begin{eq*} ( \, gh \, ; \, x_1, \, ..., \, x_m \, ) \sim ( \, g \, ; \, x_{\pi(h)^{-1}(1)}, \, ..., \, x_{\pi(h)^{-1}(m)} \, ). \end{eq*}
Using this relation we can write each object uniquely in the form $[e; x_1, ..., x_m]$, $x_i \in \{1, ..., n\}$, and we will adopt the notation $x_1 \otimes ... \otimes x_m$ for this equivalence class. In other words, we're viewing $\mathrm{Ob}(\mathbb{G}_n)$ as the monoid $\mathbb{N}^{*n}$. Also, to avoid any confusion, we'll switch to calling the generators $1, ..., n$ of this monoid $z_1, ..., z_n$ instead.

Similarly, the morphisms of $\mathbb{G}_n$ are the equivalence classes of the maps
\begin{eq*} (! ; \mathrm{id}_{x_1}, ..., \mathrm{id}_{x_m}) : ( g ; x_1, ..., x_m ) \to ( h ; x_1, ..., x_m ). \end{eq*}
Using the relation we can write each morphism uniquely in the form
\begin{eq*} [g ; \mathrm{id}_{x_1},...,\mathrm{id}_{x_m}] \, : \, x_1 \otimes ... \otimes x_m \to x_{\pi(g)^{-1}(1)} \otimes ... \otimes x_{\pi(g)^{-1}(m)},\end{eq*}
with $x_i \in \{z_1, ..., z_n \}$. The action of $\mathrm{E}G$ on objects of $\mathbb{G}_n$ is permutation and tensor product, and the action on morphisms is given by
\begin{eq*} \alpha( \, g \, : \, [h_1; \mathrm{id}_{x_1}, ..., \mathrm{id}_{x_{m_1}}], \, ..., \, [h_k; \mathrm{id}_{x_1}, ..., \mathrm{id}_{x_{m_k}}] \, ) = [ \, \mu(g;h_1, .., h_k) \, ; \, \mathrm{id}_{x_1}, \, ..., \, \mathrm{id}_{x_{m_k}} \, ] \end{eq*}
Notice that by the tensor product notation we adopted earlier the object $[e; x]$ is written as just $x$, and so $[e; \mathrm{id}_x] = \mathrm{id}_{[e;x]}$ should be written as $\mathrm{id}_x$, and hence by the above $[g; \mathrm{id}_{x_1}, ..., \mathrm{id}_{x_m}]$ is really $\alpha(g; \mathrm{id}_{x_1}, ..., \mathrm{id}_{x_m})$.

\begin{thm}\label{freealg} $\mathbb{G}_n$ is the free $\mathrm{E}G$-algebra on $n$ objects. That is, 
\begin{eq*}  F(\{1, ..., n\}) = \mathbb{G}_n \end{eq*}
\end{thm}

\subsection{The free algebra on $n$ invertible objects}

We saw in Proposition \ref{freealg} that the existence of a free $\mathrm{E}G$-algebra on $n$ objects can be proven by taking the left adjoint of a 2-functor which forgets about the algebra structure. Now we want to extend this idea into the realm of algebras on invertible objects, and so for the analogous approach we need to find a new 2-functor that lets us forget about any non-invertible objects. Then hopefully we can find its left adjoint too, and use it to freely add inverses to $\mathbb{G}_n$. First though, we need to make this concept of `forgetting non-invertible objects' a little more precise.

\begin{defi} Given an $\mathrm{E}G$-algebra $X$, we denote by $X_{\mathrm{inv}}$ the sub-$\mathrm{E}G$-algebra containing all invertible objects in $X$ and the isomorphisms between them. \end{defi}

Note that this is indeed a well-defined $\mathrm{E}G$-algebra. If $x_1, ..., x_m$ are invertible objects with inverses $x_1^*, ..., x_m^*$, then $\alpha(g; x_1, ..., x_m)$ is an invertible object with inverse $\alpha(g; x_m^*, ..., x_1^*)$, since 
\begin{eq*} \begin{array}{ll}
		& \alpha(g; x_1, ..., x_m) \otimes \alpha(g; x_m^*, ..., x_1^*) \\
		= & \big( x_{\pi(g)^{-1}(1)} \otimes ... \otimes x_{\pi(g)^{-1}(m)} \big) \otimes \big( x_{\pi(g)^{-1}(m)}^* \otimes ... \otimes x_{\pi(g)^{-1}(1)}^* \big) \\
		= & I \\
		& \\
		& \alpha(g; x_m^*, ..., x_1^*) \otimes \alpha(g; x_1, ..., x_m) \\
		= & \big( x_{\pi(g)^{-1}(m)}^* \otimes ... \otimes x_{\pi(g)^{-1}(1)}^* \big) \otimes \big( x_{\pi(g)^{-1}(1)} \otimes ... \otimes x_{\pi(g)^{-1}(m)} \big) \\
		= & I
		\end{array}.
\end{eq*}
Likewise, if $f_1, ..., f_m$ are isomorphisms from invertible objects $x_1, ..., x_m$ to invertible objects $y_1, ..., y_m$, then $\alpha(g; f_1, ..., f_m)$ is a map from the invertible object $\alpha(g; x_1, ..., x_m)$ to the invertible object $\alpha(g; y_1, ..., y_m)$, and it has an inverse $\alpha(g^{-1}; f_{\pi(g)(1)}^{-1}, ..., f_{\pi(g)(m)}^{-1})$, since
\begin{eq*} \begin{array}{ll}
		& \alpha\big( \, g^{-1} \, ; \, f_{\pi(g)(1)}^{-1}, \, ..., \, f_{\pi(g)(m)}^{-1} \, \big) \circ \alpha( \, g \, ; \, f_1, ..., f_m \,) \\
		= & \alpha\big( \, g^{-1}g \, ; \, f_1^{-1} f_1, \, ..., \, f_m^{-1} f_m \, \big) \\
		= & \mathrm{id}_{x_1 \otimes ... \otimes x_m} \\
		& \\
		& \alpha( \, g \, ; \, f_1, ..., f_m \,) \circ \alpha\big( \, g^{-1} \, ; \, f_{\pi(g)(1)}^{-1}, \, ..., \, f_{\pi(g)(m)}^{-1} \, \big) \\
		= & \alpha\big( \, gg^{-1} \, ; \, f_{\pi(g)(1)} f_{\pi(g)(1)}^{-1}, \, ..., \, f_{\pi(g)(m)} f_{\pi(g)(m)}^{-1} \, \big) \\
		= & \mathrm{id}_{y_{\pi(g)(1)} \otimes ... \otimes y_{\pi(g)(m)}}
		\end{array}.
\end{eq*}
Clearly then, $X_{\mathrm{inv}}$ is the correct algebra for our new forgetful 2-functor to send $X$ to. Knowing this, we can contruct the rest of the functor fairly easily.

\begin{prop} \label{invprop} The assignment $X \mapsto X_{\mathrm{inv}}$ can be extended to a 2-functor $(\_)_{\mathrm{inv}}: \mathrm{E}G\mathrm{Alg}_S \to \mathrm{E}G\mathrm{Alg}_S$.
\end{prop}
\begin{proof}
Let $F: X \to Y$ be a map of $\mathrm{E}G$-algebras. If $x$ is an invertible object in $X$ with inverse $x^*$, then $F(x)$ is an invertible object in $Y$ with inverse $F(x^*)$, by
\begin{eq*} F(x) \otimes F(x^*) = F(x \otimes x^*) = F(I) = I \end{eq*}
\begin{eq*} F(x^*) \otimes F(x) = F(x^* \otimes x) = F(I) = I \end{eq*}
Since $F$ sends invertible objects to invertible objects, it will also send isomorphisms of invertible objects to isomorphisms of invertible objects. In other words, the map $F: X \to Y$ can be restricted to a map $F_{\mathrm{inv}} : X_{\mathrm{inv}} \to Y_{\mathrm{inv}}$. Moreover, we have that
\begin{eq*} (F \circ G)_{\mathrm{inv}}(x) = F \circ G(x) = F_{\mathrm{inv}} \circ G_{\mathrm{inv}}(x) \end{eq*}
\begin{eq*} (F \circ G)_{\mathrm{inv}}(f) = F \circ G(f) = F_{\mathrm{inv}} \circ G_{\mathrm{inv}}(f) \end{eq*}
and so the assignment $F \mapsto F_{\mathrm{inv}}$ is clearly functorial. Next, let $\theta : F \Rightarrow G$ be an $\mathrm{E}G$-monoidal natural transformation. Choose an invertible object $x$ from $X$, and consider the component map of its inverse, $\theta_{x^*} : F(x^*) \to G(x^*)$. Since $\theta$ is monoidal, we have $\theta_{x^*} \otimes \theta_x = \theta_I = I$ and $\theta_x \otimes \theta_{x^*} = I$, or in other words that $\theta_{x^*}$ is the monoidal inverse of $\theta_x$. We can use this fact to construct a compositional inverse as well, namely $\mathrm{id}_{F(x)} \otimes \theta_{x^*} \otimes \mathrm{id}_{G(x)}$, which can be seen as follows:
\begin{eq*}  \begin{array}{rll}
		\big( \mathrm{id}_{F(x)} \otimes \theta_{x^*} \otimes \mathrm{id}_{G(x)} \big)  \circ \theta_x & = & \theta_x \otimes \theta_{x^*} \otimes \mathrm{id}_{G(x)} \\
		& = &  \mathrm{id}_{G(x)} \\
		&& \\
		\theta_x \circ  \big( \mathrm{id}_{F(x)} \otimes \theta_{x^*} \otimes \mathrm{id}_{G(x)} \big) & = & \mathrm{id}_{F(x)} \otimes \theta_{x^*} \otimes \theta_x \\
		& = &  \mathrm{id}_{F(x)} \\
		\end{array} 
\end{eq*}
Therefore, we see that all the components of our transformation on invertible objects are isomorphisms, and hence we can define a new transformation $\theta_{\mathrm{inv}}: F_{\mathrm{inv}} \Rightarrow G_{\mathrm{inv}}$ whose components are just $(\theta_{\mathrm{inv}})_x = \theta_x$. The assignment $\theta \mapsto \theta_{\mathrm{inv}}$ is also clearly functorial, and thus we have a complete 2-functor $(\_)_{\mathrm{inv}}: \mathrm{E}G\mathrm{Alg}_S \to \mathrm{E}G\mathrm{Alg}_S$.
\end{proof}

\begin{prop} The 2-functor $(\_)_{\mathrm{inv}}: \mathrm{E}G\mathrm{Alg}_S \to \mathrm{E}G\mathrm{Alg}_S$ has a left adjoint, $L: \mathrm{E}G\mathrm{Alg}_S \to \mathrm{E}G\mathrm{Alg}_S$.
\end{prop}
\begin{proof} To begin, consider the 2-monad $\mathrm{E}G(\_)$. This is a finitary monad, that is it preserves all filtered colimits, and it is a 2-monad over $\mathrm{Cat}$, which is locally finitely presentable. It follows from this that $\mathrm{E}G\mathrm{Alg}_S$ is itself locally finitely presentable. Thus if we want to prove $(\_)_{\mathrm{inv}}$ has a left adjoint, we can use the Adjoint Functor Theorem for locally finitely presentable categories, which amounts to showing that $(\_)_{\mathrm{inv}}$ preserves both limits and filtered colimits.
\begin{itemize}
\item Given an indexed collection of $\mathrm{E}G$-algebras $X_i$, the $\mathrm{E}G$-action of their product $\prod X_i$ is defined componentwise. In particular, this means that the tensor product of two objects in $\prod X_i$ is just the collection of the tensor products of their components in each of the $X_i$. An invertible object in $\prod X_i$ is thus simply a family of invertible objects from the $X_i$ --- in other words, $(\prod X_i)_{\mathrm{inv}} = \prod (X_i)_{\mathrm{inv}}$.
\item Given maps of $\mathrm{E}G$-algebras $F: X \to Z$, $G : Y \to Z$, the $\mathrm{E}G$-action of their pullback $X \times_Z Y$ is also defined componentwise. It follows that an invertible object in $X \times_Z Y$ is just a pair of invertible objects $(x, y)$ from $X$ and $Y$, such that $F(x) = G(y)$. But this is the same as asking for a pair of objects $(x, y)$ from $X_{\mathrm{inv}}$ and $Y_{\mathrm{inv}}$ such that $F_{\mathrm{inv}}(x) = G_{\mathrm{inv}}(y)$, and hence $(X \times_Z Y)_{\mathrm{inv}} = X_{\mathrm{inv}} \times_{Z_{\mathrm{inv}}} Y_{\mathrm{inv}}$.
\item Given a filtered diagram $D$ of $\mathrm{E}G$-algebras, the $\mathrm{E}G$-action of their colimit $\mathrm{colim}(D)$ is defined in the following way: use filteredness to find an algebra which contains (representatives of the classes of) all the things you want to act on, then apply the action of that algebra. In the case of tensor products this means that $[x]\otimes[y] = [x \otimes y]$, and thus an invertible object in $\mathrm{colim}(D)$ is just (the class of) an invertible object in one of the algebras of $D$. In other words, $\mathrm{colim}(D)_{\mathrm{inv}} = \mathrm{colim}(D_{\mathrm{inv}})$.
\end{itemize}
Preservation of products and pullbacks gives preservation of limits, and preservation of limits and filtered colimits gives the result.
\end{proof}

With this new 2-functor $L: \mathrm{E}G\mathrm{Alg}_S \to \mathrm{E}G\mathrm{Alg}_S$, we now have the ability to `freely add inverses to objects' in any $\mathrm{E}G$-algebra we want. The algebra $L\mathbb{G}_n$ is then a clear candidate for our free algebra on $n$ invertible objects, and indeed the proof of this is very simple.

\begin{thm} There exists a free $\mathrm{E}G$-algebra on $n$ invertible objects. Specifically, the algebra $L\mathbb{G}_n$ is such that for any other $\mathrm{E}G$-algebra $X$, we have an isomorphism of categories
\begin{eq*} \mathrm{E}G\mathrm{Alg}_S(L\mathbb{G}_n, X) \cong (X_{\mathrm{inv}})^n \end{eq*}
\end{thm}
\begin{proof}
Using the adjunction from the previous Proposition along with the one from Theorem \ref{freealg}, we see that
\begin{eq*}\begin{array}{rll}
		 U(X_{\mathrm{inv}})^n & = & \mathrm{Cat}(\{1, ..., n\}, U(X_{\mathrm{inv}}) ) \\
		& \cong & \mathrm{E}G\mathrm{Alg}_S( F(\{1, ..., n\}), X_{\mathrm{inv}}) \\
		& \cong & \mathrm{E}G\mathrm{Alg}_S( LF(\{1, ..., n\}), X)
\end{array}
 \end{eq*}
As before, $X_{\mathrm{inv}}$ and $U(X_{\mathrm{inv}})$ are obviously isomorphic as categories, and so $LF(\{1, ..., n\}) = L\mathbb{G}_n$ satisfies the requirements for the free algebra on $n$ invertible objects.
\end{proof}

\subsection{$L(X)$ as an initial algebra}

We have now proven that a free $\mathrm{E}G$-algebra on $n$ invertible objects does indeed exist. But this fact on its own is not very helpful. To be able to actually use the free algebra $L\mathbb{G}_n$, we need to know how to contruct it explicitly, in terms of its objects and morphisms. We could do this by finding a detailed characterisation of the 2-functor $L$, and then applying this to our explicit description of $\mathbb{G}_n$ from Definition \ref{Gndef}. However, this would probably be much more effort than is required, since it would involve determining the behaviour of $L$ in many situtations we aren't interested in, and we also wouldn't be leveraging $\mathbb{G}_n$'s status as a free algebra to make the calculations any easier. We will try a different strategy instead. We begin by noticing some special properties of the functor $L$.

\begin{prop} \label{linveql} For any $\mathrm{E}G$-algebra $X$, we have that $L(X)_{\mathrm{inv}} = L(X)$.
\end{prop}
\begin{proof}
From the definition of adjunctions, the isomorphisms
\begin{eq*}\mathrm{E}G\mathrm{Alg}_S(LX , Y) \cong \mathrm{E}G\mathrm{Alg}_S(X, Y_{\mathrm{inv}}) \end{eq*}
are subject to certain naturality conditions. Specifically, given $F: X' \to X$ and $G: Y \to Y'$ we get a commutative diagram
\begin{eq*} \xymatrix{
\mathrm{E}G\mathrm{Alg}_S(LX , Y) \ar[d]_{G \circ \_ \circ LF} \ar[r]^{\sim} & \mathrm{E}G\mathrm{Alg}_S(X, Y_{\mathrm{inv}}) \ar[d]^{G_{\mathrm{inv}} \circ \_ \circ F} \\
\mathrm{E}G\mathrm{Alg}_S(LX' , Y') \ar[r]^{\sim} & \mathrm{E}G\mathrm{Alg}_S(X', Y'_{\mathrm{inv}}) }.
\end{eq*}
Consider the case where $F$ is the identity map $\mathrm{id}_X : X \to X$ and $G$ is the inclusion $j: L(X)_{\mathrm{inv}} \to L(X)$. Note that because $j$ is an inclusion, the restriction $j_{\mathrm{inv}}: (L(X)_{\mathrm{inv}})_{\mathrm{inv}} \to L(X)_{\mathrm{inv}}$ is also an inclusion, but since $((\_)_{\mathrm{inv}})_{\mathrm{inv}} = (\_)_{\mathrm{inv}}$, we have that $j_{\mathrm{inv}} = id$. It follows that
\begin{eq*} \xymatrix{
\mathrm{E}G\mathrm{Alg}_S(LX , LX_{\mathrm{inv}}) \ar[d]_{j \circ \_} \ar[r]^{\sim} & \mathrm{E}G\mathrm{Alg}_S(X, LX_{\mathrm{inv}}) \ar@{=}[d] \\
\mathrm{E}G\mathrm{Alg}_S(LX , LX) \ar[r]^{\sim} & \mathrm{E}G\mathrm{Alg}_S(X, LX_{\mathrm{inv}}) }.
\end{eq*}
Therefore, for any map $f: LX \to LX$ there exists a unique $g: LX \to LX_{\mathrm{inv}}$ such that $j \circ g =f$. But this means that for any such $f$, we must have $\mathrm{im}(f) \subseteq L(X)_{\mathrm{inv}}$, and so in particular $L(X) = \mathrm{im}(\mathrm{id}_{LX}) \subseteq L(X)_{\mathrm{inv}}$. Since $L(X)_{\mathrm{inv}} \subseteq L(X)$ by definition, we obtain the result.
\end{proof}

This result is not especially surprising. Intuitively, it just says that when you freely add inverses to an algebra, every object ends up with an inverse. The next result however if far less expected.

\begin{defi} Given an $\mathrm{E}G$-algebra $X$, let $C(X)$ denote the comma category $\big( \, \mathrm{id}_X \downarrow (\_)_{\mathrm{inv}} \, \big)$. That is, $C(X)$ is the coslice category whose objects are pairs of an algebra $Y$ and an algebra map $\psi: X \to Y_{\mathrm{inv}})$, and whose morphisms $f: (Y, \psi) \to (Y', \psi')$ are algebra maps $f: Y \to Y'$ such that $f_{\mathrm{inv}} \circ \psi = \psi'$. \end{defi}

\begin{prop} $L(X)$ is an initial object in $C(X)$.
\end{prop}
\begin{proof} Consider the adjuntion isomorphism
\begin{eq*}\mathrm{E}G\mathrm{Alg}_S( LX , Y) \cong \mathrm{E}G\mathrm{Alg}_S(X, Y_{\mathrm{inv}}). \end{eq*}
Let $\psi: X \to Y_{\mathrm{inv}}$ be an arbitrary object from $C(X)$, whose image under the isomorphism we will call $f : LX \to Y$. Also, call the image of $\mathrm{id}_{LX}$ under the adjunction $\phi$. As in the proof of Proposition \ref{linveql}, we consider the naturality conditions for adjunction isomorphisms. In particular, the commutative diagram we get when we choose $f$ as our $F$ and $\mathrm{id}_{X}$ as $G$ is
\begin{eq*} \xymatrix{
\mathrm{E}G\mathrm{Alg}_S(LX , LX) \ar[d]_{f \circ \_} \ar[r]^{\sim} & \mathrm{E}G\mathrm{Alg}_S(X,  LX) \ar[d]^{f_{\mathrm{inv}} \circ \_ } \\
\mathrm{E}G\mathrm{Alg}_S(LX , Y) \ar[r]^{\sim} & \mathrm{E}G\mathrm{Alg}_S(X, Y_{\mathrm{inv}}) }.
\end{eq*}
Note that this uses $(LX)_{\mathrm{inv}} = LX$, by Proposition \ref{linveql}. Now, by starting at the top-left corner of this diagram with the map $\mathrm{id}_{LX}$ and travelling either way around the diagram, we see that $f_{\mathrm{inv}} \circ \phi = \psi$. In other words, given any object $\psi$ of $C(X)$, there exists at least one map $f$ in $C(X)$ from $\phi$ onto $\psi$. Moreover this map is unique, since $f$ is the unique choice of $F$ such that the image of $F \circ id$ under the adjunction isomorphism is $\psi$. Therefore, $\phi: X \to LX$ is an initial object in $C(X)$.
\end{proof}

Being able to view $L(\mathbb{G}_n)$ is an initial object in the comma category $C(\mathbb{G}_n)$ will prove immensely useful in the coming sections. This is because it lets us think about the properties of $L(\mathbb{G}_n)$ in terms of maps $\psi: \mathbb{G}_n \to X_{\mathrm{inv}}$, which is exactly the context where we can exploit the fact that $\mathbb{G}_n$ is a free algebra.

Also, from now on rather of writing objects in $C(\mathbb{G}_n)$ as maps $\psi: \mathbb{G}_n \to Y_{\mathrm{inv}}$ we will instead just let $X = Y_{\mathrm{inv}}$ and speak of maps $\psi: \mathbb{G}_n \to X$. This is for notational convenience, and shouldn't be a problem so long as we remember that the targets of these maps will only ever contain invertible objects and morphisms.

\subsection{Objects and morphisms of the initial algebras}

We know that the functor $L$ represents the process of `freely adding inverses to objects' of a given $\mathrm{E}G$-algebra. Therefore, it makes sense to expect the objects of $L(\mathbb{G}_n)$ to form not just a monoid but a group, and in particular be the groupification of $\mathbb{G}_n$'s monoid of objects. As we saw in Definition \ref{Gndef} the objects of $\mathbb{G}_n$ are $\mathbb{N}^{*n}$, and so we want to have that $\mathrm{Ob}(L(\mathbb{G}_n)) = \mathbb{Z}^{*n}$. This intuition is correct, and can justified as follows:

\begin{prop}\label{Zobj} Let $\phi: \mathbb{G}_n \to Z$ be an initial object in $C(\mathbb{G}_n)$. Then $\mathrm{Ob}(Z) = \mathbb{Z}^{*n}$, and the restriction of $\phi$ to objects $\phi_{\mathrm{ob}}$ is the obvious inclusion $\mathbb{N}^{*n} \to \mathbb{Z}^{*n}$.
\end{prop}
\begin{proof}
To begin, we will first construct a non-initial object in $C(\mathbb{G}_n)$ which possesses all of the required properties. Let $H$ be the $\mathrm{E}G$-algebra whose objects are $\mathrm{Ob}(H) = \mathbb{Z}^{*n}$ and which has a unique morphism between each of its objects. In order to define a map of $\mathrm{E}G$-algebras $\mathbb{G}_n \to H$ we need an underlying monoid homomorphism $\mathbb{N}^{*n} \to \mathbb{Z}^{*n}$ between their objects. There is a unique $\psi: \mathbb{G}_n \to H$ where this homomorphism $\psi_{\mathrm{ob}}$ is the obvious inclusion --- the result of $\psi$ on morphisms is just determined by their source and target. Clearly this map is an object in $C(\mathbb{G}_n)$.

Now, let $\phi: \mathbb{G}_n \to Z$ be our initial object in $C(\mathbb{G}_n)$. It follows that there is a unique algebra map $u: Z \to H$ with $u\phi = \psi$, and hence a monoid homomorphism $u_{\mathrm{ob}}$ making
\begin{eq*} \xymatrix{
& \mathbb{N}^{*n} \ar[dl]_{\phi_{\mathrm{ob}}} \ar[dr]^-{\psi_{\mathrm{ob}}} & \\
\mathrm{Ob}(Z) \ar[rr]^{u_{\mathrm{ob}}} & & \mathbb{Z}^{*n} }
\end{eq*}
commute. This fact is enough to determine much of the behaviour of $u_{\mathrm{ob}}$. For any generator $z_i$ of $\mathbb{N}^{*n}$, we have $u_{\mathrm{ob}}(\phi_{\mathrm{ob}}(z_i)) = \psi_{\mathrm{ob}}(z_i) = z_i$ and $u_{\mathrm{ob}}(\phi_{\mathrm{ob}}(z_i)^*) = z_i^*$. Since the $z_i, z_i^*$ are the generators of $\mathbb{Z}^{*n}$, it follows that $u_{\mathrm{ob}}$ is surjective, even when restricted to $\langle \, \mathrm{im}(\phi_{\mathrm{ob}}) \, \rangle$. Moreover, $\psi_{\mathrm{ob}}$ is injective, so $\phi_{\mathrm{ob}}$ is also injective, and hence $u_{\mathrm{ob}}$ is injective on $\langle \, \mathrm{im}(\phi_{\mathrm{ob}}) \, \rangle$ too. In other words, $\mathrm{Ob}(Z)$ contains a submonoid $\langle \, \mathrm{im}(\phi) \, \rangle$ which is isomorphic to $\mathbb{Z}^{*n}$. 

Finally, we wish to show that this submonoid is actually all of  $\mathrm{Ob}(Z)$. Let $J$ be the $\mathrm{E}G$-algebra with objects $\mathrm{Ob}(Z) / \langle \, \mathrm{im}(\phi_{\mathrm{ob}}) \, \rangle$ and a unique morphism between each. As before, we can construct a new object $\chi : \mathbb{G}_n \to J$ of $C(\mathbb{G}_n)$ just by defining $\chi_{\mathrm{ob}}$, since the morphisms are determined by their source and target. We choose $\chi_{\mathrm{ob}}(x) = [\phi_{\mathrm{ob}}(x)]$ for all $x \in \mathbb{N}^{*n}$, which works since $\chi_{\mathrm{ob}}(z_i) = 0$ is invertible for the generator $z_i$. Now, initiality of $\phi$ should gives us a unique $v : Z \to J$ such that 
\begin{eq*} \xymatrix{
& \mathbb{N}^{*n} \ar[dl]_{\phi_{\mathrm{ob}}} \ar[dr]^-{\chi_{\mathrm{ob}}} & \\
\mathrm{Ob}(Z) \ar[rr]^{v_{\mathrm{ob}}} & & \mathrm{Ob}(Z) / \langle \, \mathrm{im}(\phi) \, \rangle }
\end{eq*}
commutes. But there are at least two options here --- one is the $v$ whose underlying monoid homomorphism is $v_{\mathrm{ob}}(x) = [x]$, and the other is the one with $v_{\mathrm{ob}}(x) = 0$. It follows that these must be the same map, and thus $ \mathrm{Ob}(Z) /  \langle \, \mathrm{im}(\phi) \, \rangle = 0$, that is, $\mathrm{Ob}(Z) =  \langle \, \mathrm{im}(\phi) \, \rangle$. Returning to the first of our diagrams, we see now that $u_{\mathrm{ob}} : \mathrm{Ob}(Z) \to \mathbb{Z}^{*n}$ is fully injective and surjective. Therefore $ \mathrm{Ob}(Z) \cong \mathbb{Z}^{*n}$, and when viewed this way $\phi_{\mathrm{ob}}$ is the obvious inclusion $\mathbb{N}^{*n} \to \mathbb{Z}^{*n}$.
\end{proof}

Unlike with objects, there is no simple intuition for how the application of $L$ will generate the morphisms of $L\mathbb{G}_n$ from those in $\mathbb{G}_n$. Obviously we will need to add a host of new action morphisms between all of the new objects, and the various composites of these. But we will also need new action maps between old objects, based on new ways of viewing them as a tensor product. For example, we must have an $\alpha(g; \mathrm{id}_x, \mathrm{id}_{x^*})$, which will be a new automorphism of $0 = x \otimes x^* = x^* \otimes x$. Should any of these new maps actually be the equal to ones inherited from $\mathbb{G}_n$? And how do the composites of these things relate to one another? It is not immediately clear. 

However, what we do know is that all of the morphisms in $\mathbb{G}_n$ can be written as action morphisms. And there is some sense in which we shouldn't expect a free $\mathrm{E}G$-algebra contruction to add any new maps that don't come from the $\mathrm{E}G$-action. This seems to suggest that $L\mathbb{G}_n$ should also only contain action morphisms. We will need to make our reasoning much more rigourous before we can can prove this though, so we start by introducing some new terminology.

\begin{defi} \label{mgd} For an element $w$ of $\mathbb{Z}^{*n}$, let the minimal generator decomposition of $w$ be the unique finite sequence $d(w) = (d(w)_1, ..., d(w)_k)$ such that
\begin{eq*} d(w)_i \in \{z_1, z_1^*, ..., z_n, z_n^* \}, \quad \bigotimes d(w)_i = w, \quad d(w)_{i+1} \neq d(w)_i^* \end{eq*}
for all $1 \leq i \leq k$.
\end{defi}

In other words, the minimal generator decomposition of an object is the shortest way of writing it as a tensor product of generators $z_i$.

\begin{defi} Let $f_i: x_i \to y_i$, $1 \leq i \leq m$, be maps in an $\mathrm{E}G$-algebra $X$. We say that the action morphism $\alpha(g; f_1, .., f_m)$ has source sequence $(x_1, ..., x_m)$ and target sequence $(y_{\pi(g)^{-1}(1)}, ..., y_{\pi(g)^{-1}(1)})$. \end{defi}

The purpose of source and target sequences is to allow us to better talk about the composition of action morphisms. Specifically, we know that if the target of $\alpha(g; \mathrm{id}_{w_1}, ..., \mathrm{id}_{w_m})$ is the same as the source of $\alpha(g'; \mathrm{id}_{w_1'}, ..., \mathrm{id}_{w_{m'}'})$ then those two maps can be composed, just like any other morphisms. But if the target sequence of $\alpha(g; \mathrm{id}_{w_1}, ..., \mathrm{id}_{w_m})$ is the same as the source sequence of $\alpha(g'; \mathrm{id}_{w_1'}, ..., \mathrm{id}_{w_{m'}'})$, then we can go one step futher and express their composite as an action morphism as well, since from the definition of $\mathrm{E}G$-actions we know that
\begin{eq*}\begin{array}{rll}
		\alpha(g'; \mathrm{id}_{w_1'}, ..., \mathrm{id}_{w_{m'}'}) \circ \alpha(g; \mathrm{id}_{w_1}, ..., \mathrm{id}_{w_m}) & = & \alpha(g'g; \mathrm{id}_{w_1}, ..., \mathrm{id}_{w_m}) \\
		\mathrm{if} \quad w_{\pi(g)^{-1}(i)} & = & w_i' \quad \forall i 
		\end{array}.
\end{eq*}
This fact will be important for figuring out what new maps will need to appear in $L\mathbb{G}_n$. To get the most use out of it though, we'll need the following result about rearranging source and target sequences.

\begin{prop}\label{zerosubseq} Let $x_1, ..., x_m$ be a sequence of objects from $\mathrm{E}G$-algebra $X$ which has a contiguous subsequence $x_i, ..., x_j$ with $x_i \otimes ... \otimes x_j = I$. Then there exists $g \in G(m)$ such that the identity $\mathrm{id}_{x_1 \otimes ... \otimes x_m}$ can be written as an action map $\alpha(g; \mathrm{id}_{x_1}, ..., \mathrm{id}_{x_m} )$ with target sequence 
\begin{eq*} ( \, x_1, \, ..., \, x_{i-1}, \, x_{j+1}, \, ..., \, x_m, \, x_i, \, ..., \, x_j \, ). \end{eq*}
\end{prop}
\begin{proof}
To begin, choose an arbitrary element $h$ of $G(2)$ whose underlying permutation is $\pi(h) = (1 2)$. Since the map $\pi$ is surjective, such an $h$ always exists. Now consider the following manipulations of action morphisms:
\begin{eq*}\begin{array}{rll}
		\alpha(h; \mathrm{id}_y, \mathrm{id}_z) & = & \alpha(h; \mathrm{id}_y \otimes \mathrm{id}_I, \mathrm{id}_z) \\
		& = & \alpha( \, h \, ; \, \alpha(e_2; \mathrm{id}_y, \mathrm{id}_I), \, \mathrm{id}_z \,) \\
		& = & \alpha( \, \mu(h; e_2, e_1) \, ; \, \mathrm{id}_y , \, \mathrm{id}_I, \, \mathrm{id}_z \, ) \\
		&& \\
		\alpha(h; \mathrm{id}_y, \mathrm{id}_z) & = & \alpha(h; \mathrm{id}_y, \mathrm{id}_z) \otimes \mathrm{id}_I \\
		& = & \alpha( \, e_2 \, ; \, \alpha(h; \mathrm{id}_y, \mathrm{id}_z), \, \mathrm{id}_I \, ) \\
		& = & \alpha( \, \mu(e_2; h, e_1) \, ; \, \mathrm{id}_y, \, \mathrm{id}_z \, \mathrm{id}_I \, )
		\end{array}.
\end{eq*}
Since these two maps are the same, we can compose one with the inverse of the other to get the identity, $\mathrm{id}_{y \otimes z}$. However, the maps both have target sequence $(z, y, I)$, so this composite can be rephrased as an action morphism:
\begin{eq*}\begin{array}{rll}
		\mathrm{id}_{y \otimes z} & = & \alpha(h; \mathrm{id}_y, \mathrm{id}_z)^{-1} \circ \alpha(h; \mathrm{id}_y, \mathrm{id}_z) \\
		& = & \alpha\big( \, \mu(e_2; h, e_1) \, ; \, \mathrm{id}_y, \, \mathrm{id}_z, \, \mathrm{id}_I \, \big)^{-1} \circ \alpha\big( \, \mu(h; e_2, e_1) \, ; \, \mathrm{id}_y , \, \mathrm{id}_I, \, \mathrm{id}_z \, \big) \\
		& = & \alpha\big( \, \mu(e_2; h, e_1)^{-1} \, ; \, \mathrm{id}_z, \, \mathrm{id}_y, \, \mathrm{id}_I \, \big) \circ \alpha\big( \, \mu(h; e_2, e_1) \, ; \, \mathrm{id}_y , \, \mathrm{id}_I, \, \mathrm{id}_z \, \big) \\
		& = & \alpha\big( \, \mu(e_2; h^{-1}, e_1)\mu(h; e_2, e_1) \, ; \, \mathrm{id}_y, \, \mathrm{id}_I, \, \mathrm{id}_z \, \big) \\
		\end{array}.
\end{eq*}
This action map has target sequence $(y, z, I)$, so to arrive at the result we just need to use the substitutions $y = x_1 \otimes ... \otimes x_{i-1}$, $I = x_i \otimes ... \otimes x_j$, $z = x_{j+1} \otimes ... \otimes x_m$ and then expand:
\begin{eq*}\begin{array}{rl}
		& \mathrm{id}_{x_1 \otimes ... \otimes x_m} \\
		= & \alpha\big( \, \mu(e_2; h^{-1}, e_1)\mu(h; e_2, e_1) \, ; \, \mathrm{id}_{x_1 \otimes ... \otimes x_{i-1}} , \, \mathrm{id}_{x_i \otimes ... \otimes x_j}, \, \mathrm{id}_{ x_{j+1} \otimes ... \otimes x_m} \, \big) \\ 
		= & \alpha\Big( \, \mu\big( \, \mu(e_2; h^{-1}, e_1)\mu(h; e_2, e_1) \, ; \, e_{i-1}, \, e_{j-i+1}, \, e_{m-j} \big) \, ; \, \mathrm{id}_{x_1}, \, ..., \, \mathrm{id}_{x_m} \, \Big)

		\end{array}.
\end{eq*}
Therefore we see that choosing $g = \mu\big( \, \mu(e_2; h^{-1}, e_1)\mu(h; e_2, e_1) \, ; \, e_{i-1}, \, e_{j-i+1}, \, e_{m-j} \big)$ gives the required action map.
\end{proof}

With this result under our belts, we are finally ready to describe the morphisms of $L\mathbb{G}_n$ as action maps.

\begin{prop}\label{allmapsaction} Let $Z$ be an initial object in $C(\mathbb{G}_n)$. Then every morphism in $Z$ can be written as $\alpha(g; \mathrm{id}_{w_1}, ..., \mathrm{id}_{w_m})$, for some $g \in G(m)$ and $w_i \in \mathbb{Z}^{*n}$, not necessarily uniquely. 
\end{prop}
\begin{proof} Define $Z'$ to be the wide sub-$\mathrm{E}G$-algebra of $Z$ containing only morphisms of the form $\alpha(g; \mathrm{id}_{w_1}, ..., \mathrm{id}_{w_m})$. We wish to show that this is also initial, but first we need to check that it is even a well-defined $\mathrm{E}G$-algebra. The $\mathrm{E}G$-action is simple: for any maps $\alpha(h_1; \mathrm{id}_{w_{1,1}}, ..., \mathrm{id}_{w_{1,m_1}})$, ..., $\alpha(h_k; \mathrm{id}_{w_{k,1}}, ..., \mathrm{id}_{w_{k,m_k}})$, the action of $g \in G(k)$ on them is
\begin{eq*}\begin{array}{ll}
		& \alpha \big( \, g \, ; \,  \alpha(h_1; \mathrm{id}_{w_{1,1}}, ..., \mathrm{id}_{w_{1,m_1}}), \, ..., \, \alpha(h_k; \mathrm{id}_{w_{k,1}}, ..., \mathrm{id}_{w_{k,m_k}}) \, \big) \\
		= & \alpha \big( \, \mu(g; h_1, ..., h_k) \, ; \, \mathrm{id}_{w_{1,1}}, \, ..., \, \mathrm{id}_{w_{k,m_k}} \, \big)
		\end{array}.
\end{eq*}
which is in the correct form.

Composition is more subtle. Let $\alpha(g; \mathrm{id}_{w_1}, ..., \mathrm{id}_{w_m})$ and $\alpha(g'; \mathrm{id}_{w_1'}, ..., \mathrm{id}_{w_{m'}'})$ be two composable morphisms. Since they are composable, we know that the source of one must be equal to the target of the other. However, we wish to write their composite an action map itself, and we only know how to do this if they share a source and target sequence. Thus we seek a way to rewrite our two maps as action morphisms between different sequences but without changing their value. We begin by expanding the maps using minimal generator decompositions:
\begin{eq*}\begin{array}{rll}
		\alpha(g; \mathrm{id}_{w_1}, ..., \mathrm{id}_{w_m}) & = & \alpha( \, g \, ; \, \mathrm{id}_{d(w_1)_1 \otimes ... \otimes d(w_1)_{k_1}}, \, ..., \, \mathrm{id}_{d(w_m)_1 \otimes ... \otimes d(w_m)_{k_m}}) \\
		& = & \alpha( \, \mu(g; e_{k_1}, ..., e_{k_m}) \, ; \, \mathrm{id}_{d(w_1)_1} \, ..., \, \mathrm{id}_{d(w_m)_{k_m}}) \\
		& & \\
		\alpha(g'; \mathrm{id}_{w_1'}, ..., \mathrm{id}_{w_{m'}'}) & = & \alpha( \, \mu(g; e_{k_1'}, ..., e_{k_{m'}'}) \, ; \, \mathrm{id}_{d(w_1')_1} \, ..., \, \mathrm{id}_{d(w_{m'}')_{k_{m'}'}} )
		\end{array}.
\end{eq*}
Now, the target sequence of this first map and the source sequence of the second may contain different entries and may even be of different lengths. However, since our two maps are composable, we do know that 
\begin{eq*} w'_1 \otimes ... \otimes w'_m = w_{\pi(g)^{-1}(1)} \otimes ... \otimes w_{\pi(g)^{-1}(m')} \end{eq*}
and hence
\begin{eq*} d(w'_1) \otimes ... \otimes d(w'_m) = d(w_{\pi(g)^{-1}(1)}) \otimes ... \otimes d(w_{\pi(g)^{-1}(m')}). \end{eq*}
The fact that everything in these tensorings is a generator means that the sequences must differ from each other only by the presence of some contiguous subsequences which tensor to give 0, so that those parts cancel out and the products are equal. But Proposition \ref{zerosubseq} gives us a way to move contiguous tensor 0 subsequences to the end of a source or target sequence without changing the value of the map --- by composing with the identity written as a certain action morphism --- and we can use this to solve our problem. In particular, let $(c_1, ..., c_j)$ be the concatenation of the contiguous tensor 0 subequences that appear in the target sequence $d(w_{\pi(g)^{-1}(i)})$, let $(c_1', ..., c_{j'}')$ be the concatenation of contiguous tensor 0 subequences appearing in target sequence $d(w'_i)$, and let $(y_1, ...., y_{m-j})$ be the sequence that remains after removing either the $c_i$ or the $c_i'$ from their respective sequences. Applying Proposition \ref{zerosubseq} to $\alpha( \, \mu(g; e_{k_1}, ..., e_{k_m}) \, ; \, \mathrm{id}_{d(w_1)_1} \, ..., \, \mathrm{id}_{d(w_m)_{k_m}})$ lets us express that map as an action morphism on identities with target sequence $(y_1, ..., y_{m-j}, c_1, ..., c_j)$, and then tensoring on the right by $\mathrm{id}_0 = \mathrm{id}_{c_1' \otimes ... \otimes c_{j'}}$ lets us re-express the same map again in the right form but with target sequence $(y_1, ..., y_{m-j}, c_1, ..., c_j, c_1', ..., c_{j'}')$. Similarly, tensoring $\alpha( \, \mu(g; e_{k_1'}, ..., e_{k_{m'}'}) \, ; \, \mathrm{id}_{d(w_1')_1} \, ..., \, \mathrm{id}_{d(w_{m'}')_{k_{m'}'}} )$ by $\mathrm{id}_0 = \mathrm{id}_{c_1 \otimes ... \otimes c_j}$ and then applying Proposition \ref{zerosubseq} lets us express this map in a form with source sequence $(y_1, ..., y_{m-j}, c_1, ..., c_j, c_1', ..., c_{j'}')$. 

It follows that any composable action morphisms $\alpha(g; \mathrm{id}_{w_1}, ..., \mathrm{id}_{w_m})$, $\alpha(g'; \mathrm{id}_{w_1'}, ..., \mathrm{id}_{w_{m'}'})$ can be rewritten as action morphisms with a shared source and target sequence, and thus their composite can be expressed as a single action morphism of the right form. Therefore, composition is well-defined in $Z'$, and hence $Z'$ is a well-defined $\mathrm{E}G$-algebra.

By Definition \ref{Gndef}, every morphism in $\mathbb{G}_n$ can be written uniquely as $[g ; \mathrm{id}_{x_1},...,\mathrm{id}_{x_m}] = \alpha(g;  \mathrm{id}_{x_1},...,\mathrm{id}_{x_m})$, for some $g \in G(m)$ and $x_1, ..., x_m \in \{ z_1, ..., z_n \}$ generators of $\mathbb{N}^{*n}$. Using this, we can define a map $\phi' : \mathbb{G}_n \to Z'$ which acts as $\phi$ does on objects and on morphisms by
\begin{eq*} \phi(\alpha(g ; \mathrm{id}_{x_1},...,\mathrm{id}_{x_m})) = \alpha(g ; \mathrm{id}_{\phi(x_1)},...,\mathrm{id}_{\phi(x_m)}) \end{eq*}
Now, since $Z$ is initial in $C(\mathbb{G}_n)$ each object $\psi : \mathbb{G}_n \to H$ has a corresponding map $u : Z \to H$ such that $u \phi = \psi$. We can use this to define a new map $u' : Z' \to H$ with $u' \phi' = \psi$ by simply letting $u'$ be the same as $u$ on objects and setting
\begin{eq*} u'(\alpha(g; \mathrm{id}_{w_1}, ..., \mathrm{id}_{w_m})) = \alpha(g; \mathrm{id}_{u'(w_1)}, ..., \mathrm{id}_{u'(w_m)}) \end{eq*}
However, this condition is a necessary part of $u'$ being a map of $\mathrm{E}G$-algebras, so we really had no choice about what to do with the morphisms. Thus $u'$ is the unique map such that $u' \phi' = \psi$, and hence $Z'$ is an initial object in $C(\mathbb{G}_n)$. But $Z'$ was subalgebra of $Z$, also an initial object, and this is only possible if in fact $Z' = Z$.
\end{proof}

Before moving on, notice that this result lets us immediately classify the connected components of $L\mathbb{G}_n$:

\begin{prop}\label{concomp} The connected components of $\mathbb{G}_n$ are $\mathbb{N}^n$, with the assignment $[ \,\, ] : \mathbb{N}^{*n} \to \mathbb{N}^n$ of objects to their component being the quotient map of abelianisation. Also, if $Z$ is an initial object in $C(\mathbb{G}_n)$ then the connected components of $Z$ are $\mathbb{Z}^n$, with its assignment of objects to components also given by abelianisation, and with the restriction of $\phi$ to components $\phi_\pi : \mathbb{N}^n \to \mathbb{Z}^n$ being the obvious inclusion. 
\end{prop}
\begin{proof}All morphisms in $\mathbb{G}_n$ are of the form $\alpha(g; \mathrm{id}_{w_1}, ..., \mathrm{id}_{w_m})$ for some $g \in G(m)$ and $w_i \in \mathbb{N}^{*n}$. Since these have source $w_1 \otimes ... \otimes w_m$ and target $w_{\pi(g)^{-1}(1)} \otimes ... \otimes w_{\pi(g)^{-1}(m)}$, we see that two objects can be in the same connected component only if they can expanded as a tensor product in ways that are permutations of one another. Moreover, for any two objects where this is true --- say source $w = w_1 \otimes ... \otimes w_m$ and target $w' = w_{\sigma^{-1}(1)} \otimes ... \otimes w_{\sigma^{-1}(m)}$ --- we can always find a map $\alpha(g; \mathrm{id}_{w_1}, ..., \mathrm{id}_{w_m})$  between them by choosing any $g$ with $\pi(g) = \sigma$, which we can do because $\pi$ is surjective. So two objects of $\mathbb{G}_n$ are in the same connected component if and only if their expansions are permutations of each others. Therefore, the canonical map $[ \,\, ] : \mathrm{Ob}(\mathbb{G}_n) \to \pi_0(\mathbb{G}_n)$ sending each object to its connected component is just the map which forgets about these permutations, making the free product on $\mathbb{N}^{*n}$ commutative. That is, it is the quotient map for the abelianisation $q : \mathbb{N}^{*n} \to (\mathbb{N}^{*n})^{ab}$, and so $\pi_0(\mathbb{G}_n) = \mathbb{N}^n$. 

Since all morphisms in $Z$ can also be written as $\alpha(g; \mathrm{id}_{w_1}, ..., \mathrm{id}_{w_m})$, the same proof works there too, giving $\pi_0(Z) = \mathbb{Z}^n$ and $[ \,\, ]_Z = q : \mathbb{Z}^{*n} \to \mathbb{Z}^n$. Also, by Proposition \ref{Zobj} $\phi$ acts as an inclusion on objects, so we have the following commutative square:
\begin{eq*} \xymatrix{
\mathbb{N}^{*n} \ar[r]^{q} \ar[d]_{i} & \mathbb{N}^{n} \ar[d]^{\phi_\pi} \\
\mathbb{Z}^{*n} \ar[r]^{q} & \mathbb{Z}^{n} }.
\end{eq*}
Thus $\phi_\pi(q(x)) = q(x)$, and because $q$ is surjective this means that $\phi_\pi$ is an inclusion.
\end{proof}

\subsection{Refinement of $C(\mathbb{G}_n)$}

In the previous section, we learnt some important new features of the initial objects of $C(\mathbb{G}_n)$. Using this information, we can carefully make changes to the category we are finding initial objects in, hopefully in a way which will make the search for $L(\mathbb{G}_n)$ easier.

\begin{defi} Let $C_{\mathrm{i}}(\mathbb{G}_n)$ denote the subcategory of $C(\mathbb{G}_n)$ consisting of the objects $\psi: \mathbb{G}_n \to X$ and morphisms $f: X \to Y$ which have all of the following properties:
\begin{itemize}
\item injective on objects
\item injective on connected components
\item the objects in the image generate the group of objects of the target
\end{itemize}
\end{defi}

\begin{prop} Let $\phi : \mathbb{G}_n \to Z$ be an initial object of $C(\mathbb{G}_n)$. Then it is also initial in $C_{\mathrm{i}}(\mathbb{G}_n)$.
\end{prop}
\begin{proof}
By Proposition \ref{Zobj} the restriction of $\phi$ to objects is just the inclusion $\phi_{\mathrm{ob}} : \mathbb{N}^{\ast n} \to \mathbb{Z}^{\ast n}$, and by Proposition \ref{concomp} its restriction to connected components is the inclusion $\phi_\pi : \mathbb{N}^n \to \mathbb{Z}^n$. These are both clearly injective, and $\langle \, \mathrm{im}(\phi_{\mathrm{ob}}) \, \rangle =  \langle \, \mathbb{N}^{\ast n} \, \rangle = \mathbb{Z}^{\ast n}$, so $\phi$ is definitely in $C_{\mathrm{i}}(\mathbb{G}_n)$. 

For any other object $\psi: \mathbb{G}_n \to X$ of $C_{\mathrm{i}}(\mathbb{G}_n)$, we know that it is also an object of $C(\mathbb{G}_n)$, and so the initiality condition for $\phi$ tells us that there exists a unique $u : Z \to X$ in $C(\mathbb{G}_n)$ such that $\psi = u \phi$. But as we saw in the proof of Proposition \ref{Zobj}, the fact that $\phi_{\mathrm{ob}}$ and $\psi_{\mathrm{ob}}$ are injective tells us that $u_{\mathrm{ob}}$ is injective when restriced to $\langle \, \mathrm{im}(\phi_{\mathrm{ob}}) \, \rangle = \mathbb{Z}^{\ast n}$. Since we now know that this is just the entire source of $u_{\mathrm{ob}}$, it follows that $u$ is injective on objects, and the same line of reasoning but applied to $\phi_\pi$ and $\psi_\pi$ instead says that $u$ is injective on connected components as well. Lastly, we have
\begin{eq*} \langle \, \mathrm{im}(u_{\mathrm{ob}}) \, \rangle \, \supseteq \, \langle \, \mathrm{im}(u_{\mathrm{ob}}\phi_{\mathrm{ob}}) \, \rangle = \, \langle \, \mathrm{im}(\psi_{\mathrm{ob}}) \, \rangle = \mathrm{Ob}(X) \end{eq*}
and hence $u$ is a valid morphism from $\phi$ to $\psi$ in $C_{\mathrm{i}}(\mathbb{G}_n)$. Clearly there can be no other such maps, as that would violate the initiality of $\phi$ in $C(\mathbb{G}_n)$, and therefore $\phi$ is also an initial object in $C_{\mathrm{i}}(\mathbb{G}_n)$.
\end{proof}

By excluding several kinds of objects and morphisms from $C_{\mathrm{i}}(\mathbb{G}_n)$, we've effectively reduced the number of commutative diagrams that need to be checked when trying to find the initial object. This should make our task easier, but before moving on we will make one more refinement to our category. Consider that every $\mathrm{E}G$-algebra $X$ has an underlying monoidal category, the one with tensor product given by
\begin{eq*}\begin{array}{rll}
		\alpha(g; x_1, ..., x_m) & = & x_{\pi(g)^{-1}(1)} \otimes ... \otimes x_{\pi(g)^{-1}(m)} \\
		\alpha(e; f_1, ..., f_m) & = & f_1 \otimes ... \otimes f_m
		\end{array}
\end{eq*}
for objects $x_i$ and morphisms $f_i$ in $X$. Notice that the $\mathrm{E}G$-action on morphisms can always be split as
\begin{eq*} \alpha(g; f_1, ..., f_m) = \alpha(g; \mathrm{id}_{x_1}, ..., \mathrm{id}_{x_m}) \circ \alpha(e; f_1, ..., f_m) \end{eq*}
and so to recover $X$ from its underlying monoidal category the only addition information we need is the values of actions of the form $\alpha(g; \mathrm{id}_{x_1}, ..., \mathrm{id}_{x_m})$. Likewise, since a map $F : X \to Y$ between $\mathrm{E}G$-algebras is just a functor which preserves the action, all that we need to recover $F$ from its underlying monoidal functor is the actions on identities for its source and target. We can apply this fact to prove the following useful proposition:

\begin{prop} Let $C_{\mathrm{i}, \mathrm{m}}(\mathbb{G}_n)$ be the category whose objects and morphisms are the underlying monoidal functors of the objects and morphisms of $C_{\mathrm{i}}(\mathbb{G}_n)$. Then the obvious functor
\begin{eq*} C_{\mathrm{i}}(\mathbb{G}_n) \to C_{\mathrm{i}, \mathrm{m}}(\mathbb{G}_n) \end{eq*}
sending $\mathrm{E}G$-algebra maps to their underlying monoidal functors is an isomorphism of categories.  
\end{prop}
\begin{proof}
Let $\psi: \mathbb{G}_n \to X$ an arbitrary object of $C_{\mathrm{i}}(\mathbb{G}_n)$, and suppose that $\hat{\psi}: \mathbb{G}_n \to \hat{X}$ is another object that has the same underlying monoidal functor as $\psi$. As noted before this proposition, $\hat{\psi}$ can differ from $\psi$ only in the values for actions of the form $\alpha(g; \mathrm{id}, ..., \mathrm{id})$ on their source and target. Moreover, since objects of $C_{\mathrm{i}}(\mathbb{G}_n)$ are always maps out of $\mathbb{G}_n$ with its canonical $\mathrm{E}G$-algebra strucutre, we actually only need such values for $\hat{X}$.

One of the conditions these $\alpha_{\hat{X}}(g; \mathrm{id}, ..., \mathrm{id})$ must satisfy is
\begin{eq*} \hat{\psi} \big( \, \alpha_{\mathbb{G}_n}(g; \mathrm{id}_{w_1}, ..., \mathrm{id}_{w_m}) \, \big) \, = \, \alpha_{\hat{X}}(g; \mathrm{id}_{\hat{\psi}(w_1)}, ..., \mathrm{id}_{\hat{\psi}(w_m)}) \end{eq*}
for any objects $w_i$ in $\mathbb{G}_n$. Notice that given a collection of objects $x_i$ in the image of $\hat{\psi}$, this equation can be taken to be definitional --- if we choose a collection of $w_i$ in $\mathbb{G}_n$ with $\hat{\psi}(w_i) = x_i$, then $\alpha_{\hat{X}}(g; \mathrm{id}_{x_1}, ..., \mathrm{id}_{x_m})$ must then have the value $\hat{\psi}(\alpha_{\mathbb{G}_n}(g; \mathrm{id}_{w_1}, ..., \mathrm{id}_{w_m}))$. That is, the behaviour of $\alpha_{\hat{X}}$ on $\mathrm{im}(\hat{\psi})$ is fixed by the underlying monoidal functor of $\hat{\psi}$; as $\psi$ shares this functor, we also have that
\begin{eq*}\begin{array}{rll}
		\alpha_{\hat{X}}(g; \mathrm{id}_{\hat{\psi}(w_1)}, ..., \mathrm{id}_{\hat{\psi}(w_m)}) & = & \hat{\psi} \big( \, \alpha_{\mathbb{G}_n}(g; \mathrm{id}_{w_1}, ..., \mathrm{id}_{w_m}) \, \big) \\
		& = & \alpha_X(g; \mathrm{id}_{\psi(w_1)}, ..., \mathrm{id}_{\psi(w_m)})
		\end{array}
\end{eq*}

In order to understand the behaviour of $\alpha_{\hat{X}}$ outside of $\mathrm{im}(\hat{\psi})$, consider the algebra $\mathbb{G}_{2n}$. The objects of this category are the monoid $\mathbb{N}^{\ast 2n}$, with generators that we've been calling $z_1, ..., z_{2n}$. It follows that we can always construct a new map of $\mathrm{E}G$-algebras, $\hat{\psi}_2: \mathbb{G}_{2n} \to \hat{X}$, out of $\hat{\psi}$ by simply setting
\begin{eq*}
\hat{\psi}_2(z_i) =
\begin{cases}
       	\hat{\psi}(z_i) & \quad \text{if} \quad 1 \leq i \leq n \\
      	\hat{\psi}(z_{i-n})^* & \quad \text{if} \quad n+1 \leq i \leq 2n \\
\end{cases}
\end{eq*}
\begin{eq*} \hat{\psi}_2 \big( \, \alpha_{\mathbb{G}_{2n}}(g; \mathrm{id}_{w_1}, ..., \mathrm{id}_{w_m}) \, \big) \, = \, \alpha_{\hat{X}}(g; \mathrm{id}_{\hat{\psi}_2(w_1)}, ..., \mathrm{id}_{\hat{\psi}_2(w_m)}) \end{eq*}
The results of applying $\hat{\psi}_2$ to other objects of $\mathbb{G}_{2n}$ is determined from the values on generators by the need for $\hat{\psi}_2(w \otimes w') = \hat{\psi}_2(w) \otimes \hat{\psi}_2(w')$ for all objects $w, w'$. Since every morphism of $\mathbb{G}_{2n}$ can be written in the form $\alpha(g; \mathrm{id}_{w_1}, ..., \mathrm{id}_{w_m})$, not necessarily uniquely, the result of applying $\hat{\psi}_2$ to morphisms is also fully determined. The last condition also ensures that $\hat{\psi}_2$ is a well-defined $\mathrm{E}G$-algebra map, and so clearly $\hat{\psi}_2$ is an object of $C(\mathbb{G}_{2n})$. We know that this category has an initial object, $\phi: \mathbb{G}_{2n} \to Z_{2n}$, and hence there is a unique map $u: Z_{2n} \to \hat{X}$ such that $\hat{\psi}_2 = u \phi$. Remembering that $\phi$ has the properties
\begin{eq*}\phi(w) = w, \quad \quad \phi \big( \, \alpha_{\mathbb{G}_{2n}}(g; \mathrm{id}_{w_1}, ..., \mathrm{id}_{w_m}) \, \big) \, = \, \alpha_{Z_{2n}}(g; \mathrm{id}_{w_1}, ..., \mathrm{id}_{w_m}) \end{eq*}
we can conclude that $u$ satisfies
\begin{eq*}
u(z_i) =
\begin{cases}
       	\hat{\psi}(z_i) & \quad \text{if} \quad 1 \leq i \leq n \\
      	\hat{\psi}(z_{i-n})^* & \quad \text{if} \quad n+1 \leq i \leq 2n \\
\end{cases}
\end{eq*}
\begin{eq*} u \big( \, \alpha_{Z_{2n}}(g; \mathrm{id}_{w_1}, ..., \mathrm{id}_{w_m}) \, \big) \, = \, \alpha_{\hat{X}}(g; \mathrm{id}_{u(w_1)}, ..., \mathrm{id}_{u(w_m)}) \end{eq*}
Like before, we can use this last condition to fix the way that $\alpha_{\hat{X}}$ must act on the image of $u$, using only knowledge of the underlying monoidal functor of $u$. However, this time $\mathrm{im}(u)$ contains every object of $\hat{X}$, since $u_{\mathrm{ob}}$ is a group homomophism and so $\mathrm{im}(u_{\mathrm{ob}}) = \langle \, \mathrm{im}(u_{\mathrm{ob}}) \, \rangle = \mathrm{Ob}(\hat{X})$. Thus, for any collection of objects $x_i$ in the underlying categroy of $X$ we can always find some $w_i$ in $Z_{2n}$ with $u(w_i) = x_i$, meaning that
\begin{eq*}\begin{array}{rll}
		\alpha_{\hat{X}}(g; \mathrm{id}_{x_1}, ..., \mathrm{id}_{x_m}) & = & u \big( \, \alpha_{Z_{2n}}(g; \mathrm{id}_{w_1}, ..., \mathrm{id}_{w_m}) \, \big) \\
		& = & \alpha_{X}(g; \mathrm{id}_{x_1}, ..., \mathrm{id}_{x_m}) 
		\end{array}
\end{eq*}
It follows that $X$ and $\hat{X}$ are really the same $\mathrm{E}G$-algebra, and $\psi$ and $\hat{\psi}$ the same $\mathrm{E}G$-algebra map, and therefore the $\mathrm{E}G$-structure of an object of $C_{\mathrm{i}}(\mathbb{G}_n)$ is completely determined by its underlying monoidal functor.

As for the morphisms of $C_{\mathrm{i}}(\mathbb{G}_n)$, let $f: X \to Y$ be a morphism from $\psi: \mathbb{G}_n \to X$ to $\chi: \mathbb{G}_n \to Y$. Given the underlying monoidal functors of $\psi$ and $\chi$, the actions on $X$ and $Y$ are fixed, and this together with the underlying monoidal functor of $f$ is enough to determine $F$ as a map of $\mathrm{E}G$-algebras. Thus, the process of taking underlying monoidal functors of object and morphisms of $C_{\mathrm{i}}(\mathbb{G}_n)$ is one-to-one, and so $C_{\mathrm{i}}(\mathbb{G}_n) \cong C_{\mathrm{i}, \mathrm{m}}(\mathbb{G}_n)$.
\end{proof}

This result has two very important consequences. The first is that the underlying monoidal functors of the initial objects of $C_{\mathrm{i}}(\mathbb{G}_n)$ must be the initial objects of $C_{\mathrm{i}, \mathrm{m}}(\mathbb{G}_n)$. This is immediate from their isomorphism as categories, and will allows us to further use $C_{\mathrm{i}, \mathrm{m}}(\mathbb{G}_n)$ as a refinement of $C(\mathbb{G}_n)$. The second result is a more practical consideration.

\begin{cor} It is possible to recontruct the entirety of $C_{\mathrm{i}}(\mathbb{G}_n)$ from just the objects and morphisms of $C_{\mathrm{i}, \mathrm{m}}(\mathbb{G}_{n})$ and $C_{\mathrm{i}, \mathrm{m}}(\mathbb{G}_{2n})$.
\end{cor}
\begin{proof}
When we were defining the value of $\alpha_{\hat{X}}(g; \mathrm{id}_{x_1}, ..., \mathrm{id}_{x_m})$ at the end of the previous proof, we not only used the underlying monoidal functor of $u: Z_{2n} \to X$, which is a morphism in $C_{\mathrm{i}, \mathrm{m}}(\mathbb{G}_{2n})$, but also the value of $\alpha_{Z_{2n}}(g; \mathrm{id}_{w_1}, ..., \mathrm{id}_{w_m})$, something not included in the data of any $C_{\mathrm{i}, \mathrm{m}}$. However, notice that the way we defined $\hat{\psi}_2$ implies that
\begin{eq*} \mathrm{Ob}(X) = \langle \, \mathrm{im}(\psi_{\mathrm{ob}}) \, \rangle = \mathrm{im}\big( \, (\hat{\psi}_2)_{\mathrm{ob}} \, \big) = \mathrm{im}(u_{\mathrm{ob}} \phi_{\mathrm{ob}}) = u \big( \, \mathrm{im}(\phi_{\mathrm{ob}}) \, \big) \end{eq*}
Hence, when we had to pick some $w_i$ such that $u(w_i) = x_i$, we were always free to choose $w_i$ exclusively from $\mathrm{im}(\phi)$. 

But we saw earlier in the previous proof that given any object $\psi: \mathbb{G}_{n} \to X$ of $C_{\mathrm{i}}(\mathbb{G}_n)$, we can define the values of the action $\alpha_X$ on $\mathrm{im}(\psi)$ so long as we know the underlying monoidal functor of $\psi$. Applying this logic to $\phi: \mathbb{G}_{2n} \to Z_{2n}$, it follows that we can get all of the values of $\alpha_{Z_{2n}}(g; \mathrm{id}_{w_1}, ..., \mathrm{id}_{w_m})$ we need just by knowing the underlying functor of $\phi$, which is just an object of $C_{\mathrm{i}, \mathrm{m}}(\mathbb{G}_{2n})$. Therefore, we can fully recontruct an object $\psi: \mathbb{G}_n \to X$ of $C_{\mathrm{i}}(\mathbb{G}_n)$ from its underlying monoidal functor in $C_{\mathrm{i}, \mathrm{m}}(\mathbb{G}_n)$ and some extra data from $C_{\mathrm{i}, \mathrm{m}}(\mathbb{G}_{2n})$, and likewise with morphisms.
\end{proof}

\subsection{Initial algebras as groupoids}

Recall that objects $X$ in $C_{\mathrm{i}, \mathrm{m}}(\mathbb{G}_n)$ are always expressible as $Y_{\mathrm{inv}}$ for some monoidal category $Y$. From this fact, along with the definition of $\mathbb{G}_n$, it is clear that all of the structures we will be working with are not just categories, but groupoids. This is very convenient, as groupoids are often much simpler than general categories, and we can exploit this simplicity to help find initial algebras more easily. In particular, we will use the fact that for any connected component of a groupoid, all of the homsets between its objects are isomorphic. This will allow us to describe any groupoid $X$ by splitting it into two smaller pieces --- one encoding information about the objects of $X$, and the other about the morphisms of the components $X$. First, we'll look at the part which details the morphisms.

\begin{defi} Let $X$ be a monoidal groupoid. A skeleton of $X$ is a full subcategory of $X$ which contains exactly one object from each of the connected components of $X$. \end{defi}

It is well known that the canonical inclusion of a skeleton, $i: \mathrm{sk}(X) \to X$, is part of an equivalence between the two. Indeed, the skeleton of a $X$ is the smallest subcategory which is still equivalent to $X$. We shall denote weak inverse of $i$ by $R: X \to \mathrm{sk}(X)$. This inverse is actually always strict in one direction --- $R \circ i = \mathrm{id}_{\mathrm{sk}(X)}$. In the other direction we just have a natural isomorphism, which we will call $\rho: i \circ R \Rightarrow \mathrm{id}_X$. 

We will think about the objects of a particular skeleton as being a chosen set of `representatives' for each of the connected components. Under this interpretation, the functor $R$ sends each object $x$ to its isomorphism class' representative $R(x)$, and $\rho_x : R(x) \to x$ is an isomorphism which witnesses that they are indeed in the same class. Notice also that the naturality of $\rho$ serves to define the action of $R$ on morphisms:
\begin{eq*} R( \, f: x \to y \,) \, = \, \rho_y^{-1} \circ f \circ \rho_x \end{eq*}

Since $X$ is a (strict) monoidal category, the equivalence $(i, R, \rho, \mathrm{id})$ naturally induces on $\mathrm{sk}(X)$ the structure of a weak monoidal category, with product $\boxtimes$ defined by
\begin{eq*} \xymatrix{
\mathrm{sk}(X) \times \mathrm{sk}(X) \ar[d]_{i \times i} \ar[rr]^{\boxtimes} & & \mathrm{sk}(X) \\
X \times X \ar[rr]^{\otimes} & & X \ar[u]_{R} }.
\end{eq*}
This new product has unit object $R(I)$ and coherence data
\begin{eq*} \xymatrix{
(x \boxtimes y) \boxtimes z \ar[d]_{\rho_{(x \boxtimes y) \otimes z}} \ar[rrrr]^{a_{x,y,z}} & & & & x \boxtimes (y \boxtimes z) \\
(x \boxtimes y) \otimes z \ar[rr]^{\rho_{x \otimes y} \otimes \mathrm{id}_z} & & x \otimes y \otimes z \ar[rr]^{\mathrm{id}_x \otimes \rho_{y \otimes z}^{-1} } & & x \otimes (y \boxtimes z) \ar[u]_{\rho_{x \otimes (y \boxtimes z)}^{-1}} }
\end{eq*}
\begin{eq*} \xymatrix{
R(I) \boxtimes x \ar[d]_{\rho_{R(I) \otimes x}} \ar[rr]^{l_x} & & x & & x \boxtimes R(I) \ar[d]_{\rho_{x \otimes R(I)}} \ar[rr]^{r_x} & & x \\
R(I) \otimes x \ar[rr]^{\rho_I \, \otimes \, \mathrm{id}_x} & & I \otimes x \ar@{=}[u] & & x \otimes R(I) \ar[rr]^{\mathrm{id}_x \, \otimes \, \rho_I} & & x \otimes I \ar@{=}[u] }
\end{eq*}
It is also important to note that even though $\mathrm{sk}(X)$ is not strictly monoidal, the fact that
\begin{eq*} R(r \otimes s) \otimes t \quad \cong \quad r \otimes s \otimes t \quad \cong \quad r \otimes R(s \otimes t), \end{eq*}
causes $\boxtimes$ to be strictly associative on objects:
\begin{eq*} \begin{array}{rll}
 \quad (r \boxtimes s) \boxtimes t & = & R\big( R(r \otimes s) \otimes t \big) \\
		& = &  R\big( r \otimes R(s \otimes t) \big) \\
		& = & r \boxtimes (s \boxtimes t)
		\end{array}
\end{eq*}

Finally, because $\mathrm{sk}(X)$ contains one object for each of the elements of $\pi_0(X)$ there is an obvious functor, isomorphic on objects, between the two:
\begin{eq*} \begin{array}{rrrll}
		[ \, \_ \, ] & : & \mathrm{sk}(X) & \to & \pi_0(X) \\
		& : & x & \mapsto & [x] \\
		& : & f: x \to x & \mapsto & id_{[x]}
		\end{array}
\end{eq*}
This functor can easily be made weakly monoidal. Since
\begin{eq*} [x] \otimes [y] \, = \, [x \otimes y] \, = \, [ \, R(x \otimes y) \, ] \, = \, [ x \boxtimes y ] \end{eq*}
we can choose the coherence map $\mu: [ \, \_ \, ] \otimes [ \, \_ \, ] \Rightarrow [ \, \_ \, \boxtimes \, \_ \, ]$ to be the identity natural transformation $\mathrm{id}_{[ \, \_ \, ] \otimes [ \, \_ \, ]}$, and because
\begin{eq*} [I] \, = \, [ \, R(I) \, ] \end{eq*}
the unit isomorphism $\eta: [I] \to [ R(I) ]$ can just be $\mathrm{id}_{[I]}$. Then by the definition of $[ \, \_ \, ]$ on morphisms, the coherence data $a, l, r$ of $\mathrm{sk}(X)$ are all mapped onto identities too, and hence the necessary conditions for a weak monoidal functor will all be trivially satisfied.

Next, we will define the subcategory which will describe the objects of our monoidal groupoids.

\begin{defi} Let $X$ be a monoidal groupoid, and consider a new category which has the same objects as $X$, and has a unique morphism between two objects if and only if there is at least one morphism between them in $X$. We call this $\mathrm{po}(X)$, since it is always a posetal category. \end{defi}

Since $X$ is a groupoid, $\mathrm{po}(X)$ must be one too, and it also inherits a strict monoidal product durectly from $X$. Furthmore, $\mathrm{po}(X)$ has exactly the same objects and connected components that $X$ does, and so we can construct two obvious strict monoidal functors,
\begin{eq*} \begin{array}{rrrll}
		P & : & X & \to & \mathrm{po}(X) \\
		& : & x & \mapsto & x \\
		& : & f: x \to y & \mapsto & x \to y
		\end{array}
\end{eq*}
and
\begin{eq*} \begin{array}{rrrll}
		[ \, \_ \, ] & : & \mathrm{po}(X) & \to & \pi_0(X) \\
		& : & x & \mapsto & [x] \\
		& : & x \to y & \mapsto & id_{[x]}
		\end{array}.
\end{eq*}
Here we denote the morphisms of $\mathrm{po}(X)$ by giving their unique source and target, rather than assigning them a particular name. 

Putting this together with what we had for $\mathrm{sk}(X)$, we can now express precisely how we are going to split our monoidal groupoids:

\begin{prop}\label{pullback} Let $X$ be a monoidal groupoid. Then for any choice of skeleton $\mathrm{sk}(X)$, the commutative diagram
\begin{eq*} \xymatrix{
& X \ar[dl]_{R} \ar[dr]^{P} & \\
\mathrm{sk}(X) \ar[dr]_{[ \, \_ \, ]} & & \mathrm{po}(X) \ar[dl]^{[ \, \_ \, ]} \\
& \pi_0(X) & }.
\end{eq*}
is a pullback square in the category of weak monoidal categories. That is, 
\begin{eq*} X \quad \cong \quad \mathrm{sk}(X) \times_{\pi_0(X)} \mathrm{po}(X) \end{eq*}
\end{prop}
\begin{proof}
First of all, we need to check tht the above diagram actually commutes. Notice that $[ \, P( \, \_ \, ) \, ]$ and $[ \, R( \, \_ \, ) \, ]$ are both strict monoidal functors, since their coherence data lives in the category $\pi_0(X)$, which contains only identity morphisms. This means we only need to check that they commute as functors, which is simple enough:
\begin{eq*} [ \, R(x) \, ] \, = \, [x] \, = \, [ \, P(x) \, ], \quad \quad \quad [ \, R(f: x \to y) \, ] \, = \, id_{[x]} \, = \, [ \, P(f) \, ]. \end{eq*}
In other words, both  $[ \, P( \, \_ \, ) \, ]$ and $[ \, R( \, \_ \, ) \, ]$ are the obvious map $X \to \pi_0(X)$ sending objects to their connected components. 

Now, assume that we are given a pair of weak monoidal functors $S: Y \to \mathrm{sk}(X)$ and $Q: Y \to \mathrm{po}(X)$ which also form an appropriate commutative square --- that is, $[ \, S( \, \_ \, ) \, ] = [ \, Q( \, \_ \, ) \, ]$. We wish to construct a unique map $U: Y \to X$ which factors $S$ and $Q$ through $R$ and $P$ respectively:
\begin{eq*} \xymatrix{
& Y \ar@{-->}[dd]^{U} \ar@/_2pc/[dddl]_{S} \ar@/^2pc/[dddr]^{Q} \\
&& \\
& X \ar[dl]_{R} \ar[dr]^{P} & \\
\mathrm{sk}(X) \ar[dr]_{[ \, \_ \, ]} & & \mathrm{po}(X) \ar[dl]^{[ \, \_ \, ]} \\
& \pi_0(X) & }
\end{eq*}
Let $y$ be an object of $Y$. In order for the right-hand triangle in the above diagram to commute, we need that $Q(y) = PU(y) = U(y)$. If we take this to be the definition of $U$ on objects, then we see that the left-hand triangle will also commute on objects too, since
\begin{eq*} \begin{array}{rrccl}
		& [ \, S(y) \, ] & = & [ \, Q(y) \, ] & \\
		& & = & [ \, U(y) \, ] & \\
		\implies & S(y) & \cong & U(y), & S(y) \in \mathrm{sk}(X) \\
		\implies & S(y) & = & RU(y)
		\end{array}
\end{eq*}
Similarly, let $f: y \to y'$ be a morphism from $Y$. The right-hand triangle of the diagram automatically commutes, since
\begin{eq*} \begin{array}{rll}
		PU( \, f : y \to y' \, ) & = & P\big( \, U(f) : U(y) \to U(y') \, \big) \\
		& = & U(y) \to U(y') \\
		& = & Q(y) \to Q(y') \\
		& = & Q( \, f : y \to y' \, )
		\end{array}.
\end{eq*}
In order for the left-hand triangle to commute as well, we need $S(f) = RU(f) = \rho_{y'}^{-1} U(f)\rho_y$, and we can again take this to be definitional, so that $U(f) := \rho_{y'} S(f) \rho_y^{-1}$.
 
Next, we must determine the coherence data for $U$. For the right-hand triangle we need
\begin{eq*} \eta^Q = P(\eta^U) \circ \eta^P, \quad \quad \quad \mu^Q = P(\mu^U) \circ \mu^P, \end{eq*}
but since morphisms in $\mathrm{po}(X)$ are uniquely specified by their source and target, these equalities necessarily hold. Similarly, the left-hand triangle gives
\begin{eq*} \begin{array}{rrlllrll}
		& \eta^S & = & R(\eta^U) \circ \eta^R & & \mu^S & = & R(\mu^U) \circ \mu^R \\
		& & = & \rho_{U(I)}^{-1} \, \eta^U \, \rho_{I} \, \eta^R & & & = & \rho_{U(\_ \otimes \_)}^{-1} \, \mu^U \, \rho_{U(\_) \otimes U(\_)} \, \mu^R \\
		&&&&&&& \\
		\implies & \eta^U & = & \rho_{U(I)} \, \eta^S \, (\eta^R)^{-1} \, \rho_{I}^{-1} & & \mu^U & = & \rho_{U(\_ \otimes \_)} \, \mu^S \, (\mu^R)^{-1} \, \rho_{U(\_) \otimes U(\_)}^{-1}
		\end{array}.
\end{eq*}

Since this definition of $U$ has been forced on us by the requirement that it fit into a certain commutative diagram, clearly $U$ is the unique functor with that property. However, to complete the proof we also need to verify that what we have constructed is actually a well-defined weak monoidal functor. This amounts to checking that the following diagrams commute:
\begin{eq*} \xymatrix{
\big( \, U(y) \otimes U(y') \, \big) \otimes U(y'') \ar[rrr]^{a^X_{U(y), U(y'), U(y'')}} \ar[d]_{\mu^U_{y, y'} \, \otimes \, \mathrm{id}_{U(y'')}} & & & U(y) \otimes \big( \, U(y') \otimes U(y'') \, \big) \ar[d]^{\mathrm{id}_{U(y)} \, \otimes \, \mu^U_{y', y''}} \\
U(y \otimes y') \otimes U(y'') \ar[d]_{\mu^U_{y \otimes y', y''}} & & &  U(y) \otimes U(y' \otimes y'') \ar[d]^{\mu^U_{y, y' \otimes y''}} \\
U\big( \, (y \otimes y') \otimes y'' \, \big) \ar[rrr]^{U(a^Y_{y, y', y''})} & & & U\big( \, y \otimes (y' \otimes y'') \, \big) }
\end{eq*}
\begin{eq*} \xymatrix{
I \otimes U(y) \ar[r]^{l^X_{U(y)}} \ar[d]_{\eta^U \, \otimes \, \mathrm{id}_{U(y)}} & U(y) & & U(y) \otimes I \ar[r]^{r^X_{U(y)}} \ar[d]_{\mathrm{id}_{U(y)} \, \otimes \, \eta^U} & U(y) \\
 U(I) \otimes U(y) \ar[r]^{\mu^U_{I, y}} & U(I \otimes y) \ar[u]_{U(l^Y_y)} & & U(y) \otimes U(I) \ar[r]^{\mu^U_{y, I}} & U(y \otimes I) \ar[u]_{U(r^Y_y)}
 }
\end{eq*}
Note that from now on we will supress the subscripts on morphisms for the sake of clarity. In order to check the first of these conditions, consider the following diagram:
\begin{eq*} \xymatrix{
\big( \, RU(y) \boxtimes RU(y') \, \big) \boxtimes RU(y'') \ar[rr]^{a^{\mathrm{sk}(X)}} \ar[d]_{\mu^R \, \boxtimes \, \mathrm{id}} & & RU(y) \boxtimes \big( \, RU(y') \boxtimes RU(y'') \, \big) \ar[d]^{\mathrm{id} \, \boxtimes \, \mu^R} \\
R\big( \, U(y) \otimes U(y') \, \big) \boxtimes RU(y'') \ar[d]_{R(\mu^U) \, \boxtimes \, \mathrm{id}} \ar@/^6pc/[dd]^{\mu^R} & & RU(y) \boxtimes R\big( \, U(y') \otimes U(y'') \, \big) \ar[d]^{\mathrm{id} \, \boxtimes \, R(\mu^U)} \ar@/_6pc/[dd]_{\mu^R}  \\
RU(y \otimes y') \boxtimes RU(y'')  \ar@/_7pc/[dd]_{\mu^R} & & RU(y) \boxtimes RU(y' \otimes y'') \ar@/^7pc/[dd]^{\mu^R} \\
R\Big( \, \big( \, U(y) \otimes U(y') \, \big) \otimes U(y'') \, \Big) \ar[rr]^{R(a^X)} \ar[d]^{R(\mu^U \, \otimes \, \mathrm{id})} & & R\Big( \, U(y) \otimes \big( \, U(y') \otimes U(y'') \, \big) \, \Big) \ar[d]_{R(\mathrm{id} \, \otimes \, \mu^U)} \\
R\big( \, U(y \otimes y') \otimes U(y'') \, \big) \ar[d]_{R(\mu^U)} & & R\big( \, U(y) \otimes U(y' \otimes y'') \, \big) \ar[d]^{R(\mu^U)} \\
RU\big( \, (y \otimes y') \otimes y'' \, \big) \ar[rr]^{RU(a^Y)} & & RU\big( \, y \otimes (y' \otimes y'') \, \big) }
\end{eq*}
The topmost area of this diagram, enclosed by the two maps that run from $(RU(y) \boxtimes RU(y')) \boxtimes RU(y'')$ to $RU(y) \boxtimes (RU(y') \boxtimes RU(y''))$, is an example of the coherence condition for $\mu^R$, and hence commutes. The two areas on either side of the diagram also commute, by the nautrality of $\mu^R$. Lastly, the outside edges of this diagram commute because they form the coherence condition for $\mu^S = R(\mu^U) \mu^R$. As a result of this, and the fact that every edge of the diagram is invertible, it follows that the bottom rectangle the diagram commutes too. Focusing on this area, we see that is is the image under $R$ of the coherence condition for $\mu^U$. But since $R$ just acts on morphisms as $R(f) = \rho_{y'}^{-1} f \rho_y$, the full condition for $\mu^U$ is readily recoverable from this:
\begin{eq*} \begin{array}{rll}
		\mu^U \, (\mathrm{id} \otimes \mu^U) \, a^X & = & \rho \, \rho^{-1} \, \mu^U \, \rho \, \rho^{-1} \, (\mathrm{id} \otimes \mu^U) \, \rho \, \rho^{-1} \, a^X \, \rho \, \rho^{-1} \\
		& = & \rho \, R(\mu^U) \, R(\mathrm{id} \otimes \mu^U) \, R(a^X) \, \rho^{-1} \\
		& = & \rho \, RU(a^Y) \, R(\mu^U) \, R(\mu^U \otimes \mathrm{id}) \, \rho^{-1} \\
		& = & \rho \, \rho^{-1} \, U(a^Y) \, \rho \, \rho^{-1} \, \mu^U \, \rho \, \rho^{-1} \, (\mu^U \otimes \mathrm{id}) \, \rho \, \rho^{-1} \\
		& = & U(a^Y) \, \mu^U \, (\mu^U \otimes \mathrm{id})
		\end{array}
\end{eq*}
We can prove the coherence of $\eta^U$ in a very similar way. Consider the following diagram:
\begin{eq*} \xymatrix{
& RU(y) & \\
R(I) \boxtimes RU(y) \ar[ur]^{l^Y} \ar[dd]_{\eta^R \, \boxtimes \, \mathrm{id}} & & RU(I \otimes y) \ar[ul]_{RU(l^{\mathrm{sk}(X)})} \\
& R\big( \, I \otimes U(y) \, \big) \ar[uu]^{R(l^X)} \ar[dr]^{R(\eta^U \, \otimes  \, \mathrm{id})} & \\
R(I) \boxtimes RU(y) \ar[ur]^{\mu^R} \ar[dr]^{R(\eta^U) \, \otimes \mathrm{id}} & & R\big( \, U(I) \otimes U(y) \, \big) \ar[uu]_{R(\mu^U)} \\
& RU(I) \boxtimes RU(y) \ar[ur]^{\mu^R} & }
\end{eq*}
The top-left square is one of the coherence conditions for $\eta^R$; the outer edges form the same coherence condition but for $\eta^S = R(\eta^U) \eta^R$; and the bottom square follows from the naturality of $\mu^R$. Hence, all of those parts of the diagram commute, and since again all edges are invertible, we can conclude that the remaining top-right square also commutes. But this is just the image under $R$ of one of the coherence conditions for $\eta^U$, from which we can obtain the original thusly:
\begin{eq*} \begin{array}{rll}
		U(l^{\mathrm{sk}(X)}) \, \mu^U \, (\eta^U \otimes \mathrm{id}) & = & \rho \, \rho^{-1} \, U(l^{\mathrm{sk}(X)}) \, \rho \, \rho^{-1} \, \mu^U \, \rho \, \rho^{-1} \, (\eta^U \otimes \mathrm{id}) \, \rho \, \rho^{-1} \\
		& = & \rho \, RU(l^{\mathrm{sk}(X)}) \, R(\mu^U) \, R(\eta^U \otimes \mathrm{id}) \, \rho^{-1} \\
		& = & \rho \, R(l^X) \, \rho^{-1} \\
		& = & \rho \, \rho^{-1} \, l^X \, \rho \, \rho^{-1} \\
		& = & l^X
		\end{array}
\end{eq*}
The proof of the other coherence condition for $\eta^U$, involving $r$ rather than $l$, proceeds in exactly the same way. Therefore, $U$ is indeed a well-defined weak monoidal functor, and so $X$ is the required pullback.
\end{proof}

For any of the objects $\psi : \mathbb{G}_n \to X$ of $C^{\mathrm{m}}_{\mathrm{i}}(\mathbb{G}_n)$, we now have a way of breaking down their source and target into smaller groupoids representing their connected components and automorphisms. We might wonder if we can do the same sort of thing to $\psi$ itself --- show that it is equivalent to several 'smaller' maps running between corresponding terms of $\mathrm{sk}(\mathbb{G}_n) \times_{\pi_0(\mathbb{G}_n)} \mathrm{po}(\mathbb{G}_n)$ and $\mathrm{sk}(X) \times_{\pi_0(X)} \mathrm{po}(X)$. This would allow us to contruct the initial object of $C^m_{\mathrm{i}}(\mathbb{G}_n)$ by finding the initial versions of these reduced maps, greatly simplifying the problem. It turns out that this procedure is relatively straightforward.

\begin{prop}\label{factprop} Let $\psi : \mathbb{G}_n \to X$ be an object of $C_{\mathrm{i}, \mathrm{m}}(\mathbb{G}_n)$, and $\mathrm{sk}(\mathbb{G}_n)$ a choice of skeleton for $\mathbb{G}_n$. Then there exists a choice of skeleton $\mathrm{sk}(X)$ and a unique pair of weak monoidal functors
\begin{eq*} \psi_{\mathrm{s}} : \mathrm{sk}(\mathbb{G}_n) \to \mathrm{sk}(X), \quad \quad \psi_{\mathrm{p}} : \mathrm{po}(\mathbb{G}_n) \to \mathrm{po}(X) \end{eq*}
making the following diagram commutes:
\begin{eq*} \xymatrix{
\mathbb{G}_n \ar[rrrr]^{\psi} \ar[ddd]_{R_{\mathbb{G}_n}} \ar[dr]_{P_{\mathbb{G}_n}} & & & & X \ar[ddd]^{R_X} \ar[dl]^{P_X} \\
& \mathrm{po}(\mathbb{G}_n) \ar[rr]^{\psi_{\mathrm{p}}} \ar[d]_{[ \, \_ \, ]} & & \mathrm{po}(X) \ar[d]^{[ \, \_ \, ]} & \\
& \pi_0(\mathbb{G}_n) \ar[rr]^{\psi_\pi} & & \pi_0(X) & \\
\mathrm{sk}(\mathbb{G}_n) \ar[rrrr]^{\psi_{\mathrm{s}}} \ar[ur]^{[ \, \_ \, ]} & & & & \mathrm{sk}(X) \ar[ul]_{[ \, \_ \, ]} }
\end{eq*}
\end{prop}
\begin{proof}
To begin, we have to make a choice of $\mathrm{sk}(X)$. Consider the commutative diagram below:
\begin{eq*} \begin{tikzcd}
\mathrm{sk}(\mathbb{G}_n) \ar[r] \ar[d, hook, "i"'] & \mathrm{im}(\psi i) \ar[d, hook] \\
\mathbb{G}_n \ar[r, "\psi"] \ar[d] & X \ar[d] \\
\pi_0(\mathbb{G}_n) \ar[r, "\psi_\pi"] & \pi_0(X) 
\end{tikzcd} \end{eq*}
The bottom square here is just the definition of $\psi_\pi$, the restriction of $\psi$ to connected components. For the top square, we have factored the functor $\psi \circ i : \mathrm{sk}(\mathbb{G}_n) \to \mathbb{G}_n \to X$ through its image. This image is a clearly subcategory of $X$, but the rest of the diagram tells us more than just that. Because $\psi$ is an object of $C^{\mathrm{m}}_{\mathrm{i}}(\mathbb{G}_n)$, it is injective on connected components, or in other words the bottom edge of the diagram, $\psi_\pi$, is injective on objects. Also, since $\mathrm{sk}(\mathbb{G}_n)$ contains one object from each connected component of $\mathbb{G}_n$, the lefthand edge must be bijective on objects. Lastly, the top edge is surjective on objects, by the definition of the image of $\psi i$. It follows from this that the righthand edge of the diagram must be injective, and this in turn means that the subcategory $\mathrm{im}(\psi i)$ will contain at most one object from each connected component of $X$. 

The upshot of all of this is that $\mathrm{im}(\psi i)$ can be extended to a skeleton of $X$, though not necessarily uniquely --- given any set of representatives that includes every object of $\mathrm{im}(\psi i)$, the full subcategory of $X$ on those representatives will be a skeleton $\mathrm{sk}(X)$ with $\mathrm{im}(\psi i) \subseteq \mathrm{sk}(X)$. We will proceed by picking any one of these possibilities to be our chosen $\mathrm{sk}(X)$. Recalling the diagram given in the statement of the proposition, it follows immediately from Proposition \ref{pullback} that the left- and righthand squares will commute.

We also need to choose a natural isomorphism $\rho^X : i^X \circ R^X \to \mathrm{id}_X$ which witnesses that our $\mathrm{sk}(X)$ is indeed equivalent to $X$. For this, we'll pick one that satisfies
\begin{eq*} \rho^X_{\psi(w)} \, = \, \psi( \rho^{\mathbb{G}_n}_w ) \end{eq*}









Next, we need to show that there is at least one pair of maps $\psi_{\mathrm{s}}$, $\psi_{\mathrm{p}}$ satisfying the necessary conditions. For $\psi_{\mathrm{p}}$, we will choose the strict monoidal functor induced by the restriction of $\psi$ to objects:
\begin{eq*} \begin{array}{rcccl}
		\psi_{\mathrm{p}} & : &  \mathrm{po}(\mathbb{G}_n) & \to & \mathrm{po}(X) \\
		& : & w & \mapsto & \psi(w) \\
		& : & w \to v & \mapsto & \psi(w) \to \psi(v)
		\end{array}
\end{eq*}
With $\psi_{\mathrm{p}}$ defined this way, look at the top and centre squares of the diagram that appears in the proposition. They now contain only strictly monoidal functors, and so the following is sufficient to show that they commute:
\begin{eq*} \begin{array}{rllrrlll}
		P_X \psi(w) & = & \psi(w) & \quad & P_X \psi( \, f: w \to v \, ) & = & P_X \big( \, \psi(w) : \psi(w) \to \psi(v) \, \big) \\
		& = & \psi_{\mathrm{p}}(w) & & & = & \psi(w) \to \psi(v) \\
		& = & \psi_{\mathrm{p}} P_{\mathbb{G}_n}(w) & & & = & \psi_{\mathrm{p}}(w \to v) \\
		& & & & & = & \psi_{\mathrm{p}} P_{\mathbb{G}_n}(f: w \to v) \\
		& & & & & & \\
		\left[ \, \psi_{\mathrm{p}}(w) \, \right] & = & [ \, \psi(w) \, ] & & [ \, \psi_{\mathrm{p}}(w \to v) \, ] & = & [ \, \psi(w) \to \psi(v) \, ] \\
		& = & \psi_{\pi}\big( \, [w] \, \big) & & & = & \mathrm{id}_{[ \psi(w)]} \\
		& & & & & = & \mathrm{id}_{\psi_{\pi}([w])} \\
		& & & & & = & \psi_{\pi}(\mathrm{id}_{[w]}) \\
		& & & & & = & \psi_{\pi}\big( \, [w \to v] \, \big) \\
		\end{array}
 \end{eq*}
For $\psi_{\mathrm{s}}$, we will choose the composite given by the top edge of the diagram below:
\begin{eq*} \begin{tikzcd}
\mathrm{sk}(\mathbb{G}_n) \ar[r] \ar[d, hook, "i_{\mathbb{G}_n}"'] & \mathrm{im}(\psi i) \ar[d, hook] \ar[r, hook] & \mathrm{sk}(X) \ar[d, hook, "i_X"] \\
\mathbb{G}_n \ar[r, "\psi"] \ar[d, "R_{\mathbb{G}_n}"'] & X \ar[r, equal] & X \ar[d, "R_X"] \\
\mathrm{sk}(\mathbb{G}_n) \ar{d}[']{[ \, \_ \, ]} & & \mathrm{sk}(X) \ar{d}{[ \, \_ \, ]} \\
\pi_0(\mathbb{G}_n) \ar[rr, "\psi_\pi"] & & \pi_0(X) 
\end{tikzcd} \end{eq*} 
This diagram is pasted together out of three smaller diagrams: the top-left square is the definition of $\mathrm{im}(\psi i)$; the top-right represents the fact that $\mathrm{sk}(X)$ contains $\mathrm{im}(\psi i)$; and the bottom is the definition of $\psi_\pi$, with the canonical maps $\mathbb{G}_n \to \pi_0(\mathbb{G}_n)$ and $X \to \pi_0(X)$ expanded as $[ \, R(\_) \, ]$ per the proof of Proposition \ref{pullback}. Since all of these commute, the outer edges of diagram also commutes, which are
\begin{eq*} \begin{tikzcd}
\mathrm{sk}(\mathbb{G}_n) \ar{d}[']{[ \, \_ \, ]} \ar[r, "\psi_s"] & \mathrm{sk}(X) \ar{d}{[ \, \_ \, ]} \\
\pi_0(\mathbb{G}_n) \ar[r, "\psi_\pi"] & \pi_0(X) 
\end{tikzcd} \end{eq*} 
due to the fact that $R \circ i = \mathrm{id}$. This is now the bottom square from the commutative diagram that was stated in the proposition. Moreover, if we embed this square back into the rectangle representing the definition of $\psi_\pi$,
\begin{eq*} \begin{tikzcd}
\mathbb{G}_n \ar[r, "\psi"] \ar[d, "R_{\mathbb{G}_n}"'] & X \ar[d, "R_X"] \\
\mathrm{sk}(\mathbb{G}_n) \ar{d}[']{[ \, \_ \, ]} \ar[r, "\psi_s"] & \mathrm{sk}(X) \ar{d}{[ \, \_ \, ]} \\
\pi_0(\mathbb{G}_n) \ar[r, "\psi_\pi"] & \pi_0(X) 
\end{tikzcd} \end{eq*}
we see that the top square of this diagram is the outer edge of the diagram from the proposition, the last part that we need to show commutes. 

.

.

.

To begin, assume we are given an object $\psi: \mathbb{G}_n \to X$ of $C_{\mathrm{i, m}}(\mathbb{G}_n)$. By Proposition \ref{zerotree}, we know that our choice of representatives on $\mathbb{G}_n$ defines an isomorphism $F: \mathbb{G}_n \to \coprod \mathrm{B}\mathbb{G}_n(r, r) \times \mathrm{E}[r]$. Next, we choose a tree of representatives $(R', \rho')$ for $X$ as well, but we do it in such a way that the restriction of $(R', \rho')$ to the image of $\psi$ is the same as the image of $(R, \rho)$ under $\psi$. In other words, we need to make sure that $\psi$ sends representatives to representatives, so $\psi(R) \subseteq R'$, and sends their maps to the corresponding maps, so $\rho'_{\psi(w)} = \psi(\rho_{w})$. It is always possible to make a choice of $(R', \rho')$ like this, because $\psi$ is injective on connected components and so for different representatives $r, s \in R$
\begin{eq*} r \neq s \quad \implies \quad [r] \neq [s] \quad \implies \quad [\psi(r)] \neq [\psi(s)]. \end{eq*}
From $(R', \rho')$, Proposition \ref{zerotree} then gives us a second isomorphism, $F': X \to \mathrm{B}X(I,I) \times \coprod \mathrm{E}[r']$. Notice that while $F'$ may differ based on our particular choice of $R'$, the induced map $\psi' = F' \psi F^{-1}$ will be the same regardless, since the representatives in the image of $\phi$ are all fixed.

Unpacking the definition of $\psi'$, we see that on objects,
\begin{eq*} \begin{array}{rll}
		\psi'(\, r, \, w \, ) & = & F' \psi F^{-1}(\, r, \, w \, ) \\
		& = & F' \psi( \, w \, )  \\
		& = & \big( \,  I, \, \psi(w) \, \big) \\
		& = & \big( \,  I, \, \psi_{\mathrm{ob}}(w) \, \big)
		\end{array}.
\end{eq*}
Thus the restriction of $\psi'$ to objects is determined by the same restriction of $\psi$ --- the map $\psi_{\mathrm{ob}}$. Furthermore, by Corollary \ref{tenscor}
\begin{eq*} \begin{array}{rll}
		\psi'(\, (r, w) \otimes (s, v) \, ) & = & \psi'(\, r \boxtimes s, \, w \otimes v \, ) \\
		& = & \big( \,  I, \, \psi_{\mathrm{ob}}(w \otimes v) \, \big) \\
		&& \\
		\psi'(\, r, \, w \, ) \otimes \psi'(\, s, \, v \, ) & = & \big( \,  I, \, \psi_{\mathrm{ob}}(w) \, \big) \otimes \big( \,  I, \, \psi_{\mathrm{ob}}(v) \, \big) \\
		& = & \big( \,  I, \, \psi_{\mathrm{ob}}(w) \otimes \psi_{\mathrm{ob}}(v) \, \big) \\
		\end{array}.
\end{eq*}
and so the monoidality of $\psi'$ (proven in Lemma \ref{indmap}) is entirely accounted for here by the fact that $\psi_{\mathrm{ob}}$ is a monoid homomorphism. This will be our first map $a$.

The rest of the definition of $\psi'$ is that
\begin{eq*} \begin{array}{rll}
		\psi'( \, f: r \to r, \, w \to w' \, ) & = & F' \psi F^{-1}( \, f, \, w \to w' \, ) \\
		& = & F' \psi( \, \rho_{w'} \circ f  \circ \rho_{w}^{-1} \, )  \\
		& = & F'\big( \, \psi(\rho_{w'}) \circ \psi(f) \circ \psi(\rho_w^{-1}) \, \big)  \\
		& = & F'\big( \, \rho'_{\psi(w')} \circ \psi(f) \circ \rho'^{-1}_{\psi(w)} \, \big)  \\
		& = & \big( \, \psi(f) \otimes \mathrm{id}_{\psi(r)^*}, \, \psi(w) \to \psi(w') \, \big) \\
		& = & \big( \, \psi(f) \otimes \mathrm{id}_{\psi(r)^*}, \, \psi_{\mathrm{ob}}(w) \to \psi_{\mathrm{ob}}(w') \, \big) \\
		\end{array}.
\end{eq*}
As before, the second entry in the pair is written entirely in terms of $\psi_{\mathrm{ob}}$. Notice however that the map $\psi_{\mathrm{ob}}(w) \to \psi_{\mathrm{ob}}(w')$ is not something which is defined for general $w, w' \in \mathrm{Ob}(X)$. For our purposes we want such a morphism to exist whenever we have $w \to w'$ in $\coprod \mathrm{E}[r]$, and so we need to know that $\psi_{\mathrm{ob}}$ maps objects from the same connected component into the same connected component. In other words, if $\psi_\pi : \pi_0(\mathbb{G}_n) \to \pi_0(X)$ is the restriction of $\psi$ to components, then $\psi_{\mathrm{ob}}$ must satisfy
\begin{eq*} \xymatrix{
\mathrm{Ob}(\mathbb{G}_n) \ar[r]^{[\_]} \ar[d]_{\psi_{\mathrm{ob}}} & \pi_0(\mathbb{G}_n)  \ar[d]^{\psi_\pi} \\
\mathrm{Ob}(X) \ar[r]^{[\_]} & \pi_0(X)  }.
\end{eq*}
Thus, the injective homomorphism $\psi_\pi$ will be our second map $b$.

The remainder of what $\psi'$ does is described by an $R$-indexed family of maps, $\psi( \, \_ \, ) \otimes \mathrm{id}_{\psi(r)^*} : \mathbb{G}_n(r,r) \to X(I,I)$ which we'll denote $\psi_r$. We can actually define a map like this for any object $w$ of $\mathbb{G}_n$, not just representatives, and while this is more data than is strictly needed to specify $\psi'$, keeping track of the whole $\mathrm{Ob}(\mathbb{G}_n)$-indexed family will make the unpcoming calculations easier to perform. Because $\psi$ is a functor, it follows that 
\begin{eq*}\begin{array}{rll}
		\psi_w(g \circ f) & = & \psi(g \circ f) \otimes \mathrm{id}_{\psi(w)^*} \\
		& = & \big( \, \psi(g) \circ \psi(f) \, \big) \otimes \mathrm{id}_{\psi(w)^*} \\
		& = & \big( \, \psi(g) \otimes \mathrm{id}_{\psi(w)^*} \, \big) \circ \big( \, \psi(f) \otimes \mathrm{id}_{\psi(w)^*} \, \big) \\
		& = & \psi_w(g) \circ \psi_w(f) \\
		&& \\
		\psi_w(f^{-1}) & = & \psi(f^{-1}) \otimes \mathrm{id}_{\psi(w)^*} \\
		& = & \psi(f)^{-1} \otimes (\mathrm{id}_{\psi(w)^*})^{-1} \\
		& = & \big( \psi(f) \otimes \mathrm{id}_{\psi(w)^*} \big)^{-1} \\
		& = & \psi_w(f)^{-1}
		\end{array}
\end{eq*}
and hence these maps are all group homomorphisms. Moreover, for any $f \in \mathbb{G}_n(w, w)$ and $v \in \mathrm{Ob}(\mathbb{G}_n)$,
\begin{eq*}\begin{array}{rll}
		\psi_w(f) & = & \psi(f) \otimes \mathrm{id}_{\psi(w)^*} \\
		& = & \psi(f) \otimes \mathrm{id}_{\psi(v)} \otimes \mathrm{id}_{\psi(v)^*} \otimes \mathrm{id}_{\psi(w)^*} \\
		& = & \psi(f \otimes \mathrm{id}_v) \otimes \mathrm{id}_{(\psi(v)^* \otimes \psi(w))^*} \\
		& = & \psi(f \otimes \mathrm{id}_v) \otimes \mathrm{id}_{(\psi(w \otimes v))^*} \\
		& = & \psi_{w \otimes v}(f \otimes \mathrm{id}_v)
		\end{array}.
\end{eq*}
In other words, all of the diagrams of the form
\begin{eq*} \xymatrix{
\mathbb{G}_n(w,w) \ar[dr]_{\psi_w} \ar[rr]^-{\_ \otimes \mathrm{id}_v} & & \mathbb{G}_n(w \otimes v, w \otimes v) \ar[dl]^{\psi_{w \otimes v}} \\
& X(I,I) & }
\end{eq*}
commute. Thus $X(I,I)$ is actually a cocone of a diagram $D$, and since the colimit of a diagram is its universal cocone, we obtain a new homorphism of groups $\psi_D: \mathrm{colim}(D) \to X(I,I)$. This map condenses all of the information given by the $\psi_w$, in the sense that if $i_w : \mathbb{G}_n(w,w) \to \mathrm{colim}(D)$ are the relevant injections, then $\psi_D$ is the unique group homomorphism with $\psi_w = \psi_D \circ i_w$. Also, since $X(I,I)$ has two binary operations, $\circ$ and $\otimes$, which are both unital and satisfy interchange, an Eckmann-Hilton argument tells us that $X(I,I)$ is an abelian group, and together with the previous Eckmann-Hilton argument in $\ref{colimab}$ this means that $\psi_D$ is an abelian group homomorphism. We will use $\psi_D$ as our last map, $c$.

Finally, using Corollary \ref{tenscor} again, we consider the effect of $\psi'$ on tensor products of morphisms:
\begin{eq*} \begin{array}{rll}
		&& \psi'( \, ( \, f: r \to r, \, w \to w' \, ) \otimes ( \, g: s \to s, \, v \to v' \, ) \, ) \\
		& = & \psi'\big( \, \rho_{w' \otimes v'}^{-1} \circ ( \rho_{w'} \otimes \rho_{v'} ) \circ ( f \otimes g) \circ ( \rho_{w}^{-1} \otimes \rho_{v}^{-1} ) \circ \rho_{w \otimes v} \, , \, w \otimes v \to w' \otimes v' \, \big) \\
		& = & \Big( \, \psi\big(\rho_{w' \otimes v'}^{-1} \circ ( \rho_{w'} \otimes \rho_{v'} ) \circ ( f \otimes g) \circ ( \rho_{w}^{-1} \otimes \rho_{v}^{-1} ) \circ \rho_{w \otimes v} \big) \otimes \mathrm{id}_{\psi_{\mathrm{ob}}(r \boxtimes s)^*} \, , \\
		&& \quad \psi_{\mathrm{ob}}(w \otimes v) \to \psi_{\mathrm{ob}}(w' \otimes v') \, \Big) \\
		& = & \Big( \, \big(\rho_{\psi_{\mathrm{ob}}(w' \otimes v')}^{-1} \circ ( \rho_{\psi_{\mathrm{ob}}(w')} \otimes \rho_{\psi_{\mathrm{ob}}(v')} ) \circ \psi(f \otimes g) \circ ( \rho_{\psi_{\mathrm{ob}}(w)}^{-1} \otimes \rho_{\psi_{\mathrm{ob}}(v)}^{-1} ) \circ \rho_{\psi_{\mathrm{ob}}(w \otimes v)} \big) \\
		&& \quad \otimes \, \,  \mathrm{id}_{\psi_{\mathrm{ob}}(r \boxtimes s)^*} \, , \, \psi_{\mathrm{ob}}(w \otimes v) \to \psi_{\mathrm{ob}}(w' \otimes v') \, \Big) \\
		&& \\
		&& \psi'( \, f: r \to r, \, w \to w' \, ) \otimes \psi'( \, g: s \to s, \, v \to v' \, ) \\
		& = & \big( \, \psi(f) \otimes \mathrm{id}_{\psi(r)^*}, \, \psi_{\mathrm{ob}}(w) \to \psi_{\mathrm{ob}}(w') \, \big) \otimes \big( \, \psi(g) \otimes \mathrm{id}_{\psi(s)^*}, \, \psi_{\mathrm{ob}}(v) \to \psi_{\mathrm{ob}}(v') \, \big) \\
		& = & \Big( \, \big( \, \rho_{\psi_{\mathrm{ob}}(w') \otimes \psi_{\mathrm{ob}}(v')}^{-1} \circ ( \rho_{\psi_{\mathrm{ob}}(w')} \otimes \rho_{\psi_{\mathrm{ob}}(v')} ) \circ ( \psi(f) \otimes \psi(g) \, ) \circ ( \rho_{\psi_{\mathrm{ob}}(w)}^{-1} \otimes \rho_{\psi_{\mathrm{ob}}(v)}^{-1} ) \\
		&& \quad \circ \, \, \rho_{\psi_{\mathrm{ob}}(w) \otimes \psi_{\mathrm{ob}}(v)} \, \big) \otimes \mathrm{id}_{(\psi_{\mathrm{ob}}(r) \boxtimes \psi_{\mathrm{ob}}(s))^*} \, , \, \psi_{\mathrm{ob}}(w) \otimes \psi_{\mathrm{ob}}(v) \to \psi_{\mathrm{ob}}(w') \otimes \psi_{\mathrm{ob}}(v') \, \Big)
		\end{array}
\end{eq*}
Comparing the two, we see that $\psi'$ being a monoidal funtor once again comes from the monoidality of $\psi_{\mathrm{ob}}$, but also from the fact that $\psi(f \otimes g) = \psi(f) \otimes \psi(g)$ for endomorphisms of representatives $f: r \to r$, $g: s \to s$. We can re-express this condition in terms of the data $\psi_{\mathrm{ob}}$ and $\psi_D$ by using our previous definitions --- since we know that for any $h: w \to w'$,
\begin{eq*} \begin{array}{rll}
		\psi(h) & = & \rho'_{w'} \circ \psi'( \, \rho_{w'}^{-1} \circ h \circ \rho_w \, ) \circ \rho_w^{\prime \, -1} \\
		& = & \rho'_{w'} \circ \psi_r( \, \rho_{w'}^{-1} \circ h  \circ \rho_w \, ) \circ \rho_w^{\prime \, -1} \\
		& = & \rho'_{w'} \circ \big( \, \psi_D i_r ( \, \rho_{w'}^{-1} \circ h \circ \rho_w \, ) \otimes \mathrm{id}_{\psi_{\mathrm{ob}}(r)} \, \big) \circ  \rho_w^{\prime \, -1} 
		\end{array}
\end{eq*}
it follows that requiring $\psi(f \otimes g) = \psi(f) \otimes \psi(g)$ is same as asking for the outer rectangle of the following diagram to commute:
\begin{eq*} \xymatrix{
\mathbb{G}_n(r,r) \times \mathbb{G}_n(s,s) \ar[rr]^{\otimes} \ar@{=}[d] && \mathbb{G}_n( r \otimes s, r \otimes s)  \ar[d]^{\rho_{r \otimes s}^{-1} \circ \, \_ \, \circ \rho_{r \otimes s}} \\
\mathbb{G}_n(r,r) \times \mathbb{G}_n(s,s) \ar[rr]^{\boxtimes} \ar[d]_{i_r \times i_s} && \mathbb{G}_n( r \boxtimes s, r \boxtimes s)  \ar[d]^{i_{r \boxtimes s}} \\
\mathrm{colim}(D) \times \mathrm{colim}(D) \ar[rr]^{\otimes} \ar[d]_{\psi_D \times \psi_D} && \mathrm{colim}(D) \ar[d]^{\psi_D} \\
X(I,I) \times X(I,I) \ar[rr]^{\otimes} \ar[d]_{(\_ \otimes \mathrm{id}_{\psi(r)}) \times (\_ \otimes \mathrm{id}_{\psi(s)})} && X(I,I) \ar[d]^{\_ \otimes \mathrm{id}_{\psi(r \boxtimes s)}} \\
X\big( \psi(r), \psi(r) \big) \times X\big( \psi(s), \psi(s) \big) \ar[rr]^{\boxtimes} \ar@{=}[d] && X\big(\psi(r \boxtimes s), \psi(r \boxtimes s)\big)  \ar[d]^{\rho'_{\psi(r \otimes s)} \circ \, \_ \, \circ \rho_{\psi(r \otimes s)}^{\prime \, -1}} \\
X\big( \psi(r), \psi(r) \big) \times X\big( \psi(s), \psi(s) \big) \ar[rr]^{\otimes} && X\big(\psi(r \otimes s), \psi(r \otimes s) \big)  }.
\end{eq*}
However, notice that each of the smaller rectangles in the diagram is already known to commute. The top and bottom rectangles are simply the definition of the product $\boxtimes$ on morphisms, applied to $\mathbb{G}_n$ and $X$ respectively; the second rectangle is the definition of the tensor product of $\mathrm{colim}(D)$ from Lemma \ref{colimab}; the third is just the condition that $\psi_D$ is a group homomorphism; and the fourth follows from the fact that $X$ is a spacial category (see Lemma \ref{spacelem}):
\begin{eq*} \begin{array}{rll}
		f \otimes g \otimes \mathrm{id}_{\psi(r \boxtimes s)} & = & f \otimes g \otimes (\rho_{\psi(r \otimes s)}^{\prime \, -1} \circ \rho'_{\psi(r \otimes s)} ) \\
		& = & \rho_{\psi(r \otimes s)}^{\prime \, -1} \circ (f \otimes g \otimes \mathrm{id}_{\psi(r \otimes s)}) \circ \rho'_{\psi(r \otimes s)} \\
		& = & \rho_{\psi(r \otimes s)}^{\prime \, -1} \circ (f \otimes g \otimes \mathrm{id}_{\psi(r)} \otimes \mathrm{id}_{\psi(s)}) \circ \rho'_{\psi(r \otimes s)} \\
		& = & \rho_{\psi(r \otimes s)}^{\prime \, -1} \circ (f \otimes \mathrm{id}_{\psi(r)} \otimes g \otimes \mathrm{id}_{\psi(s)}) \circ \rho'_{\psi(r \otimes s)} \\
		& = & (f \otimes \mathrm{id}_{\psi(r)}) \boxtimes (g \otimes \mathrm{id}_{\psi(s)})
		\end{array}.
\end{eq*}

 Therefore, the fact that $\psi(f \otimes g) = \psi(f) \otimes \psi(g)$ is actually automatic, given what we already know of $\psi_{\mathrm{ob}}$ and $\psi_D$.

Now we can easily reverse this process. Assume that we are are given maps
\begin{eq*} a : \mathrm{Ob}(\mathbb{G}_n) \to \mathrm{Ob}(X), \quad b : \pi_0(\mathbb{G}_n) \to \pi_0(X), \quad c : \mathrm{colim}(D) \to X(I,I) \end{eq*}
satisfying all of the appropriate conditions, for some known $X$. Choose a tree of representatives $(R', \rho')$ for $X$ in such a way that $a(R) \subseteq R'$. This is always possible because $b$ is injective, and so for any $r, s \in R$,
\begin{eq*} r \neq s \quad \implies \quad [r] \neq [s] \quad \implies \quad b([r]) \neq b([s]) \quad \implies \quad [a(r)] \neq [a(s)]. \end{eq*}
Define a functor
\begin{eq*}\begin{array}{rlrcl}
		\psi' & : & \coprod \mathrm{B}\mathbb{G}_n(r, r) \times \mathrm{E}[r] & \to & \mathrm{B}X(I,I) \times \coprod \mathrm{E}[r'] \\
		& : & (r, w) & \mapsto & \big( \, I, \, a(w) \, \big) \\
		& : & ( \, f: r \to r, \, w \to w' \, ) & \mapsto & \big( \, c \circ i_r(f), \, a(w) \to a(w') \, \big)
		\end{array}.
\end{eq*}
Since 
\begin{eq*} [w] = [w'] \implies b([w]) = b([w']) \implies [a(w)] = [a(w')], \end{eq*}
the map $a(w) \to a(w')$ exists whenever $ w \to w'$ does, and so the map $\psi'$ is well-defined. Moreover, by Corollary \ref{tenscor}
\begin{eq*}\begin{array}{rll}
		\psi' \big( \, (r, w) \otimes (s, v) \, \big) & = & \psi'( \, r \boxtimes s, \, w \otimes v \, ) \\
		& = & \big( \, I, \, a(w \otimes v) \, \big) \\
		& = & \big( \, I, \, a(w) \otimes a(v) \, \big) \\
		& = & \big( \, I, \, a(w) \, \big) \otimes \big( \, I, \, a(v) \, \big) \\
		& = & \psi'(r, w) \otimes \psi'(s, v)
		\end{array}
\end{eq*}
and
\begin{eq*}\begin{array}{rll}
		&& \psi' \big( \, ( \, f: r \to r, \, w \to w' \, ) \otimes ( \, g: s \to s, \, v \to v' \, ) \, \big) \\
		& = & \psi' \big( \, \rho_{w' \otimes v'}^{-1} \circ ( \rho_{w'} \otimes \rho_{v'} ) \circ ( f \otimes g) \circ ( \rho_{w}^{-1} \otimes \rho_{v}^{-1} ) \circ \rho_{w \otimes v} \, , \, w \otimes v \to w' \otimes v' \, \big) \\
		& = & \Big( \,  c i_{r \boxtimes s} \big( \, \rho_{w' \otimes v'}^{-1} \circ ( \rho_{w'} \otimes \rho_{v'} ) \circ ( f \otimes g) \circ ( \rho_{w}^{-1} \otimes \rho_{v}^{-1} ) \circ \rho_{w \otimes v} \, \big) \, , \, a(w \otimes v) \to a(w' \otimes v') \, \Big) \\
		& = & \\
		& = & \Big( \, \big( \, \rho_{a(w' \otimes v')}^{-1} \circ ( \rho_{a(w')} \otimes \rho_{a(v')} ) \circ ( c i_r(f) \otimes \mathrm{id}_{a(r)} \otimes  c i_s(g) \otimes \mathrm{id}_{a(s)} ) \circ ( \rho_{a(w)}^{-1} \otimes \rho_{a(v')}^{-1} ) \circ \rho_{a(w \otimes v)} \, \big) \otimes \mathrm{id}_{a(r \boxtimes s)^*} \, , \\
		&& \quad a(w) \otimes a(v) \to a(w') \otimes a(v') \Big) \\	
		& = & \Big( \, \big( \, \rho_{a(w') \otimes a(v')}^{-1} \circ ( \rho_{a(w')} \otimes \rho_{a(v')} ) \circ ( c i_r(f) \otimes \mathrm{id}_{a(r)} \otimes  c i_s(g) \otimes \mathrm{id}_{a(s)} ) \circ ( \rho_{a(w)}^{-1} \otimes \rho_{a(v')}^{-1} ) \circ \rho_{a(w) \otimes a(v)} \, \big) \otimes \mathrm{id}_{a(r) \boxtimes a(w)^*} \, , \\
		&& \quad a(w) \otimes a(v) \to a(w') \otimes a(v') \Big) \\
		& = & \big( \, c i_r(f), \, a(w) \to a(w') \, \big) \otimes \big( \, c i_s(g), \, a(v) \to a(v') \, \big) \\
		& = & \psi'( \, f: r \to r, \, w \to w' \, ) \otimes \psi'( \, g: s \to s, \, v \to v' \, )
		\end{array}
\end{eq*}

 Then we can construct an object $\psi : \mathbb{G}_n \to X$ of $C_{\mathrm{i, m}}(\mathbb{G}_n)$ by

\end{proof}



\begin{defi} Let $D$ be the diagram in category of groups whose vertices are the endomorphism groups $\mathbb{G}_n(r, r)$ of the representing objects $r \in R$, and which has an edge
\begin{eq*}\_ \otimes \mathrm{id}_w : \mathbb{G}_n(r, r) \to \mathbb{G}_n( s, s ) \end{eq*}
whenever $s = r \otimes w$ for some $r, s \in R$, $w \in \mathrm{Ob}(\mathbb{G}_n)$, and
\begin{eq*}\mathrm{id}_w \otimes \_ : \mathbb{G}_n(r, r) \to \mathbb{G}_n( s, s) \end{eq*}
whenever $s = w \otimes r$, for $r, s \in R$, $w \in \mathrm{Ob}(\mathbb{G}_n)$. 
\end{defi}

\begin{lem}\label{colimab} The monoidal structure $\boxtimes$ of $R$ induces a monoidal structure on the colimit of the diagram $D$. This structure makes $\mathrm{colim}(D)$ into an abelian group. \end{lem}
\begin{proof}
We can define a tensor product on $\mathrm{colim}(D)$ by simply setting
\begin{eq*} i_r(f) \boxtimes i_s(g) := i_{r \boxtimes s}(f \boxtimes g) \end{eq*}
for any $f : r \to r$ and $g: s \to s$, where $i_r: \mathbb{G}_n(r, r) \to \mathrm{colim}(D)$, $i_s: \mathbb{G}_n(s, s) \to \mathrm{colim}(D)$ are the appropriate injections. The unit object in $\mathrm{colim}(D)$ is then clearly $i_I( \mathrm{id}_I)$, since $I$ and $\mathrm{id}_I$ are the units of $\boxtimes$ on objects and morphisms respectively. To see that this product is well-defined, we need to check that whenever $i_r(f) = i_{r'}(f')$ we can substitute one for the other in our formula for $\boxtimes$ without changing the result. Notice that, given the form that the edges of $D$ take, it suffices to show that if $w \otimes r \otimes x \in R$ and $y \otimes s \otimes z \in R$ for some $w, x, y, z \in \mathrm{Ob}(\mathbb{G}_n)$, then
\begin{eq*}	i_{r \boxtimes s}(f \boxtimes g) =  i_{(w \otimes r \otimes x) \boxtimes (y \otimes s \otimes z)}\big( \, (\mathrm{id}_w \otimes f \otimes \mathrm{id}_x) \boxtimes (\mathrm{id}_y \otimes g \otimes \mathrm{id}_z) \, \big) .\end{eq*}
But this is indeed the case, since
\begin{eq*}\begin{array}{rll}
		(\mathrm{id}_w \otimes f \otimes \mathrm{id}_x) \boxtimes (\mathrm{id}_y \otimes g \otimes \mathrm{id}_z) & = & \rho^{-1}_{(wrx) \boxtimes (ysz)} \circ (\mathrm{id}_w \otimes f \otimes \mathrm{id}_x \otimes \mathrm{id}_y \otimes g \otimes \mathrm{id}_z) \circ \rho_{(wrx) \boxtimes (ysz)}\\
		& = &  (\mathrm{id}_x \otimes f \otimes \mathrm{id}_y \otimes \mathrm{id}_v \otimes \mathrm{id}_z) \circ (\mathrm{id}_x \otimes \mathrm{id}_w \otimes \mathrm{id}_y \otimes g \otimes \mathrm{id}_z) \\
		& = & (\mathrm{id}_x \otimes f \otimes \mathrm{id}_{yvz}) \circ (\mathrm{id}_{xwy} \otimes g \otimes \mathrm{id}_z) \\
		&& \\
		i_{xwyvz}(\mathrm{id}_x \otimes f \otimes \mathrm{id}_y \otimes g \otimes \mathrm{id}_z) & = &  i_{xwyvz}\big( (\mathrm{id}_x \otimes f \otimes \mathrm{id}_{yvz}) \circ (\mathrm{id}_{xwy} \otimes g \otimes \mathrm{id}_z) \big) \\
		& = &  i_{xwyvz}(\mathrm{id}_x \otimes f \otimes \mathrm{id}_{yvz}) \circ  i_{xwyvz}(\mathrm{id}_{xwy} \otimes g \otimes \mathrm{id}_z) \\
		& = &  i_{w}(f) \circ  i_{v}(g) \\
		& = &  i_{wv}(f \otimes \mathrm{id}_{v}) \circ  i_{wv}(\mathrm{id}_{w} \otimes g) \\
		& = &  i_{wv}\big( (f \otimes \mathrm{id}_{v}) \circ (\mathrm{id}_{w} \otimes g) \big) \\
		& = &  i_{wv}(f \otimes g)
		.\end{array}
\end{eq*}
Next we apply the classic Eckmann-Hilton argument (see \cite{eckhil} for their original paper). The group $\mathrm{colim}(D)$ has two binary operations on it; the tensor product $\otimes$ we've just defined, and the group multiplication that comes from its definition as a colimit in the category of groups, $\circ$, which is related to the compositions in each of the vertex groups $\mathbb{G}_n(w, w)$. Both of these operations are unital --- indeed they share the unit $\mathrm{id}_I$ --- and so it remains to show that they satisfy an interchange law. Let $f: w \to w$, $g: x \to x$, $h: y \to y$ and $j: z \to z$ all be maps in $\mathbb{G}_n$. Then
\begin{eq*}\begin{array}{rll}
		&& \big( \, i_{w}(f) \circ  i_{x}(g) \, \big)  \, \otimes \, \big( \,  i_{y}(h) \circ  i_{z}(j) \, \big) \\
		& = &  \big( \, i_{wx}(f \otimes \mathrm{id}_x) \circ  i_{wx}(\mathrm{id}_w \otimes g) \, \big)  \, \otimes \, \big( \, i_{yz}(h \otimes \mathrm{id}_z) \circ  i_{yz}(\mathrm{id}_y \otimes j) \, \big) \\
		& = & i_{wx}(f \otimes g) \, \otimes \, i_{yz}(h \otimes j) \\
		& = & i_{wxyz}(f \otimes g \otimes h \otimes j) \\
		& = & i_{wxyz}(f \otimes \mathrm{id}_{x} \otimes h \otimes \mathrm{id}_{z}) \, \circ \, i_{wxyz}(\mathrm{id}_{w} \otimes g \otimes \mathrm{id}_{y} \otimes j) \\
		& = & \big( \, i_{wx}(f \otimes \mathrm{id}_{x}) \otimes i_{yz}(h \otimes \mathrm{id}_{z}) \, \big) \, \circ \, \big( \, i_{wx}(\mathrm{id}_{w} \otimes g) \otimes i_{yz}(\mathrm{id}_{y} \otimes j) \, \big) \\
		& = & \big( \, i_{w}(f) \otimes i_{y}(h) \, \big) \, \circ \, \big( \, i_{x}(g) \otimes i_{z}(j) \, \big)
		.\end{array}
\end{eq*}
Since we have unitality and interchange, the Eckmann-Hilton argument tells us that $\otimes$ and $\circ$ must in fact be the same binary operation, and that that operation is commutative. Therefore $\mathrm{colim}(D)$ is actually an abelian group.
\end{proof}








\epnote{Everything from here to next section needs to be rephrased in terms of new notation / paper reordering; ignore for the moment}

\begin{thm}\label{colimthm}  Let $\phi : \mathbb{G}_n \to Z$ be an initial object in $C_{\mathrm{i}}(\mathbb{G}_n)$. Then 
\begin{eq*} Z \quad \cong \quad \mathbb{Z}^n \times \mathrm{B} \, \mathrm{colim} \Big( \mathbb{G}_n(w,w) \to \mathbb{G}_n(w \otimes v, w\otimes v) \Big)^{\mathrm{ab}} \times \mathrm{E}[\mathbb{Z}^{*n}, \mathbb{Z}^{*n}]. \end{eq*}
\end{thm}
\begin{proof}

From Proposition \ref{concomp}, we know that $\pi_0(Z)$ is in fact just $\mathbb{Z}^n$. It also tells us that the canonical map $[ \, \, ] : \mathbb{Z}^{*n} \to \mathbb{Z}^n$ sending objects of $Z$ to their connected component is the quotient map of abelianisation, and so $[0]$ is just the kernel of this map, the commutator subgroup $[\mathbb{Z}^{*n}, \mathbb{Z}^{*n}]$.

\end{proof}

\subsection{Examples}

\subsection{The free algebra on $n$ weakly invertible objects}

Up until now, we've been working under the convention that by `invertible' objects we mean stictly invertible --- $x \otimes x^* = I$. As an additional exercise, we can ask ourselves how all of this would change if we permitted our objects to be only weakly invertible, that is $x \otimes x^* \cong I$. The situation is actually quite elegant, in that the effect of weakening in our objects can be offset completely by the effect of also weakening our algebra homomorphisms, such that we won't need to calculate any new free algebras other than those given by Theorem \ref{colimthm}. Before proving this though, we first to need to set out some definitions.

\begin{defi} Given an $\mathrm{E}G$-algebra $X$, we denote by $X_{\mathrm{wkinv}}$ the category whose
\begin{itemize}
\item objects are tuples $(x, x^*, \eta, \epsilon)$, where $x$ and $x^*$ are objects of $X$ and $\eta: I \to x^* \otimes x$ and $\epsilon : x \otimes x^* \to I$ are morphisms such that the composites
\begin{eq*} \xymatrix{
x \ar[r]^-{id \otimes \eta} & x \otimes x^* \otimes x \ar[r]^-{\epsilon \otimes id} & x &
x^* \ar[r]^-{\eta \otimes id} & x^* \otimes x \otimes x^* \ar[r]^-{id \otimes \epsilon} & x^* }
\end{eq*}
are identity morphisms.
\item maps $(f, f^*): (x, x^*, \eta_x, \epsilon_x) \to (y, y^*, \eta_y, \epsilon_y)$ are pairs $f: x \to y$, $f^* : x^* \to y^*$ of morphisms such that the diagrams
\begin{eq*} \xymatrix{
& I \ar[dl]_{\eta_x} \ar[dr]^{\eta_y} & & x \otimes x^* \ar[rr]^{f \otimes f^*} \ar[dr]_{\epsilon_x} & & y \otimes y^* \ar[dl]^{\epsilon_y} \\
x^* \otimes x \ar[rr]_{f^* \otimes f} & & y \otimes y^* & & I & }
\end{eq*}
commute.
\end{itemize}
\end{defi}

\begin{defi}\label{weakmonfunc} Let $(X, \alpha)$ and $(Y, \beta)$ be $\mathrm{E}G$-algebras. A weak $\mathrm{E}G$-algebra homorphism between them is a weak monoidal functor $\psi: X \to Y$ such that all diagrams of the form
\begin{eq*} \xymatrix{
\psi( x_1 \otimes ... \otimes x_m) \ar[r]^-{\sim} \ar[d]_{\psi(\alpha(g; h_1, ... h_m))} &  \psi(x_1) \otimes ... \otimes \psi(x_m) \ar[d]^{\beta(g; \psi(h_1), ... \psi(h_m))} \\
\psi( y_{\pi(g)^{-1}(1)} \otimes ... \otimes y_{\pi(g)^{-1}(m)}) \ar[r]^-{\sim} &  \psi(y_{\pi(g)^{-1}(1)}) \otimes ... \otimes \psi(y_{\pi(g)^{-1}(m)}) }
\end{eq*}
commute.
\end{defi}

\begin{defi} We denote by $\mathrm{E}G\mathrm{Alg}_W$ the 2-category of $\mathrm{E}G$-algebras, weak $\mathrm{E}G$-algebra homomorphisms, and weak monoidal transformations.\end{defi}

Now we can properly express what we mean by the free algebras on weakly invertible objects being the same as those in the strict case.

\begin{thm} The algebra $L\mathbb{G}_n$ is also the free $\mathrm{E}G$-algebra on $n$ weakly invertible objects. Specifically, for any other $\mathrm{E}G$-algebra $X$ there is an equivalence of categories
\begin{eq*} \mathrm{E}G\mathrm{Alg}_W(L\mathbb{G}_n, X) \simeq (X_{\mathrm{wkinv}})^n \end{eq*}
\end{thm}
\begin{proof}
We begin by defining a functor $F : \mathrm{E}G\mathrm{Alg}_W(L\mathbb{G}_n, X) \to (X_{\mathrm{wkinv}})^n$. On weak maps, $F$ acts as 
\begin{eq*} F( \, \psi: L\mathbb{G}_n \to X \, ) = \big\{ \, ( \, \psi(z_i), \, \psi(z_i^*), \, I \xrightarrow{\sim} \psi(I) \xrightarrow{\sim} \psi(z_i^*)\psi(z_i), \, \psi(z_i)\psi(z_i^*) \xrightarrow{\sim} \psi(I) \xrightarrow{\sim} I \, ) \, \big\}_{i \in \{ 1, ..., n \} } \end{eq*}
where the $z_i$ are the generators of $\mathbb{Z}^{*n}$ and the isomorphisms are those given by $\psi$ being a weak moniodal functor. On weak monoidal transformations, $F$ acts as
\begin{eq*} F( \, \theta : \psi \to \chi \, ) = \big\{ \, ( \, \theta_{z_i}, \, \theta_{z_i^*} \, ) \, \big\}_{i \in \{ 1, ..., n \} }. \end{eq*}
This choice does satisfy the condition on morphisms of $(X_{\mathrm{wkinv}})^n$, since we can build the required commuting diagrams out of smaller ones given by $\theta$ being a weak monoidal transfomation:
\begin{eq*} \xymatrix{
& I \ar[dl]_{\sim} \ar[dr]^{\sim} & & \psi(z_i) \otimes \psi(z_i^*) \ar[rr]^{\theta_{z_i} \otimes \theta_{z_i^*}} \ar[d]_{\sim} & & \chi(z_i) \otimes \chi(z_i^*) \ar[d]^{\sim} \\
\psi(I) \ar[d]_{\sim} \ar[rr]^{\theta_I} & & \chi(I) \ar[d]^{\sim} & \psi(I) \ar[dr]^{\sim} \ar[rr]^{\theta_I} & & \chi(I) \ar[dl]_{\sim} \\
\psi(z_i^*) \otimes \psi(z_i) \ar[rr]^{\theta_{z_i^*} \otimes \theta_{z_i}} & & \chi(z_i^*) \otimes \chi(z_i) & & I & }.
\end{eq*}

Now we need to check if $F$ is an equivalence of categories. First, let $\big\{ ( x_i, x_i^*, \eta_i, \epsilon_i ) \big\}_{i \in \{1, ..., n \} }$ be an arbitrary object of $(X_{\mathrm{wkinv}})^n$. We can construct a weak algebra map $\psi: L\mathbb{G}_n \to X$ from it as follows. Define
\begin{eq*} \psi(I) = I, \quad \psi(z_i) = x_i, \quad \psi(z_i^*) = x_i^* \end{eq*}
and choose the isomorphisms
\begin{eq*} \begin{array}{rllllll}
		\psi_I & : & I \to \psi(I) & = & \mathrm{id}_I & : & I \to I \\
		\psi_{z_i, z_i^*} & : & \psi(z_i) \otimes \psi(z_i^*) \to \psi(I) & = & \epsilon_i & : & x_i \otimes x_i^* \to I \\
		\psi_{z_i^*, z_i} & : & \psi(z_i^*) \otimes \psi(z_i) \to \psi(I) & = & \eta_i^{-1} & : & x_i^* \otimes x_i \to I
		\end{array} .
\end{eq*}
Then for any $w, w' \in \mathrm{Ob}(L\mathbb{G}_n)$ such that $d(w \otimes w') = d(w) \otimes d(w')$, where $d(-)$ is the minimal generator decomposition from Definition \ref{mgd}, set 
\begin{eq*} \psi(w \otimes w') = \psi(w) \otimes \psi(w'), \quad \quad \psi_{w, w'} = \mathrm{id}_{\psi(w) \otimes \psi(w')} \end{eq*}
This is enough to determine the value of $\psi$ on all of the remaining objects, via successive decompositions. For the isomorphisms, first note that the ones we have already defined satisfy the associativity and unitality required of weak monoidal functors. Now consider some $w, w'$ with $d(w \otimes w') \neq d(w) \otimes d(w')$. The fact that they differ implies that tensoring $w$ with $w'$ causes some cancellation of inverses to occur where the end of one sequence meets the beginning of another. In particular, if we let $b$ be the last term in the minimal generator decomposition of $w$, and let $c = w'$, then we conclude that the length $d(b \otimes c)$ is smaller than the length of $d(c)$. Let $a$ be the product of the rest of $d(w)$, so that $a \otimes b = w$. Then we can use requirement for associativity,
\begin{eq*} \xymatrix{
\psi(a) \otimes \psi(b) \otimes \psi(c) \ar[rr]^{id \otimes \psi_{b, c}} \ar[d]_{\psi_{a, b} \otimes id} && \psi(a) \otimes \psi(b \otimes c) \ar[d]^{\psi_{a, b \otimes c}} \\
\psi(a \otimes b) \otimes \psi(c) \ar[rr]_{\psi_{a \otimes b, c}} && \psi(a \otimes b \otimes c) },
\end{eq*}
to define $\psi_{w, w'} = \psi{a\otimes b, c}$ in terms of three other isomorphisms that each have strictly smaller decompositions. Repeating this process will therefore eventually yield a definition in terms of our previous isomorphisms.

By Proposition \ref{allmapsaction}, every morphism in $L\mathbb{G}_n$ can be written as $\alpha(g; \mathrm{id}_{w_1}, ..., \mathrm{id}_{w_m})$ for some $g \in G(m)$, $w_i \in \mathbb{Z}^{*n}$. The action of $\psi$ on morphisms is thus determined by the diagram in Definition \ref{weakmonfunc}, that is
\begin{eq*} \psi(\alpha(g; w_1, ... w_m)) \, = \, \psi_{\mathbf{w}_{\pi(g)^{-1}}} \circ \beta(\, g \, ; \, \mathrm{id}_{\psi(w_1)}, \, ..., \, \mathrm{id}_{\psi(w_m)}\, ) \circ \psi_{\mathbf{w}}^{-1}. \end{eq*} 
However, morphisms do not have a unique representation of this form, so we must check that whenever we have different representations of the same morphism
\begin{eq*} \alpha(g; \mathrm{id}_{w_1}, ..., \mathrm{id}_{w_m}) = \alpha(g'; \mathrm{id}_{w_1'}, ..., \mathrm{id}_{w_{m'}'}) \end{eq*}
their diagrams give the same image under $\psi$. There are two cases to consider here;
\begin{eq*} \alpha(g; \mathrm{id}_{w_1}, ..., \mathrm{id}_{w_m}) = \alpha( \, g \otimes e_k \, ; \, \mathrm{id}_{w_1}, \, ..., \, \mathrm{id}_{w_m}, \, \mathrm{id}_{v_1}, \, ..., \, \mathrm{id}_{v_k} \, ) \end{eq*}
when $v_1 \otimes ... \otimes v_k = 0$, which comes from the edges of the colimit diagram $D$ in Theorem \ref{colimthm}; and
\begin{eq*} \begin{array}{rll}
		\alpha(g; \mathrm{id}_{w_1}, ..., \mathrm{id}_{w_m}) & = & \alpha(\, h \, ; \, \mathrm{id}_{w_1'}, \, ..., \, \mathrm{id}_{w_{m'}} \, ) \\
		&& \circ \, \, \alpha(\, j \, ; \, \mathrm{id}_{w_1''}, \, ..., \, \mathrm{id}_{w_{m''}''} \, ) \\
		&& \circ \, \, \alpha(\, h^{-1} \, ; \, \mathrm{id}_{w_1'}, \, ..., \, \mathrm{id}_{w_{m'}'} \, ) \\
		&& \circ \, \, \alpha(\, j^{-1} \, ; \, \mathrm{id}_{w_1''}, \, ..., \, \mathrm{id}_{w_{m''}''} \, ) \\
		& = & \mathrm{id}_{w_1 \otimes ... \otimes w_m} 
		\end{array}.
\end{eq*}
for $ \alpha(\, h \, ; \, \mathrm{id}_{w_1'}, \, ..., \, \mathrm{id}_{w_{m'}} \, ), \alpha(\, j \, ; \, \mathrm{id}_{w_1''}, \, ..., \, \mathrm{id}_{w_{m''}''} \, ) \in \mathbb{G}_n(w_1 \otimes ... \otimes w_m,  w_1 \otimes ... \otimes w_m)$, which comes from the abelianisation of the vertices of $D$. All other ways for a morphism to have different representations must be generated by successive examples of these cases, since otherwise they wouldn't be coequalised by the colimit in Theorem \ref{colimthm}. In the first case we just have
\begin{eq*} \begin{array}{rl}
		& \psi( \, \alpha( \, g \otimes e_k \, ; \, \mathrm{id}_{w_1}, \, ..., \, \mathrm{id}_{w_m}, \, \mathrm{id}_{v_1}, \, ..., \, \mathrm{id}_{v_k} \, ) \, ) \\
		= & \psi_{\mathbf{w}_{\pi(g)^{-1}}, \mathbf{v}} \circ \beta(\, g \otimes e_k \, ; \, \mathrm{id}_{\psi(w_1)}, \, ..., \, \mathrm{id}_{\psi(w_m)}, \, \mathrm{id}_{\psi(v_1)}, \, ..., \, \mathrm{id}_{\psi(v_k)} \, ) \circ \psi_{\mathbf{w}, \mathbf{v}}^{-1} \\
		= & \big( \psi_{\mathbf{w}_{\pi(g)^{-1}}} \otimes \psi_{\mathbf{v}} \big) \circ \big( \beta( g ; \mathrm{id}_{\psi(w_1)}, ..., \mathrm{id}_{\psi(w_m)}) \otimes \mathrm{id}_{\psi(\mathbf{v})} \big) \circ \big( \psi_{\mathbf{w}}^{-1} \otimes \psi_{\mathbf{v}}^{-1} \big) \\
		= & \big( \psi_{\mathbf{w}_{\pi(g)^{-1}}} \circ \beta( g ; \mathrm{id}_{\psi(w_1)}, ..., \mathrm{id}_{\psi(w_m)}) \circ \psi_{\mathbf{w}}^{-1} \big) \otimes \big( \psi_{\mathbf{v}} \circ \mathrm{id}_{\psi(\mathbf{v})} \circ \psi_{\mathbf{v}}^{-1} \big) \\
		= & \psi_{\mathbf{w}_{\pi(g)^{-1}}} \circ \beta( g ; \mathrm{id}_{\psi(w_1)}, ..., \mathrm{id}_{\psi(w_m)}) \circ \psi_{\mathbf{w}}^{-1} \\
		=& \psi( \, \alpha(g; \mathrm{id}_{w_1}, ..., \mathrm{id}_{w_m}) \, )
		\end{array}.
\end{eq*}
as required. The second case is more subtle. We begin by expanding
\begin{eq*} \begin{array}{rl}
		& \psi( \, \alpha( \, g \, ; \, \mathrm{id}_{w_1}, \, ..., \, \mathrm{id}_{w_m} \, ) \\
		= & \psi( \, \alpha(\, h \, ; \, \mathrm{id}_{w_1'}, \, ..., \, \mathrm{id}_{w_{m'}} \, ) \, ) \\
		& \circ \, \, \psi( \, \alpha(\, j \, ; \, \mathrm{id}_{w_1''}, \, ..., \, \mathrm{id}_{w_{m''}''} \, ) \, ) \\
		& \circ \, \, \psi( \, \alpha(\, h^{-1} \, ; \, \mathrm{id}_{w_1'}, \, ..., \, \mathrm{id}_{w_{m'}'} \, ) \, ) \\
		&\circ \, \, \psi( \, \alpha(\, j^{-1} \, ; \, \mathrm{id}_{w_1''}, \, ..., \, \mathrm{id}_{w_{m''}''} \, ) \, ) \\
		= & \psi_{\mathbf{w'}} \circ \beta(\, h \, ; \, \mathrm{id}_{\psi(w_1')}, \, ..., \, \mathrm{id}_{\psi(w_{m'})} \, ) \circ \psi_{\mathbf{w'}}^{-1} \\
		& \circ \, \, \psi_{\mathbf{w''}} \circ\beta(\, j \, ; \, \mathrm{id}_{\psi(w_1'')}, \, ..., \, \mathrm{id}_{\psi(w_{m''}'')} \, ) \circ \psi_{\mathbf{w''}}^{-1} \\
		& \circ \, \, \psi_{\mathbf{w'}} \circ \beta(\, h^{-1} \, ; \, \mathrm{id}_{\psi(w_1')}, \, ..., \, \mathrm{id}_{\psi(w_{m'}')} \, ) \circ \psi_{\mathbf{w'}}^{-1}  \\
		&\circ \, \, \psi_{\mathbf{w''}} \circ \beta(\, j^{-1} \, ; \, \mathrm{id}_{\psi(w_1'')}, \, ..., \, \mathrm{id}_{\psi(w_{m''}'')} \, ) \circ \psi_{\mathbf{w''}}^{-1} \\
		\end{array}.
\end{eq*}
Here the objects $w_i, w_i', w_i''$ are all in $\mathbb{G}_n \subseteq L\mathbb{G}_n$, and so we know their minimal generator decompositions are also in $\mathbb{G}_n$. It follows that $d(w_i \otimes w_j) = d(w_i) \otimes d(w_j)$ for all $i,j$, and hence by our definition of $\psi$ we have $\psi(w_i \otimes w_j) = \psi(w_i) \otimes \psi(w_j)$ and also $\psi_{\mathbf{w}_{\sigma}} = id$ for any permuation $\sigma$ --- and the same for $\mathbf{w'}$ and $\mathbf{w''}$. Also, note that since we are working in $\mathbb{G}_n(w_1 \otimes ... \otimes w_m,  w_1 \otimes ... \otimes w_m)$, all of the action morphisms in the above composite have the same source and target, $\psi(w_1 \otimes ...\otimes w_m)$. This object is weakly invertible, because each of the $w_i$ are invertible. However, the automorphisms of any weakly invertible object are isomorphic to the automorphisms of the unit object, as in the proof of Proposition \ref{zerotree}, and hence form an abelian group, by an Eckmann-Hilton argument like in the proof of Theorem \ref{colimthm}. Therefore we may permute these action morphisms freely, and so
\begin{eq*} \begin{array}{rl}
& \psi( \, \alpha( \, g \, ; \, \mathrm{id}_{w_1}, \, ..., \, \mathrm{id}_{w_m} \, ) \\
		= & \beta(\, h \, ; \, \mathrm{id}_{\psi(w_1')}, \, ..., \, \mathrm{id}_{\psi(w_{m'})} \, ) \\
		& \circ \, \, \beta(\, h^{-1} \, ; \, \mathrm{id}_{\psi(w_1')}, \, ..., \, \mathrm{id}_{\psi(w_{m'}')} \, )  \\
		& \circ \, \, \beta(\, j \, ; \, \mathrm{id}_{\psi(w_1'')}, \, ..., \, \mathrm{id}_{\psi(w_{m''}'')} \, ) \\
		& \circ \, \, \beta(\, j^{-1} \, ; \, \mathrm{id}_{\psi(w_1'')}, \, ..., \, \mathrm{id}_{\psi(w_{m''}'')} \, ) \\
		= & \mathrm{id}_{\psi(w_1) \otimes ... \otimes \psi(w_m)} \\
		= & \psi_{\mathbf{w}} \circ \beta(\, e_m \, ; \, \mathrm{id}_{\psi(w_1)}, \, ..., \, \mathrm{id}_{\psi(w_{m})} \, ) \circ \psi_{\mathbf{w}}^{-1}
		\end{array}
\end{eq*}
as required.

With $\psi$ now fully defined, notice that
\begin{eq*} \begin{array}{rll}
		F(\psi) & = & \big\{ \, ( \, \psi(z_i), \, \psi(z_i^*), \, I \xrightarrow{\sim} \psi(I) \xrightarrow{\sim} \psi(z_i^*)\psi(z_i), \, \psi(z_i)\psi(z_i^*) \xrightarrow{\sim} \psi(I) \xrightarrow{\sim} I \, ) \, \big\}_{i \in \{ 1, ..., n \} } \\
		& = & \big\{ \, ( \, x_i, \, x_i^*, \, \eta_i, \, \epsilon_i \, ) \, \big\}_{i \in \{ 1, ..., n \} } \\
		\end{array}
\end{eq*}
which was our arbitrary object in $(X_{\mathrm{wkinv}})^n$. Therefore, $F$ is surjective on objects.

Next, choose an arbitrary monoidal transformation $\theta : \psi \to \chi$ from $\mathrm{E}G\mathrm{Alg}_W(L\mathbb{G}_n, X)$. By naturality, for any $w, w' \in \mathrm{Ob}(L\mathbb{G}_n)$ we have that
\begin{eq*} \xymatrix{
\psi(w) \otimes \psi(w') \ar[r]^-{\sim} \ar[d]_{\theta_w \otimes \theta_{w'}} & \psi(w \otimes w') \ar[d]^{\theta_{w \otimes w'}} \\
\chi(w) \otimes \chi(w') \ar[r]^-{\sim} & \chi(w \otimes w') }
\end{eq*}
or equivalently, $\theta_{w \otimes w'} = \chi_{w, w'} \circ (\theta_w \otimes \theta_{w'}) \circ \psi_{w, w'}^{-1}$. It follows from this that the components of $\theta$ are generated by the components on the generators of $\mathrm{Ob}(L\mathbb{G}_n)$, namely $\{ \, ( \, \theta_{z_i}, \, \theta_{z_i^*} \, ) \, \}_{i \in \{ 1, ..., n \} }$. But this is just $F(\theta)$, and thus any monoidal transformation $\theta$ is determined uniquely by its image under $F$, or in other words $F$ is faithful.

Finally, let $\psi, \chi$ be objects of $\mathrm{E}G\mathrm{Alg}_W(L\mathbb{G}_n, X)$, and choose an arbitrary map $\{ \, ( \, f_i, \, f^*_i \, ) \, \}_{i \in \{ 1, ..., n \} } : F(\psi) \to F(\chi)$ from $(X_{\mathrm{wkinv}})^n$. We can use this to construct a monoidal transformation $\theta : \psi \to \chi$ via the reverse of process we just used. Specifically, if we define
\begin{eq*} \theta_I = \chi_I \circ \psi_I^{-1}, \quad \quad \theta_{z_i} =  f_i, \quad \quad \theta_{z_i^*} = f_i^*\end{eq*}
then these will automatically form the naturality squares
\begin{eq*} \xymatrix{
\psi(z_i) \otimes \psi(z_i^*) \ar[rr]^-{\psi_{z_i, z_i^*}} \ar[dd]_{f_i \otimes f_i^*} & & \psi(I) \ar[d]^{\psi_I^{-1}} & \psi(z_i^*) \otimes \psi(z_i) \ar[rr]^-{\psi_{z_i^*, z_i}} \ar[dd]_{f_i^* \otimes f_i} & & \psi(I) \ar[d]^{\psi_I^{-1}} \\
& & I \ar[d]^{\chi_I} & & & I \ar[d]^{\chi_I}\\
\chi(z_i) \otimes \chi(z_i^*) \ar[rr]^-{\chi_{z_i, z_i^*}} & & \chi(I) & \chi(z_i^*) \otimes \chi(z_i) \ar[rr]^-{\chi_{z_i^*, z_i}} & & \chi(I)}
\end{eq*}
since these are just the conditions for $\{ \, ( \, f_i, \, f^*_i \, ) \, \}_{i \in \{ 1, ..., n \} }$ to be a map $F(\psi) \to F(\chi)$ in $(X_{\mathrm{wkinv}})^n$. Repeatedly applying the naturality condition $\theta_{w \otimes w'} = \chi_{w, w'} \circ (\theta_w \otimes \theta_{w'}) \circ \psi_{w, w'}^{-1}$ will then generate all of the other components of $\theta$, in a way that clearly satisfies naturality. Thus we have a well-defined monoidal transformation $\theta : \psi \to \chi$, and applying $F$ to it gives
\begin{eq*} \begin{array}{rll}
		F(\theta) & = & \big\{ \, ( \, \theta_{z_i}, \, \theta_{z_i^*} \, ) \, \big\}_{i \in \{ 1, ..., n \} } \\
		& = & \big\{ \, ( \, f_i, \, f_i^* \, ) \, \big\}_{ i \in \{ 1, ..., n \} },
		\end{array}
\end{eq*}
our arbitrary map. Therefore $F$ is full and, putting this together with the previous results, is an equivalence of categories.
\end{proof}

\begin{thebibliography}{1}

\bibitem{graphicalmon}
Peter Selinger.
\it{A survey of graphical languages for monoidal categories}.
http://www.mscs.dal.ca/~selinger/papers/graphical.pdf

\bibitem{operadborel} 
Nick Gurski. 
\it{Operads, tensor products, and the categorical Borel construction}. 
 arXiv:1508.04050 [math.CT].

\bibitem{eckhil}
Eckmann, B.; Hilton, P. J. 
\it{Group-like structures in general categories. I. Multiplications and comultiplications}
Mathematische Annalen, 145 (3): 227–255

\end{thebibliography}


\end{document}