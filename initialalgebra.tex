\section{Free $\mathrm{E}G$-algebras}

Our ultimate goal for this chapter is to understand the free braided monoidal category on an finite number of invertible objects. Thus, now that we have a firm grasp on action operads and their algebras, we should begin to think about the various free constructions they can form. 

\subsection{The free algebra on $n$ objects} 

We begin with the simplest case, which we will use extensively when calculating the invertible case later on. In the paper \cite{operadborel}, Gurski establishes how to contruct free $G$-operad algebras through the use of the monad $\mathrm{E}G$. What follows in this section is a quick summary of the results which will be useful for our purposes. For a more detailed treatment please refer to \cite{operadborel}.

\begin{prop}\label{freealg} There exists a free $\mathrm{E}G$-algebra on $n$ objects. That is, there is an $\mathrm{E}G$-algebra $Y$ such that for any other $\mathrm{E}G$-algebra $X$, we have an isomorphism of categories
\begin{eq*} \mathrm{E}G\mathrm{Alg}_S(Y, X) \cong X^n \end{eq*}
\end{prop}
\begin{proof}
There is an obvious forgetful 2-functor \( U: \mathrm{E}G\mathrm{Alg}_S \to \mathrm{Cat}\) sending $\mathrm{E}G$-algebras to their underlying categories. $U$ has a left adjoint, which we call the free 2-functor \( F : \mathrm{Cat} \to \mathrm{E}G\mathrm{Alg}_S \) adjoint to it. It follows immediately that
\begin{eq*}\begin{array}{rll}
		U(X)^n & = & \mathrm{Cat}(\{z_1, ..., z_n\}, U(X) ) \\
		& \cong & \mathrm{E}G\mathrm{Alg}_S( F(\{z_1, ..., z_n\}), X) 
		\end{array}
\end{eq*}
where $\{z_1, ..., z_n\}$ is any set with $n$ distinct elements. Since $X$ and $U(X)$ are obviously isomorphic as categories, this shows that $F(\{z_1, ..., z_n\})$ is the free algebra on $n$ objects as required. 
\end{proof}

\begin{defn}\label{Gndef} Let $\{ z_1, ..., z_n \}$ be an $n$-object set, which we will also consider as a discrete category. Then we will denote by $\mathbb{G}_n$ the $\mathrm{E}G$-algebra whose underlying category is $\mathrm{E}G( \{ z_1, ..., z_n \})$ and whose action
\begin{eq*} \alpha : \mathrm{E}G\big( \, \mathrm{E}G( \{ z_1, ..., z_n \}) \, \big) \to \mathrm{E}G( \{ z_1, ..., z_n \}) \end{eq*}
is the appropriate component of the multiplication natural transformation $\mu: \mathrm{E}G \circ \mathrm{E}G \to \mathrm{E}G$ of the 2-monad $\mathrm{E}G$.
\end{defn}

\begin{thm} $\mathbb{G}_n$ is the free $\mathrm{E}G$-algebra on $n$ objects. That is,
\begin{eq*}  F(\{z_1, ..., z_n\}) = \mathbb{G}_n \end{eq*}
\end{thm}
\begin{proof}
\end{proof}

\cref{Gndef} is a fairly opaque definition, so we'll spend a little time upacking it. Recall from \cref{monaddef} that $\mathrm{E}G( \{ z_1, ..., z_n \})$ is the coequalizer of the maps
\begin{eq*} \begin{tikzcd}
\coprod_{m \geq 0} \mathrm{E}G(m) \times G(m) \times \{ z_1, ..., z_n \}^m \ar[r, shift left] \ar[r, shift right] & \coprod_{m \geq 0} \mathrm{E}G(m) \times \{ z_1, ..., z_n \}^m
\end{tikzcd} \end{eq*}
that comes from the action of $G(m)$ on $\mathrm{E}G(m)$ by multiplication on the right,
\begin{eq*} \begin{array}{rll}
		\mathrm{E}G(m) \times G(m) & \to & \mathrm{E}G(m) \\
		(g, h) & \mapsto & gh \\
		( \, !: g \to g', \mathrm{id}_h \, ) & \mapsto & !: gh \to g'h
		\end{array}
\end{eq*}
and the action of $G(m)$ on $\{ z_1, ..., z_n \}^m$ by permutation,
\begin{eq*} \begin{array}{rll}
		G(m) \times \{ z_1, ..., z_n \}^m & \to & \{ z_1, ..., z_n \}^m \\
		( \, h \, ; \, x_1, ..., x_m \, ) & \mapsto & (x_{\pi(h^{-1})(1)}, ..., x_{\pi(h^{-1})(m)}) \\
		 \, (\mathrm{id}_h \, ; \, \mathrm{id}_{(x_1, ..., x_m)} \, ) & \mapsto & \mathrm{id}_{(x_{\pi(h^{-1})(1)}, ..., x_{\pi(h^{-1})(m)})}
		\end{array}
\end{eq*}

First, objects in this algebra are equivalence classes of tuples $(g; x_1, ..., x_m)$, for $g \in G(m)$ and $x_i \in \{z_1, ..., z_n\}$, under the relation
\begin{eq*} ( \, gh \, ; \, x_1, \, ..., \, x_m \, ) \sim ( \, g \, ; \, x_{\pi(h)^{-1}(1)}, \, ..., \, x_{\pi(h)^{-1}(m)} \, )\end{eq*}
Notice that using this relation we can rewrite any object uniquely in the form $[e; x_1, ..., x_m]$ for some $m \in \mathbb{N}$ and $x_i \in \{z_1, ..., z_n\}$. This means that each equivalence class is just the tensor product $x_1 \otimes ... \otimes x_m$ in the underlying monoidal category of $\mathbb{G}_n$, for some unique sequence of generators. That is, we can view the objects of $\mathbb{G}_n$ as elements of the monoid freely generated by each of the $z_i$, or in other words:

\begin{lem} \label{Gnobj} $\mathrm{Ob}(\mathbb{G}_n)$ is the free monoid on $n$ generators, $\mathbb{N}^{\ast n}$, the free product of $n$ copies of $\mathbb{N}$. \end{lem}

Similarly, the morphisms of $\mathbb{G}_n$ are the maps
\begin{eq*} (! ; \mathrm{id}_{x_1}, ..., \mathrm{id}_{x_m}) : ( g ; x_1, ..., x_m ) \to ( g' ; x_1, ..., x_m )\end{eq*}
with $g, g' \in G(m)$ and $x_i \in \{z_1, ..., z_n \}$. Using the relation $\sim$ on objects we can rewrite each of these morphisms in the form
\begin{eq*} [h ; \mathrm{id}_{y_1},...,\mathrm{id}_{y_m}] \, : \, y_1 \otimes ... \otimes y_m \, \to \, y_{\pi(h^{-1})(1)} \otimes ... \otimes y_{\pi(h^{-1})(m)} \end{eq*}
where
\begin{eq*} h = g' g^{-1}, \quad \quad y_i = x_{\pi(g^{-1})(i)} \end{eq*}
 The $\mathrm{E}G$-action of $\mathbb{G}_n$ is permutation and tensor product, and the action on morphisms is given by
\begin{eq*} \alpha( \, g \, ; \, [h_1; \mathrm{id}_{x_1}, ..., \mathrm{id}_{x_{m_1}}], \, ..., \, [h_k; \mathrm{id}_{x_1}, ..., \mathrm{id}_{x_{m_k}}] \, ) = [ \, \mu(g;h_1, .., h_k) \, ; \, \mathrm{id}_{x_1}, \, ..., \, \mathrm{id}_{x_{m_k}} \, ] \end{eq*}
Notice that using tensor product notation the object $[e; x]$ is simply $x$, and so $[e; \mathrm{id}_x] = \mathrm{id}_{[e;x]}$ should be written as $\mathrm{id}_x$. Hence by the above $[g; \mathrm{id}_{x_1}, ..., \mathrm{id}_{x_m}]$ is really just $\alpha(g; \mathrm{id}_{x_1}, ..., \mathrm{id}_{x_m})$, and so we have the following:

\begin{lem} \label{Gnmapsaction} Each morphism of $\mathbb{G}_n$ can be expressed uniquely as an action morphism $\alpha(g; \mathrm{id}_{x_1}, ..., \mathrm{id}_{x_m})$, for some $g, g' \in G(m)$ and $x_i \in \{z_1, ..., z_n \}$. \end{lem}

From this, we can also determine $\mathbb{G}_n$'s connected components. 

\begin{prop}\label{Gnconcomp} The connected components of $\mathbb{G}_n$ are $\mathbb{N}^n$, with the assignment $[ \,\, ] : \mathbb{N}^{*n} \to \mathbb{N}^n$ of objects to their component being the quotient map of abelianisation.
\end{prop}
\begin{proof}
By \cref{Gnmapsaction}, all morphisms in $\mathbb{G}_n$ can be written uniquely as $\alpha(g; \mathrm{id}_{x_1}, ..., \mathrm{id}_{x_m})$, for some $g \in G(m)$ and $x_i \in \{z_1, ..., z_n \}$, the set of generators of $\mathbb{N}^{*n}$. Since maps of this form have source $x_1 \otimes ... \otimes x_m$ and target $x_{\pi(g^{-1})(1)} \otimes ... \otimes x_{\pi(g^{-1})(m)}$, we see that there can only exist a morphism between two objects if they can expanded as tensor products which are permutations of one another. Moreover, for any two objects where this is true --- say source $x_1 \otimes ... \otimes x_m$ and target $x_{\sigma^{-1}(1)} \otimes ... \otimes x_{\sigma^{-1}(m)}$ --- we can always find a map $\alpha(g; \mathrm{id}_{x_1}, ..., \mathrm{id}_{x_m})$ between them, because $\pi: G(m) \to S_m$ is surjective and so there must exist at least one $g$ with $\pi(g) = \sigma$. Thus two objects of $\mathbb{G}_n$ share a connected component if and only if they are tensor products that differ by a permutation, and therefore the canonical map $[ \,\, ] : \mathrm{Ob}(\mathbb{G}_n) \to \pi_0(\mathbb{G}_n)$ sending each object to its connected component is just the map which forgets about these permutations, making the free product on $\mathbb{N}^{*n}$ commutative. That is, $[ \,\, ]$ is the quotient map for the abelianisation $\mathrm{ab} : \mathbb{N}^{*n} \to (\mathbb{N}^{*n})^{\mathrm{ab}}$, and so $\pi_0(\mathbb{G}_n) = \mathbb{N}^n$.
\end{proof}

\subsection{The free algebra on $n$ invertible objects}

We saw in \cref{freealg} that the existence of a free $\mathrm{E}G$-algebra on $n$ objects can be proven by taking the left adjoint of a 2-functor which forgets about the algebra structure. Now we want to extend this idea into the realm of algebras on invertible objects. For the analogous approach, we will need to find a new 2-functor that lets us forget about non-invertible objects, and then hopefully we can find its left adjoint too, and use it to freely add inverses to $\mathbb{G}_n$. First though, we need to make this concept of `forgetting non-invertible objects' a little more precise.

\begin{defn} Given an $\mathrm{E}G$-algebra $X$, we denote by $X_{\mathrm{inv}}$ the sub-$\mathrm{E}G$-algebra containing all invertible objects in $X$ and the isomorphisms between them. \end{defn}

Note that this is indeed a well-defined $\mathrm{E}G$-algebra. If $x_1, ..., x_m$ are invertible objects with inverses $x_1^*, ..., x_m^*$, then $\alpha(g; x_1, ..., x_m)$ is an invertible object with inverse $\alpha(g; x_m^*, ..., x_1^*)$, since 
\begin{eq*} \begin{array}{ll}
		& \alpha(g; x_1, ..., x_m) \otimes \alpha(g; x_m^*, ..., x_1^*) \\
		= & \big( x_{\pi(g)^{-1}(1)} \otimes ... \otimes x_{\pi(g)^{-1}(m)} \big) \otimes \big( x_{\pi(g)^{-1}(m)}^* \otimes ... \otimes x_{\pi(g)^{-1}(1)}^* \big) \\
		= & I \\
		& \\
		& \alpha(g; x_m^*, ..., x_1^*) \otimes \alpha(g; x_1, ..., x_m) \\
		= & \big( x_{\pi(g)^{-1}(m)}^* \otimes ... \otimes x_{\pi(g)^{-1}(1)}^* \big) \otimes \big( x_{\pi(g)^{-1}(1)} \otimes ... \otimes x_{\pi(g)^{-1}(m)} \big) \\
		= & I
		\end{array}
\end{eq*}
Likewise, if $f_1, ..., f_m$ are isomorphisms from invertible objects $x_1, ..., x_m$ to invertible objects $y_1, ..., y_m$, then $\alpha(g; f_1, ..., f_m)$ is a map from the invertible object $\alpha(g; x_1, ..., x_m)$ to the invertible object $\alpha(g; y_1, ..., y_m)$, and it has an inverse $\alpha(g^{-1}; f_{\pi(g)(1)}^{-1}, ..., f_{\pi(g)(m)}^{-1})$, since
\begin{eq*} \begin{array}{ll}
		& \alpha\big( \, g^{-1} \, ; \, f_{\pi(g)(1)}^{-1}, \, ..., \, f_{\pi(g)(m)}^{-1} \, \big) \circ \alpha( \, g \, ; \, f_1, ..., f_m \,) \\
		= & \alpha\big( \, g^{-1}g \, ; \, f_1^{-1} f_1, \, ..., \, f_m^{-1} f_m \, \big) \\
		= & \mathrm{id}_{x_1 \otimes ... \otimes x_m} \\
		& \\
		& \alpha( \, g \, ; \, f_1, ..., f_m \,) \circ \alpha\big( \, g^{-1} \, ; \, f_{\pi(g)(1)}^{-1}, \, ..., \, f_{\pi(g)(m)}^{-1} \, \big) \\
		= & \alpha\big( \, gg^{-1} \, ; \, f_{\pi(g)(1)} f_{\pi(g)(1)}^{-1}, \, ..., \, f_{\pi(g)(m)} f_{\pi(g)(m)}^{-1} \, \big) \\
		= & \mathrm{id}_{y_{\pi(g)(1)} \otimes ... \otimes y_{\pi(g)(m)}}
		\end{array}
\end{eq*}
Clearly then, $X_{\mathrm{inv}}$ is the correct algebra for our new forgetful 2-functor to send $X$ to. Knowing this, we can contruct the rest of the functor fairly easily.

\begin{prop} \label{invprop} The assignment $X \mapsto X_{\mathrm{inv}}$ can be extended to a 2-functor $(\_)_{\mathrm{inv}}: \mathrm{E}G\mathrm{Alg}_S \to \mathrm{E}G\mathrm{Alg}_S$.
\end{prop}
\begin{proof}
Let $F: X \to Y$ be a (strict) map of $\mathrm{E}G$-algebras. If $x$ is an invertible object in $X$ with inverse $x^*$, then $F(x)$ is an invertible object in $Y$ with inverse $F(x^*)$, by
\begin{eq*} F(x) \otimes F(x^*) = F(x \otimes x^*) = F(I) = I \end{eq*}
\begin{eq*} F(x^*) \otimes F(x) = F(x^* \otimes x) = F(I) = I \end{eq*}
Since $F$ sends invertible objects to invertible objects, it will also send isomorphisms of invertible objects to isomorphisms of invertible objects. In other words, the map $F: X \to Y$ can be restricted to a map $F_{\mathrm{inv}} : X_{\mathrm{inv}} \to Y_{\mathrm{inv}}$. Moreover, we have that
\begin{eq*} (F \circ G)_{\mathrm{inv}}(x) = F \circ G(x) = F_{\mathrm{inv}} \circ G_{\mathrm{inv}}(x) \end{eq*}
\begin{eq*} (F \circ G)_{\mathrm{inv}}(f) = F \circ G(f) = F_{\mathrm{inv}} \circ G_{\mathrm{inv}}(f) \end{eq*}
and so the assignment $F \mapsto F_{\mathrm{inv}}$ is clearly functorial. Next, let $\theta : F \Rightarrow G$ be an $\mathrm{E}G$-monoidal natural transformation. Choose an invertible object $x$ from $X$, and consider the component map of its inverse, $\theta_{x^*} : F(x^*) \to G(x^*)$. Since $\theta$ is monoidal, we have $\theta_{x^*} \otimes \theta_x = \theta_I = I$ and $\theta_x \otimes \theta_{x^*} = I$, or in other words that $\theta_{x^*}$ is the monoidal inverse of $\theta_x$. We can use this fact to construct a compositional inverse as well, namely $\mathrm{id}_{F(x)} \otimes \theta_{x^*} \otimes \mathrm{id}_{G(x)}$, which can be seen as follows:
\begin{eq*}  \begin{array}{rll}
		\big( \mathrm{id}_{F(x)} \otimes \theta_{x^*} \otimes \mathrm{id}_{G(x)} \big)  \circ \theta_x & = & \theta_x \otimes \theta_{x^*} \otimes \mathrm{id}_{G(x)} \\
		& = &  \mathrm{id}_{G(x)} \\
		&& \\
		\theta_x \circ  \big( \mathrm{id}_{F(x)} \otimes \theta_{x^*} \otimes \mathrm{id}_{G(x)} \big) & = & \mathrm{id}_{F(x)} \otimes \theta_{x^*} \otimes \theta_x \\
		& = &  \mathrm{id}_{F(x)} \\
		\end{array} 
\end{eq*}
Therefore, we see that all the components of our transformation on invertible objects are isomorphisms, and hence we can define a new transformation $\theta_{\mathrm{inv}}: F_{\mathrm{inv}} \Rightarrow G_{\mathrm{inv}}$ whose components are just $(\theta_{\mathrm{inv}})_x = \theta_x$. The assignment $\theta \mapsto \theta_{\mathrm{inv}}$ is also clearly functorial, and thus we have a complete 2-functor $(\_)_{\mathrm{inv}}: \mathrm{E}G\mathrm{Alg}_S \to \mathrm{E}G\mathrm{Alg}_S$.
\end{proof}

\begin{prop} The 2-functor $(\_)_{\mathrm{inv}}: \mathrm{E}G\mathrm{Alg}_S \to \mathrm{E}G\mathrm{Alg}_S$ has a left adjoint, $L: \mathrm{E}G\mathrm{Alg}_S \to \mathrm{E}G\mathrm{Alg}_S$.
\end{prop}
\begin{proof} To begin, consider the 2-monad $\mathrm{E}G(\_)$. This is a finitary monad, that is it preserves all filtered colimits, and it is a 2-monad over $\mathrm{Cat}$, which is locally finitely presentable. It follows from this that $\mathrm{E}G\mathrm{Alg}_S$ is itself locally finitely presentable. Thus if we want to prove $(\_)_{\mathrm{inv}}$ has a left adjoint, we can use the Adjoint Functor Theorem for locally finitely presentable categories, which amounts to showing that $(\_)_{\mathrm{inv}}$ preserves both limits and filtered colimits.
\begin{itemize}
\item Given an indexed collection of $\mathrm{E}G$-algebras $X_i$, the $\mathrm{E}G$-action of their product $\prod X_i$ is defined componentwise. In particular, this means that the tensor product of two objects in $\prod X_i$ is just the collection of the tensor products of their components in each of the $X_i$. An invertible object in $\prod X_i$ is thus simply a family of invertible objects from the $X_i$ --- in other words, $(\prod X_i)_{\mathrm{inv}} = \prod (X_i)_{\mathrm{inv}}$.
\item Given maps of $\mathrm{E}G$-algebras $F: X \to Z$, $G : Y \to Z$, the $\mathrm{E}G$-action of their pullback $X \times_Z Y$ is also defined componentwise. It follows that an invertible object in $X \times_Z Y$ is just a pair of invertible objects $(x, y)$ from $X$ and $Y$, such that $F(x) = G(y)$. But this is the same as asking for a pair of objects $(x, y)$ from $X_{\mathrm{inv}}$ and $Y_{\mathrm{inv}}$ such that $F_{\mathrm{inv}}(x) = G_{\mathrm{inv}}(y)$, and hence $(X \times_Z Y)_{\mathrm{inv}} = X_{\mathrm{inv}} \times_{Z_{\mathrm{inv}}} Y_{\mathrm{inv}}$.
\item Given a filtered diagram $D$ of $\mathrm{E}G$-algebras, the $\mathrm{E}G$-action of their colimit $\mathrm{colim}(D_n)$ is defined in the following way: use filteredness to find an algebra which contains (representatives of the classes of) all the things you want to act on, then apply the action of that algebra. In the case of tensor products this means that $[x]\otimes[y] = [x \otimes y]$, and thus an invertible object in $\mathrm{colim}(D_n)$ is just (the class of) an invertible object in one of the algebras of $D$. In other words, $\mathrm{colim}(D_n)_{\mathrm{inv}} = \mathrm{colim}(D_{\mathrm{inv}})$.
\end{itemize}
Preservation of products and pullbacks gives preservation of limits, and preservation of limits and filtered colimits gives the result.
\end{proof}

With this new 2-functor $L: \mathrm{E}G\mathrm{Alg}_S \to \mathrm{E}G\mathrm{Alg}_S$, we now have the ability to `freely add inverses to objects' in any $\mathrm{E}G$-algebra we want. The algebra $L\mathbb{G}_n$ is then a clear candidate for our free algebra on $n$ invertible objects, and indeed the proof of this is very simple.

\begin{thm} There exists a free $\mathrm{E}G$-algebra on $n$ invertible objects. Specifically, the algebra $L\mathbb{G}_n$ is such that for any other $\mathrm{E}G$-algebra $X$, we have an isomorphism of categories
\begin{eq*} \mathrm{E}G\mathrm{Alg}_S(L\mathbb{G}_n, X) \cong (X_{\mathrm{inv}})^n \end{eq*}
\end{thm}
\begin{proof}
Using the adjunction from the previous Proposition along with the one from \cref{freealg}, we see that
\begin{eq*}\begin{array}{rll}
		 U(X_{\mathrm{inv}})^n & = & \mathrm{Cat}(\{z_1, ..., z_n\}, U(X_{\mathrm{inv}}) ) \\
		& \cong & \mathrm{E}G\mathrm{Alg}_S( F(\{z_1, ..., z_n\}), X_{\mathrm{inv}}) \\
		& \cong & \mathrm{E}G\mathrm{Alg}_S( LF(\{z_1, ..., z_n\}), X)
\end{array}
 \end{eq*}
As before, $X_{\mathrm{inv}}$ and $U(X_{\mathrm{inv}})$ are obviously isomorphic as categories, and so \( LF(\{z_1, ..., z_n\}) = L\mathbb{G}_n \) satisfies the requirements for the free algebra on $n$ invertible objects.
\end{proof}

\subsection{$L\mathbb{G}_n$ as an initial algebra}

We have now proven that a free $\mathrm{E}G$-algebra on $n$ invertible objects does indeed exist. But this fact on its own is not very helpful. To be able to actually use the free algebra $L\mathbb{G}_n$, we need to know how to contruct it explicitly, in terms of its objects and morphisms. We could do this by finding a detailed characterisation of the 2-functor $L$, and then applying this to our explicit description of $\mathbb{G}_n$ from \cref{Gndef}. However, this would probably be much more effort than is required, since it would involve determining the behaviour of $L$ in many situtations we aren't interested in, and we also wouldn't be leveraging $\mathbb{G}_n$'s status as a free algebra to make the calculations any easier. We will try a different strategy instead. We begin by noticing a special property of the functor $L$.

\begin{prop} \label{linveql} For any $\mathrm{E}G$-algebra $X$, we have $L(X)_{\mathrm{inv}} = L(X)$.
\end{prop}
\begin{proof}
From the definition of adjunctions, the isomorphisms
\begin{eq*}\mathrm{E}G\mathrm{Alg}_S(LX , Y) \cong \mathrm{E}G\mathrm{Alg}_S(X, Y_{\mathrm{inv}}) \end{eq*}
are subject to certain naturality conditions. Specifically, given $F: X' \to X$ and $G: Y \to Y'$ we get a commutative diagram
\begin{eq*} \begin{tikzcd}
\mathrm{E}G\mathrm{Alg}_S(LX , Y) \ar[dd, "G \circ \_ \circ LF"'] \ar[r, "\sim"] & \mathrm{E}G\mathrm{Alg}_S(X, Y_{\mathrm{inv}}) \ar[dd, "G_{\mathrm{inv}} \circ \_ \circ F"] \\
& \\
\mathrm{E}G\mathrm{Alg}_S(LX' , Y') \ar[r, "\sim"] & \mathrm{E}G\mathrm{Alg}_S(X', Y'_{\mathrm{inv}})
\end{tikzcd} \end{eq*}
Consider the case where $F$ is the identity map $\mathrm{id}_X : X \to X$ and $G$ is the inclusion $j: L(X)_{\mathrm{inv}} \to L(X)$. Note that because $j$ is an inclusion, the restriction $j_{\mathrm{inv}}: (L(X)_{\mathrm{inv}})_{\mathrm{inv}} \to L(X)_{\mathrm{inv}}$ is also an inclusion, but since $((\_)_{\mathrm{inv}})_{\mathrm{inv}} = (\_)_{\mathrm{inv}}$, we have that $j_{\mathrm{inv}} = \mathrm{id}$. It follows that
\begin{eq*} \begin{tikzcd}
\mathrm{E}G\mathrm{Alg}_S(LX , LX_{\mathrm{inv}}) \ar[dd, "j \circ \_"'] \ar[r, "\sim"] & \mathrm{E}G\mathrm{Alg}_S(X, LX_{\mathrm{inv}}) \ar[dd, equal] \\
& \\
\mathrm{E}G\mathrm{Alg}_S(LX , LX) \ar[r, "\sim"] & \mathrm{E}G\mathrm{Alg}_S(X, LX_{\mathrm{inv}})
\end{tikzcd} \end{eq*}
Therefore, for any map $f: LX \to LX$ there exists a unique $g: LX \to LX_{\mathrm{inv}}$ such that $j \circ g =f$. But this means that for any such $f$, we must have $\mathrm{im}(f) \subseteq L(X)_{\mathrm{inv}}$, and so in particular $L(X) = \mathrm{im}(\mathrm{id}_{LX}) \subseteq L(X)_{\mathrm{inv}}$. Since $L(X)_{\mathrm{inv}} \subseteq L(X)$ by definition, we obtain the result.
\end{proof}

This result is not especially surprising. Intuitively, it just says that when you freely add inverses to an algebra, every object ends up with an inverse. But the upshot of this is that we now have another way of thinking about $L(X)$: as the target object of the unit of our adjunction, $\eta_X: X \to L(X)_{\mathrm{inv}}$. This means that we don't really need to know the entirety of $L$ in order to determine the free algebra $L\mathbb{G}_n$, just its unit. To find this unit directly, we can turn to the following fact about adjunctions, for which a proof can be found in Lemma 2.3.5 of Leinster's \textit{Basic Category Theory} \cite{bct}.

\begin{prop}\label{initial} Let $F \dashv G: A \to B$ be an adjunction with unit $\eta$. For any object $a$ in $A$, let $(a \downarrow G)$ denote the comma category whose objects are pairs $(b, f)$ consisting of an object $b$ from $B$ and a morphism $f: a \to G(b)$ from $A$, and whose morphisms $h: (b, f) \to (b', f')$ are morphisms $f: b \to b'$ from $B$ such that $G(f) \circ f = f'$. Then the pair $\big(F(a), \eta_a: a \to GF(a) \big)$ is an initial object of $(a \downarrow G)$.
\end{prop}

\begin{cor} If $\phi: \mathbb{G}_n \to Z$ is an initial object of $(\mathbb{G}_n \downarrow \mathrm{inv})$, then 
\begin{eq*} Z \, \cong \, (L\mathbb{G}_n)_{\mathrm{inv}} \, = \, L\mathbb{G}_n \end{eq*}
\end{cor}

Before moving on, we'll make a small change in notation. From now on, rather than writing objects in $(\mathbb{G}_n \downarrow \mathrm{inv})$ as maps $\psi: \mathbb{G}_n \to Y_{\mathrm{inv}}$, we will instead just let $X = Y_{\mathrm{inv}}$ and speak of maps $\psi: \mathbb{G}_n \to X$. This is purely to prevent the notation from becoming cluttered, and shouldn't be a problem so long as we always remember that the targets of these maps only ever contain invertible objects and morphisms.

\subsection{$L\mathbb{G}_n$ as a quotient}

Being able to view $L\mathbb{G}_n$ as the initial object in the comma category $(\mathbb{G}_n \downarrow \mathrm{inv})$ will prove immensely useful in the coming sections. This is because it lets us think about the properties of $L\mathbb{G}_n$ in terms of maps $\psi: \mathbb{G}_n \to X$, and this is exactly the context where we can exploit $\mathbb{G}_n$'s status as a free algebra. 

However, this not the only way of thinking about $L\mathbb{G}_n$. Consider for a moment the free $\mathrm{E}G$-algebra on $2n$ objects, $\mathbb{G}_{2n}$. Intuitively, if we were to take this algebra and then enforce upon it the relations $z_{n+1} = z_1^*, ..., z_{2n} = z_n^*$, we would be changing it from a structure with $2n$ independent generators into one with $n$ indepedent generators and their inverses. That is, there seems to be a natural way to think about $L\mathbb{G}_n$ as a quotient of the larger algebra $\mathbb{G}_{2n}$. \cref{quotient} makes this idea precise.

\begin{prop}\label{quotient} Let $i: \mathbb{G}_n \to \mathbb{G}_{2n}$ be the obvious inclusion of $\mathrm{E}G$-algebras defined on generators by
\begin{eq*} i(z_i) \, = \, z_i \end{eq*}
and likewise let $d: \mathbb{G}_n \to \mathbb{G}_{2n}$ be the $\mathrm{E}G$-algebra map defined by
\begin{eq*} d(z_i) \, = \, z_i \otimes z_{i+n} \end{eq*}
Denote by $q: \mathbb{G}_{2n} \to \mathbb{G}_{2n}/\mathbb{G}_n$ the cokernel of the map $d$. Then $i \circ q$ is the initial object $\eta$ of $(\mathbb{G}_n \downarrow \mathrm{inv})$, and so in particular
\begin{eq*} L\mathbb{G}_n \, = \, \mathbb{G}_{2n}/\mathbb{G}_n \end{eq*}
\end{prop}
\begin{proof}
First, note that the above proposition does actually make sense. The descriptions of $i$ and $d$ are enough to specify them uniquely, because $\mathbb{G}_n$ is the free $\mathrm{E}G$-algebra on $n$ objects and so maps $\mathbb{G}_n \to \mathbb{G}_{2n}$ are canonically isomorphic to functors $\{z_1, ..., z_n\} \to \mathbb{G}_{2n}$. Also we can be sure that the map $q$ exists, because the cokernel of $d$ is just its pushout along the terminal object $\mathrm{E}G$-algebra $0$,
\begin{eq*} \begin{tikzcd}
\mathbb{G}_n \ar[r, "d"] \ar[d] \ar[dr, phantom, "\ulcorner", very near end] & \mathbb{G}_{2n} \ar[d, "q"] \\
0 \ar[r] & \mathbb{G}_{2n}/\mathbb{G}_n
\end{tikzcd} \end{eq*}
and $\mathrm{E}G\mathrm{Alg}_S$ is a locally finitely presentable category, so it has all finite colimits. Equivalently, because $0$ is both the terminal and initial object in $\mathrm{E}G\mathrm{Alg}_S$ the cokernel of $d$ will be the coequalizer of $d$ and the zero map $\mathbb{G}_n to 0 \to \mathbb{G}_{2n}$, which is also a finite colimit. 

Now, let $\psi: \mathbb{G}_n \to X$ be an arbitrary object of $(\mathbb{G}_n \downarrow \mathrm{inv})$. The map $\psi^*: \mathbb{G}_n \to X$ which takes values
\begin{eq*} \psi^*(z_i) \, = \, \psi(z_i)^* \end{eq*}
is also an object of $(\mathbb{G}_n \downarrow \mathrm{inv})$, and using these two we can define a new map $\psi + \psi^*$ using the universal property of the colimit:
\begin{eq*} \begin{tikzcd}
& \mathbb{G}_n + \mathbb{G}_n \ar[dd, "\psi + \psi^*"] & \\
\mathbb{G}_n \ar[ur, hookrightarrow, "i"] \ar[dr, "\psi"'] & & \mathbb{G}_n \ar[ul, hookrightarrow, "i'"'] \ar[dl, "\psi^*"] \\
& X & 
\end{tikzcd} \end{eq*}
But $\mathbb{G}_n$ is the free $\mathrm{E}G$-algebra on $n$ objects, and the free functor $F : \mathrm{Cat} \to \mathrm{E}G\mathrm{Alg}_S$ preserves colimits because it is a left adjoint, so clearly
\begin{eq*} \begin{array}{rll}
		\mathbb{G}_n + \mathbb{G}_n & = & F(\{ z_1, ..., z_n\}) + F(\{ z'_1, ..., z'_n\}) \\
		& = & F( \, \{ z_1, ..., z_n\} + \{ z'_1, ..., z'_n\} \, ) \\
		& = & F(\{ z_1, ..., z_{2n} \}) \\
		& = & \mathbb{G}_{2n} 
		\end{array}
\end{eq*}
This means that we can compose $\psi + \psi^*: \mathbb{G}_{2n} \to X$ with the map $d:  \mathbb{G}_n \to  \mathbb{G}_{2n} $, though we need to be a little careful and actually specify which inclusions we really used in the definition of $\psi + \psi^*$. Suppose that the lefthand inclusion is $i$, the one given in the statement of the proposition, and the other is defined by the assignment $z_i \mapsto z_{i+n}$. Then
\begin{eq*}	(\psi + \psi^*) \circ d(z_i) \, = \, (\psi + \psi^*)(z_i \otimes z_{i+n}) \, = \, \psi(z_i) \otimes \psi(z_i)^* \, = \, I \end{eq*}
and hence $(\psi + \psi^*) \circ d$ is just the zero map from $ \mathbb{G}_n$ to $X$. However, we've already defined $q: \mathbb{G}_{2n} \to \mathbb{G}_{2n}/\mathbb{G}_n$ to be the coequalizer of $d$ and the zero morphism, and so it is the universal map such that its composite with $d$ is zero. Therefore, there must exist a unique map $u: \mathbb{G}_{2n}/\mathbb{G}_n \to X$ making the righthand triangle in following diagram commute:
\begin{eq*} \begin{tikzcd}
\mathbb{G}_n \ar[r, "i"] \ar[ddr, "\psi"'] & \mathbb{G}_{2n} \ar[r, "q"] \ar[dd, "\psi + \psi^*", near start] & \mathbb{G}_{2n}/\mathbb{G}_n \ar[ddl, "u"] \\
& & \\
& X &
\end{tikzcd} \end{eq*}
The other triangle commutes by the definition of $\psi + \psi^*$, and so together the diagram tells us that for any object $\psi$ of $(\mathbb{G}_n \downarrow \mathrm{inv})$, there exists at least one morphism $u$ from $q \circ i$ onto $\psi$. 

Finally, let $v: \mathbb{G}_{2n}/\mathbb{G}_n \to X$ be an arbitrary morphism from $q \circ i$ to $\psi$ in $(\mathbb{G}_n \downarrow \mathrm{inv})$. By definition, this means that
\begin{eq*} vqi \, = \, \psi \quad \implies \quad  \psi + \psi^* \, = \, vqi + (vqi)^* \end{eq*}
and since for all $n+1 \leq i \leq 2n$,
\begin{eq*} q(z_{i-n}) \otimes q(z_i) \, = \, q(z_{i-n} \otimes z_i) \, = \, qd(z_{i-n}) \, = \, I  \end{eq*}
\begin{eq*} \implies q(z_{i-n}) \, = \, q(z_i)^* \end{eq*}
it follows that
\begin{eq*}\begin{array}{rll}
		(\psi + \psi^*)(z_i) & = & \big( vqi + (vqi)^* \big)(z_i) \\
		& = &
			\begin{cases}
       				vqi(z_i) & \quad \text{if} \quad 1 \leq i \leq n \\
      				vqi(z_i)^* & \quad \text{if} \quad n+1 \leq i \leq 2n \\
			\end{cases} \\ 
		& = & 
			\begin{cases}
       				vq(z_i) & \quad \text{if} \quad 1 \leq i \leq n \\
      				vq(z_{i-n})^* & \quad \text{if} \quad n+1 \leq i \leq 2n \\
			\end{cases} \\
		& = & 
			\begin{cases}
       				vq(z_i) & \quad \text{if} \quad 1 \leq i \leq n \\
      				v\big( q(z_i)^* \big)^* & \quad \text{if} \quad n+1 \leq i \leq 2n \\
			\end{cases} \\
		& = & vq(z_i)
		\end{array}
\end{eq*}
In other words, the diagram
\begin{eq*} \begin{tikzcd}
\mathbb{G}_{2n} \ar[r, "q"] \ar[dd, "\psi + \psi^*"'] & \mathbb{G}_{2n}/\mathbb{G}_n \ar[ddl, "v"] \\
& \\
X &
\end{tikzcd} \end{eq*}
will commute for any morphism $v: q \circ i \to \psi$ in $(\mathbb{G}_n \downarrow \mathrm{inv})$. But this is the property that the map $u$ was supposed to satisfy uniquely, and thus there is exactly one morphism from $q \circ i$ onto any other object of $(\mathbb{G}_n \downarrow \mathrm{inv})$ --- that is, $q \circ i$ is initial. Therefore $q \circ i$ is isomorphic to any other initial object, inculding $\eta$, and hence their targets $\mathbb{G}_{2n}/\mathbb{G}_n$ and $L\mathbb{G}_n$ are isomorphic $\mathrm{E}G$-algebras.
\end{proof}

It's worth noting that we have not given a method for actually taking quotients of $\mathrm{E}G$-algebras, and so \cref{quotient} doesn't immediately provide an explicit description of $L\mathbb{G}_n$. Nevertheless, we will be able to tease many important facts from this new perspective, which would not be so obvious from the initial object point of view. First of all, we can find the objects of $L\mathbb{G}_n$:

\begin{prop}\label{Zobj} The object monoid of $L\mathbb{G}_n$ is $\mathbb{Z}^{*n}$, and the restriction of $\eta$ to objects $\eta_{\mathrm{ob}}$ is the obvious inclusion $\mathbb{N}^{*n} \to \mathbb{Z}^{*n}$.
\end{prop}
\begin{proof}
Recall that given a monoid $M$, the category $\mathrm{E}M$ is the one whose monoid of objects is $M$ and which has a unique isomorphism between any two objects. We can view $\mathrm{E}M$ as not just a category but an $\mathrm{E}G$-algebra, by letting the action on morphisms take the only possible values it can, given the required source and target. Similarly, for any monoid homomorphisms $h: M \to M'$ we can define a map of $\mathrm{E}G$-algebras
\begin{eq*} \begin{array}{rlrll}
		\mathrm{E}h & : & \mathrm{E}M & \to & \mathrm{E}M' \\
		& : & m & \mapsto & h(m) \\
		& : & m \to m' & \mapsto & h(m) \to h(m')
		\end{array}
\end{eq*}
This definition of $\mathrm{E}h$ respects composition and identities, and so together with $\mathrm{E}M$ it describes a functor $\mathrm{E}: \mathrm{Mon} \to \mathrm{E}G\mathrm{Alg}_S$.

Now, for any $\mathrm{E}G$-algebra $X$, a map $F: X \to \mathrm{E}M$ will be determined entirely by its restriction on objects, the monoid homomorphism $F_{\mathrm{ob}} : \mathrm{Ob}(X) \to M$. This is because functorality ensures that any $f: x \to x'$ in $X$ must be sent to a map $F(x) \to F(x')$ in $\mathrm{E}M$, and there is exactly one of these. In other words, we have an isomorphism between homsets
\begin{eq*} \mathrm{Mon}( \, \mathrm{Ob}(X), M \, ) \quad \cong \quad \mathrm{E}G\mathrm{Alg}_S( \, X, \mathrm{E}M \, ) \end{eq*}
Additionally, if we let $\mathrm{Ob}: \mathrm{E}G\mathrm{Alg}_S \to \mathrm{Mon}$ be the functor that which sends algebras $X$ to their monoid of objects $\mathrm{Ob}(X)$ and algebra maps $F$ to their underlying monoid homomorphism $F_{\mathrm{ob}}$, then this homset isomorphism is natural in both coordinates. That is, for any $G: X \to X'$ in $\mathrm{E}G\mathrm{Alg}_S$ and $h : M \to M'$ in $\mathrm{Mon}$, the diagram
\begin{eq*} \begin{tikzcd}
\mathrm{Mon}(\mathrm{Ob}(X), M) \ar[dd, "h \circ \_ \circ G_{\mathrm{ob}}"'] \ar[r, "\sim"] & \mathrm{E}G\mathrm{Alg}_S(X, \mathrm{E}M) \ar[dd, "\mathrm{E}h \circ \_ \circ G"] \\
& \\
\mathrm{Mon}(\mathrm{Ob}(X'), M') \ar[r, "\sim"] &  \mathrm{E}G\mathrm{Alg}_S(X', \mathrm{E}M')
\end{tikzcd} \end{eq*}
commutes, because
\begin{eq*} ( \, \mathrm{E}h \circ F \circ G \, )_{\mathrm{ob}} \quad = \quad h \circ F_{\mathrm{ob}} \circ G_{\mathrm{ob}} \end{eq*}
Therefore, $\mathrm{Ob}$ is a left adjoint to the functor $\mathrm{E}$.

Now, take the algebra map $d: \mathbb{G}_n \to \mathbb{G}_{2n}$ and consider its underlying monoid homomorphism, $d_{\mathrm{ob}}: \mathbb{N}^{\ast n} \to \mathbb{N}^{\ast 2n}$. We can take the colimit of this map in the category $\mathrm{Mon}$, which will give the canonical map from $\mathbb{N}^{\ast 2n}$ onto its quotient by the free monoid generated by objects of the form $z_i \otimes z_{i+n}$, $1 \leq i \leq n$:
\begin{eq*} \mathbb{N}^{\ast 2n}/\langle z_i \otimes z_{i+n} \rangle \, = \, \mathbb{Z}^{\ast n} \end{eq*}
But $\mathrm{Ob}$ is a left adjoint and hence preserves colimits, so it follows that the quotient map of $d_{\mathrm{ob}}$ is just the underlying monoid homomorphism of the quotient map $q$ of $d$. Therefore, $\mathrm{Ob}(L\mathbb{G}_n) = \mathbb{Z}^{\ast n}$. 

Moreover, the free monoid $\langle z_i \otimes z_{i+n} \rangle$ does not contain any of the same elements as the free monoid on the generators $z_i$, $1 \leq i \leq n$. Thus when we take the quotient of $\mathbb{N}^{\ast 2n}$ by $\langle z_i \otimes z_{i+n} \rangle$, the submonoid $\langle z_i: 1 \leq i \leq n \rangle = \mathbb{N}^{\ast n}$ will be left unaffected. Therefore, for any $z_i$ with $1 \leq i \leq n$,
\begin{eq*} \eta_{\mathrm{ob}}(z_i) \, = (q \circ i)_{\mathrm{ob}}(z_i) \, = \, q_{\mathrm{ob}} i_{\mathrm{ob}}(z_i) \, = \, i_{\mathrm{ob}}(z_i) \, = \, z_i \end{eq*}
and so the underlying monoid homomorphism of $\eta$ is just the evident inclusion.
\end{proof}

This result makes concrete the sense in which the functor $L$ represents `freely adding inverses' to objects. Specifically, in order to get the object monoid of $L\mathbb{G}_n$ from the objects $\mathbb{N}^{\ast n}$ of $\mathbb{G}_n$, we use the process of \emph{group completion}. Applying this same logic to connected components as well, it seems reasonable to expect that $\pi_0(L\mathbb{G}_n)$ should be $\mathbb{Z}^n$, the group completion of $\pi_0(\mathbb{G}_n) = \mathbb{N}^n$. This is indeed the case, and proof is largely analagous to \cref{Zobj}.

\begin{prop}\label{Zconcomp} The connected components of $L\mathbb{G}_n$ are $\mathbb{Z}^n$, with the assignment $[ \,\, ] : \mathbb{Z}^{*n} \to \mathbb{Z}^n$ of objects to their component being the quotient map of abelianisation, and the restriction of $\eta$ to components $\eta_\pi : \mathbb{N}^n \to \mathbb{Z}^n$ being the obvious inclusion. 
\end{prop}
\begin{proof}
There exists an inclusion of 2-categories $\mathrm{Set} \to \mathrm{Cat}$ which allows us to view any set $X$ as a discrete category, whose objects are elements of $X$ and whose morphisms are all identities. If a given set also happens to be a monoid then there is an obvious way to see its discete category as a monoidal category, and so we have a similar inclusion $\mathrm{Mon} \to \mathrm{MonCat}$. Likewise, if a given monoid happens to be commutative, then there is a unique way to assign an $\mathrm{E}G$-action to its discete category, which yields an inclusion $\mathrm{D}: \mathrm{CMon} \to \mathrm{E}G\mathrm{Alg}_S$, where $\mathrm{CMon}$ is the category of commutative monoids and monoid homomorphisms between them. This works because for any elements $c_i$ in a commutive monoid $C$, the morphism $\alpha(g; \mathrm{id}_{c_1}, ..., \mathrm{id}_{c_m})$ must have source and target $c_1 \otimes ... \otimes c_m = c_{\pi(g^{-1})(1)} \otimes ... \otimes c_{\pi(g^{-1})(m)}$, and therefore can only be $\mathrm{id}_{c_1 \otimes ... \otimes c_m}$.

Now, consider a map $F: X \to \mathrm{D}C$ from some $\mathrm{E}G$-algebra $X$ to the discrete $\mathrm{E}G$-algebra on a commutative monoid $C$. $F$ is determined entirely by its restriction to connected components, the monoid homomorphism $F_{\pi} : \pi_0(X) \to C$, because by functorality any $f: x \to x'$ in $X$ must be sent to an identity map in $\mathrm{D}C$, and so $x, x'$ being in the same component implies $F(x) = F(x')$. In other words, we have an isomorphism between homsets
\begin{eq*} \mathrm{CMon}( \, \pi_0(X), C \, ) \quad \cong \quad \mathrm{E}G\mathrm{Alg}_S( \, X, \mathrm{D}C \, ) \end{eq*}
and this homset isomorphism is also natural in both coordinates, because for any $G: X \to X'$ in $\mathrm{E}G\mathrm{Alg}_S$ and $h : C \to C'$ in $\mathrm{CMon}$, 
\begin{eq*} ( \, \mathrm{D}h \circ F \circ G \, )_{\pi} \quad = \quad h \circ F_{\pi} \circ G_{\pi} \end{eq*}
and so the diagram
\begin{eq*} \begin{tikzcd}
\mathrm{CMon}(\pi_0(X), C) \ar[dd, "h \circ \_ \circ G_{\pi}"'] \ar[r, "\sim"] & \mathrm{E}G\mathrm{Alg}_S(X, \mathrm{D}C) \ar[dd, "\mathrm{D}h \circ \_ \circ G"] \\
& \\
\mathrm{CMon}(\pi_0(X'), C') \ar[r, "\sim"] &  \mathrm{E}G\mathrm{Alg}_S(X', \mathrm{D}C')
\end{tikzcd} \end{eq*}
commutes. Therefore, $\pi_0$ is a left adjoint to the functor $i$.

Now, take the algebra map $d: \mathbb{G}_n \to \mathbb{G}_{2n}$ and consider its restriction to connected components, which by \cref{Gnconcomp} is a monoid homomorphism $d_\pi: \mathbb{N}^n \to \mathbb{N}^{2n}$. We can take the colimit of this map in the category $\mathrm{CMon}$ to get the canonical map from $\mathbb{N}^{2n}$ onto its quotient by objects of the form $z_i \otimes z_{i+n}$, $1 \leq i \leq n$,
\begin{eq*} \mathbb{N}^{2n}/\langle z_i \otimes z_{i+n} \rangle \, = \, \mathbb{Z}^n \end{eq*}
and since $\pi_0$ as a left adjoint preserves colimits, this quotient map of $d_{\pi}$ is just the restriction to connected components of the quotient map $q$ of $d$. Therefore, $\pi_0(L\mathbb{G}_n) = \mathbb{Z}^n$. 

Like in \cref{Zobj}, the free commutative monoids $\langle z_i \otimes z_{i+n} \rangle$ and $\langle z_i \rangle$ for $1 \leq i \leq n$ are disjoint, and thus taking the quotient of $\mathbb{N}^{2n}$ by $\langle z_i \otimes z_{i+n} \rangle$ will leave $\langle z_i: 1 \leq i \leq n \rangle = \mathbb{N}^n$ unaffected. Therefore, for any $z_i$ with $1 \leq i \leq n$,
\begin{eq*} \eta_\pi(z_i) \, = (q \circ i)_\pi(z_i) \, = \, q_\pi i_\pi(z_i) \, = \, i_\pi(z_i) \, = \, z_i \end{eq*}
and so the restriction of $\eta$ to connected components is just the inclusion $\mathbb{N}^n \to \mathbb{Z}^n$.
\end{proof}

\subsection{$L\mathbb{G}_n$ as a coequalizer}

\cref{Zobj,Zconcomp} both contain statements about the partial surjectivity of the map $q$; since $q_{\mathrm{ob}}$ is the quotient map of monoids $\mathbb{N}^{\ast 2n} \to \mathbb{Z}^{\ast n}$, every object of $L\mathbb{G}_n$ is the image under $q$ of some object of $\mathbb{G}_{2n}$, and likewise for connected components. From this one might guess that $q$ is will turn out to be a surjective map of $\mathrm{E}G$-algebras. Indeed this is the case, and much as \cref{Zobj,Zconcomp} are analogues of \cref{Gnobj,Gnconcomp} respectively, the fact that $q$ is surjective on morphisms will provide us with a result analagous to \cref{Gnmapsaction}. That is, since every morphism of $\mathbb{G}_{2n}$ is an action morphism, and since $\mathrm{E}G$-algebra maps always send action morphisms to action morphisms, if $q$ is surjective then every morphism of $L\mathbb{G}_n$ is also an action morphism.

However, the proof of this is not so straightforward. In particular, if every morphism in $L\mathbb{G}_n$ were an action morphisms of the form $\alpha(g; \mathrm{id}_{x_1}, ..., \mathrm{id}_{x_m})$, $x_i \in \{ z_1, ..., z_n, z^*_1, ..., z^*_n \}$, then this would mean that the set of such action morphisms would have to be closed under composition, which is not at all obvious. For $\mathbb{G}_{2n}$ we know this is true because two maps $\alpha(g; \mathrm{id}_{x_1}, ..., \mathrm{id}_{x_m})$, $\alpha(g'; \mathrm{id}_{x'_1}, ..., \mathrm{id}_{x'_m})$, $x_i, x'_i \in \{ z_1, ..., z_{2n} \}$ are only composable if the target of the first is equal to the source of the second, which means that
\begin{eq*}\begin{array}{rrll}
		& x_{\pi(g^{-1})(1)} \otimes ... \otimes x_{\pi(g^{-1})(m)} & = & x'_1 \otimes ... \otimes x'_m, \quad \quad \, x_i, x'_i \in \{ z_1, ..., z_{2n} \} \\
		\implies & x_{\pi(g^{-1})(i)} & = & x'_i, \quad \quad \quad \quad \quad \quad \quad 1 \le i \le m \\
		\implies & \alpha(gg'; \mathrm{id}_{x_1}, ..., \mathrm{id}_{x_m}) & = & \alpha(g; \mathrm{id}_{x_1}, ..., \mathrm{id}_{x_m}) \circ \alpha(g'; \mathrm{id}_{x'_1}, ..., \mathrm{id}_{x'_m})
		\end{array}
\end{eq*}
However, in $L\mathbb{G}_n$ this line of reasoning does not work, because for example
\begin{eq*} z_1 \otimes z^*_1 \quad = \quad I \quad = \quad z_2 \otimes z^*_2 \end{eq*}
but this does not imply that $z_1 = z_2$. 

In order to solve this problem, we will need to adopt a slightly different view of the map $q$ --- no longer seeing it as a cokernel, but just as a general coequalizer.

\begin{lem} Recall that $\mathbb{G}_n$ is the free $\mathrm{E}G$-algebra on $n$ objects so that $\mathbb{G}_{3n}$ is the coproduct of $\mathbb{G}_{2n}$ and $\mathbb{G}_n$. Using this, define the maps $\delta, \zeta : \mathbb{G}_{3n} \to \mathbb{G}_{2n}$ using the universal property of the coproduct:
\begin{eq*} \begin{tikzcd}
& \mathbb{G}_{3n} \ar[dd, "\delta"] & & & \mathbb{G}_{3n} \ar[dd, "\zeta"] & \\
\mathbb{G}_n \ar[ur, hookrightarrow] \ar[dr, "d"'] & & \mathbb{G}_{2n} \ar[ul, hookrightarrow] \ar[dl, "\mathrm{id}"] & \mathbb{G}_n \ar[ur, hookrightarrow] \ar[dr, "0"'] & & \mathbb{G}_{2n} \ar[ul, hookrightarrow] \ar[dl, "\mathrm{id}"] \\
& \mathbb{G}_{2n} & & & \mathbb{G}_{2n} & \\
\end{tikzcd} \end{eq*}
Then $q$ is the coequalizer of $\delta$ and $\zeta$.
\end{lem}
\begin{proof}
Let $i: \mathbb{G}_n \to \mathbb{G}_{3n}$ be the inclusion defined by sending the generators $z_i$ to $z_{i+2n}$, and let $p: \mathbb{G}_{2n} \to X$ be any map of $\mathrm{E}G$-algebras. If $p \circ \delta = p \circ \zeta$, then
\begin{eq*} \begin{array}{rll}
		p \circ d & = & p \circ \delta \circ i \\
		& = & p \circ \zeta \circ i \\
		& = & p \circ 0 \\
		& = & 0 
		\end{array}
\end{eq*}
and conversely if $p \circ d = 0$ then
\begin{eq*} \begin{array}{rll}
		p \circ \delta & = & p \circ (\mathrm{id}_{\mathbb{G}_{2n}} + d) \\
		& = & p + p \circ d \\
		& = & p + 0 \\
		& = & p \circ (\mathrm{id}_{\mathbb{G}_{2n}} + 0) \\
		& = & p \circ \zeta
		\end{array}
\end{eq*}
That is, $p \circ d = 0$ and $p \circ \delta = p \circ \zeta$ are equivalent properties, and so since $q$ is the universal map with the first property, it must also be the universal map with the second. Therefore, $q$ is the coequalizer of $\delta$ and $\zeta$.
\end{proof}

\begin{cor} $q$ is a reflexive coequalizer. \end{cor}
\begin{proof}
A reflexive coequalizer is a map which is the coequalizer of a reflexive pair, which is in turn a pair of parallel maps which share a right-inverse. If we let $i: \mathbb{G}_{2n} \to \mathbb{G}_{3n}$ be the inclusion defined on generators by $z_i \mapsto z_i$, then $i$ is a right-inverse of both $\delta$ and $\zeta$:
\begin{eq*} \delta \circ i \, = \, (\mathrm{id}_{\mathbb{G}_{2n}} + d) \circ i \, = \, \mathrm{id}_{\mathbb{G}_{2n}}, \quad \quad \zeta \circ i \, = \, (\mathrm{id}_{\mathbb{G}_{2n}} + 0) \circ i \, = \, \mathrm{id}_{\mathbb{G}_{2n}} \end{eq*}
Therefore, $(\delta, \zeta)$ is a reflexive pair, and $q$ a reflexive coequalizer.
\end{proof}

Using this new perspective of $q$, we can prove the following result about the composites of morphisms in $L \mathbb{G}_n$:

\begin{prop} For any two morphisms $f: v \to w$, $f' : w' \to v'$ in $\mathbb{G}_{2n}$,
\begin{eq*} q(w) = q(w') \quad \implies \quad \exists \, h: v \to v' \quad \mathrm{s.t.} \quad q(h) = q(f) \circ q(f') \end{eq*}
\end{prop}
\begin{proof}
Let $w, w'$ be objects of $\mathbb{G}_{2n}$ for which $q(w) = q(w')$. By \cref{Gnconcomp}, their connected components $[w], [w']$ are just elements of $\mathbb{N}^{2n}$, and so we can use them to define a new element
\begin{eq*} m := (m_1, ..., m_{2n}) \in \mathbb{N}^{2n}, \quad \quad m_i := \mathrm{min}\big( \, [w]_i, \, [w']_i \, \big) \end{eq*}
Because of how we've contructed $m$, the expressions $[w]_i - m_i$, $[w']_i - m_i$ are well-defined, despite subtraction not being defined in general on $\mathbb{N}$, and at least one of the two will always equal 0. Thus we can also create two new elements $a, a'$ of $\mathbb{N}^{2n}$ by setting
\begin{eq*} a_i \, := \, [w]_i - m_i, \quad \quad a'_i  \, := \, [w']_i - m_i \end{eq*}
and these will be the unique elements with the property that $m+a = [w]$ and $m+a' = [w']$. Also, note that because of how we defined $m_i$ we will always have at least one of $a_i$ and $a'_i$ equal to 0.

Now, since $q_\pi : \mathbb{N}^{2n} \to \mathbb{Z}^n$ is the restriction of $q$ to connected components, we have
\begin{eq*}\begin{array}{rlrll}
		q(w) = q(w') & \implies & q_\pi([w]) & = & q_\pi([w']) \\
		& \implies & q_\pi(m+a) & = & q_\pi(m+a') \\
		& \implies & q_\pi(m)+q_\pi(a) & = & q_\pi(m)+q_\pi(a') \\
		& \implies & q_\pi(a) & = & q_\pi(a') \\
		& \implies & q_\pi(a)_i & = & q_\pi(a')_i \quad \quad 1 \le i \le n
		\end{array}
\end{eq*}
Moreover, by \cref{Zconcomp} $q_\pi$ is the map that quotients out the submonoid $\langle z_i + z_{i+n} \rangle$ of $\mathbb{N}^{2n}$, and so
\begin{eq*} a_i - a_{i+n} \, = \, q_\pi(a)_i \, = \, q_\pi(a')_i \, = \, a'_i - a'_{i+n}, \quad \quad 1 \le i \le n \end{eq*}
But at least one out of $a_i$ and $a'_i$ is 0, and likewise with $a_i, a'_i$, and so there are four possibilities to consider:
\begin{eq*}\begin{array}{rll}
		a_i = a_{i+n} = 0 & \implies & a'_i - a'_{i+n} = 0 \\
		& \implies & a'_i = a'_{i+n} \\
		& & \\
		a'_i = a'_{i+n} = 0 & \implies & a_i = a_{i+n} \\
		& & \\
		a_i = a'_{i+n} = 0 & \implies & - a_{i+n} = a'_i, \quad \quad a_{i+n}, a'_i \in \mathbb{N} \\
		& \implies & a_{i+n} = a'_i = 0 \\
		& & \\
		a'_i = a_{i+n} = 0 & \implies & a_i = a'_{i+n} = 0
		\end{array} 
\end{eq*}
In every case, we can conclude that $a_i = a_{i+n}$ and $a'_i = a'_{i+n}$ for all $1 \le i \le n$.

Next, let $b, b'$ be the following elements of $\mathbb{N}^{\ast n}$,
\begin{eq*} b \, := \, z_1^{\otimes a_1} \otimes ... \otimes z_n^{\otimes a_n}, \quad \quad b' \, := \, z_1^{\otimes a'_1} \otimes ... \otimes z_n^{\otimes a'_n} \end{eq*}
and consider their images under $d_{ob}: \mathbb{N}^{\ast n} \to \mathbb{N}^{\ast 2n}$,
\begin{eq*}\begin{array}{rll}
		d(b) & = & d( \,  z_1^{\otimes a_1} \otimes ... \otimes z_n^{\otimes a_n} \, ) \\
		& = & d(z_1)^{\otimes a_1} \otimes ... \otimes d(z_n)^{\otimes a_n} \\
		& = & (z_1 \otimes z_{1+n})^{\otimes a_1} \otimes ... \otimes (z_n \otimes z_{2n})^{\otimes a_n} \\
		& & \\
		d(b') & = & (z_1 \otimes z_{1+n})^{\otimes a'_1} \otimes ... \otimes (z_n \otimes z_{2n})^{\otimes a'_n}
		\end{array}
\end{eq*}
By \cref{Gnconcomp}, the map sending objects of $\mathbb{G}_{2n}$ to their connected components is the quotient of abelianisation $\mathbb{N}^{\ast 2n} \to \mathbb{N}^{2n}$, the one making the tensor product $\otimes$ commutative. Therefore, the components of $d(b)$ and $d(b')$ are $a$ and $a'$ respectively:
\begin{eq*} \begin{array}{rll}
		[ \, d(b) \, ] & = & [ \, (z_1 \otimes z_{1+n})^{\otimes a_1} \otimes ... \otimes (z_n \otimes z_{2n})^{\otimes a_n} \, ] \\
		& = & z_1^{\otimes a_1} \otimes ... \otimes z_n^{\otimes a_n} \otimes z_{1+n}^{\otimes a_1} \otimes ... \otimes z_{2n}^{\otimes a_n} \\
		& = & (a_1, ...., a_n, a_1, ..., a_n) \\
		& = & a \\
		& & \\
		\protect [ \, d(b') \, ] & = & a' 
		\end{array}
\end{eq*}
It follows that the objects $w \otimes d(b')$ and $d(b') \otimes w'$ share a connected component of $\mathbb{G}_{2n}$,
\begin{eq*} \begin{array}{rll}
		[ \, w \otimes d(b') \, ] & = & [w] + [ \, d(b') \, ] \\
		& = & a + m + a' \\
		& = & [ \, d(b) \, ] + [w'] \\
		& = & [ \, d(b') \otimes w' \, ]
		\end{array}
\end{eq*}
Furthermore, these two objects have the same image under $q$, which is also the image of $w$ and $w'$, because of how $q$ is the cokernel of $d$:
\begin{eq*} \begin{array}{rll}
		q\big( \, w \otimes d(b') \, \big) & = & q(w) \otimes qd(b') \\
		& = & q(w) \\
		& = & q(w') \\
		& = & qd(b) \otimes q(w') \\
		& = & q\big( \, d(b') \otimes w' \, \big)
		\end{array}
\end{eq*}

Finaly, we return to our morphisms $f: v \to w$, $f' : w' \to v'$ in $\mathbb{G}_{2n}$. 
\end{proof}

As an immediate consequence, we can learn the following about the morphisms $L\mathbb{G}_n$:

\begin{prop}\label{allmapsaction} The quotient map $q: \mathbb{G}_{2n} \to L\mathbb{G}_n$ is surjective. Therefore, every morphism in $L\mathbb{G}_n$ can be expressed as $\alpha_{L\mathbb{G}_n}(g; \mathrm{id}_{x_1}, ..., \mathrm{id}_{x_m})$ for some $g \in G(m)$ and $x_i \in \{z_1, ..., z_n, z^*_1, ..., z^*_n  \}$.
\end{prop}
\begin{proof}
Let $f$ be an arbitrary morphism in $L\mathbb{G}_n$. By surjectivity there exists at least one morphism $f'$ in $\mathbb{G}_{2n}$ such that $q(f') = f$, and from \cref{Gnmapsaction} we know that this $f'$ can be expressed uniquely as $\alpha(g; \mathrm{id}_{x'_1}, ..., \mathrm{id}_{x'_m})$ for some $g \in G(m)$ and $x'_i \in \{z_1, ..., z_{2n} \}$. Thus, because $q$ is a map of $\mathrm{E}G$-algebras, we will have
\begin{eq*} f \, = \, q(f') \, = \, q \big( \, \alpha_{\mathbb{G}_{2n}}(g; \mathrm{id}_{x'_1}, ..., \mathrm{id}_{x'_m}) \, \big)  \, = \, \alpha_{L\mathbb{G}_n}(g; \mathrm{id}_{q(x'_1)}, ..., \mathrm{id}_{q(x'_m)}) \end{eq*}
But for every generator $z_i$ from $\mathbb{G}_{2n}$, $q(z_i)$ is either the generator $z_i$ of $\mathbb{G}_n$ or the object $z^*_{i-n}$, depending on the value of the subscript. Therefore, there is at least one collection of $x_i = q(x'_i)$ for which the proposition holds. 
\end{proof}

Like the proposition before it, \cref{allmapsaction} is formalising a certain intuition about how the functor $L$ should act on algebras. This time it is the idea that a `free' structure really shouldn't have any `superfluous' components, only whatever data is absolutely required for it to be well-defined. In the case of $L\mathbb{G}_n$, we have proven that the only morphisms contained in the free $\mathrm{E}G$-algebra on invertible objects are $\mathrm{E}G$-action morphisms. This is very similar to what we have in the non-invertible case, though it should be stressed that unlike with $\mathbb{G}_n$, \cref{allmapsaction} does \emph{not} prove that the morphisms of $L\mathbb{G}_n$ have \emph{unique} representations $\alpha(g; \mathrm{id}_{w_1}, ..., \mathrm{id}_{w_m})$.

Now that we understand a little of the morphisms $L\mathbb{G}_n$, we can also show how they relate to the objects.

This result is very similar to \cref{Zobj}, in that it shows that some of the behaviour of $L\mathbb{G}_n$ can be obtained from the behaviour of $\mathbb{G}_n$ through the process of group completion.

Finally, there is one more important application of \cref{quotient}, relating to the only part of the $\mathrm{E}G$-algebra $L\mathbb{G}_n$ that we have yet to examine --- the action.

\begin{prop}\label{LGaction} The action of $L\mathbb{G}_n$ is determined entirely by the action of $\mathbb{G}_{2n}$ and the underlying monoidal functor of $q$.
\end{prop}
\begin{proof}
The action of $L\mathbb{G}_n$ on objects is simply
\begin{eq*} \alpha(g; w_1, ..., w_m) \, = \,  w_{\pi(g)^{-1}(1)}, ..., w_{\pi(g)^{-1}(m)} \end{eq*}
which can be understood from $L\mathbb{G}_n$'s underlying monoidal category. For morphisms, notice that the $\mathrm{E}G$-action can always be split as
\begin{eq*}\begin{array}{rll}
		\alpha(g; f_1, ..., f_m) & = & \alpha(g; \mathrm{id}_{w_1}, ..., \mathrm{id}_{w_m}) \circ \alpha(e; f_1, ..., f_m) \\
		& = & \alpha(g; \mathrm{id}_{w_1}, ..., \mathrm{id}_{w_m}) \circ (f_1 \otimes ... \otimes f_m)
		\end{array}
\end{eq*}
and so the only additional information we need to recover the action from the underlying monoidal category of $L\mathbb{G}_n$ are the values of the action morphisms $\alpha(g; \mathrm{id}_{w_1}, ..., \mathrm{id}_{w_m})$. 

But since $q: \mathbb{G}_{2n} \to L\mathbb{G}_n$ is a cokernel we know it must be essentially surjective; for each $w_i$ there exists at least one object $w'_i$ in $\mathbb{G}_{2n}$ such that $q(w'_i) = w_i$. Since $q$ is also a map of $\mathrm{E}G$-algebras, it then follows that
\begin{eq*}\begin{array}{rll}
		\alpha_{L\mathbb{G}_n}(g; \mathrm{id}_{w_1}, ..., \mathrm{id}_{w_m}) & = & \alpha_{L\mathbb{G}_n}(g; \mathrm{id}_{q(w'_1)}, ..., \mathrm{id}_{q(w'_m)}) \\
		& = &  q \big( \, \alpha_{\mathbb{G}_{2n}}(g; \mathrm{id}_{w'_1}, ..., \mathrm{id}_{w'_m}) \, \big)
		\end{array}
\end{eq*}
Therefore, $\alpha_{L\mathbb{G}_n}$ is completely determined by $\alpha_{\mathbb{G}_{2n}}$, and the values that $q$ takes on objects and morphisms.
\end{proof}

In order to describe an $\mathrm{E}G$-algebra like $L\mathbb{G}_n$ in full, we must have an understanding three things: its objects, its morphisms, and its $\mathrm{E}G$-action. We achieved the first of these in \cref{Zobj}, where we learnt that $\mathrm{Ob}(L\mathbb{G}_n) = \mathbb{Z}^{\ast n}$. Now \cref{LGaction} is telling us that for the third one we will need the action of $\mathbb{G}_{2n}$ and the values that $q$ takes on objects, which we do know, and the the values that $q$ takes on morphisms, which we do not. Therefore, our aim  going forward will be to work towards a description of the morphisms of $L\mathbb{G}_n$ and how the functor $q$ maps onto them, from which we will recover the whole of the $\mathrm{E}G$-algebra structure. For this task, it will be convenient to readopt our original point of view of $L\mathbb{G}_n$ as part of the initial object $\eta$; we will return to the question of $q$ later. 

\subsection{Refinement of $(\mathbb{G}_n \downarrow \mathrm{inv})$}

In the previous section, we learnt several important features of the initial object $\eta$ of $(\mathbb{G}_n \downarrow \mathrm{inv})$. We can use this information to make the job of finding $\eta$ easier, by carefully making changes to the category we are finding initial objects in. In particular, consider the category $C(\mathbb{G}_n)$ defined as follows:

\begin{defn}\label{Cdef} Let $C(\mathbb{G}_n)$ denote the full subcategory of $(\mathbb{G}_n \downarrow \mathrm{inv})$ on the objects $\psi: \mathbb{G}_n \to X$ for which
\begin{itemize}
\item $X$ has $\mathbb{Z}^{\ast n}$ as its monoid of objects
\item the restriction of $\psi$ to objects, $\psi_{\mathrm{ob}}: \mathbb{N}^{\ast n} \to \mathbb{Z}^{\ast n}$, is an injective monoid homomorphism
\item every morphism of $X$ can be written as $\alpha_X(g; \mathrm{id}_{w_1}, ..., \mathrm{id}_{w_m})$, for some $g \in G(m)$ and $w_i \in \mathbb{Z}^{\ast n}$, not necessarily uniquely
\end{itemize}
\end{defn}

Essentially, the objects of $C(\mathbb{G}_n)$ are exactly those objects of $(\mathbb{G}_n \downarrow \mathrm{inv})$ which possess all of the properties that we proved $\eta$ possesses in the previous section. \cref{Cdef} may only list the parallels of \cref{Zobj} and \cref{allmapsaction} explicitly, but the rest can be derived from these in exactly the same way that they were for $\eta$:

\begin{lem} Given any object $\psi: \mathbb{G}_n \to X$ of $C(\mathbb{G}_n)$, the following statements hold:
\begin{itemize}
\item the connected components of $X$ are $\mathbb{Z}^n$
\item the canonical map $[ \, \, ]: \mathrm{Ob}(X) \to \pi_0(X)$ sending objects to their components is the quotient map of abelianisation $\mathrm{ab}: \mathbb{Z}^{\ast n} \to \mathbb{Z}^n$
\item the restriction of $\psi$ to connected components, $\psi_\pi: \mathbb{N}^n \to \mathbb{Z}^n$, is the abelianisation of $\psi_{\mathrm{ob}}: \mathbb{N}^{\ast n} \to \mathbb{Z}^{\ast n}$, the restriction of $\psi$ on objects
\end{itemize}
Additionally, given any morphism $f:X \to Y$ from $\psi: \mathbb{G}_n \to X$ to $\chi: \mathbb{G}_n \to Y$ in $C(\mathbb{G}_n)$, the following statements hold:
\begin{itemize}
\item the restriction of $f$ to objects, $f_{\mathrm{ob}}: \mathbb{Z}^{\ast n} \to \mathbb{Z}^{\ast n}$, is injective
\item the restriction of $f$ to connected components, $f_\pi: \mathbb{Z}^n \to \mathbb{Z}^n$, is the abelianisation of $f_{\mathrm{ob}}$
\end{itemize}
\end{lem}
\begin{proof}
The three statements about the objects $\psi$ are exact parallels of the facts that we proved about $\eta$ in \cref{concomp}. The proofs of these statements is likewise exactly analagous to the proof of \cref{concomp}, with the definition of $C(\mathbb{G}_n)$ taking the place of \cref{Zobj} and \cref{allmapsaction}.

Next we'll check the statements about morphisms $f: \psi \to \chi$. From \cref{Cdef}, we know that both $\psi_{\mathrm{ob}}$ and $\chi_{\mathrm{ob}}$ are injective. Since
$f_{\mathrm{ob}} \psi_{\mathrm{ob}} = \chi_{\mathrm{ob}}$, it follows that $f_{\mathrm{ob}}$ is injective when restricted to the image of $\psi_{\mathrm{ob}}$. However, $f_{\mathrm{ob}}$ is a group homomorphism, which means that
\begin{eq*} f_{\mathrm{ob}}(w^*) \, \neq \, f_{\mathrm{ob}}(v^*) \quad \iff \quad f_{\mathrm{ob}}(w) \, \neq \, f_{\mathrm{ob}}(v) \end{eq*}
Thus if $f_{\mathrm{ob}}$ is injective on $\mathrm{im}(\psi_{\mathrm{ob}}) = \mathbb{N}^{\ast n}$ then it must also be injective on the free group generated by that image, $\langle \, \mathrm{im}(\psi_{\mathrm{ob}}) \, \rangle = \langle \, \mathbb{N}^{\ast n} \, \rangle = \mathbb{Z}^{\ast n}$. But this is the entire source of $f_{\mathrm{ob}}$, and so $f$ is just injective on objects. Moving on to connected components, we also see that
\begin{eq*} f_\pi \psi_\pi \, = \, \chi_\pi \, = \, (\chi_{\mathrm{ob}})^{\mathrm{ab}} \, = \, (f_{\mathrm{ob}})^{\mathrm{ab}} (\psi_{\mathrm{ob}})^{\mathrm{ab}} \, =  \,  (f_{\mathrm{ob}})^{\mathrm{ab}} \psi_\pi \end{eq*}
This means that $f_ \pi$ is equal to $(f_{\mathrm{ob}})^{\mathrm{ab}}$ on the image of $\chi_\pi$. But by the same line of reasoning as before, since $f_\pi$ is a group homomorphism we actually get equality on the whole group $\langle \, \mathrm{im}(\psi_\pi) \, \rangle = \langle \, \mathbb{N}^n \, \rangle = \mathbb{Z}^n$, which says that $f_ \pi$ and $(f_{\mathrm{ob}})^{\mathrm{ab}}$ are really equal everywhere.
\end{proof} 

Now, this $C(\mathbb{G}_n)$ that we've constructed is greatly restricted in comparison to $(\mathbb{G}_n \downarrow \mathrm{inv})$, with both fewer objects to worry about and more handy properties to exploit. Thus it would certainly be much easier for us to find the initial objects of in this newer category. But since we constructed $C(\mathbb{G}_n)$ precisely so that $\eta$ would still be included, it must be initial here too:

\begin{lem}\label{initialab} Let $\eta : \mathbb{G}_n \to Z$ be an initial object of $(\mathbb{G}_n \downarrow \mathrm{inv})$. Then it is also initial in $C(\mathbb{G}_n)$.
\end{lem}
\begin{proof}
By \cref{Zobj}, the objects of $Z$ are $\mathbb{Z}^{\ast n}$ and the restriction of $\eta$ to objects is the inclusion $\eta_{\mathrm{ob}} : \mathbb{N}^{\ast n} \to \mathbb{Z}^{\ast n}$. By \cref{allmapsaction}, every morphism of $Z$ can be written in the form $\alpha_Z(g; \mathrm{id}, ..., \mathrm{id})$, and by \ref{concomp} the restriction of $\eta$ to connected components is the inclusion $\eta_\pi = (\eta_{\mathrm{ob}})^{\mathrm{ab}} : \mathbb{N}^n \to \mathbb{Z}^n$. Thus, by design, $\eta$ is definitely an object of $C(\mathbb{G}_n)$. 

For any other object $\psi: \mathbb{G}_n \to X$ of $C(\mathbb{G}_n)$, we know that it is also an object of $(\mathbb{G}_n \downarrow \mathrm{inv})$, and so the initiality condition for $\eta$ tells us that there exists a unique $u : Z \to X$ from $\eta$ to $\psi$ in $(\mathbb{G}_n \downarrow \mathrm{inv})$. Since $C(\mathbb{G}_n)$ is a full subcategory of $(\mathbb{G}_n \downarrow \mathrm{inv})$, $u$ is also the unique morphism $\eta \to \psi$ in $C(\mathbb{G}_n)$, and so $\eta$ is initial there too.
\end{proof}

As a result of this, from now on we will work exclusively with $C(\mathbb{G}_n)$ instead of $(\mathbb{G}_n \downarrow \mathrm{inv})$.

\subsection{$L\mathbb{G}_n$ as a monoidal groupoid}

Recall that for objects $\psi: \mathbb{G}_n \to X$ of $C(\mathbb{G}_n)$, we always have that $X = X_{\mathrm{inv}}$. From this fact, along with the definition of $\mathbb{G}_n$, it is clear that all of the structures we will be working with are not just categories, but groupoids. This is very convenient, as groupoids are often much simpler than general categories, and we can exploit this simplicity to help find initial algebras more easily. In particular, we will use the fact that for any connected component of a groupoid, all of the homsets between its objects are isomorphic. This will allow us to describe any groupoid $X$ by splitting it into two smaller pieces --- one encoding information about the objects of $X$, and the other about the morphisms of the components $X$. First, we'll look at the part which details the morphisms.

\begin{defn} Let $X$ be a monoidal groupoid. A \emph{skeleton} of $X$ is a full subcategory of $X$, normally denoted $\mathrm{sk}(X)$, which contains exactly one object from each of the connected components of $X$. \end{defn}

It is well known that the canonical inclusion of a skeleton, $i: \mathrm{sk}(X) \to X$, is part of an equivalence between the two. Indeed, a skeleton of $X$ is the smallest subcategory for which this is true, in the sense that any proper subcategory of $\mathrm{sk}(X)$ will no longer be equivalent to $X$. We shall denote weak inverse of $i$ by $R: X \to \mathrm{sk}(X)$. This inverse is actually always strict in one direction --- $R \circ i = \mathrm{id}_{\mathrm{sk}(X)}$. In the other direction we have several natural isomorphisms, $\rho: i \circ R \Rightarrow \mathrm{id}_X$. 

We will think about the objects of a particular skeleton of $X$ as being a chosen set of `representatives' for each of the connected components of $X$. For this reason we will normally be loose with notation and just write $s \in \mathrm{Ob}(X)$ when talking about $i(s) \in \mathrm{Ob}(X)$ for $s \in \mathrm{sk}(X)$. Under this interpretation, the functor $R$ sends each object $x$ to its isomorphism class' representative $R(x)$, and $\rho_x : R(x) \to x$ is an isomorphism which witnesses that they are indeed in the same class. Notice also that the naturality of $\rho$ serves to define the action of $R$ on morphisms:
\begin{eq*} R( \, f: x \to y \,) \, = \, \rho_y^{-1} \circ f \circ \rho_x \end{eq*}

Since $X$ is a (strict) monoidal category, the equivalence $(i, R, \rho, \mathrm{id})$ naturally induces on $\mathrm{sk}(X)$ the structure of a weak monoidal category:

\begin{defn}\label{boxtimes} Let $\boxtimes$ be the weak monoidal product defined on $\mathrm{sk}(X)$ by
\begin{eq*} \begin{tikzcd}
\mathrm{sk}(X) \times \mathrm{sk}(X) \ar[d, "i \times i"'] \ar[rr, "\boxtimes"] & & \mathrm{sk}(X) \\
X \times X \ar[rr, "\otimes"] & & X \ar[u, "R"']
\end{tikzcd} \end{eq*}
This new product has unit object $R(I)$ and coherence data
\begin{eq*} \begin{tikzcd}
(s \boxtimes s') \boxtimes s'' \ar[d, equal] \ar[rr, "a^{\mathrm{sk}(X)}_{s,s',s''}"] & & s \boxtimes (s' \boxtimes s') \\
R\big( \, R(s \otimes s') \otimes s'' \, \big) \ar[d, "R( \rho_{s \otimes s'} \otimes \mathrm{id}_{s''})"'] & & R\big( \, s \otimes R(s' \otimes s'') \, \big) \ar[u, equal] \\
R\big( \, (s \otimes s') \otimes s'' \, \big) \ar[rr, "R(a^X_{s,s',s''})", equal] & & R\big( \, s \otimes (s' \otimes s'') \, \big) \ar[u, " R(\mathrm{id}_{s} \otimes \rho_{s' \otimes s''})^{-1}"']
\end{tikzcd} \end{eq*}
\begin{eq*} \begin{tikzcd}
R(I) \boxtimes s \ar[d, equal] \ar[rr, "l^{\mathrm{sk}(X)}_s"] & & s & & s \boxtimes R(I) \ar[d, equal] \ar[rr, "r^{\mathrm{sk}(X)}_s"] & & s \\
R\big( \, R(I) \otimes s \, ) \ar[d, "R(\rho_I \otimes \mathrm{id}_s)"'] & & & & R\big( \, s \otimes R(I) \, ) \ar[d, "R(\mathrm{id}_s \otimes \rho_I)"'] & & \\
R(I \otimes s) \ar[rr, "R(l^X_s)", equal] & & R(s) \ar[uu, equal] & & R(s \otimes I) \ar[rr, "R(r^X_s)", equal] & & R(s) \ar[uu, equal] 
\end{tikzcd} \end{eq*}
\end{defn}

Moreover, as this weak monoidal structure $\boxtimes$ was induced on $\mathrm{sk}(X)$ by the equivalence $(i, R, \rho, \mathrm{id})$, it follows immediately that $(i, R, \rho, \mathrm{id})$ is a weak monoidal equivalence with respect to $\boxtimes$. Explicitly:

\begin{cor}\label{iRdata} There exists coherence data
\begin{eq*} \begin{array}{rclrrcl}
		\mu^R_{x, x'} & = & R( \rho_x \otimes \rho_{x'} ) & \quad & \eta^R & = & \mathrm{id}_{R(I)} \\
		\mu^i_{s, s'} & = & \rho^{-1}_{s \otimes s'} & & \eta^i & = & \rho^{-1}_I \\
		\end{array}
\end{eq*}
making $R, i$ into weak monoidal functors and $\rho$ into a monoidal natural transformation. 
\end{cor}

It is important to note that while the product $\boxtimes$ is strictly associative and unital
\begin{eq*} \begin{array}{rcl}
		(s \boxtimes s') \boxtimes s'' & = & R\big( \, R(s \otimes s') \otimes s'' \, \big)  \\
		& = & R(s \otimes s' \otimes s'') \\
		& = & R\big(\, s \otimes R(s' \otimes s'') \, \big) \\
		& = & s \boxtimes (s'' \boxtimes s'') 
		\end{array}
\end{eq*}
\begin{eq*} \begin{array}{rcllrcl}
		R(I) \boxtimes s & = & R\big( \, R(I) \otimes s \,) & \quad & s \boxtimes R(I) & = & R\big( \, s \otimes R(I) \, ) \\
		& = & R(I \otimes s) & & & = & R(s \otimes I) \\
		& = & R(s) & & & = & R(s) \\
		& = & s & & & = & s \\
		\end{array}
\end{eq*}
nevertheless the associator and unitors of $\mathrm{sk}(X)$ are in general \emph{not} identities.

Finally, because $\mathrm{sk}(X)$ contains one object for each of the elements of $\pi_0(X)$ there is an obvious functor, isomorphic on objects, between the two:
\begin{eq*} \begin{array}{rrrll}
		[ \, \_ \, ] & : & \mathrm{sk}(X) & \to & \pi_0(X) \\
		& : & s & \mapsto & [s] \\
		& : & f: s \to s & \mapsto & id_{[s]}
		\end{array}
\end{eq*}
This functor is clearly weak monoidal; the only morphisms we have to pick the coherence maps $\mu^{[ \, \_ \, ]}_{s, s'}$ and $\eta^{[ \, \_ \, ]}$ from are identities, which is well-defined because
\begin{eq*} [s] \otimes [s'] \, = \, [s' \otimes s'] \, = \, [ \, R(s' \otimes s') \, ] \, = \, [ s \boxtimes s' ], \quad \quad [I] \, = \, [ \, R(I) \, ] \end{eq*}
and since the coherence data $a, l, r$ of $\mathrm{sk}(X)$ will all be mapped onto identities too the conditions for a weak monoidal functor will be satisfied trivially.

Next, we will define the subcategory which will describe the objects of our monoidal groupoids.

\begin{defn} Let $X$ be a monoidal groupoid, and consider a new category which has the same objects as $X$, and has a unique morphism between two objects if and only if there is at least one morphism between them in $X$. We call this $\mathrm{po}(X)$, since it is always a posetal category. \end{defn}

Because $X$ is a groupoid $\mathrm{po}(X)$ must be one too, and it also inherits a strict monoidal product directly from $X$. Furthermore, $\mathrm{po}(X)$ has exactly the same objects and connected components that $X$ does, and so we can construct two obvious strict monoidal functors,
\begin{eq*} \begin{array}{rrrll}
		P & : & X & \to & \mathrm{po}(X) \\
		& : & x & \mapsto & x \\
		& : & f: x \to y & \mapsto & x \to y
		\end{array}
\end{eq*}
and
\begin{eq*} \begin{array}{rrrll}
		[ \, \_ \, ] & : & \mathrm{po}(X) & \to & \pi_0(X) \\
		& : & x & \mapsto & [x] \\
		& : & x \to y & \mapsto & id_{[x]}
		\end{array}
\end{eq*}
Here we denote the morphisms of $\mathrm{po}(X)$ by giving their unique source and target, rather than assigning them any particular names. 

Putting this together with what we had for $\mathrm{sk}(X)$, we can now express precisely how we are going to split our monoidal groupoids:

\begin{prop}\label{pullback} Let $X$ be a monoidal groupoid. Then for any choice of skeleton $\mathrm{sk}(X)$, the commutative diagram
\begin{eq*} \begin{tikzcd}
& X \ar[dl, "R"'] \ar[dr, "P"] & \\
\mathrm{sk}(X) \ar[dr, "\lbrack \, \_ \, \rbrack"'] & & \mathrm{po}(X) \ar[dl, "\lbrack \, \_ \, \rbrack"] \\
& \pi_0(X) &
\end{tikzcd} \end{eq*}
is a pullback square. That is, 
\begin{eq*} X \quad \cong \quad \mathrm{sk}(X) \times_{\pi_0(X)} \mathrm{po}(X) \end{eq*}
\end{prop}
\begin{proof}
First of all, we need to check that the above diagram actually commutes. Notice that $[ \, P( \, \_ \, ) \, ]$ and $[ \, R( \, \_ \, ) \, ]$ are both strict monoidal functors, since their coherence data lives in the category $\pi_0(X)$, which contains only identity morphisms. This means we only need to check that their underlying functors are equal, which is simple enough:
\begin{eq*} [ \, R(x) \, ] \, = \, [x] \, = \, [ \, P(x) \, ], \quad \quad \quad [ \, R(f: x \to y) \, ] \, = \, id_{[x]} \, = \, [ \, P(f) \, ]\end{eq*}
In other words, both  $[ \, P( \, \_ \, ) \, ]$ and $[ \, R( \, \_ \, ) \, ]$ are the obvious map $[ \, \_ \, ]: X \to \pi_0(X)$ sending objects to their connected components. 

Now, assume that we are given a pair of weak monoidal functors $S: Y \to \mathrm{sk}(X)$ and $Q: Y \to \mathrm{po}(X)$ which also form an appropriate commutative square --- that is, $[ \, S( \, \_ \, ) \, ] = [ \, Q( \, \_ \, ) \, ]$. We wish to construct a unique map $U: Y \to X$ which factors $S$ and $Q$ through $R$ and $P$ respectively:
\begin{eq*} \begin{tikzcd}
& Y \ar[dd, dashrightarrow, "U"] \ar[dddl, bend right, "S"'] \ar[dddr, bend left, "Q"] & \\
&& \\
& X \ar[dl, "R"'] \ar[dr, "P"] & \\
\mathrm{sk}(X) \ar[dr, "\lbrack \, \_ \, \rbrack"'] & & \mathrm{po}(X) \ar[dl, "\lbrack \, \_ \, \rbrack"] \\
& \pi_0(X) &
\end{tikzcd} \end{eq*}
Let $y$ be an object of $Y$. In order for the right-hand triangle in the above diagram to commute, we need that $Q(y) = PU(y) = U(y)$. If we take this to be the definition of $U$ on objects, then we see that the left-hand triangle will also commute on objects too, since
\begin{eq*} \begin{array}{rrccl}
		& [ \, S(y) \, ] & = & [ \, Q(y) \, ] & \\
		& & = & [ \, U(y) \, ] & \\
		\implies & S(y) & \cong & U(y), & S(y) \in \mathrm{sk}(X) \\
		\implies & S(y) & = & RU(y)
		\end{array}
\end{eq*}
Similarly, let $f: y \to y'$ be a morphism from $Y$. Since the right-hand triangle of the diagram commutes on objects and $P$ is determined entirely by its values on objects, we know that $PU(f) = Q(f)$ regardless of what $U$ does to morphisms. In order for the left-hand triangle to commute as well, we need $S(f) = RU(f) = \rho_{y'}^{-1} U(f)\rho_y$, and we can again take this to be definitional, so that $U(f) := \rho_{y'} S(f) \rho_y^{-1}$.
 
Next, we must determine the coherence data for $U$. For the right-hand triangle we need
\begin{eq*} \eta^Q = P(\eta^U) \circ \eta^P, \quad \quad \quad \mu^Q = P(\mu^U) \circ \mu^P, \end{eq*}
but since morphisms in $\mathrm{po}(X)$ are uniquely specified by their source and target, these equalities necessarily hold. Similarly, the left-hand triangle gives
\begin{eq*} \begin{array}{rrlllrll}
		& \eta^S & = & R(\eta^U) \circ \eta^R & & \mu^S & = & R(\mu^U) \circ \mu^R \\
		& & = & \rho_{U(I)}^{-1} \, \eta^U \, \rho_{I} \, \eta^R & & & = & \rho_{U(\_ \otimes \_)}^{-1} \, \mu^U \, \rho_{U(\_) \otimes U(\_)} \, \mu^R \\
		&&&&&&& \\
		\implies & \eta^U & = & \rho_{U(I)} \, \eta^S \, (\eta^R)^{-1} \, \rho_{I}^{-1} & & \mu^U & = & \rho_{U(\_ \otimes \_)} \, \mu^S \, (\mu^R)^{-1} \, \rho_{U(\_) \otimes U(\_)}^{-1} \\
		& & = & \rho_{U(I)} \, \eta^S \, (\eta^R)^{-1} \, \rho_{I}^{-1} & & \mu^U & = & \rho_{U(\_ \otimes \_)} \, \mu^S \, (\mu^R)^{-1} \, \rho_{U(\_) \otimes U(\_)}^{-1} \\
		\end{array}
\end{eq*}

Since this definition of $U$ has been forced on us by the requirement that it fit into a certain commutative diagram, clearly $U$ is the unique functor with that property. However, to complete the proof we also need to verify that what we have constructed is actually a well-defined weak monoidal functor. To see that $U$ respects associators, consider the following diagram:
\begin{eq*} \begin{tikzcd}
& \bullet \ar[rrrr, "a^{\mathrm{sk}(X)}"] \ar[d, "\mu^R \, \boxtimes \, \mathrm{id}"'] & & & & \bullet \ar[d, "\mathrm{id} \, \boxtimes \, \mu^R"] & \\
& \bullet \ar[dl, "R(\mu^U) \, \boxtimes \, \mathrm{id}"'] \ar[dr, "\mu^R"] & & & & \bullet \ar[dr, "\mathrm{id} \, \boxtimes \, R(\mu^U)"] \ar[dl, "\mu^R"'] & \\
\bullet \ar[dr, "\mu^R"'] & & \bullet \ar[rr, "R(a^X)"] \ar[dl, "R(\mu^U \, \otimes \, \mathrm{id})"] & & \bullet \ar[dr, "R(\mathrm{id} \, \otimes \, \mu^U)"'] & & \bullet \ar[dl, "\mu^R"] \\
& \bullet \ar[d, "R(\mu^U)"'] & & & &\bullet \ar[d, "R(\mu^U)"] & \\
& \bullet \ar[rrrr, "RU(a^Y)"] & & & & \bullet &
\end{tikzcd} \end{eq*}
Here the vertices have been left unlabeled because otherwise the long object names, like $\big( \, RU(y) \boxtimes RU(y') \, \big) \boxtimes RU(y'')$, cause the diagram to be too cluttered to read easily. Objects can be inferred from which morphisms they are the source and target of.

The topmost region in this diagram is just the associativity coherence condition for $R$, and hence it commutes. Likewise, the outside edges of this diagram commute because they form the associativity condition for $S$, via $\mu^S = R(\mu^U) \mu^R$. Lastly, the two areas on either side of the diagram commute by naturality of $\mu^R$. As a result of this, and the fact that every edge of the diagram is invertible, it follows that the bottom rectangle the diagram commutes too. Focusing on this area, we see that is is the image under $R$ of the associativity condition for $U$. But since $R$ just acts on morphisms by $f \mapsto \rho_{y'}^{-1} f \rho_y$, which is bijective, the full condition for $U$ is immediately recoverable. 

We can prove that $U$ respects unitors in a similar way. Consider the following diagram:
\begin{eq*} \begin{tikzcd}
& & \bullet & & \\
\bullet \ar[urr, "l^Y"] \ar[dd, "\eta^R \, \boxtimes \, \mathrm{id}"'] & & & & \bullet \ar[ull, "RU(l^{\mathrm{sk}(X)})"'] \\
& & \bullet \ar[uu, "R(l^X)"'] \ar[drr, "R(\eta^U \, \otimes  \, \mathrm{id})"'] & & \\
\bullet \ar[urr, "\mu^R"] \ar[drr, "R(\eta^U) \, \boxtimes \mathrm{id}"'] & & & & \bullet \ar[uu, "R(\mu^U)"'] \\
& & \bullet \ar[urr, "\mu^R"'] & &
\end{tikzcd} \end{eq*}
The top-left square is the left unitality coherence condition for $R$; the outer edges form the same coherence condition but for $S$ with $\eta^S = R(\eta^U) \eta^R$; and the bottom square follows from the naturality of $\mu^R$. Hence, all of those parts of the diagram commute, and since again all edges are invertible, we can conclude that the remaining top-right square also commutes. But this is just the image under $R$ of left unitality for $\eta^U$, and again we can obtain the original from this by bijectivity. The proof of the right unitality coherence condition for $U$ proceeds in exactly the same way, except involving $r$ instead of $l$. Therefore, $U$ is indeed a well-defined weak monoidal functor, and hence $X$ is the required pullback.
\end{proof}

For any of the objects $\psi : \mathbb{G}_n \to X$ of $C(\mathbb{G}_n)$, we now have a way of breaking down their source and target into smaller groupoids representing their connected components and automorphisms. We might wonder if we can do the same sort of thing to $\psi$ itself --- show that it is equivalent to several `smaller' maps running between corresponding terms of $\mathrm{sk}(\mathbb{G}_n) \times_{\pi_0(\mathbb{G}_n)} \mathrm{po}(\mathbb{G}_n)$ and $\mathrm{sk}(X) \times_{\pi_0(X)} \mathrm{po}(X)$. This would allow us to contruct the initial object of $C(\mathbb{G}_n)$ by finding the initial versions of these reduced maps, greatly simplifying the problem. It turns out that this procedure is relatively straightforward.

\begin{prop}\label{factor1} Let $X$ and $Y$ be monoidal groupoids with chosen skeletons $\mathrm{sk}(X)$ and $\mathrm{sk}(Y)$ respectively. Then for any weak monoidal functor $F : X \to Y$ there exists a unique pair of weak monoidal functors
\begin{eq*} F_{\mathrm{sk}} : \mathrm{sk}(X) \to \mathrm{sk}(Y), \quad \quad F_{\mathrm{po}} : \mathrm{po}(X) \to \mathrm{po}(Y) \end{eq*}
such that the diagrams
\begin{eq*} \begin{tikzcd}
X \ar[r, "F"] \ar[d, "R^X"'] & Y \ar[d, "R^Y"] & & X \ar[r, "F"] \ar[d, "P^X"'] & Y \ar[d, "P^Y"] \\
\mathrm{sk}(X) \ar[r, "F_{\mathrm{sk}}"] \ar[d, "\lbrack \, \_ \, \rbrack"'] & \mathrm{sk}(Y) \ar[d, "\lbrack \, \_ \, \rbrack"] & & \mathrm{po}(X) \ar[r, "F_{\mathrm{po}}"] \ar[d, "\lbrack \, \_ \, \rbrack"'] & \mathrm{po}(Y) \ar[d, "\lbrack \, \_ \, \rbrack"] \\
\pi_0(X) \ar[r, "F_\pi"] & \pi_0(Y) & & \pi_0(X) \ar[r, "F_\pi"] & \pi_0(Y)
\end{tikzcd} \end{eq*}
commute, where $F_{\pi}$ is the restriction of $F$ on connected components.
\end{prop}
\begin{proof}
To begin, recall that we saw in the proof of \cref{pullback} that $[ \, R^X \, ]$, $[ \, R^Y \, ]$, $[ \, P^X \, ]$, and $[ \, P^Y \, ]$ are just the canonical maps $\mathrm{Ob}(X) \to \pi_0(X)$ or $\mathrm{Ob}(Y) \to \pi_0(Y)$ sending objects to their connected components. It follows that the outside edges of both diagrams will commute by the definition of $F_{\pi}$, regardless of our choice of $F_{\mathrm{sk}}$ and $F_{\mathrm{po}}$.

Next, notice that if a particular choice of $F_{\mathrm{sk}}$ makes the top square of its diagram commute, then the bottom square will automatically do so too. This is because if we have such an $F_{\mathrm{sk}}$, a quick diagram chase yields
\begin{eq*} [ \, F_{\mathrm{sk}} R^X \, ] \, = \, F_{\pi}[ \, R^X \, ]  \end{eq*}
and precomposing this by $i^X$ reduces it to
\begin{eq*} [ \, F_{\mathrm{sk}} \, ] \, = \, [ \, F_{\mathrm{sk}} R^X  i^X \, ] \, = \, F_{\pi}[ \, R^X i^X \, ] \, = \, F_{\pi}[ \, \_ \, ] \end{eq*}
However, the exact same method of precomposing by $i^X$ shows that there is only one choice of $F_{\mathrm{sk}}$ for which the top square of the diagram commutes:  
\begin{eq*} F_{\mathrm{sk}} R^X \, = \, R^Y F \quad \implies \quad F_{\mathrm{sk}} \, = \, F_{\mathrm{sk}} R^X i^X \, = \, R^Y F i^X \end{eq*}
Therefore, this is the unique weak monoidal functor $F_{\mathrm{sk}}$ we are looking for.

Lastly, consider $F_{\mathrm{po}}$. The value it takes on objects follows immediately from the requirement that the top square of its diagram commutes:
\begin{eq*} F_{\mathrm{po}}(x) \, = \,  F_{\mathrm{po}} P^X(x) \, = \, P^Y F(x)  \, = \, F(x) \end{eq*}
Its action on morphisms is then fixed by the fact that morphisms of $\mathrm{po}(X)$ and $\mathrm{po}(Y)$ are uniquely determined by their sources and targets. Moreover, since $\mathrm{po}(Y)$ is a strictly monoidal category, the coherence data $\eta$, $\mu$ for the weak monoidal functor $F_{\mathrm{po}}$ will have components which are all automorphisms of certain objects. Thus, uniqueness of morphisms on a given source and target also tells us that these components are identities, and hence that $F_{\mathrm{po}}$ is strict.

With everything about $F_{\mathrm{po}}$ now known, the following is sufficient to show that the bottom square of its diagram commutes as well:
\begin{eq*} \begin{array}{rllrrlll}
		\left[ \, F_{\mathrm{po}}(x) \, \right] & = & [ \, F(x) \, ] & \quad \quad & [ \, F_{\mathrm{po}}(x \to x') \, ] & = & [ \, F(x) \to F(x') \, ] \\
		& = & F_{\pi}\big( \, [x] \, \big) & & & = & \mathrm{id}_{[ F(x) ]} \\
		& & & & & = & \mathrm{id}_{F_{\pi}([x])} \\
		& & & & & = & F_{\pi}(\mathrm{id}_{[x]}) \\
		& & & & & = & F_{\pi}\big( \, [x \to x'] \, \big) \\
		\end{array}
 \end{eq*}
Therefore, we have a found the unique $F_{\mathrm{sk}}$ and $F_{\mathrm{po}}$ with the required properties.
\end{proof}

\begin{prop} \label{factor2} Let $X$ and $Y$ be monoidal groupoids with chosen skeletons $\mathrm{sk}(X)$ and $\mathrm{sk}(Y)$ respectively. Then for any pair of weak monoidal functors
\begin{eq*} F_{\mathrm{sk}} : \mathrm{sk}(X) \to \mathrm{sk}(Y), \quad \quad F_{\mathrm{po}} : \mathrm{po}(X) \to \mathrm{po}(Y) \end{eq*}
that have the same restriction to connected components, $F_{\pi}$, there exists a unique weak monoidal functor $F : X \to Y$ such that the following diagrams commute:
\begin{eq*} \begin{tikzcd}
X \ar[r, "F"] \ar[d, "R^X"'] & Y \ar[d, "R^Y"] & & X \ar[r, "F"] \ar[d, "P^X"'] & Y \ar[d, "P^Y"] \\
\mathrm{sk}(X) \ar[r, "F_{\mathrm{sk}}"] & \mathrm{sk}(Y) & & \mathrm{po}(X) \ar[r, "F_{\mathrm{po}}"] & \mathrm{po}(Y)
\end{tikzcd} \end{eq*}
\end{prop}
\begin{proof}
The fact that $F_{\pi}$ is the underlying map on connected components for both $F_{\mathrm{sk}}$ and $F_{\mathrm{po}}$ means that
\begin{eq*} [ \, F_{\mathrm{sk}} \, ] \, = \, F_{\pi}[ \, \_ \, ] \, = \, [ \, F_{\mathrm{po}} \, ] \end{eq*}
and, by \cref{pullback}, we also know
\begin{eq*} [ \, R^X \, ] \, = \, [ \, P^X \, ] \end{eq*}
It follows from these that
\begin{eq*} [ \, F_{\mathrm{sk}} R^X \, ] \, = \, F_{\pi}[ \, R^X \, ] \, = \, F_{\pi}[ \, P^X \, ] \, = \, [ \, F_{\mathrm{po}} P^X \, ] \end{eq*}
or, in other words, the outside edges of the following diagram commute:
\begin{eq*} \begin{tikzcd}
& X \ar[dd, dashrightarrow, "F"] \ar[dddl, bend right, "F_{\mathrm{sk}} R^X"'] \ar[dddr, bend left, "F_{\mathrm{po}} P^X"] & \\
&& \\
& Y \ar[dl, "R^Y"'] \ar[dr, "P^Y"] & \\
\mathrm{sk}(Y) \ar[dr, "\lbrack \, \_ \, \rbrack"'] & & \mathrm{po}(Y) \ar[dl, "\lbrack \, \_ \, \rbrack"] \\
& \pi_0(Y) &
\end{tikzcd} \end{eq*}
But the bottom region of this diagram is a pullback square, again by \cref{pullback}, and so there must exist a unique map $F: X \to Y$ as shown making the top left and top right regions of the diagram commute, as required.
\end{proof}

Propositions \ref{factor1} and \ref{factor2} together tell us that we can break apart any object $\psi : \mathbb{G}_n \to X$ of $C(\mathbb{G}_n)$ into two simpler maps $\psi_{\mathrm{sk}}$ and $\psi_{\mathrm{po}}$, and that we can always recover $\psi$ from them again. Thus, if we want to find the initial objects $\eta : \mathbb{G}_n \to Z$ of $C(\mathbb{G}_n)$, it will suffice to calculate $\eta_{\mathrm{sk}}$ and $\eta_{\mathrm{po}}$. But as it happens, we have already proven enough about the initial algebra to get the second of these:

\begin{lem}\label{polem} Let $\eta : \mathbb{G}_n \to Z$ be the initial object of $C(\mathbb{G}_n)$. Then $\eta_{\mathrm{po}}$ is the inclusion
\begin{eq*} \bigsqcup_{w \in \mathbb{N}^n} \mathrm{E}\big( \, q_{\mathbb{N}^{\ast n}}^{-1}(w) \, \big) \quad \hookrightarrow \quad \bigsqcup_{w \in \mathbb{Z}^n} \mathrm{E}\big( \, q_{\mathbb{Z}^{\ast n}}^{-1}(w) \, \big) \end{eq*}
where the $q$ are the appropriate quotients of abelianisation, and the monoidal products are inherited from $\mathbb{N}^{\ast n}$ and $\mathbb{Z}^{\ast n}$ in the obvious way.
\end{lem}
\begin{proof}
We know from \cref{Zobj} that action of $\eta$ on objects is given by the inclusion map $\mathbb{N}^{\ast n} \to \mathbb{Z}^{\ast n}$, and in the proof of \cref{factor1} we saw that $\eta_{\mathrm{po}}$ is a strict monoidal functor which is completely determined by $\eta_{\mathrm{ob}}$. Thus, $\eta_{\mathrm{po}}$ is the inclusion $\mathrm{po}(\mathbb{G}_n) \to \mathrm{po}(Z)$. By \cref{concomp} the connected components of $\mathbb{G}_n$ are $\mathbb{N}^n$ and for $Z$ they are $\mathbb{Z}^n$, each assigned by abelianisation. Thus $\mathrm{po}(\mathbb{G}_n)$ and $\mathrm{po}(Z)$ are isomorphic as categories to the coproducts given in the lemma, with $\otimes$ induced by $\eta_{\mathrm{ob}}: \mathbb{N}^{\ast n} \to \mathbb{Z}^{\ast n}$.
\end{proof}

Therefore, the final step is to try to find $\eta_{\mathrm{sk}}$. For that purpose, we will define one last new category to work in.

\begin{defn} Fix a choice of skeleton $\mathrm{sk}(\mathbb{G}_n)$ of $\mathbb{G}_n$. Then we define the category $C_{\mathrm{sk}}(\mathbb{G}_n)$ in the following way:
\begin{itemize}
\item an object of $C_{\mathrm{sk}}(\mathbb{G}_n)$ is any weak monoidal functor $\chi: \mathrm{sk}(\mathbb{G}_n) \to \mathrm{sk}(X)$ whose target is a choice of skeleton for some monoidal category $X$ that is the target of at least one object $\psi: \mathbb{G}_n \to X$ of $C(\mathbb{G}_n)$
\item a morphism $f: \chi \to \chi'$ in $C_{\mathrm{sk}}(\mathbb{G}_n)$ between two objects with targets $\mathrm{sk}(X)$ and $\mathrm{sk}(X')$ respectively is a weak monoidal functor $f: \mathrm{sk}(X) \to \mathrm{sk}(X')$ such that $\chi' = f \circ \chi$
\end{itemize}
\end{defn}

This definition may seem a little strange at first, but the following lemma should show why it is a useful one:

\begin{lem}\label{Csklem} A functor $\chi$ is an object of $C_{\mathrm{sk}}(\mathbb{G}_n)$ if and only if there exists an object $\psi$ of $C(\mathbb{G}_n)$ such that $\psi_{\mathrm{sk}} = \chi$.
\end{lem}
\begin{proof}
It should be clear from the previous definition that if $\psi$ is in $C(\mathbb{G}_n)$ then $\psi_{\mathrm{sk}}$ is in $C_{\mathrm{sk}}(\mathbb{G}_n)$. For the converse, let $\chi: \mathrm{sk}(\mathbb{G}_n) \to \mathrm{sk}(X)$ be an object of $C_{\mathrm{sk}}(\mathbb{G}_n)$ and consider the composite
\begin{eq*} \begin{tikzcd}
\mathrm{Ob}(\mathbb{G}_n) \ar[r, "\lbrack \, \_ \, \rbrack"] & \pi_0(\mathbb{G}_n) \ar[d, equal] & & & \\
& \pi_0\big( \mathrm{sk}(\mathbb{G}_n) \big) \ar[r, "\chi_{\pi}"] & \pi_0\big( \mathrm{sk}(X) \big) \ar[r, "\lbrack \, \_ \, \rbrack^{-1}"] & \mathrm{Ob}\big(\mathrm{sk}(X) \big) \ar[r, "i^X"] & \mathrm{Ob}(X)
\end{tikzcd} \end{eq*}
Note that this makes sense because $\mathrm{sk}(X)$ is skeletal, and so $[ \, \_ \, ] : \mathrm{Ob}(\mathrm{sk}(X)) \to \pi_0(\mathrm{sk}(X))$ an isomorphism on objects. From this composite we can define a unique map $\chi': \mathrm{po}(\mathbb{G}_n) \to \mathrm{po}(X)$, since such maps are completely determined by their behaviour on objects. But because of how $\chi'$ was constructed, $\chi'_{\pi} = \chi_{\pi}$, and so we can apply \cref{factor2} to $\chi, \chi'$ to obtain an object $\psi : \mathbb{G}_n \to X$ of $C(\mathbb{G}_n)$ for which $\psi_{\mathrm{sk}} = \chi$.
\end{proof}

The exact same reasoning also shows that the morphism of $C_{\mathrm{sk}}(\mathbb{G}_n)$ are just the maps $f_{\mathrm{sk}}: \mathrm{sk}(X) \to \mathrm{sk}(Y)$ for each morphism $f: X \to Y$ in $C(\mathbb{G}_n)$. Now we can put this new definition to use.

\begin{prop}\label{initialsk} For any initial object $\eta: \mathbb{G}_n \to Z$ of $C(\mathbb{G}_n)$, the functor $\eta_{\mathrm{sk}}$ is an initial object of $C_{\mathrm{sk}}(\mathbb{G}_n)$.
\end{prop}
\begin{proof}
Let $\chi: \mathrm{sk}(\mathbb{G}_n) \to \mathrm{sk}(X)$ be an arbitrary object of $C_{\mathrm{sk}}(\mathbb{G}_n)$. If we wish to show that $\eta_{\mathrm{sk}}$ is initial, we need to construct the unique morphism $u: \mathrm{sk}(Z) \to \mathrm{sk}(X)$ in $C_{\mathrm{sk}}(\mathbb{G}_n)$ such that $\chi = u \circ \eta_{\mathrm{sk}}$.

To begin, choose any object $\psi: \mathbb{G}_n \to X$ of $C(\mathbb{G}_n)$ with the property that $\psi_{\mathrm{sk}} = \chi$. By \cref{Csklem} at least one such $\psi$ must exist. We can use the fact that $\eta$ is an initial object of $C(\mathbb{G}_n)$ to find a unique morphism $v: Z \to X$ such that $\psi = v \circ \eta$, and then we can apply \cref{factor1} to obtain from it a morphism $v_{\mathrm{sk}}: \mathrm{sk}(Z) \to \mathrm{sk}(X)$ of $C_{\mathrm{sk}}(\mathbb{G}_n)$. This is all summarised by the following diagram:
\begin{eq*} \begin{tikzcd}
& \mathbb{G}_n \ar[dd, "R^{\mathbb{G}_n}" near end] \ar[ld, "\eta"'] \ar[rd, "\psi"] & \\
Z \ar[dd, "R^Z"'] \ar[rr, "v" near start, crossing over]& & X \ar[dd, "R^X"] \\
& \mathrm{sk}(\mathbb{G}_n)\ar[ld, "\eta_{\mathrm{sk}}"'] \ar[rd, "\chi"] & \\
\mathrm{sk}(Z) \ar[rr, "v_{\mathrm{sk}}"]& & \mathrm{sk}(X)
\end{tikzcd} \end{eq*}
Every face of this diagram is known to commute except for the bottom one, and so it follows that
\begin{eq*} \chi R^{\mathbb{G}_n} \, = \, v_{\mathrm{sk}} \eta_{\mathrm{sk}} R^{\mathbb{G}_n} \end{eq*}
and hence
\begin{eq*} \chi \, = \, \chi R^{\mathbb{G}_n} i^{\mathbb{G}_n} \, = \, v_{\mathrm{sk}} \eta_{\mathrm{sk}} R^{\mathbb{G}_n} i^{\mathbb{G}_n} \, = \, v_{\mathrm{sk}} \eta_{\mathrm{sk}} \end{eq*}
Therefore, $v_{\mathrm{sk}}$ shows that there is at least one morphism of $C_{\mathrm{sk}}(\mathbb{G}_n)$ which satisfies the condition that we want $u$ to. 

To complete the proof, assume now that $u, u': \mathrm{sk}(Z) \to \mathrm{sk}(X)$ are both morphisms of $C_{\mathrm{sk}}(\mathbb{G}_n)$ with $\chi = u \circ \eta_{\mathrm{sk}} = u' \circ \eta_{\mathrm{sk}}$. From $u$, define a new map $w: \mathrm{po}(Z) \to \mathrm{po}(X)$ which is the unique such map that acts on objects as the composite
\begin{eq*} \begin{tikzcd}
\mathrm{Ob}(Z) \ar[r, "\lbrack \, \_ \, \rbrack"] & \pi_0(Z) \ar[d, equal] & & & \\
& \pi_0\big( \mathrm{sk}(Z) \big) \ar[r, "u_{\pi}"]  & \pi_0\big( \mathrm{sk}(X) \big) \ar[r, "\lbrack \, \_ \, \rbrack^{-1}"] & \mathrm{Ob}\big(\mathrm{sk}(X) \big) \ar[r, "i^X"] & \mathrm{Ob}(X)
\end{tikzcd} \end{eq*}
We'll also define a map $w': \mathrm{po}(Z) \to \mathrm{po}(X)$ from $u'$ in the precisely analogous way. Now, by design $w, w'$ act on connected components in the same way that $u, u'$ do respectively --- that is, $w_{\pi} = u_{\pi}$ and $w'_{\pi} = u'_{\pi}$. Thus by \cref{factor2} we can combine $u$ and $w$ into a unique morphism $v: Z \to X$ from $C(\mathbb{G}_n)$, and likewise get a unique $v': Z \to X$ from $u', w'$. Moreover, by applying \cref{factor1} to break $v$ and $v'$ back down again, we see that $v = v'$ if and only if $u = u'$. As before, we have diagrams
\begin{eq*} \begin{tikzcd}
& \mathbb{G}_n \ar[dd, "R^{\mathbb{G}_n}" near end] \ar[ld, "\eta"'] \ar[rd, "\psi"] & & & \mathbb{G}_n \ar[dd, "R^{\mathbb{G}_n}" near end] \ar[ld, "\eta"'] \ar[rd, "\psi"] & \\
Z \ar[dd, "R^Z"'] \ar[rr, "v" near start, crossing over]& & X \ar[dd, "R^X"] & Z \ar[dd, "R^Z"'] \ar[rr, "v'" near start, crossing over]& & X \ar[dd, "R^X"] \\
& \mathrm{sk}(\mathbb{G}_n)\ar[ld, "\eta_{\mathrm{sk}}"'] \ar[rd, "\chi"] & & & \mathrm{sk}(\mathbb{G}_n)\ar[ld, "\eta_{\mathrm{sk}}"'] \ar[rd, "\chi"] & \\
\mathrm{sk}(Z) \ar[rr, "u"]& & \mathrm{sk}(X) & \mathrm{sk}(Z) \ar[rr, "u'"]& & \mathrm{sk}(X)
\end{tikzcd} \end{eq*}
but this time all of the faces are known to commute except for the top ones. It follows from these that
\begin{eq*} R^X \psi \, = \, R^X v \eta \, = \, R^X v' \eta \end{eq*}
However, recall that we constructed $w$ by using a composite whose last part was the inclusion $i^X$, and as a result $\mathrm{im}(w) \subseteq \mathrm{po}(\mathrm{sk}(X)) \subseteq \mathrm{po}(X)$. Obviously $\mathrm{im}(u) \subseteq \mathrm{sk}(X) = \mathrm{sk}(\mathrm{sk}(X))$ as well, and so if we think of $u, w$ as being maps $u: \mathrm{sk}(\mathbb{G}_n) \to \mathrm{sk}(\mathrm{sk}(X))$ and $w: \mathrm{po}(\mathbb{G}_n) \to \mathrm{po}(\mathrm{sk}(X))$ we can apply \cref{factor2} and conclude that $\mathrm{im}(v) \subseteq \mathrm{sk}(X)$. Thus $R^X v$ is actually just $v$, and so --- after making all of the same arguments again for $u', w', v'$ --- we get that
\begin{eq*} R^X v \eta \, = \, R^X v' \eta \implies v \eta \, = \, v' \eta \end{eq*}
But by initiality of $\eta$ there should be only one morphism $f$ with the property that $v \circ \eta = f \circ \eta$, and so we can conclude that $v = v'$ and $u = u'$. 

Thus there is also at most one morphism $u$ in $C_{\mathrm{sk}}(\mathbb{G}_n)$ such that $\chi = u \circ \eta_{\mathrm{sk}}$. Therefore $u$ must be unique, and hence $\eta_{\mathrm{sk}}$ is indeed an initial object of $C_{\mathrm{sk}}(\mathbb{G}_n)$.
\end{proof}

\subsection{Initial algebras as a colimit}

Finally, we have reached the point where in order to find the initial object of $(\mathbb{G}_n \downarrow \mathrm{inv})$ it will suffice to find an initial object in the much more restrictive category $C_{\mathrm{sk}}(\mathbb{G}_n)$. With so much of the unnecessary structure of $(\mathbb{G}_n \downarrow \mathrm{inv})$ now removed, this task is at last appproachable.

\begin{defn} \label{Ddef} Let $D_n$ be a diagram in category of groups whose vertices are the endomorphism groups $\mathbb{G}_n(s, s)$ of the representing objects $s$ in $\mathrm{sk}(\mathbb{G}_n)$, and which has edges
\begin{eq*} \begin{array}{rrrll}
		\_ \boxtimes \mathrm{id}_{s'} & : & \mathbb{G}_n(s, s) & \to & \mathbb{G}_n(s \boxtimes s', s \boxtimes s') \\
		\mathrm{id}_{s'} \boxtimes \_ & : & \mathbb{G}_n(s, s) & \to & \mathbb{G}_n(s' \boxtimes s, s' \boxtimes s) 
		\end{array}
\end{eq*}
for all $s, s' \in \mathbb{N}^{\ast n} = \mathrm{Ob}\big(\mathrm{sk}(\mathbb{G}_n)\big)$.
\end{defn}

\begin{prop}\label{colimD} Let $\eta_{\mathrm{sk}} : \mathrm{sk}(\mathbb{G}_n) \to \mathrm{sk}(Z)$ be the initial object of $C_{\mathrm{sk}}(\mathbb{G}_n)$. Then $\mathrm{sk}(Z)$ is the weak monoidal category with underlying category $\mathbb{Z}^n \times \mathrm{colim}(D_n)$ and coherence data
\begin{eq*} \begin{array}{rll}
		a^{\mathrm{sk}(Z)}_{s, s', s''} & = & j_{s \boxtimes s' \boxtimes s''}(a^{\mathrm{sk}(\mathbb{G}_n)}_{s, s', s''}) \\
		l^{\mathrm{sk}(Z)}_s & = & j_s(l^{\mathrm{sk}(\mathbb{G}_n)}_s) \\
		r^{\mathrm{sk}(Z)}_s & = & j_s(r^{\mathrm{sk}(\mathbb{G}_n)}_s) \\
		\end{array}
\end{eq*} 
and $\eta_{\mathrm{sk}}$ is the weak monoidal functor
\begin{eq*}\begin{array}{rrrll}
		\eta_{\mathrm{sk}} & : & \mathrm{sk}(\mathbb{G}_n) & \to & \mathbb{Z}^n \times \mathrm{B} \, \mathrm{colim}(D_n) \\
		& : & s & \mapsto & s \\
		& : & f: s \to s & \mapsto & \big( \, \mathrm{id}_s, j_s(f) \, \big) \\
		& & \mu^\chi_{s, s'} & = & \mathrm{id}_{s \boxtimes s'} \\
		& & \eta^\chi & = & \mathrm{id}_{R(I)}
		\end{array}
\end{eq*}
where the $j_s$ are the canonical maps $\mathbb{G}_n(s, s) \to \mathrm{colim}(D_n)$.
\end{prop}
\begin{proof}
By unraveling its definition, recall an object $\chi$ of $C_{\mathrm{sk}}(\mathbb{G}_n)$ is a monoidal functor whose source is a chosen skeleton $\mathrm{sk}(\mathbb{G}_n)$ of $\mathbb{G}_n$, whose target is a skeleton $\mathrm{sk}(X)$ of some $\mathrm{E}$G-algebra $X$ with objects $\mathbb{Z}^{\ast n}$ assigned to connected components though abelianisation, and whose restriction $\chi_{\pi}$ is the inclusion $\mathbb{N}^n \to \mathbb{Z}^n$. In other words, $\eta_{\mathrm{sk}}$ is part of the initial diagram amongst those of the form
\begin{eq*} \begin{tikzcd}
\mathrm{sk}(\mathbb{G}_n) \ar[r, "\chi"] \ar[d, "\lbrack \, \_ \, \rbrack"'] & \mathrm{sk}(X) \ar[d, "\lbrack \, \_ \, \rbrack"] \\
\mathbb{N}^n \ar[r, hookrightarrow] & \mathbb{Z}^n
\end{tikzcd} \end{eq*}
where all the maps $[ \, \_ \, ]$ do is 
\begin{eq*} [s] \, = \, s \quad \quad [f: s \to s] \, = \, \mathrm{id}_s \end{eq*}
However, from now on it will prove more useful to adopt a different perspective, where we will stop thinking of objects like $\chi$ as weak monoidal functors, but instead as structure-preserving maps between collections of morphisms. Since the categories in question are skeletons, their sets of morphisms are just
\begin{eq*} \mathrm{Mor}\big( \, \mathrm{sk}(\mathbb{G}_n) \, ) \, = \, \bigsqcup_s \mathbb{G}_n(s, s), \quad \quad \mathrm{Mor}\big( \, \mathrm{sk}(X) \, ) \, = \, \bigsqcup_s X(s, s) \end{eq*}
where the disjoint unions are indexed over the representative objects $s$. These sets then possess extra structure which $\chi$ must respect --- the binary operations $\boxtimes$ and $\circ$, and the coherence data, $a_{s, s', s''}$, $l_s$, $r_s$.

Now, since the objects of $\mathrm{sk}(X)$ form the monoid $\mathbb{Z}^n$ under $\boxtimes$, each object $s$ has an inverse $s^*$, and so we have some isomorphisms of sets
\begin{eq*}\begin{array}{rll}
		X(s, s) & \to & X(R(I), R(I)) \\
		f & \mapsto & f \boxtimes \mathrm{id}_{s^*} 
		\end{array}
\end{eq*}
We can combine all of these into a larger isomorphism
\begin{eq*}\begin{array}{rrrll}
		\theta & : & \bigsqcup X(s, s) & \to & \mathbb{Z}^n \times X(R(I), R(I)) \\
		& & f: s \to s & \mapsto & (\mathrm{id}_s, f \boxtimes \mathrm{id}_{s^*})
		\end{array}
\end{eq*}
which then allows us to split the function $\chi: \bigsqcup \mathbb{G}_n(s, s) \to \bigsqcup X(s, s)$ into two independent pieces, namely
\begin{eq*}\begin{array}{rrrrrll}
		p_1 \circ \theta \circ \chi & =: & \chi_{\mathbb{Z}} & : & \bigsqcup \mathbb{G}_n(s, s) & \to & \mathbb{Z}^n \\
		& & & : & f: s \to s & \mapsto & \mathrm{id}_{s}
		\end{array}
\end{eq*}
and
\begin{eq*}\begin{array}{rrrrrll}
		p_2 \circ \theta \circ \chi & =: & \chi_I & : & \bigsqcup \mathbb{G}_n(s, s) & \to & X(R(I), R(I)) \\
		& & & : & f: s \to s & \mapsto & \chi(f) \boxtimes \mathrm{id}_{s^*}
		\end{array}
\end{eq*}
This will in turn separate the diagram for $\chi$ into two independent diagrams
\begin{eq*} \begin{tikzcd}
\bigsqcup_s \mathbb{G}_n(s, s) \ar[r, "\chi_{\mathbb{Z}}"] \ar[d, "\lbrack \, \_ \, \rbrack"'] & \mathbb{Z}^n \ar[d, "\lbrack \, \_ \, \rbrack", equal] & & \bigsqcup_s \mathbb{G}_n(s, s) \ar[r, "\chi_I"] \ar[d, "\lbrack \, \_ \, \rbrack"'] &  X(R(I), R(I)) \ar[d, "\lbrack \, \_ \, \rbrack"]\\
\mathbb{N}^n \ar[r, hookrightarrow] & \mathbb{Z}^n & & \mathbb{N}^n \ar[r, hookrightarrow] & \mathbb{Z}^n
\end{tikzcd} \end{eq*}
with the corresponding diagrams for $\eta_{\mathrm{sk}}$ being the initial ones of each kind. But diagram on the left only commutes for one value of $\chi_{\mathbb{Z}}$, so all such maps are equal and trivially. Also, since $R(I)$ is the unit of $\mathrm{sk}(\mathbb{G}_n)$ the map $[ \, \_ \, ]: X(R(I), R(I)) \to \mathbb{Z}^n$ will always send everything to 0, rendering the bottom part of the righthand diagram uninteresting. Therefore, to find $\eta_{\mathrm{sk}}$ all that remains is to construct the initial map among all of the $\chi_I$.

So, what effect do the properties of $\chi$ have on $\chi_I$? Firstly, preservation of composition is inherited directly:
\begin{eq*}\begin{array}{rll}
		\chi_I( \, f \circ f' : s \to s \, ) & = & \chi(f \circ f') \boxtimes \mathrm{id}_{s^*} \\
		& = & \big( \, \chi(f) \circ \chi(f') \, \big) \boxtimes (\mathrm{id}_{s^*} \circ \mathrm{id}_{s^*}) \\
		& = & \big( \, \chi(f) \boxtimes \mathrm{id}_{s^*} \, \big) \circ \big( \, \chi(f') \boxtimes \mathrm{id}_{s^*} \, \big)  \\
		& = & \psi_I(f) \circ \psi_I(f')
		\end{array}
\end{eq*}
In other words, $\chi_I$ can be seen as the disjoint union of family of group homomorphisms $\chi_s: \mathbb{G}_n(s, s) \to X(R(I), R(I))$, indexed by the objects of $\mathrm{sk}(X)$. 

Secondly, part of the coherence data for the weak monoidal functor $\chi$ is the natural isomorphism $\mu^{\chi}$, and so by naturality we have
\begin{eq*}\begin{array}{rll}
		\chi_I(f  \boxtimes f') & = & \chi(f \boxtimes f') \boxtimes \mathrm{id}_{(s \boxtimes s)^*} \\
		& = & \big( \, \mu^{\chi}_{s, s'} \circ \big( \, \chi(f) \boxtimes \chi(f') \, \big) \circ (\mu^{\chi}_{s, s'})^{-1} \, \big) \boxtimes \mathrm{id}_{(s \boxtimes s)^*} \\
		& = & \big( \, \mu^{\chi}_{s, s'} \boxtimes \mathrm{id}_{(s \boxtimes s)^*} \, \big) \circ \big( \, \chi(f) \boxtimes \chi(f') \boxtimes \mathrm{id}_{(s \boxtimes s)^*} \, \big) \circ \big( \, \mu^{\chi}_{s, s'} \boxtimes \mathrm{id}_{(s \boxtimes s)^*} \, \big)^{-1} 
		\end{array}
\end{eq*}
However, we can simplify this condition greatly, due to an important fact about $X(R(I), R(I))$. Recall that even though $\mathrm{sk}(X)$ is a weak monoidal category, its product $\boxtimes$ is strictly associative and unital, like $\circ$. These operations also have the same unit, $\mathrm{id}_{R(I)}$, and obey an interchange law,
\begin{eq*} (f \circ f') \boxtimes (g \circ g') \, = \, (f \boxtimes f') \circ (g \boxtimes g') \end{eq*}
and thus by applying the classic Eckmann-Hilton argument (see \cite{eckhil} for their original paper) we can conclude that $X(R(I), R(I))$ is a commutative monoid with $\boxtimes = \circ$. Indeed, this means that it is even an abelian group, since all morphisms in $X$ have compositional inverses. Using this, and the fact that $X$ is spacial (\cref{spacial}), we get
\begin{eq*}\begin{array}{rll}
		\chi_I(f  \boxtimes f') & = & \big( \, \mu^{\chi}_{s, s'} \boxtimes \mathrm{id}_{(s \boxtimes s)^*} \, \big) \circ \big( \, \chi(f) \boxtimes \chi(f') \boxtimes \mathrm{id}_{(s \boxtimes s)^*} \, \big) \circ \big( \, \mu^{\chi}_{s, s'} \boxtimes \mathrm{id}_{(s \boxtimes s')^*} \, \big)^{-1} \\
		& = & \big( \, \mu^{\chi}_{s, s'} \boxtimes \mathrm{id}_{(s \boxtimes s')^*} \, \big) \circ \big( \, \mu^{\chi}_{s, s'} \boxtimes \mathrm{id}_{(s \boxtimes s')^*} \, \big)^{-1} \circ \big( \, \chi(f) \boxtimes \chi(f') \boxtimes \mathrm{id}_{(s \boxtimes s')^*} \, \big) \\
		& = & \chi(f) \boxtimes \chi(f') \boxtimes \mathrm{id}_{(s \boxtimes s')^*} \\
		& = & \chi(f) \boxtimes \chi(f') \boxtimes \mathrm{id}_{(s')^*} \boxtimes \mathrm{id}_{s^*} \\
		& = & \chi(f) \boxtimes \mathrm{id}_{s^*} \boxtimes \chi(f') \boxtimes \mathrm{id}_{(s')^*} \\
		& = & \chi_I(f) \boxtimes \chi_I(f')
		\end{array}
\end{eq*}
So the effect that $\mu^{\chi}$ has on $\chi_I$ is to make it preserve $\boxtimes$. Moreover, since
\begin{eq*}\begin{array}{rll}
		\chi_I(f \boxtimes f') & = & \chi_I\big( \, (f \boxtimes \mathrm{id}_{s'}) \circ (\mathrm{id}_s \boxtimes f') \, \big) \\
		& = & \chi_I(f \boxtimes \mathrm{id}_{s'}) \circ \chi_I(\mathrm{id}_s \boxtimes f') \\
		& = & \chi_I(f \boxtimes \mathrm{id}_{s'}) \boxtimes \chi_I(\mathrm{id}_s \boxtimes f') \\
		\end{array}
\end{eq*}
we can recover the full condition $\chi_I(f \boxtimes f') = \chi_I(f) \boxtimes \chi_I(f')$ from two of its subcases,
\begin{eq*}\begin{array}{rllrrll}
		\chi_I(f \boxtimes \mathrm{id}_{s'}) & = & \chi_I(f) \boxtimes \chi_I(\mathrm{id}_{s'}) & \quad & \chi_I(\mathrm{id}_{s'} \boxtimes f) & = & \chi_I(\mathrm{id}_{s'}) \boxtimes \chi_I(f) \\
		& = & \chi_I(f) \boxtimes \mathrm{id}_{R(I)} & & & = & \mathrm{id}_{R(I)} \boxtimes \chi_I(f) \\
		& = & \chi_I(f) & & & = & \chi_I(f)
		\end{array} \end{eq*}
and so these are really the more fundamental property of $\chi_I$. Returning to the view of $\chi_I$ as a family of group homomorphisms again, what we've proven is that both $\chi_{s \boxtimes s'}(\_ \boxtimes \mathrm{id}_{s'})$ and $\chi_{s' \boxtimes s}(\mathrm{id}_{s'} \boxtimes \_)$ are equal to $\chi_s$. In the language of \cref{Ddef}, the pair $(X(R(I), R(I)), \chi_I)$ is a cocone of the diagram $D_n$.

The last few properties that $\chi_I$ will inherit come from the coherence conditions governing how $\chi$ and its natural isomorphisms $\mu^\chi, \eta^\chi$ interact with the data $a, l, r$ of $\mathrm{sk}(\mathbb{G}_n)$ and $\mathrm{sk}(X)$. However, it turns out that that these can be safely ignored when determining the initial object $\eta_{\mathrm{sk}}$. To see this, let $\chi': \mathrm{sk}(\mathbb{G}_n) \to \mathrm{sk}(X)'$ be the weak monoidal functor with the same underlying functor as $\chi$, but coherence data 
\begin{eq*} \begin{array}{rll}
		a^{\mathrm{sk}(X)'}_{s, s', s''} & = & \chi'(a^{\mathrm{sk}(\mathbb{G}_n)}_{s, s', s''}) \\
		l^{\mathrm{sk}(X)'}_s & = & \chi'(l^{\mathrm{sk}(\mathbb{G}_n)}_s) \\
		r^{\mathrm{sk}(X)'}_s & = & \chi'(r^{\mathrm{sk}(\mathbb{G}_n)}_s) \\
		\mu^\chi_{s, s'} & = & \mathrm{id}_{s \boxtimes s'} \\
		\eta^\chi & = & \mathrm{id}_{R(I)}
		\end{array}
\end{eq*} 
It is easy to verify that this is a well-defined object of $C_{\mathrm{sk}}(\mathbb{G}_n)$ which produces the same cocone $\chi_I$ of $D_n$ that $\chi$ does. Thus any weak monoidal functor $f: \mathrm{sk}(X)' \to \mathrm{sk}(X)$ such that $\chi = f \circ \chi'$ must have the identity as its underlying functor, and coherence data satisfying
\begin{eq*} \begin{array}{rclrrcl}
		\mu^f_{s, s'} & = & f(\mu^{\chi'}_{s, s'})^{-1} \circ \mu^{\chi}_{s, s'} & \quad & \eta^f_{s, s'} & = & f(\eta^{\chi'}_{s, s'})^{-1} \circ \eta^{\chi}_{s, s'} \\
		& = & \mu^\chi_{s, s'} & \quad & & = & \eta^\chi_{s, s'}
		\end{array}
\end{eq*} 
This weak monoidal functor is well-defined --- the proof involves similar steps to those taken during the proof of \cref{pullback} --- and so there exists a unique morphism from $\chi'$ to $\chi$ in $C_{\mathrm{sk}}(\mathbb{G}_n)$. Therefore, the unique morphism from the initial object $\eta_{\mathrm{sk}}$ onto $\chi$ can always be recovered from another map with the same cocone, and hence the cocones produced by the $\chi_I$ alone will provide sufficient information to determine $\eta_{\mathrm{sk}, I}$.

Furthermore, this argument also showns that for any cocone $\xi_s: \mathbb{G}_n(s, s) \to A$ of $D_n$, there exist some object of $C_{\mathrm{sk}}(\mathbb{G}_n)$ with $\xi_s$ as its associated cocone. Specifically, if we combine the $\xi_s$ into a functor 
\begin{eq*}\begin{array}{rrrll}
		\chi & : & \mathrm{sk}(\mathbb{G}_n) & \to & \mathbb{Z}^n \times \mathrm{B}A \\
		& : & s & \mapsto & s \\
		& : & f: s \to s & \mapsto & \big( \, \mathrm{id}_s, \xi_s(f) \, \big)
		\end{array}
\end{eq*}
then the weak monoidal functor $\chi'$ we get from it by the method given above will have $\chi'_s = \xi_s$. Therefore, $\eta_{\mathrm{sk}} : \mathrm{sk}(\mathbb{G}_n) \to \mathrm{sk}(Z)$ is the initial object of $C_{\mathrm{sk}}(\mathbb{G}_n)$, then the cocone obtained from $\psi_{\mathrm{sk}, I}$ is initial cocone of $D_n$ --- the colimit. In other words, the cocone $(Z(R(I), R(I)), \psi_{\mathrm{sk}, I})$ is just the pair $(\mathrm{colim}(D_n), j)$. Working backwards, from this cocone we can then reconstruct a $\eta_{\mathrm{sk}}: \mathrm{sk}(\mathbb{G}_n) \to \mathrm{sk}(Z)$ which produces it, again by setting
\begin{eq*} \begin{array}{rll}
		\mathrm{sk}(Z) & = & \mathbb{Z}^n \times \mathrm{B} \, \mathrm{colim}(D_n) \\
		a^{\mathrm{sk}(Z)}_{s, s', s''} & = & j_{s \boxtimes s' \boxtimes s''}(a^{\mathrm{sk}(\mathbb{G}_n)}_{s, s', s''}) \\
		l^{\mathrm{sk}(Z)}_s & = & j_s(l^{\mathrm{sk}(\mathbb{G}_n)}_s) \\
		r^{\mathrm{sk}(Z)}_s & = & j_s(r^{\mathrm{sk}(\mathbb{G}_n)}_s) \\
		\mu^\chi_{s, s'} & = & \mathrm{id}_{s \boxtimes s'} \\
		\eta^\chi & = & \mathrm{id}_{R(I)}
		\end{array}
\end{eq*} 
\end{proof}

With this proposition proven, the results in this chapter now collectively describe how to construct free $\mathrm{E}G$-algebras on $n$ invertible objects. Since the argument is obviously arranged in a rather piecemeal fashion, it would be best to restate the general conclusion all in one place.

\begin{thm}\label{freeinvalg} Let $\mathbb{G}_n$ be the free $\mathrm{E}G$-algebra on $n$ objects. Choose a skeleton of $\mathbb{G}_n$ and an equivalence $\rho$ between the two, and use them to define the diagram $D_n$ as in \cref{Ddef}, with colimit $(\mathrm{colim}(D_n), j)$. Also do the same with $\mathbb{G}_{2n}$, and let $u: \mathrm{colim}(D_{2n}) \to \mathrm{colim}(D_n)$ be the unique group homomorphism induced by the cocone of maps
\begin{eq*} \begin{tikzcd} \mathbb{G}_{2n}(w, w) \ar[r, "j + j^*"] & \mathrm{colim}(D_n) \end{tikzcd} \end{eq*}
Lastly, denote by $q_n$ the the quotient of abelianisation $\mathbb{Z}^{\ast n} \to \mathbb{Z}^n$, and define the function $h: \mathbb{Z}^{\ast n} \to \mathbb{N}^{\ast 2n}$ using minimal generator decompositions:
\begin{eq*} h(z_i) \, = \, z_i, \quad \quad h(z_i^*) \, = \, z_{i+n}, \quad \quad d\big( \, h(w) \, \big) \, = \, h\big( \, d(w) \, \big) \end{eq*}
Then the free $\mathrm{E}G$-algebra on $n$ invertible objects, $L\mathbb{G}_n$, is the algebra described by
\begin{eq*}\begin{array}{rll}
		& & \\
		\mathrm{Ob}(L\mathbb{G}_n) & = & \mathbb{Z}^{\ast n} \\
		& & \\
		 L\mathbb{G}_n(w, w') & = & \begin{cases}
     	  		\mathrm{colim}(D_n) & \quad \text{if} \quad q_n(w) \, = \, q_n(w') \\
      			\emptyset & \quad \text{otherwise}
			\end{cases} \\
		& & \\
		\alpha_{L\mathbb{G}_n}( \, e \, ; \, f_1, ..., f_m \, ) & = & f_1 \cdot ... \cdot f_m \\
		& & \\
		\alpha_{L\mathbb{G}_n}( \, g \, ; \, \mathrm{id}_{w_1}, ..., \mathrm{id}_{w_m} \, ) & = & u j_{q_{2n}(w)}\big( \, \rho_{w'}^{-1} \alpha_{\mathbb{G}_{2n}}( \, g \, ; \, \mathrm{id}_{h(w_1)}, ..., \mathrm{id}_{h(w_m)} \, ) \rho_{w} \, \big) \\
		& &
		\end{array}		
\end{eq*}
Note that this means that the underlying monoidal category of $L\mathbb{G}_n$ is
\begin{eq*}L\mathbb{G}_n \quad = \quad \mathrm{B} \, \mathrm{colim}(D_n) \times \bigsqcup_{w \in \mathbb{Z}^n} \mathrm{E}\big( \, q_{\mathbb{Z}^{\ast n}}^{-1}(w) \, \big) \end{eq*}
with the monoidal product inherited from $\mathbb{Z}^{\ast n}$.
\end{thm}
\begin{proof}
\cref{colimD} details the initial object of $C_{\mathrm{sk}}(\mathbb{G}_n)$. \cref{initialsk} tells us that this is this one of the functors $\eta_{\mathrm{sk}}$ obtained from the initial object $\eta_n$ of $C(\mathbb{G}_n)$ by applying \cref{factor1}. The other one, $\eta_{\mathrm{po}}$, is given by \cref{polem}. \cref{pullback} then lets us recover the target of $\eta_n$ as the pullback
\begin{eq*} \begin{array}{rll}
		 Z & \cong & \big( \, \mathbb{Z}^n \times \mathrm{B} \, \mathrm{colim}(D_n) \, \big) \times_{\mathbb{Z}^n} \Big( \, \bigsqcup_{\mathbb{Z}^n} \mathrm{E}\big( \, q_{\mathbb{Z}^{\ast n}}^{-1}(w) \, \big) \, \Big) \\
		& = & \mathrm{B} \, \mathrm{colim}(D_n) \times \bigsqcup_{\mathbb{Z}^n} \mathrm{E}\big( \, q_{\mathbb{Z}^{\ast n}}^{-1}(w) \, \big)
		\end{array}
\end{eq*}
and from Propositions \ref{initial}, \ref{initialab} and \ref{initialmon} we can conclude that this target is indeed the free algebra $L\mathbb{G}_n$. Similarly, after making any choice of skeleton for $\mathbb{G}_n$, \cref{factor2} can combine $\eta_{\mathrm{sk}}$ and $\eta_{\mathrm{po}}$ into $\eta$ by means of a pullback:
\begin{eq*}\begin{array}{rrrll}
		\eta_n & : & \mathbb{G}_n & \to & \mathrm{B} \, \mathrm{colim}(D_n) \times \bigsqcup_{\mathbb{Z}^n} \mathrm{E}\big( \, q_{\mathbb{Z}^{\ast n}}^{-1}(w) \, \big) \\
		& : & w & \mapsto & w \\
		& : & f: w \to w' & \mapsto & \Big( \, j_{q_{\mathbb{Z}^{\ast n}}(w)}\big( \, (\rho^{\mathbb{G}_n}_{w'})^{-1} f \rho^{\mathbb{G}_n}_{w} \, \big), \, w \to w' \, \Big)
		\end{array}
\end{eq*}
Finally, we can recontruct the action of $L\mathbb{G}_n$ using \cref{initialact}. Specifically, let $\eta'_n: \mathbb{G}_{2n} \to L\mathbb{G}_n$ be the composite
\begin{eq*} \begin{tikzcd}
\mathbb{G}_{2n} \ar[r, "\sim"] & \mathbb{G}_n + \mathbb{G}_n \ar[r, "\eta_n + \eta_n"] & L\mathbb{G}_n + L\mathbb{G}_n \ar[r, "\mathrm{id} + \delta"] & L\mathbb{G}_n + L\mathbb{G}_n \ar[r, "i + i"] & L\mathbb{G}_n
\end{tikzcd} \end{eq*}
On generating objects, this functor acts by
\begin{eq*}\begin{array}{rllll}
		\eta'_n(z_i) & = &
			\begin{cases}
       				z_i & \quad \text{if} \quad 1 \leq i \leq n \\
      				z_{i-n}^* & \quad \text{if} \quad n+1 \leq i \leq 2n \\
			\end{cases}
		\end{array}
\end{eq*}
and so we have $\eta'_n h(w) = w$ for any element $w$ of $\mathbb{Z}^{\ast n}$. Moreover, $\eta'_n$ is an object of $(\mathbb{G}_{2n} \downarrow (\_)_{\mathrm{inv}})$, and so if $\eta_{2n}$ is the initial object of that category then there must exist some $u: L\mathbb{G}_{2n} \to L\mathbb{G}_n$ such that $\eta'_n = u \circ \eta_{2n}$. The values that this $u$ takes on objects are given by
\begin{eq*} u(w) \, = \, u \eta_{2n}(w) \, = \, \eta'_n(w) \end{eq*}
and its values on morphisms are determined by the fact that the maps
\begin{eq*} \begin{tikzcd} \mathbb{G}_{2n}(w, w) \ar[r, "\sim"] & (\mathbb{G}_n + \mathbb{G}_n)(w, w) \ar[r, "j + j"] & \mathrm{colim}(D_n) + \mathrm{colim}(D_n) \ar[r, "\mathrm{id} + \_^*"] & \mathrm{colim}(D_n) \end{tikzcd} \end{eq*}
form a cocone over $D_{2n}$. Therefore,
\begin{eq*} \begin{array}{rll}
		\alpha_{L\mathbb{G}_n}( \, g \, ; \, \mathrm{id}_{w_1}, ..., \mathrm{id}_{w_m} \, ) & = & \alpha_{L\mathbb{G}_n}( \, g \, ; \, \mathrm{id}_{\eta'_n h(w_1)}, ..., \mathrm{id}_{\eta'_n h(w_m)} \, ) \\
		& = & \eta'_n \big( \, \alpha_{\mathbb{G}_{2n}}( \, g \, ; \, \mathrm{id}_{h(w_1)}, ..., \mathrm{id}_{h(w_m)} \, ) \, \big) \\
		& = & u \eta_{2n} \big( \, \alpha_{\mathbb{G}_{2n}}( \, g \, ; \, \mathrm{id}_{h(w_1)}, ..., \mathrm{id}_{h(w_m)} \, ) \, \big) \\
		& = & u j_{q_{2n}(w)}\big( \, \rho^{-1} \alpha_{\mathbb{G}_{2n}}( \, g \, ; \, \mathrm{id}_{h(w_1)}, ..., \mathrm{id}_{h(w_m)} \, ) \rho_{w} \, \big)
		\end{array}
\end{eq*}
\end{proof}

\subsection{Examples}

With \cref{freeinvalg} proven we can now finally achieve the primary goal of this chapter --- to describe the free braided monoidal category on $n$ invertible objects. In addition, this section will provide a few other simple applications of the theorem, in an effort to build up to the main result more gently. The definition of $L\mathbb{G}_n$ given in \ref{freeinvalg} is after all a little difficult to parse on first reading, because of the fairly abstract way it is presented, and hopefully the following concrete examples should allow the braided case to be properly understood.

\begin{prop} Let $G$ be any action operad, and denote by $G(\infty)$ the colimit of the sequence 
\begin{eq*} G(0) \, \hookrightarrow \, G(1) \, \hookrightarrow G(2) \, \hookrightarrow \, ... \end{eq*}
 Then the free $\mathrm{E}G$-algebra on a single invertible object is the category
\begin{eq*} L\mathbb{G}_1 \quad = \quad \mathrm{B}G(\infty)^{\mathrm{ab}} \times \mathbb{Z} \end{eq*}
equipped with the action
\begin{eq*}\alpha_{L\mathbb{G}_1}( \, g \, ; \, f_1, ..., f_m \, ) \, = \, g \cdot f_1 \cdot ... \cdot f_m \end{eq*}
\end{prop}
\begin{proof}
Each object of $\mathbb{G}_1$ is isomorphic only to itself, and so it is its own skeleton. We are therefore free to choose $\rho$ to be the identity transformation, in which case the product $\boxtimes$ in our `skeleton' will just be the normal $\otimes$. The diagram $D_1$ from \cref{Ddef} then has edges
\begin{eq*}\begin{tikzcd} \mathbb{G}_1(n) \ar[r, "\, \_ \, \otimes \mathrm{id}_m"] & \mathbb{G}_1(n + m), & \mathbb{G}_1(n) \ar[r, "\mathrm{id}_m \otimes \, \_ \,"] & \mathbb{G}_1(m+n)  \end{tikzcd}\end{eq*}
for each $m, n \in \mathbb{N}$. Given the structure of $\mathbb{G}_1$, these are really
\begin{eq*}\begin{tikzcd} G(n) \ar[r, "\, \_ \, \otimes \mathrm{id}_m"] & G(n + m), & \mathbb{G}_1(n) \ar[r, "\mathrm{id}_m \otimes \, \_ \,"] & G(m+n)  \end{tikzcd}\end{eq*}
which are all composites of the specific cases
\begin{eq*}\begin{tikzcd} G(n) \ar[r, "\, \_ \, \otimes \mathrm{id}_1"] & G(n + 1), & \mathbb{G}_1(n) \ar[r, "\mathrm{id}_1 \otimes \, \_ \,"] & G(n+1)  \end{tikzcd}\end{eq*}
Moreover, since we know that $\mathrm{colim}(D_1)$ is always an abelian group, and each of the above pairs has the same image under abelianisation, we can simply choose one of them to be the canonical inclusion $G(n) \hookrightarrow G(n+1)$, and then only abelianise after we've taken the colimit. Hence $L\mathbb{G}_1$ has the underlying category
\begin{eq*}\mathrm{Ob}(L\mathbb{G}_1) \, = \, \mathbb{Z}, \quad \quad L\mathbb{G}_1(m, n) \, = \, \begin{cases}
     	  		G(\infty)^{\mathrm{ab}} & \quad \text{if} \quad m \, = \, n \\
      			\emptyset & \quad \text{otherwise}
			\end{cases}
\end{eq*}
as required.

In order to find the action $\alpha_{L\mathbb{G}_1}$ as well, we now also need to consider some of the structure of $L\mathbb{G}_2$. In particular, 
\end{proof}

\subsection{The free algebra on $n$ weakly invertible objects}

Up until now, we've been working under the convention that by `invertible' objects we mean stictly invertible --- $x \otimes x^* = I$. As an additional exercise, we can ask ourselves how all of this would change if we permitted our objects to be only weakly invertible, that is $x \otimes x^* \cong I$. The situation is actually quite elegant, in that the effect of weakening in our objects can be offset completely by the effect of also weakening our algebra homomorphisms, such that we won't need to calculate any new free algebras other than those given by \cref{freeinvalg}. Before proving this though, we first to need to set out some definitions.

\begin{defn} Given an $\mathrm{E}G$-algebra $X$, we denote by $X_{\mathrm{wkinv}}$ the category whose
\begin{itemize}
\item objects are tuples $(x, x^*, \eta, \epsilon)$, where $x$ and $x^*$ are objects of $X$ and $\eta: I \to x^* \otimes x$ and $\epsilon : x \otimes x^* \to I$ are morphisms such that the composites
\begin{eq*} \begin{tikzcd}
x \ar[r, "\mathrm{id} \otimes \eta"] & x \otimes x^* \otimes x \ar[r, "\epsilon \otimes \mathrm{id}"] & x &
x^* \ar[r, "\eta \otimes \mathrm{id}"] & x^* \otimes x \otimes x^* \ar[r, "\mathrm{id} \otimes \epsilon"] & x^* 
\end{tikzcd} \end{eq*}
are identity morphisms.
\item maps $(f, f^*): (x, x^*, \eta_x, \epsilon_x) \to (y, y^*, \eta_y, \epsilon_y)$ are pairs $f: x \to y$, $f^* : x^* \to y^*$ of morphisms such that the diagrams
\begin{eq*} \begin{tikzcd}
& I \ar[dl, "\eta_x"'] \ar[dr, "\eta_y"] & & x \otimes x^* \ar[rr, "f \otimes f^*"] \ar[dr, "\epsilon_x"'] & & y \otimes y^* \ar[dl, "\epsilon_y"] \\
x^* \otimes x \ar[rr, "f^* \otimes f"] & & y \otimes y^* & & I &
\end{tikzcd} \end{eq*}
commute.
\end{itemize}
\end{defn}

\begin{defn}\label{weakmonfunc} Let $(X, \alpha)$ and $(Y, \beta)$ be $\mathrm{E}G$-algebras. A \emph{weak $\mathrm{E}G$-algebra homorphism} between them is a weak monoidal functor $\psi: X \to Y$ such that all diagrams of the form
\begin{eq*} \begin{tikzcd}
\psi( x_1 \otimes ... \otimes x_m) \ar[r, "\sim"] \arrow{d}[']{\psi(\alpha(g; h_1, ... , h_m))} & \psi(x_1) \otimes ... \otimes \psi(x_m) \arrow{d}{\beta(g; \psi(h_1), ..., \psi(h_m))} \\
\psi( y_{\pi(g)^{-1}(1)} \otimes ... \otimes y_{\pi(g)^{-1}(m)}) \ar[r, "\sim"] & \psi(y_{\pi(g)^{-1}(1)}) \otimes ... \otimes \psi(y_{\pi(g)^{-1}(m)})
\end{tikzcd} \end{eq*}
commute.
\end{defn} 

\begin{defn} We denote by $\mathrm{E}G\mathrm{Alg}_W$ the 2-category of $\mathrm{E}G$-algebras, weak $\mathrm{E}G$-algebra homomorphisms, and weak monoidal transformations.\end{defn}

Now we can properly express what we mean by the free algebras on weakly invertible objects being the same as those in the strict case.

\begin{thm} The algebra $L\mathbb{G}_n$ is also the free $\mathrm{E}G$-algebra on $n$ weakly invertible objects. Specifically, for any other $\mathrm{E}G$-algebra $X$ there is an equivalence of categories
\begin{eq*} \mathrm{E}G\mathrm{Alg}_W(L\mathbb{G}_n, X) \simeq (X_{\mathrm{wkinv}})^n \end{eq*}
\end{thm}
\begin{proof}
We begin by defining a functor $F : \mathrm{E}G\mathrm{Alg}_W(L\mathbb{G}_n, X) \to (X_{\mathrm{wkinv}})^n$. On weak maps, $F$ acts as 
\begin{eq*} F( \, \psi: L\mathbb{G}_n \to X \, ) = \big\{ \, ( \, \psi(z_i), \, \psi(z_i^*), \, I \xrightarrow{\sim} \psi(I) \xrightarrow{\sim} \psi(z_i^*)\psi(z_i), \, \psi(z_i)\psi(z_i^*) \xrightarrow{\sim} \psi(I) \xrightarrow{\sim} I \, ) \, \big\}_{i \in \{z_1, ..., z_n\} } \end{eq*}
where the $z_i$ are the generators of $\mathbb{Z}^{*n}$ and the isomorphisms are those given by $\psi$ being a weak moniodal functor. On weak monoidal transformations, $F$ acts as
\begin{eq*} F( \, \theta : \psi \to \chi \, ) = \big\{ \, ( \, \theta_{z_i}, \, \theta_{z_i^*} \, ) \, \big\}_{i \in \{z_1, ..., z_n\} }\end{eq*}
This choice does satisfy the condition on morphisms of $(X_{\mathrm{wkinv}})^n$, since we can build the required commuting diagrams out of smaller ones given by $\theta$ being a weak monoidal transfomation:
\begin{eq*} \begin{tikzcd}
& I \ar[dl, "\sim"'] \ar[dr, "\sim"] & & \psi(z_i) \otimes \psi(z_i^*) \ar[rr, "\theta_{z_i} \otimes \theta_{z_i^*}"] \ar[d, "\sim"'] & & \chi(z_i) \otimes \chi(z_i^*) \ar[d, "\sim"] \\
\psi(I) \ar[d, "\sim"'] \ar[rr, "\theta_I"] & & \chi(I) \ar[d, "\sim"] & \psi(I) \ar[dr, "\sim"'] \ar[rr, "\theta_I"] & & \chi(I) \ar[dl, "\sim"] \\
\psi(z_i^*) \otimes \psi(z_i) \ar[rr, "\theta_{z_i^*} \otimes \theta_{z_i}"] & & \chi(z_i^*) \otimes \chi(z_i) & & I & 
\end{tikzcd} \end{eq*}

Now we need to check if $F$ is an equivalence of categories. First, let $\big\{ ( x_i, x_i^*, \eta_i, \epsilon_i ) \big\}_{i \in \{z_1, ..., z_n\} }$ be an arbitrary object of $(X_{\mathrm{wkinv}})^n$. We can construct a weak algebra map $\psi: L\mathbb{G}_n \to X$ from it as follows. Define
\begin{eq*} \psi(I) = I, \quad \psi(z_i) = x_i, \quad \psi(z_i^*) = x_i^* \end{eq*}
and choose the isomorphisms
\begin{eq*} \begin{array}{rllllll}
		\psi_I & : & I \to \psi(I) & = & \mathrm{id}_I & : & I \to I \\
		\psi_{z_i, z_i^*} & : & \psi(z_i) \otimes \psi(z_i^*) \to \psi(I) & = & \epsilon_i & : & x_i \otimes x_i^* \to I \\
		\psi_{z_i^*, z_i} & : & \psi(z_i^*) \otimes \psi(z_i) \to \psi(I) & = & \eta_i^{-1} & : & x_i^* \otimes x_i \to I
		\end{array} .
\end{eq*}
Then for any $w, w' \in \mathrm{Ob}(L\mathbb{G}_n)$ such that $d(w \otimes w') = d(w) \otimes d(w')$, where $d(-)$ is the minimal generator decomposition from \cref{mgd}, set 
\begin{eq*} \psi(w \otimes w') = \psi(w) \otimes \psi(w'), \quad \quad \psi_{w, w'} = \mathrm{id}_{\psi(w) \otimes \psi(w')} \end{eq*}
This is enough to determine the value of $\psi$ on all of the remaining objects, via successive decompositions. For the isomorphisms, first note that the ones we have already defined satisfy the associativity and unitality required of weak monoidal functors. Now consider some $w, w'$ with $d(w \otimes w') \neq d(w) \otimes d(w')$. The fact that they differ implies that tensoring $w$ with $w'$ causes some cancellation of inverses to occur where the end of one sequence meets the beginning of another. In particular, if we let $b$ be the last term in the minimal generator decomposition of $w$, and let $c = w'$, then we conclude that the length $d(b \otimes c)$ is smaller than the length of $d(c)$. Let $a$ be the product of the rest of $d(w)$, so that $a \otimes b = w$. Then we can use requirement for associativity,
\begin{eq*} \begin{tikzcd}
\psi(a) \otimes \psi(b) \otimes \psi(c) \ar[rr, "\mathrm{id} \otimes \psi_{b, c}"] \ar[d, "\psi_{a, b} \otimes \mathrm{id}"'] & & \psi(a) \otimes \psi(b \otimes c) \ar[d, "\psi_{a, b \otimes c}"] \\
\psi(a \otimes b) \otimes \psi(c) \ar[rr, "\psi_{a \otimes b, c}"] && \psi(a \otimes b \otimes c)
\end{tikzcd} \end{eq*}
to define $\psi_{w, w'} = \psi{a\otimes b, c}$ in terms of three other isomorphisms that each have strictly smaller decompositions. Repeating this process will therefore eventually yield a definition in terms of our previous isomorphisms.

By \cref{allmapsaction}, every morphism in $L\mathbb{G}_n$ can be written as $\alpha(g; \mathrm{id}_{w_1}, ..., \mathrm{id}_{w_m})$ for some $g \in G(m)$, $w_i \in \mathbb{Z}^{*n}$. The action of $\psi$ on morphisms is thus determined by the diagram in \cref{weakmonfunc}, that is
\begin{eq*} \psi(\alpha(g; w_1, ... w_m)) \, = \, \psi_{\mathbf{w}_{\pi(g)^{-1}}} \circ \beta(\, g \, ; \, \mathrm{id}_{\psi(w_1)}, \, ..., \, \mathrm{id}_{\psi(w_m)}\, ) \circ \psi_{\mathbf{w}}^{-1}\end{eq*} 
However, morphisms do not have a unique representation of this form, so we must check that whenever we have different representations of the same morphism
\begin{eq*} \alpha(g; \mathrm{id}_{w_1}, ..., \mathrm{id}_{w_m}) = \alpha(g'; \mathrm{id}_{w_1'}, ..., \mathrm{id}_{w_{m'}'}) \end{eq*}
their diagrams give the same image under $\psi$. There are two cases to consider here;
\begin{eq*} \alpha(g; \mathrm{id}_{w_1}, ..., \mathrm{id}_{w_m}) = \alpha( \, g \otimes e_k \, ; \, \mathrm{id}_{w_1}, \, ..., \, \mathrm{id}_{w_m}, \, \mathrm{id}_{v_1}, \, ..., \, \mathrm{id}_{v_k} \, ) \end{eq*}
when $v_1 \otimes ... \otimes v_k = 0$, which comes from the edges of the colimit diagram $D_n$ in \cref{colimthm}; and
\begin{eq*} \begin{array}{rll}
		\alpha(g; \mathrm{id}_{w_1}, ..., \mathrm{id}_{w_m}) & = & \alpha(\, h \, ; \, \mathrm{id}_{w_1'}, \, ..., \, \mathrm{id}_{w_{m'}} \, ) \\
		&& \circ \, \, \alpha(\, j \, ; \, \mathrm{id}_{w_1''}, \, ..., \, \mathrm{id}_{w_{m''}''} \, ) \\
		&& \circ \, \, \alpha(\, h^{-1} \, ; \, \mathrm{id}_{w_1'}, \, ..., \, \mathrm{id}_{w_{m'}'} \, ) \\
		&& \circ \, \, \alpha(\, j^{-1} \, ; \, \mathrm{id}_{w_1''}, \, ..., \, \mathrm{id}_{w_{m''}''} \, ) \\
		& = & \mathrm{id}_{w_1 \otimes ... \otimes w_m} 
		\end{array}
\end{eq*}
for $ \alpha(\, h \, ; \, \mathrm{id}_{w_1'}, \, ..., \, \mathrm{id}_{w_{m'}} \, ), \alpha(\, j \, ; \, \mathrm{id}_{w_1''}, \, ..., \, \mathrm{id}_{w_{m''}''} \, ) \in \mathbb{G}_n(w_1 \otimes ... \otimes w_m,  w_1 \otimes ... \otimes w_m)$, which comes from the abelianisation of the vertices of $D_n$. All other ways for a morphism to have different representations must be generated by successive examples of these cases, since otherwise they wouldn't be coequalised by the colimit in \cref{colimthm}. In the first case we just have
\begin{eq*} \begin{array}{rl}
		& \psi( \, \alpha( \, g \otimes e_k \, ; \, \mathrm{id}_{w_1}, \, ..., \, \mathrm{id}_{w_m}, \, \mathrm{id}_{v_1}, \, ..., \, \mathrm{id}_{v_k} \, ) \, ) \\
		= & \psi_{\mathbf{w}_{\pi(g)^{-1}}, \mathbf{v}} \circ \beta(\, g \otimes e_k \, ; \, \mathrm{id}_{\psi(w_1)}, \, ..., \, \mathrm{id}_{\psi(w_m)}, \, \mathrm{id}_{\psi(v_1)}, \, ..., \, \mathrm{id}_{\psi(v_k)} \, ) \circ \psi_{\mathbf{w}, \mathbf{v}}^{-1} \\
		= & \big( \psi_{\mathbf{w}_{\pi(g)^{-1}}} \otimes \psi_{\mathbf{v}} \big) \circ \big( \beta( g ; \mathrm{id}_{\psi(w_1)}, ..., \mathrm{id}_{\psi(w_m)}) \otimes \mathrm{id}_{\psi(\mathbf{v})} \big) \circ \big( \psi_{\mathbf{w}}^{-1} \otimes \psi_{\mathbf{v}}^{-1} \big) \\
		= & \big( \psi_{\mathbf{w}_{\pi(g)^{-1}}} \circ \beta( g ; \mathrm{id}_{\psi(w_1)}, ..., \mathrm{id}_{\psi(w_m)}) \circ \psi_{\mathbf{w}}^{-1} \big) \otimes \big( \psi_{\mathbf{v}} \circ \mathrm{id}_{\psi(\mathbf{v})} \circ \psi_{\mathbf{v}}^{-1} \big) \\
		= & \psi_{\mathbf{w}_{\pi(g)^{-1}}} \circ \beta( g ; \mathrm{id}_{\psi(w_1)}, ..., \mathrm{id}_{\psi(w_m)}) \circ \psi_{\mathbf{w}}^{-1} \\
		=& \psi( \, \alpha(g; \mathrm{id}_{w_1}, ..., \mathrm{id}_{w_m}) \, )
		\end{array}
\end{eq*}
as required. The second case is more subtle. We begin by expanding
\begin{eq*} \begin{array}{rl}
		& \psi( \, \alpha( \, g \, ; \, \mathrm{id}_{w_1}, \, ..., \, \mathrm{id}_{w_m} \, ) \\
		= & \psi( \, \alpha(\, h \, ; \, \mathrm{id}_{w_1'}, \, ..., \, \mathrm{id}_{w_{m'}} \, ) \, ) \\
		& \circ \, \, \psi( \, \alpha(\, j \, ; \, \mathrm{id}_{w_1''}, \, ..., \, \mathrm{id}_{w_{m''}''} \, ) \, ) \\
		& \circ \, \, \psi( \, \alpha(\, h^{-1} \, ; \, \mathrm{id}_{w_1'}, \, ..., \, \mathrm{id}_{w_{m'}'} \, ) \, ) \\
		&\circ \, \, \psi( \, \alpha(\, j^{-1} \, ; \, \mathrm{id}_{w_1''}, \, ..., \, \mathrm{id}_{w_{m''}''} \, ) \, ) \\
		= & \psi_{\mathbf{w'}} \circ \beta(\, h \, ; \, \mathrm{id}_{\psi(w_1')}, \, ..., \, \mathrm{id}_{\psi(w_{m'})} \, ) \circ \psi_{\mathbf{w'}}^{-1} \\
		& \circ \, \, \psi_{\mathbf{w''}} \circ\beta(\, j \, ; \, \mathrm{id}_{\psi(w_1'')}, \, ..., \, \mathrm{id}_{\psi(w_{m''}'')} \, ) \circ \psi_{\mathbf{w''}}^{-1} \\
		& \circ \, \, \psi_{\mathbf{w'}} \circ \beta(\, h^{-1} \, ; \, \mathrm{id}_{\psi(w_1')}, \, ..., \, \mathrm{id}_{\psi(w_{m'}')} \, ) \circ \psi_{\mathbf{w'}}^{-1}  \\
		&\circ \, \, \psi_{\mathbf{w''}} \circ \beta(\, j^{-1} \, ; \, \mathrm{id}_{\psi(w_1'')}, \, ..., \, \mathrm{id}_{\psi(w_{m''}'')} \, ) \circ \psi_{\mathbf{w''}}^{-1} \\
		\end{array}
\end{eq*}
Here the objects $w_i, w_i', w_i''$ are all in $\mathbb{G}_n \subseteq L\mathbb{G}_n$, and so we know their minimal generator decompositions are also in $\mathbb{G}_n$. It follows that $d(w_i \otimes w_j) = d(w_i) \otimes d(w_j)$ for all $i,j$, and hence by our definition of $\psi$ we have $\psi(w_i \otimes w_j) = \psi(w_i) \otimes \psi(w_j)$ and also $\psi_{\mathbf{w}_{\sigma}} = id$ for any permuation $\sigma$ --- and the same for $\mathbf{w'}$ and $\mathbf{w''}$. Also, note that since we are working in $\mathbb{G}_n(w_1 \otimes ... \otimes w_m,  w_1 \otimes ... \otimes w_m)$, all of the action morphisms in the above composite have the same source and target, $\psi(w_1 \otimes ...\otimes w_m)$. This object is weakly invertible, because each of the $w_i$ are invertible. However, the automorphisms of any weakly invertible object are isomorphic to the automorphisms of the unit object, as in the proof of \cref{zerotree}, and hence form an abelian group, by an Eckmann-Hilton argument like in the proof of \cref{colimthm}. Therefore we may permute these action morphisms freely, and so
\begin{eq*} \begin{array}{rl}
& \psi( \, \alpha( \, g \, ; \, \mathrm{id}_{w_1}, \, ..., \, \mathrm{id}_{w_m} \, ) \\
		= & \beta(\, h \, ; \, \mathrm{id}_{\psi(w_1')}, \, ..., \, \mathrm{id}_{\psi(w_{m'})} \, ) \\
		& \circ \, \, \beta(\, h^{-1} \, ; \, \mathrm{id}_{\psi(w_1')}, \, ..., \, \mathrm{id}_{\psi(w_{m'}')} \, )  \\
		& \circ \, \, \beta(\, j \, ; \, \mathrm{id}_{\psi(w_1'')}, \, ..., \, \mathrm{id}_{\psi(w_{m''}'')} \, ) \\
		& \circ \, \, \beta(\, j^{-1} \, ; \, \mathrm{id}_{\psi(w_1'')}, \, ..., \, \mathrm{id}_{\psi(w_{m''}'')} \, ) \\
		= & \mathrm{id}_{\psi(w_1) \otimes ... \otimes \psi(w_m)} \\
		= & \psi_{\mathbf{w}} \circ \beta(\, e_m \, ; \, \mathrm{id}_{\psi(w_1)}, \, ..., \, \mathrm{id}_{\psi(w_{m})} \, ) \circ \psi_{\mathbf{w}}^{-1}
		\end{array}
\end{eq*}
as required. 

With $\psi$ now fully defined, notice that
\begin{eq*} \begin{array}{rll}
		F(\psi) & = & \big\{ \, ( \, \psi(z_i), \, \psi(z_i^*), \, I \xrightarrow{\sim} \psi(I) \xrightarrow{\sim} \psi(z_i^*)\psi(z_i), \, \psi(z_i)\psi(z_i^*) \xrightarrow{\sim} \psi(I) \xrightarrow{\sim} I \, ) \, \big\}_{i \in \{z_1, ..., z_n\} } \\
		& = & \big\{ \, ( \, x_i, \, x_i^*, \, \eta_i, \, \epsilon_i \, ) \, \big\}_{i \in \{z_1, ..., z_n\} } \\
		\end{array}
\end{eq*}
which was our arbitrary object in $(X_{\mathrm{wkinv}})^n$. Therefore, $F$ is surjective on objects.

Next, choose an arbitrary monoidal transformation $\theta : \psi \to \chi$ from $\mathrm{E}G\mathrm{Alg}_W(L\mathbb{G}_n, X)$. By naturality, for any $w, w' \in \mathrm{Ob}(L\mathbb{G}_n)$ we have that
\begin{eq*} \begin{tikzcd}
\psi(w) \otimes \psi(w') \ar[r, "\sim"] \ar[d, "\theta_w \otimes \theta_{w'}"'] & \psi(w \otimes w') \ar[d, "\theta_{w \otimes w'}"] \\
\chi(w) \otimes \chi(w') \ar[r, "\sim"] & \chi(w \otimes w')
\end{tikzcd} \end{eq*}
or equivalently, $\theta_{w \otimes w'} = \chi_{w, w'} \circ (\theta_w \otimes \theta_{w'}) \circ \psi_{w, w'}^{-1}$. It follows from this that the components of $\theta$ are generated by the components on the generators of $\mathrm{Ob}(L\mathbb{G}_n)$, namely $\{ \, ( \, \theta_{z_i}, \, \theta_{z_i^*} \, ) \, \}_{i \in \{z_1, ..., z_n\} }$. But this is just $F(\theta)$, and thus any monoidal transformation $\theta$ is determined uniquely by its image under $F$, or in other words $F$ is faithful.

Finally, let $\psi, \chi$ be objects of $\mathrm{E}G\mathrm{Alg}_W(L\mathbb{G}_n, X)$, and choose an arbitrary map $\{ \, ( \, f_i, \, f^*_i \, ) \, \}_{i \in \{z_1, ..., z_n\} } : F(\psi) \to F(\chi)$ from $(X_{\mathrm{wkinv}})^n$. We can use this to construct a monoidal transformation $\theta : \psi \to \chi$ via the reverse of process we just used. Specifically, if we define
\begin{eq*} \theta_I = \chi_I \circ \psi_I^{-1}, \quad \quad \theta_{z_i} =  f_i, \quad \quad \theta_{z_i^*} = f_i^*\end{eq*}
then these will automatically form the naturality squares
\begin{eq*} \begin{tikzcd}
\psi(z_i) \otimes \psi(z_i^*) \ar[rr, "\psi_{z_i, z_i^*}"] \ar[dd, "f_i \otimes f_i^*"'] & & \psi(I) \ar[d, "\psi_I^{-1}"] & \psi(z_i^*) \otimes \psi(z_i) \ar[rr, "\psi_{z_i^*, z_i}"] \ar[dd, "f_i^* \otimes f_i"'] & & \psi(I) \ar[d, "\psi_I^{-1}"] \\
& & I \ar[d, "\chi_I"] & & & I \ar[d, "\chi_I"] \\
\chi(z_i) \otimes \chi(z_i^*) \ar[rr, "\chi_{z_i, z_i^*}"] & & \chi(I) & \chi(z_i^*) \otimes \chi(z_i) \ar[rr, "\chi_{z_i^*, z_i}"] & & \chi(I)
\end{tikzcd} \end{eq*}
since these are just the conditions for $\{ \, ( \, f_i, \, f^*_i \, ) \, \}_{i \in \{z_1, ..., z_n\} }$ to be a map $F(\psi) \to F(\chi)$ in $(X_{\mathrm{wkinv}})^n$. Repeatedly applying the naturality condition $\theta_{w \otimes w'} = \chi_{w, w'} \circ (\theta_w \otimes \theta_{w'}) \circ \psi_{w, w'}^{-1}$ will then generate all of the other components of $\theta$, in a way that clearly satisfies naturality. Thus we have a well-defined monoidal transformation $\theta : \psi \to \chi$, and applying $F$ to it gives
\begin{eq*} \begin{array}{rll}
		F(\theta) & = & \big\{ \, ( \, \theta_{z_i}, \, \theta_{z_i^*} \, ) \, \big\}_{i \in \{z_1, ..., z_n\} } \\
		& = & \big\{ \, ( \, f_i, \, f_i^* \, ) \, \big\}_{ i \in \{z_1, ..., z_n\} },
		\end{array}
\end{eq*}
our arbitrary map. Therefore $F$ is full and, putting this together with the previous results, is an equivalence of categories.
\end{proof}