\documentclass{amsart} % default font size is 10pt
\setcounter{tocdepth}{3}
\setcounter{secnumdepth}{3}
\usepackage{latexsym}
\usepackage{amsfonts}
\usepackage{amssymb}
\usepackage{amsmath}
\usepackage{amscd}
\usepackage{graphicx}
\usepackage{eucal}
\usepackage{amsthm}
\usepackage[all]{xy}
\usepackage{tikz-cd}
\usepackage{pdfsync}
\usepackage{xspace}
\usepackage[unicode=true, pdfusetitle,
 bookmarks=true,bookmarksnumbered=false,
 breaklinks=false,
 backref=false,
 colorlinks=true,
 %linkcolor=blue!70!black,
 citecolor=black,
 urlcolor=blue!78!red,
 final
]{hyperref}
\usepackage[capitalise]{cleveref}
\newcommand{\fref}{\cref}
\newcommand{\Fref}{\Cref}
\newcommand{\prettyref}{\cref}
\newcommand{\newrefformat}[2]{}

\usepackage[color=orange!80,bordercolor=black,textwidth=3cm,textsize=small,colorinlistoftodos]{todonotes}
\makeatletter \providecommand\@dotsep{5}
\makeatother
\newcommand{\amsartlistoftodos}{\makeatother \listoftodos\relax}

%% E.P. notes
\newcommand{\epnote}[1]{\todo[color=blue!40,linecolor=blue!40!black,size=\tiny]{#1}}
\newcommand{\epmpar}[1]{\todo[noline,color=blue!40,linecolor=blue!40!black,size=\tiny]{#1}}
\newcommand{\epnoteil}[1]{\todo[inline,color=blue!40,linecolor=blue!40!black,size=\normalsize]{#1}}

%% N.G. notes
\newcommand{\ngnote}[1]{\todo[color=red!40,linecolor=red!40!black,size=\tiny]{#1}}
\newcommand{\ngmpar}[1]{\todo[noline,color=red!40,linecolor=red!40!black,size=\tiny]{#1}}
\newcommand{\ngnoteil}[1]{\todo[inline,color=red!40,linecolor=red!40!black,size=\normalsize]{#1}}

% Cleveref definitions
\crefname{prop}{Proposition}{Propositions}
\crefname{thm}{Theorem}{Theorems}
\crefname{defn}{Definition}{Definitions}
\crefname{notn}{Notation}{Notations}
\crefname{construction}{Construction}{Constructions}
\crefname{lem}{Lemma}{Lemmas}
\crefname{rem}{Remark}{Remarks}
\crefname{cor}{Corollary}{Corollaries}
\crefname{scholium}{Scholium}{Scholia}
\crefname{figure}{Figure}{Figures}
\crefname{equation}{Display}{Displays}
\crefname{eq}{Display}{Displays}
\crefname{eqn}{Display}{Displays}

\newenvironment{eq}{\begin{equation}}{\end{equation}}
\newenvironment{eqn}{\begin{equation}}{\end{equation}}
\newenvironment{eq*}{\begin{equation*}}{\end{equation*}}
\newenvironment{eqn*}{\begin{equation*}}{\end{equation*}}

%\usepackage{natbib}
\newcommand{\bs}{\boldsymbol}
\newcommand{\mb}{\mathbf}
\renewcommand{\dot}{\centerdot}
\newcommand{\R}{\mathbb{R}}
\newcommand{\m}[1]{\mathcal{#1}}
\renewcommand{\SS}{\mathcal{S}}
\newcommand{\colim}{\textrm{colim }}
\newcommand{\f}[1]{\ensuremath{\mathcal{#1}}\xspace}
\newcommand{\g}[1]{\ensuremath{\mathbb{#1}}\xspace}
\newcommand{\Cat}{\ensuremath{\textnormal{Cat}}\xspace}
\newcommand{\cd}[2][]{\vcenter{\hbox{\xymatrix#1{#2}}}}
\newcommand{\Set}{\ensuremath{\textnormal{Set}}\xspace}
\newcommand{\twocat}{\ensuremath{\textnormal{2-Cat}}\xspace}
\newcommand{\Icon}{\ensuremath{\textnormal{Icon}}\xspace}
%\newcommand{\to}{\rightarrow}

%\pdfshift

\begin{document}
\numberwithin{equation}{section}% [if desired]
\newtheorem{thm}[equation]{Theorem}
\newtheorem{prop}[equation]{Proposition}
\newtheorem{lem}[equation]{Lemma}
\newtheorem{cor}[equation]{Corollary}

\newtheoremstyle{example}{\topsep}{\topsep}%
     {}%         Body font
     {}%         Indent amount (empty = no indent, \parindent = para indent)
     {\bfseries}% Thm head font
     {.}%        Punctuation after thm head
     {2pt}%     Space after thm head (\newline = linebreak)
     {\thmname{#1}\thmnumber{ #2}\thmnote{ #3}}%         Thm head spec

   \theoremstyle{example}
   \newtheorem{nota}[equation]{Notation}
   \newtheorem{example}[equation]{Example}
   \newtheorem{defn}[equation]{Definition}
   \newtheorem{rem}[equation]{Remark}
	\newtheorem{comment}[equation]{Comment}

\tableofcontents

\begin{defn} A strict monoidal category $X$ is said to be \emph{spacial} if, for any object $x \in \mathrm{Ob}(X)$ and any endomorphism of the unit object $f: I \to I$, 
\begin{eq*} f \otimes \mathrm{id}_x = \mathrm{id}_x \otimes f \end{eq*}
\end{defn}

The motivation for the name `spacial' comes from the context of string diagrams \cite{graphicalmon}. In a string diagram, the act of tensoring two strings together is represented by placing those strings side by side. Since the defining feature of the unit object is that tensoring it with other objects should have no effect, the unit object is therefore represented diagrammatically by the absense of a string. An endomorphism of the unit thus appears as an entity with no input or output strings, detached from the rest of the diagram. In a real-world version of these diagrams, made out of physical strings arranged in real space, we could use this detachedness to grab these endomorphisms and slide them over or under any strings we please, without affecting anything else in the diagram. This ability is embodied algebraically by the equation above, and hence categories which obey it are called `spacial'.

\begin{lem}\label{spacial} All $\mathrm{E}G$-algebras are spacial. \end{lem}
\begin{proof}
Let $X$ be an $\mathrm{E}G$-algebra, and fix $x \in \mathrm{Ob}(X)$ and \( f: I \to I \). From the surjectivity of \( \pi : G(2) \to S_2 \) we know that the set $\pi^{-1}( \, (1 \, 2) \, )$ is non-empty, and from the rules for composition of action morphisms we see that for any such $g \in \pi^{-1}( \, (1 \, 2) \, )$,
\begin{eq*}\begin{array}{rll}
		\alpha( \, g \, ; \, \mathrm{id}_x, \, \mathrm{id}_I \, ) \circ \alpha( \, e_2 \, ; \, \mathrm{id}_x, \, f \, ) & = & \alpha( \, g \, ; \, \mathrm{id}_x, \, f \, ) \\
		& = & \alpha( \, e_2 \, ; \, f, \, \mathrm{id}_x \, ) \circ \alpha( \, g \, ; \, \mathrm{id}_x, \, \mathrm{id}_I \, ) \\
		\end{array}
\end{eq*}
Thus in order to obtain the result we're after, it will suffice to find a particular $g \in \pi^{-1}( \, (1 \, 2) \, )$ for which
\begin{eq*}\alpha( \, g \, ; \, \mathrm{id}_x, \, \mathrm{id}_I \, ) = \mathrm{id}_x \end{eq*}
However, since
\begin{eq*}\begin{array}{rll}
		\alpha( \, g \, ; \, \mathrm{id}_x, \, \mathrm{id}_I \, ) & = & \alpha( \, g \, ; \, \mathrm{id}_x, \, \alpha( e_0; - ) \, ) \\
		& = & \alpha( \, \mu(g; e_1, e_0) \, ; \, \mathrm{id}_x \, )
		\end{array}
\end{eq*}
all we really need is to find a $g \in \pi^{-1}( \, (1 \, 2) \, )$ for which
\begin{eq*} \mu(g; e_1, e_0) = e_1 \end{eq*}
To this end, choose an arbtrary element $h \in \pi^{-1}( \, (1 \, 2) \, )$. This $h$ probably won't obey the above equation, but we can use it to construct a new element $g$ which does. Specifically, define
\begin{eq*} k \, := \, \mu( \, h \ ; \, e_1, \, e_0 \, ) \end{eq*}
and then consider
\begin{eq*} g \, := \, h \cdot \mu(e_2; k^{-1}, e_1) \end{eq*}
To see that this is the correct choice of $g$, first note that we must have \( \pi(k) = e_1 \), since this is the only element of $S_1$. Following from that, we have 
\begin{eq*}\begin{array}{rll}
		\pi \big( \, \mu(e_2; k^{-1}, e_1) \, \big) & = & \mu \big( \, \pi(e_2) \ ; \, \pi(k^{-1}), \, \pi(e_1) \, \big) \\
		& = & \mu \big( \, e_2  \ ; \, e_1, \, e_1 \, \big) \\
		& = & e_2
		\end{array}
\end{eq*}
and hence
\begin{eq*}\begin{array}{rll}
		\pi(g) & = & \pi \big( h \cdot \mu(e_2; k^{-1}, e_1) \big) \\
		& = & \pi(h) \cdot \pi \big(\mu(e_2; k^{-1}, e_1) \big) \\
		& = & (1 \, 2) \cdot e_2 \\
		& = & (1 \, 2)
		.\end{array}
\end{eq*}
So $g$ is indeed in $\pi^{-1}( \, (1 \, 2) \, )$, and furthermore
\begin{eq*}\begin{array}{rll}
		\mu(g; e_1, e_0) & = & \mu \big( \, h \cdot \mu(e_2; k^{-1}, e_1) \ ; \, e_1, \, e_0 \, \big) \\
		& = & \mu( \, h \ ; \, e_1, \, e_0 \, ) \cdot \mu \big( \, \mu(e_2; k^{-1}, e_1) \ ; \, e_1, \, e_0 \, \big) \\
		& = & \mu( \, h \ ; \, e_1, \, e_0 \, ) \cdot \mu \big( \, e_2 \ ; \, \mu(k^{-1}; e_1), \, \mu(e_1; e_0) \, \big) \\
		& = & \mu( \, h \ ; \, e_1, \, e_0 \, ) \cdot \mu( \, e_2 \ ; \, k^{-1}, e_0 \, ) \\
		& = & k \cdot k^{-1} \\
		& = & e_1
		\end{array}
\end{eq*}
Therefore, $h \cdot \mu(e_2; k^{-1}, e_1)$ is exactly the $g$ we were looking for, and so working backwards through the proof we obtain the required result:
\begin{eq*} \begin{array}{rll}
		\mu(g; e_1, e_0) & = & e_1 \\
		\implies \quad \alpha( \, g \, ; \, \mathrm{id}_x, \, \mathrm{id}_I \, ) & = & \mathrm{id}_x \\
		& & \\
		\alpha( \, g \, ; \, \mathrm{id}_x, \, \mathrm{id}_I \, ) \circ \alpha( \, e_2 \, ; \, \mathrm{id}_x, \, f \, ) & = & \alpha( \, e_2 \, ; \, f, \, \mathrm{id}_x \, ) \circ \alpha( \, g \, ; \, \mathrm{id}_x, \, \mathrm{id}_I \, ) \\
		\implies \quad \alpha( \, e_2 \, ; \, \mathrm{id}_x, \, f \, ) & = & \alpha( \, e_2 \, ; \, f, \, \mathrm{id}_I \, )
		\end{array}
\end{eq*}
\end{proof}

\section{Free algebras of action operads}

Our ultimate goal for this chapter is to understand the free braided monoidal category on an finite number of invertible objects. Thus, now that we have a firm grasp on action operads and their algebras, we should begin to think about the various free constructions they can form. 

\subsection{The free algebra on $n$ objects} 

We begin with the simplest case, which we will use extensively when calculating the invertible case later on. In the paper \cite{operadborel}, Gurski establishes how to contruct certain free action operad algebras through the use of the Borel construction. What follows in this section is a quick summary of the results which will be useful for our purposes. For a more detailed treatment please refer to \cite{operadborel}.

\begin{prop}\label{freealg} There exists a free $\mathrm{E}G$-algebra on $n$ objects. That is, there is an $\mathrm{E}G$-algebra $Y$ such that for any other $\mathrm{E}G$-algebra $X$, we have an isomorphism of categories
\begin{eq*} \mathrm{E}G\mathrm{Alg}_S(Y, X) \cong X^n \end{eq*}
\end{prop}
\begin{proof}
There is an obvious forgetful 2-functor \( U: \mathrm{E}G\mathrm{Alg}_S \to \mathrm{Cat}\) sending $\mathrm{E}G$-algebras to their underlying categories. $U$ has a left adjoint, which we call the free 2-functor \( F : \mathrm{Cat} \to \mathrm{E}G\mathrm{Alg}_S \) adjoint to it. It follows immediately that
\begin{eq*}\begin{array}{rll}
		U(X)^n & = & \mathrm{Cat}(\{1, ..., n\}, U(X) ) \\
		& \cong & \mathrm{E}G\mathrm{Alg}_S( F(\{1, ..., n\}), X) 
		\end{array}
\end{eq*}
Since $X$ and $U(X)$ are obviously isomorphic as categories, this shows that $F(\{1, ..., n\})$ is the free algebra on $n$ objects as required. 
\end{proof}

\begin{defn}\label{Gndef} Let $\mathbb{G}_n$ denote the $\mathrm{E}G$-algebra
\begin{eq*} \mathbb{G}_n = \coprod_{m \geq 0} \mathrm{E}G(m) \times_{G(m)} \{ z_1, ..., z_n \}^m \end{eq*}
\end{defn}

Objects in this algebra are equivalence classes of tuples $(g; x_1, ..., x_m)$, for $g \in G(m)$ and $x_i \in \{z_1, ..., z_n\}$, under the relation
\begin{eq*} ( \, gh \, ; \, x_1, \, ..., \, x_m \, ) \sim ( \, g \, ; \, x_{\pi(h)^{-1}(1)}, \, ..., \, x_{\pi(h)^{-1}(m)} \, )\end{eq*}
Using this relation we can write each object uniquely in the form $[e; x_1, ..., x_m]$ for some $m \in \mathbb{N}$ and $x_i \in \{z_1, ..., z_n\}$, and we will adopt the notation $x_1 \otimes ... \otimes x_m$ for this equivalence class. In other words, we're viewing $\mathrm{Ob}(\mathbb{G}_n)$ as the free monoid on $n$ generators, $\mathbb{N}^{*n}$, which is just the coproduct of the monoids $\mathbb{N}$ freely generated by each of the $z_i$.

Similarly, the morphisms of $\mathbb{G}_n$ are the equivalence classes of the maps
\begin{eq*} (! ; \mathrm{id}_{x_1}, ..., \mathrm{id}_{x_m}) : ( g ; x_1, ..., x_m ) \to ( h ; x_1, ..., x_m )\end{eq*}
Using the relation we can write each morphism uniquely in the form
\begin{eq*} [g ; \mathrm{id}_{x_1},...,\mathrm{id}_{x_m}] \, : \, x_1 \otimes ... \otimes x_m \to x_{\pi(g)^{-1}(1)} \otimes ... \otimes x_{\pi(g)^{-1}(m)},\end{eq*}
with $x_i \in \{z_1, ..., z_n \}$. The action of $\mathrm{E}G$ on objects of $\mathbb{G}_n$ is permutation and tensor product, and the action on morphisms is given by
\begin{eq*} \alpha( \, g \, ; \, [h_1; \mathrm{id}_{x_1}, ..., \mathrm{id}_{x_{m_1}}], \, ..., \, [h_k; \mathrm{id}_{x_1}, ..., \mathrm{id}_{x_{m_k}}] \, ) = [ \, \mu(g;h_1, .., h_k) \, ; \, \mathrm{id}_{x_1}, \, ..., \, \mathrm{id}_{x_{m_k}} \, ] \end{eq*}
Notice that by the tensor product notation we adopted earlier the object $[e; x]$ is written as just $x$, and so $[e; \mathrm{id}_x] = \mathrm{id}_{[e;x]}$ should be written as $\mathrm{id}_x$, and hence by the above $[g; \mathrm{id}_{x_1}, ..., \mathrm{id}_{x_m}]$ is really $\alpha(g; \mathrm{id}_{x_1}, ..., \mathrm{id}_{x_m})$.

\begin{thm} $\mathbb{G}_n$ is the free $\mathrm{E}G$-algebra on $n$ objects. That is, 
\begin{eq*}  F(\{1, ..., n\}) = \mathbb{G}_n \end{eq*}
\end{thm}

\epnote{need link to proof}

\subsection{The free algebra on $n$ invertible objects}

We saw in \cref{freealg} that the existence of a free $\mathrm{E}G$-algebra on $n$ objects can be proven by taking the left adjoint of a 2-functor which forgets about the algebra structure. Now we want to extend this idea into the realm of algebras on invertible objects. For the analogous approach, we will need to find a new 2-functor that lets us forget about non-invertible objects, and then hopefully we can find its left adjoint too, and use it to freely add inverses to $\mathbb{G}_n$. First though, we need to make this concept of `forgetting non-invertible objects' a little more precise.

\begin{defn} Given an $\mathrm{E}G$-algebra $X$, we denote by $X_{\mathrm{inv}}$ the sub-$\mathrm{E}G$-algebra containing all invertible objects in $X$ and the isomorphisms between them. \end{defn}

Note that this is indeed a well-defined $\mathrm{E}G$-algebra. If $x_1, ..., x_m$ are invertible objects with inverses $x_1^*, ..., x_m^*$, then $\alpha(g; x_1, ..., x_m)$ is an invertible object with inverse $\alpha(g; x_m^*, ..., x_1^*)$, since 
\begin{eq*} \begin{array}{ll}
		& \alpha(g; x_1, ..., x_m) \otimes \alpha(g; x_m^*, ..., x_1^*) \\
		= & \big( x_{\pi(g)^{-1}(1)} \otimes ... \otimes x_{\pi(g)^{-1}(m)} \big) \otimes \big( x_{\pi(g)^{-1}(m)}^* \otimes ... \otimes x_{\pi(g)^{-1}(1)}^* \big) \\
		= & I \\
		& \\
		& \alpha(g; x_m^*, ..., x_1^*) \otimes \alpha(g; x_1, ..., x_m) \\
		= & \big( x_{\pi(g)^{-1}(m)}^* \otimes ... \otimes x_{\pi(g)^{-1}(1)}^* \big) \otimes \big( x_{\pi(g)^{-1}(1)} \otimes ... \otimes x_{\pi(g)^{-1}(m)} \big) \\
		= & I
		\end{array}
\end{eq*}
Likewise, if $f_1, ..., f_m$ are isomorphisms from invertible objects $x_1, ..., x_m$ to invertible objects $y_1, ..., y_m$, then $\alpha(g; f_1, ..., f_m)$ is a map from the invertible object $\alpha(g; x_1, ..., x_m)$ to the invertible object $\alpha(g; y_1, ..., y_m)$, and it has an inverse $\alpha(g^{-1}; f_{\pi(g)(1)}^{-1}, ..., f_{\pi(g)(m)}^{-1})$, since
\begin{eq*} \begin{array}{ll}
		& \alpha\big( \, g^{-1} \, ; \, f_{\pi(g)(1)}^{-1}, \, ..., \, f_{\pi(g)(m)}^{-1} \, \big) \circ \alpha( \, g \, ; \, f_1, ..., f_m \,) \\
		= & \alpha\big( \, g^{-1}g \, ; \, f_1^{-1} f_1, \, ..., \, f_m^{-1} f_m \, \big) \\
		= & \mathrm{id}_{x_1 \otimes ... \otimes x_m} \\
		& \\
		& \alpha( \, g \, ; \, f_1, ..., f_m \,) \circ \alpha\big( \, g^{-1} \, ; \, f_{\pi(g)(1)}^{-1}, \, ..., \, f_{\pi(g)(m)}^{-1} \, \big) \\
		= & \alpha\big( \, gg^{-1} \, ; \, f_{\pi(g)(1)} f_{\pi(g)(1)}^{-1}, \, ..., \, f_{\pi(g)(m)} f_{\pi(g)(m)}^{-1} \, \big) \\
		= & \mathrm{id}_{y_{\pi(g)(1)} \otimes ... \otimes y_{\pi(g)(m)}}
		\end{array}
\end{eq*}
Clearly then, $X_{\mathrm{inv}}$ is the correct algebra for our new forgetful 2-functor to send $X$ to. Knowing this, we can contruct the rest of the functor fairly easily.

\begin{prop} \label{invprop} The assignment $X \mapsto X_{\mathrm{inv}}$ can be extended to a 2-functor $(\_)_{\mathrm{inv}}: \mathrm{E}G\mathrm{Alg}_S \to \mathrm{E}G\mathrm{Alg}_S$.
\end{prop}
\begin{proof}
Let $F: X \to Y$ be a (strict) map of $\mathrm{E}G$-algebras. If $x$ is an invertible object in $X$ with inverse $x^*$, then $F(x)$ is an invertible object in $Y$ with inverse $F(x^*)$, by
\begin{eq*} F(x) \otimes F(x^*) = F(x \otimes x^*) = F(I) = I \end{eq*}
\begin{eq*} F(x^*) \otimes F(x) = F(x^* \otimes x) = F(I) = I \end{eq*}
Since $F$ sends invertible objects to invertible objects, it will also send isomorphisms of invertible objects to isomorphisms of invertible objects. In other words, the map $F: X \to Y$ can be restricted to a map $F_{\mathrm{inv}} : X_{\mathrm{inv}} \to Y_{\mathrm{inv}}$. Moreover, we have that
\begin{eq*} (F \circ G)_{\mathrm{inv}}(x) = F \circ G(x) = F_{\mathrm{inv}} \circ G_{\mathrm{inv}}(x) \end{eq*}
\begin{eq*} (F \circ G)_{\mathrm{inv}}(f) = F \circ G(f) = F_{\mathrm{inv}} \circ G_{\mathrm{inv}}(f) \end{eq*}
and so the assignment $F \mapsto F_{\mathrm{inv}}$ is clearly functorial. Next, let $\theta : F \Rightarrow G$ be an $\mathrm{E}G$-monoidal natural transformation. Choose an invertible object $x$ from $X$, and consider the component map of its inverse, $\theta_{x^*} : F(x^*) \to G(x^*)$. Since $\theta$ is monoidal, we have $\theta_{x^*} \otimes \theta_x = \theta_I = I$ and $\theta_x \otimes \theta_{x^*} = I$, or in other words that $\theta_{x^*}$ is the monoidal inverse of $\theta_x$. We can use this fact to construct a compositional inverse as well, namely $\mathrm{id}_{F(x)} \otimes \theta_{x^*} \otimes \mathrm{id}_{G(x)}$, which can be seen as follows:
\begin{eq*}  \begin{array}{rll}
		\big( \mathrm{id}_{F(x)} \otimes \theta_{x^*} \otimes \mathrm{id}_{G(x)} \big)  \circ \theta_x & = & \theta_x \otimes \theta_{x^*} \otimes \mathrm{id}_{G(x)} \\
		& = &  \mathrm{id}_{G(x)} \\
		&& \\
		\theta_x \circ  \big( \mathrm{id}_{F(x)} \otimes \theta_{x^*} \otimes \mathrm{id}_{G(x)} \big) & = & \mathrm{id}_{F(x)} \otimes \theta_{x^*} \otimes \theta_x \\
		& = &  \mathrm{id}_{F(x)} \\
		\end{array} 
\end{eq*}
Therefore, we see that all the components of our transformation on invertible objects are isomorphisms, and hence we can define a new transformation $\theta_{\mathrm{inv}}: F_{\mathrm{inv}} \Rightarrow G_{\mathrm{inv}}$ whose components are just $(\theta_{\mathrm{inv}})_x = \theta_x$. The assignment $\theta \mapsto \theta_{\mathrm{inv}}$ is also clearly functorial, and thus we have a complete 2-functor $(\_)_{\mathrm{inv}}: \mathrm{E}G\mathrm{Alg}_S \to \mathrm{E}G\mathrm{Alg}_S$.
\end{proof}

\begin{prop} The 2-functor $(\_)_{\mathrm{inv}}: \mathrm{E}G\mathrm{Alg}_S \to \mathrm{E}G\mathrm{Alg}_S$ has a left adjoint, $L: \mathrm{E}G\mathrm{Alg}_S \to \mathrm{E}G\mathrm{Alg}_S$.
\end{prop}
\begin{proof} To begin, consider the 2-monad $\mathrm{E}G(\_)$. This is a finitary monad, that is it preserves all filtered colimits, and it is a 2-monad over $\mathrm{Cat}$, which is locally finitely presentable. It follows from this that $\mathrm{E}G\mathrm{Alg}_S$ is itself locally finitely presentable. Thus if we want to prove $(\_)_{\mathrm{inv}}$ has a left adjoint, we can use the Adjoint Functor Theorem for locally finitely presentable categories, which amounts to showing that $(\_)_{\mathrm{inv}}$ preserves both limits and filtered colimits.
\begin{itemize}
\item Given an indexed collection of $\mathrm{E}G$-algebras $X_i$, the $\mathrm{E}G$-action of their product $\prod X_i$ is defined componentwise. In particular, this means that the tensor product of two objects in $\prod X_i$ is just the collection of the tensor products of their components in each of the $X_i$. An invertible object in $\prod X_i$ is thus simply a family of invertible objects from the $X_i$ --- in other words, $(\prod X_i)_{\mathrm{inv}} = \prod (X_i)_{\mathrm{inv}}$.
\item Given maps of $\mathrm{E}G$-algebras $F: X \to Z$, $G : Y \to Z$, the $\mathrm{E}G$-action of their pullback $X \times_Z Y$ is also defined componentwise. It follows that an invertible object in $X \times_Z Y$ is just a pair of invertible objects $(x, y)$ from $X$ and $Y$, such that $F(x) = G(y)$. But this is the same as asking for a pair of objects $(x, y)$ from $X_{\mathrm{inv}}$ and $Y_{\mathrm{inv}}$ such that $F_{\mathrm{inv}}(x) = G_{\mathrm{inv}}(y)$, and hence $(X \times_Z Y)_{\mathrm{inv}} = X_{\mathrm{inv}} \times_{Z_{\mathrm{inv}}} Y_{\mathrm{inv}}$.
\item Given a filtered diagram $D$ of $\mathrm{E}G$-algebras, the $\mathrm{E}G$-action of their colimit $\mathrm{colim}(D_n)$ is defined in the following way: use filteredness to find an algebra which contains (representatives of the classes of) all the things you want to act on, then apply the action of that algebra. In the case of tensor products this means that $[x]\otimes[y] = [x \otimes y]$, and thus an invertible object in $\mathrm{colim}(D_n)$ is just (the class of) an invertible object in one of the algebras of $D$. In other words, $\mathrm{colim}(D_n)_{\mathrm{inv}} = \mathrm{colim}(D_{\mathrm{inv}})$.
\end{itemize}
Preservation of products and pullbacks gives preservation of limits, and preservation of limits and filtered colimits gives the result.
\end{proof}

With this new 2-functor $L: \mathrm{E}G\mathrm{Alg}_S \to \mathrm{E}G\mathrm{Alg}_S$, we now have the ability to `freely add inverses to objects' in any $\mathrm{E}G$-algebra we want. The algebra $L\mathbb{G}_n$ is then a clear candidate for our free algebra on $n$ invertible objects, and indeed the proof of this is very simple.

\begin{thm} There exists a free $\mathrm{E}G$-algebra on $n$ invertible objects. Specifically, the algebra $L\mathbb{G}_n$ is such that for any other $\mathrm{E}G$-algebra $X$, we have an isomorphism of categories
\begin{eq*} \mathrm{E}G\mathrm{Alg}_S(L\mathbb{G}_n, X) \cong (X_{\mathrm{inv}})^n \end{eq*}
\end{thm}
\begin{proof}
Using the adjunction from the previous Proposition along with the one from \cref{freealg}, we see that
\begin{eq*}\begin{array}{rll}
		 U(X_{\mathrm{inv}})^n & = & \mathrm{Cat}(\{1, ..., n\}, U(X_{\mathrm{inv}}) ) \\
		& \cong & \mathrm{E}G\mathrm{Alg}_S( F(\{1, ..., n\}), X_{\mathrm{inv}}) \\
		& \cong & \mathrm{E}G\mathrm{Alg}_S( LF(\{1, ..., n\}), X)
\end{array}
 \end{eq*}
As before, $X_{\mathrm{inv}}$ and $U(X_{\mathrm{inv}})$ are obviously isomorphic as categories, and so \( LF(\{1, ..., n\}) = L\mathbb{G}_n \) satisfies the requirements for the free algebra on $n$ invertible objects.
\end{proof}

\subsection{$L(X)$ as an initial algebra}

We have now proven that a free $\mathrm{E}G$-algebra on $n$ invertible objects does indeed exist. But this fact on its own is not very helpful. To be able to actually use the free algebra $L\mathbb{G}_n$, we need to know how to contruct it explicitly, in terms of its objects and morphisms. We could do this by finding a detailed characterisation of the 2-functor $L$, and then applying this to our explicit description of $\mathbb{G}_n$ from \cref{Gndef}. However, this would probably be much more effort than is required, since it would involve determining the behaviour of $L$ in many situtations we aren't interested in, and we also wouldn't be leveraging $\mathbb{G}_n$'s status as a free algebra to make the calculations any easier. We will try a different strategy instead. We begin by noticing a special property of the functor $L$.

\begin{prop} \label{linveql} For any $\mathrm{E}G$-algebra $X$, we have $L(X)_{\mathrm{inv}} = L(X)$.
\end{prop}
\begin{proof}
From the definition of adjunctions, the isomorphisms
\begin{eq*}\mathrm{E}G\mathrm{Alg}_S(LX , Y) \cong \mathrm{E}G\mathrm{Alg}_S(X, Y_{\mathrm{inv}}) \end{eq*}
are subject to certain naturality conditions. Specifically, given $F: X' \to X$ and $G: Y \to Y'$ we get a commutative diagram
\begin{eq*} \xymatrix{
\mathrm{E}G\mathrm{Alg}_S(LX , Y) \ar[d]_{G \circ \_ \circ LF} \ar[r]^{\sim} & \mathrm{E}G\mathrm{Alg}_S(X, Y_{\mathrm{inv}}) \ar[d]^{G_{\mathrm{inv}} \circ \_ \circ F} \\
\mathrm{E}G\mathrm{Alg}_S(LX' , Y') \ar[r]^{\sim} & \mathrm{E}G\mathrm{Alg}_S(X', Y'_{\mathrm{inv}}) }.
\end{eq*}
Consider the case where $F$ is the identity map $\mathrm{id}_X : X \to X$ and $G$ is the inclusion $j: L(X)_{\mathrm{inv}} \to L(X)$. Note that because $j$ is an inclusion, the restriction $j_{\mathrm{inv}}: (L(X)_{\mathrm{inv}})_{\mathrm{inv}} \to L(X)_{\mathrm{inv}}$ is also an inclusion, but since $((\_)_{\mathrm{inv}})_{\mathrm{inv}} = (\_)_{\mathrm{inv}}$, we have that $j_{\mathrm{inv}} = id$. It follows that
\begin{eq*} \xymatrix{
\mathrm{E}G\mathrm{Alg}_S(LX , LX_{\mathrm{inv}}) \ar[d]_{j \circ \_} \ar[r]^{\sim} & \mathrm{E}G\mathrm{Alg}_S(X, LX_{\mathrm{inv}}) \ar@{=}[d] \\
\mathrm{E}G\mathrm{Alg}_S(LX , LX) \ar[r]^{\sim} & \mathrm{E}G\mathrm{Alg}_S(X, LX_{\mathrm{inv}}) }.
\end{eq*}
Therefore, for any map $f: LX \to LX$ there exists a unique $g: LX \to LX_{\mathrm{inv}}$ such that $j \circ g =f$. But this means that for any such $f$, we must have $\mathrm{im}(f) \subseteq L(X)_{\mathrm{inv}}$, and so in particular $L(X) = \mathrm{im}(\mathrm{id}_{LX}) \subseteq L(X)_{\mathrm{inv}}$. Since $L(X)_{\mathrm{inv}} \subseteq L(X)$ by definition, we obtain the result.
\end{proof}

This result is not especially surprising. Intuitively, it just says that when you freely add inverses to an algebra, every object ends up with an inverse. But the upshot of this is that we now have another way of thinking about $L(X)$: as the target object of the unit of our adjunction, $\eta_X: X \to L(X)_{\mathrm{inv}}$. This means that we don't really need to know the entirety of $L$ in order to determine the free algebra $L\mathbb{G}_n$, just its unit. To find this unit directly, we can turn to the following fact about adjunctions, for which a proof can be found in Lemma 2.3.5 of Leinster's \textit{Basic Category Theory} \cite{bct}.

\begin{prop}\label{initial} Let $F \dashv G: A \to B$ be an adjunction with unit $\eta$. For any object $a$ in $A$, let $(a \downarrow G)$ denote the comma category whose objects are pairs $(b, f)$ consisting of an object $b$ from $B$ and a morphism $f: a \to G(b)$ from $A$, and whose morphisms $h: (b, f) \to (b', f')$ are morphisms $f: b \to b'$ from $B$ such that $G(f) \circ f = f'$. Then the pair $\big(F(a), \eta_a: a \to GF(a) \big)$ is an initial object of $(a \downarrow G)$.
\end{prop}

\begin{cor} If $\phi: \mathbb{G}_n \to Z$ is an initial object of $(\mathbb{G}_n \downarrow \mathrm{inv})$, then 
\begin{eq*} Z \, \cong \, (L\mathbb{G}_n)_{\mathrm{inv}} \, = \, L\mathbb{G}_n \end{eq*}
\end{cor}

Being able to view $L\mathbb{G}_n$ as the initial object in the comma category $(\mathbb{G}_n \downarrow \mathrm{inv})$ will prove immensely useful in the coming sections. This is because it lets us think about the properties of $L\mathbb{G}_n$ in terms of maps $\psi: \mathbb{G}_n \to X_{\mathrm{inv}}$, which is exactly the context where we can exploit the fact that $\mathbb{G}_n$ is a free algebra.

Also, from now on rather than writing objects in $(\mathbb{G}_n \downarrow \mathrm{inv})$ as maps $\psi: \mathbb{G}_n \to Y_{\mathrm{inv}}$, we will instead just let $X = Y_{\mathrm{inv}}$ and speak of maps $\psi: \mathbb{G}_n \to X$. This is purely for notational convenience, and shouldn't be a problem so long as we always remember that the targets of these maps only ever contain invertible objects and morphisms.

\subsection{Objects and morphisms of the initial algebras}

We know that the functor $L$ represents the process of `freely adding inverses to objects' of a given $\mathrm{E}G$-algebra. Therefore, it makes sense to expect the objects of $L\mathbb{G}_n$ to form not just a monoid but a group, and in particular be the group completion of $\mathbb{G}_n$'s monoid of objects. As we saw in \cref{Gndef} the objects of $\mathbb{G}_n$ are $\mathbb{N}^{*n}$, and so we want to have that $\mathrm{Ob}(L\mathbb{G}_n) = \mathbb{Z}^{*n}$. This intuition is correct, and can justified as follows:

\begin{prop}\label{Zobj} Let $\phi: \mathbb{G}_n \to Z$ be an initial object in $(\mathbb{G}_n \downarrow \mathrm{inv})$. Then $\mathrm{Ob}(Z) = \mathbb{Z}^{*n}$, and the restriction of $\phi$ to objects $\phi_{\mathrm{ob}}$ is the obvious inclusion $\mathbb{N}^{*n} \to \mathbb{Z}^{*n}$.
\end{prop}
\begin{proof}
To begin, we will first construct a non-initial object in $(\mathbb{G}_n \downarrow \mathrm{inv})$ which possesses all of the required properties. Let $H$ be the $\mathrm{E}G$-algebra whose objects are $\mathrm{Ob}(H) = \mathbb{Z}^{*n}$ and which has a unique morphism between each of its objects. In order to define a map of $\mathrm{E}G$-algebras $\mathbb{G}_n \to H$, all we need an underlying monoid homomorphism $\mathbb{N}^{*n} \to \mathbb{Z}^{*n}$ between their objects. There is a unique $\psi: \mathbb{G}_n \to H$ where this homomorphism $\psi_{\mathrm{ob}}$ is the obvious inclusion --- the result of $\psi$ on morphisms will just be determined by their source and target. Clearly this map is an object in $(\mathbb{G}_n \downarrow \mathrm{inv})$.

Now, let $\phi: \mathbb{G}_n \to Z$ be our initial object in $(\mathbb{G}_n \downarrow \mathrm{inv})$. It follows that there is a unique algebra map $u: Z \to H$ with $u\phi = \psi$, and hence a monoid homomorphism $u_{\mathrm{ob}}$ making
\begin{eq*} \xymatrix{
& \mathbb{N}^{*n} \ar[dl]_{\phi_{\mathrm{ob}}} \ar[dr]^-{\psi_{\mathrm{ob}}} & \\
\mathrm{Ob}(Z) \ar[rr]^{u_{\mathrm{ob}}} & & \mathbb{Z}^{*n} }
\end{eq*}
commute. This fact is enough to determine much of the behaviour of $u_{\mathrm{ob}}$. If we use the inclusion $\mathbb{N}^{*n} \to \mathbb{Z}^{*n}$ to see the generators $z_1, ..., z_n$ of $\mathbb{N}^{*n}$ as the generators of $\mathbb{Z}^{*n}$ as well, then for any such $z_i$ we have
\begin{eq*} u_{\mathrm{ob}}\big( \, \phi_{\mathrm{ob}}(z_i) \, \big) \, = \, \psi_{\mathrm{ob}}(z_i) \, = \, z_i, \quad \quad u_{\mathrm{ob}}\big( \, \phi_{\mathrm{ob}}(z_i)^* \, \big) \, = \, z_i^* \end{eq*}
Since any element of $\mathbb{Z}^{*n}$ is a product of the generators $z_i, z_i^*$, it follows that any element of $\mathbb{Z}^{*n}$ is a product of things in the image of $u_{\mathrm{ob}}$, and hence that the homomorphism $u_{\mathrm{ob}}$ is surjective. Indeed, this is clearly true even when $u_{\mathrm{ob}}$ restricted to $\langle \, \mathrm{im}(\phi_{\mathrm{ob}}) \, \rangle$, the free group generated by elements under $\phi_{\mathrm{ob}}$. Moreover, $\psi_{\mathrm{ob}}$ is injective, so $\phi_{\mathrm{ob}}$ is also injective, and hence $u_{\mathrm{ob}}$ is injective on $\langle \, \mathrm{im}(\phi_{\mathrm{ob}}) \, \rangle$ too. In other words, $\mathrm{Ob}(Z)$ contains a submonoid $\langle \, \mathrm{im}(\phi) \, \rangle$ which is isomorphic to $\mathbb{Z}^{*n}$. 

Finally, we wish to show that this submonoid is actually all of  $\mathrm{Ob}(Z)$. Let $J$ be the $\mathrm{E}G$-algebra with objects $\mathrm{Ob}(Z) / \langle \, \mathrm{im}(\phi_{\mathrm{ob}}) \, \rangle$ and a unique morphism between each. As before, we can construct a new object $\chi : \mathbb{G}_n \to J$ of $(\mathbb{G}_n \downarrow \mathrm{inv})$ just by defining $\chi_{\mathrm{ob}}$, since the morphisms are determined by their source and target. We choose $\chi_{\mathrm{ob}}(x) = [\phi_{\mathrm{ob}}(x)]$ for all $x \in \mathbb{N}^{*n}$, which works since $\chi_{\mathrm{ob}}(z_i) = 0$ is invertible for the generator $z_i$. Now, initiality of $\phi$ should gives us a unique $v : Z \to J$ such that 
\begin{eq*} \xymatrix{
& \mathbb{N}^{*n} \ar[dl]_{\phi_{\mathrm{ob}}} \ar[dr]^-{\chi_{\mathrm{ob}}} & \\
\mathrm{Ob}(Z) \ar[rr]^{v_{\mathrm{ob}}} & & \mathrm{Ob}(Z) / \langle \, \mathrm{im}(\phi) \, \rangle }
\end{eq*}
commutes. But there are at least two options here --- one is the $v$ whose underlying monoid homomorphism is $v_{\mathrm{ob}}(x) = [x]$, and the other is the one with $v_{\mathrm{ob}}(x) = 0$. It follows that these must be the same map, and thus $ \mathrm{Ob}(Z) /  \langle \, \mathrm{im}(\phi) \, \rangle = 0$, that is, $\mathrm{Ob}(Z) =  \langle \, \mathrm{im}(\phi) \, \rangle$. Returning to the first of our diagrams, we see now that $u_{\mathrm{ob}} : \mathrm{Ob}(Z) \to \mathbb{Z}^{*n}$ is both injective and surjective on its whole domain. Therefore $ \mathrm{Ob}(Z) \cong \mathbb{Z}^{*n}$, and when viewed this way $\phi_{\mathrm{ob}}$ is the obvious inclusion $\mathbb{N}^{*n} \to \mathbb{Z}^{*n}$.
\end{proof}

Unlike with objects, there is no simple intuition for how the application of $L$ will generate the morphisms of $L\mathbb{G}_n$ from those in $\mathbb{G}_n$. Obviously we will need to add a host of new action morphisms between all of the new objects, and the various composites of these. But we will also need new action maps between old objects, based on new ways of viewing them as a tensor product. For example, we must have an $\alpha(g; \mathrm{id}_x, \mathrm{id}_{x^*})$, which will be a new automorphism of $0 = x \otimes x^* = x^* \otimes x$. Should any of these new maps actually be the equal to ones inherited from $\mathbb{G}_n$? And how do the composites of these things relate to one another? It is not immediately clear. 

However, what we do know is that all of the morphisms in $\mathbb{G}_n$ can be written as action morphisms. And there is some sense in which we shouldn't expect a free $\mathrm{E}G$-algebra contruction to add any new maps that don't come from the $\mathrm{E}G$-action. This seems to suggest that $L\mathbb{G}_n$ should also only contain action morphisms. We will need to make our reasoning much more rigourous before we can can prove this though, so we start by introducing some new terminology.

\begin{defn} \label{mgd} For an element $w$ of $\mathbb{Z}^{*n}$, let the \emph{minimal generator decomposition} of $w$ be the unique finite sequence $d(w) = (d(w)_1, ..., d(w)_k)$ such that
\begin{eq*} d(w)_i \in \{z_1, z_1^*, ..., z_n, z_n^* \}, \quad \bigotimes d(w)_i = w, \quad d(w)_{i+1} \neq d(w)_i^* \end{eq*}
for all $1 \leq i \leq k$.
\end{defn}

In other words, the minimal generator decomposition of an object is the shortest way of writing it as a tensor product of generators $z_i$.

\begin{defn} Let $f_i: x_i \to y_i$, $1 \leq i \leq m$, be maps in an $\mathrm{E}G$-algebra $X$. Then we say that $(x_1, ..., x_m)$ is the \emph{source sequence} of the morphism $\alpha(g; f_1, .., f_m)$, and that $(y_{\pi(g)^{-1}(1)}, ..., y_{\pi(g)^{-1}(1)})$ is its \emph{target sequence}. \end{defn}

The purpose of source and target sequences is to allow us to better talk about the composition of action morphisms. Specifically, we know that if the target of $\alpha(g; \mathrm{id}_{w_1}, ..., \mathrm{id}_{w_m})$ is the same as the source of $\alpha(g'; \mathrm{id}_{w_1'}, ..., \mathrm{id}_{w_{m'}'})$ then those two maps can be composed, just like any other morphisms. But if the target sequence of $\alpha(g; \mathrm{id}_{w_1}, ..., \mathrm{id}_{w_m})$ is the same as the source sequence of $\alpha(g'; \mathrm{id}_{w_1'}, ..., \mathrm{id}_{w_{m'}'})$, then we can go one step futher and express their composite as an action morphism as well, since from the definition of $\mathrm{E}G$-actions we know that
\begin{eq*}\begin{array}{rll}
		\alpha(g'; \mathrm{id}_{w_1'}, ..., \mathrm{id}_{w_{m'}'}) \circ \alpha(g; \mathrm{id}_{w_1}, ..., \mathrm{id}_{w_m}) & = & \alpha(g'g; \mathrm{id}_{w_1}, ..., \mathrm{id}_{w_m}) \\
		\mathrm{if} \quad w_{\pi(g)^{-1}(i)} & = & w_i' \quad \forall i 
		\end{array}
\end{eq*}
This fact will be important for figuring out what new maps will need to appear in $L\mathbb{G}_n$. To get the most use out of it though, we'll need the following result about rearranging source and target sequences.

\begin{prop}\label{zerosubseq} Let $x_1, ..., x_m$ be a sequence of objects from $\mathrm{E}G$-algebra $X$ which has a contiguous subsequence $x_i, ..., x_j$ with $x_i \otimes ... \otimes x_j = I$. Then there exists $g \in G(m)$ such that the identity $\mathrm{id}_{x_1 \otimes ... \otimes x_m}$ can be written as an action map $\alpha(g; \mathrm{id}_{x_1}, ..., \mathrm{id}_{x_m} )$ with target sequence 
\begin{eq*} ( \, x_1, \, ..., \, x_{i-1}, \, x_{j+1}, \, ..., \, x_m, \, x_i, \, ..., \, x_j \, )\end{eq*}
\end{prop}
\begin{proof}
To begin, choose an arbitrary element $h$ of $G(2)$ whose underlying permutation is $\pi(h) = (1 2)$. Since the map $\pi$ is surjective, such an $h$ always exists. Now consider the following manipulations of action morphisms:
\begin{eq*}\begin{array}{rll}
		\alpha(h; \mathrm{id}_y, \mathrm{id}_z) & = & \alpha(h; \mathrm{id}_y \otimes \mathrm{id}_I, \mathrm{id}_z) \\
		& = & \alpha( \, h \, ; \, \alpha(e_2; \mathrm{id}_y, \mathrm{id}_I), \, \mathrm{id}_z \,) \\
		& = & \alpha( \, \mu(h; e_2, e_1) \, ; \, \mathrm{id}_y , \, \mathrm{id}_I, \, \mathrm{id}_z \, ) \\
		&& \\
		\alpha(h; \mathrm{id}_y, \mathrm{id}_z) & = & \alpha(h; \mathrm{id}_y, \mathrm{id}_z) \otimes \mathrm{id}_I \\
		& = & \alpha( \, e_2 \, ; \, \alpha(h; \mathrm{id}_y, \mathrm{id}_z), \, \mathrm{id}_I \, ) \\
		& = & \alpha( \, \mu(e_2; h, e_1) \, ; \, \mathrm{id}_y, \, \mathrm{id}_z \, \mathrm{id}_I \, )
		\end{array}
\end{eq*}
Since these two maps are the same, we can compose one with the inverse of the other to get the identity, $\mathrm{id}_{y \otimes z}$. However, the maps both have target sequence $(z, y, I)$, so this composite can be rephrased as an action morphism:
\begin{eq*}\begin{array}{rll}
		\mathrm{id}_{y \otimes z} & = & \alpha(h; \mathrm{id}_y, \mathrm{id}_z)^{-1} \circ \alpha(h; \mathrm{id}_y, \mathrm{id}_z) \\
		& = & \alpha\big( \, \mu(e_2; h, e_1) \, ; \, \mathrm{id}_y, \, \mathrm{id}_z, \, \mathrm{id}_I \, \big)^{-1} \circ \alpha\big( \, \mu(h; e_2, e_1) \, ; \, \mathrm{id}_y , \, \mathrm{id}_I, \, \mathrm{id}_z \, \big) \\
		& = & \alpha\big( \, \mu(e_2; h, e_1)^{-1} \, ; \, \mathrm{id}_z, \, \mathrm{id}_y, \, \mathrm{id}_I \, \big) \circ \alpha\big( \, \mu(h; e_2, e_1) \, ; \, \mathrm{id}_y , \, \mathrm{id}_I, \, \mathrm{id}_z \, \big) \\
		& = & \alpha\big( \, \mu(e_2; h^{-1}, e_1)\mu(h; e_2, e_1) \, ; \, \mathrm{id}_y, \, \mathrm{id}_I, \, \mathrm{id}_z \, \big) \\
		\end{array}
\end{eq*}
This action map has target sequence $(y, z, I)$, so to arrive at the result we just need to use the substitutions $y = x_1 \otimes ... \otimes x_{i-1}$, $I = x_i \otimes ... \otimes x_j$, $z = x_{j+1} \otimes ... \otimes x_m$ and then expand:
\begin{eq*}\begin{array}{rl}
		& \mathrm{id}_{x_1 \otimes ... \otimes x_m} \\
		= & \alpha\big( \, \mu(e_2; h^{-1}, e_1)\mu(h; e_2, e_1) \, ; \, \mathrm{id}_{x_1 \otimes ... \otimes x_{i-1}} , \, \mathrm{id}_{x_i \otimes ... \otimes x_j}, \, \mathrm{id}_{ x_{j+1} \otimes ... \otimes x_m} \, \big) \\ 
		= & \alpha\Big( \, \mu\big( \, \mu(e_2; h^{-1}, e_1)\mu(h; e_2, e_1) \, ; \, e_{i-1}, \, e_{j-i+1}, \, e_{m-j} \big) \, ; \, \mathrm{id}_{x_1}, \, ..., \, \mathrm{id}_{x_m} \, \Big)

		\end{array}
\end{eq*}
Therefore we see that choosing $g = \mu\big( \, \mu(e_2; h^{-1}, e_1)\mu(h; e_2, e_1) \, ; \, e_{i-1}, \, e_{j-i+1}, \, e_{m-j} \big)$ gives the required action map.
\end{proof}

With this result under our belts, we are finally ready to describe the morphisms of $L\mathbb{G}_n$ as action maps.

\begin{prop}\label{allmapsaction} Let $Z$ be an initial object in $(\mathbb{G}_n \downarrow \mathrm{inv})$. Then every morphism in $Z$ can be written as $\alpha(g; \mathrm{id}_{w_1}, ..., \mathrm{id}_{w_m})$, for some $g \in G(m)$ and $w_i \in \mathbb{Z}^{*n}$, not necessarily uniquely. 
\end{prop}
\begin{proof} Define $Z'$ to be the wide sub-$\mathrm{E}G$-algebra of $Z$ containing only morphisms of the form $\alpha(g; \mathrm{id}_{w_1}, ..., \mathrm{id}_{w_m})$. We wish to show that this is also initial, but first we need to check that it is even a well-defined $\mathrm{E}G$-algebra. The $\mathrm{E}G$-action is simple: for any maps $\alpha(h_1; \mathrm{id}_{w_{1,1}}, ..., \mathrm{id}_{w_{1,m_1}})$, ..., $\alpha(h_k; \mathrm{id}_{w_{k,1}}, ..., \mathrm{id}_{w_{k,m_k}})$, the action of $g \in G(k)$ on them is
\begin{eq*}\begin{array}{ll}
		& \alpha \big( \, g \, ; \,  \alpha(h_1; \mathrm{id}_{w_{1,1}}, ..., \mathrm{id}_{w_{1,m_1}}), \, ..., \, \alpha(h_k; \mathrm{id}_{w_{k,1}}, ..., \mathrm{id}_{w_{k,m_k}}) \, \big) \\
		= & \alpha \big( \, \mu(g; h_1, ..., h_k) \, ; \, \mathrm{id}_{w_{1,1}}, \, ..., \, \mathrm{id}_{w_{k,m_k}} \, \big)
		\end{array}
\end{eq*}
which is in the correct form.

Composition is more subtle. Let $\alpha(g; \mathrm{id}_{w_1}, ..., \mathrm{id}_{w_m})$ and $\alpha(g'; \mathrm{id}_{w_1'}, ..., \mathrm{id}_{w_{m'}'})$ be two composable morphisms. Since they are composable, we know that the source of one must be equal to the target of the other. However, we wish to write their composite an action map itself, and we only know how to do this if they share a source and target sequence. Thus we seek a way to rewrite our two maps as action morphisms between different sequences but without changing their value. We begin by expanding the maps using minimal generator decompositions:
\begin{eq*}\begin{array}{rll}
		\alpha(g; \mathrm{id}_{w_1}, ..., \mathrm{id}_{w_m}) & = & \alpha( \, g \, ; \, \mathrm{id}_{d(w_1)_1 \otimes ... \otimes d(w_1)_{k_1}}, \, ..., \, \mathrm{id}_{d(w_m)_1 \otimes ... \otimes d(w_m)_{k_m}}) \\
		& = & \alpha( \, \mu(g; e_{k_1}, ..., e_{k_m}) \, ; \, \mathrm{id}_{d(w_1)_1} \, ..., \, \mathrm{id}_{d(w_m)_{k_m}}) \\
		& & \\
		\alpha(g'; \mathrm{id}_{w_1'}, ..., \mathrm{id}_{w_{m'}'}) & = & \alpha( \, \mu(g; e_{k_1'}, ..., e_{k_{m'}'}) \, ; \, \mathrm{id}_{d(w_1')_1} \, ..., \, \mathrm{id}_{d(w_{m'}')_{k_{m'}'}} )
		\end{array}
\end{eq*}
Now, the target sequence of this first map and the source sequence of the second may contain different entries and may even be of different lengths. However, since our two maps are composable, we do know that 
\begin{eq*} w'_1 \otimes ... \otimes w'_m = w_{\pi(g)^{-1}(1)} \otimes ... \otimes w_{\pi(g)^{-1}(m')} \end{eq*}
and hence
\begin{eq*} d(w'_1) \otimes ... \otimes d(w'_m) = d(w_{\pi(g)^{-1}(1)}) \otimes ... \otimes d(w_{\pi(g)^{-1}(m')})\end{eq*}
The fact that everything in these tensorings is a generator means that the sequences must differ from each other only by the presence of some contiguous subsequences which tensor to give 0, so that those parts cancel out and the products are equal. But \cref{zerosubseq} gives us a way to move contiguous tensor 0 subsequences to the end of a source or target sequence without changing the value of the map --- by composing with the identity written as a certain action morphism --- and we can use this to solve our problem. In particular, let $(c_1, ..., c_j)$ be the concatenation of the contiguous tensor 0 subequences that appear in the target sequence $d(w_{\pi(g)^{-1}(i)})$, let $(c_1', ..., c_{j'}')$ be the concatenation of contiguous tensor 0 subequences appearing in target sequence $d(w'_i)$, and let $(y_1, ...., y_{m-j})$ be the sequence that remains after removing either the $c_i$ or the $c_i'$ from their respective sequences. Applying \cref{zerosubseq} to $\alpha( \, \mu(g; e_{k_1}, ..., e_{k_m}) \, ; \, \mathrm{id}_{d(w_1)_1} \, ..., \, \mathrm{id}_{d(w_m)_{k_m}})$ lets us express that map as an action morphism on identities with target sequence $(y_1, ..., y_{m-j}, c_1, ..., c_j)$, and then tensoring on the right by $\mathrm{id}_0 = \mathrm{id}_{c_1' \otimes ... \otimes c_{j'}}$ lets us re-express the same map again in the right form but with target sequence $(y_1, ..., y_{m-j}, c_1, ..., c_j, c_1', ..., c_{j'}')$. Similarly, tensoring $\alpha( \, \mu(g; e_{k_1'}, ..., e_{k_{m'}'}) \, ; \, \mathrm{id}_{d(w_1')_1} \, ..., \, \mathrm{id}_{d(w_{m'}')_{k_{m'}'}} )$ by $\mathrm{id}_0 = \mathrm{id}_{c_1 \otimes ... \otimes c_j}$ and then applying \cref{zerosubseq} lets us express this map in a form with source sequence $(y_1, ..., y_{m-j}, c_1, ..., c_j, c_1', ..., c_{j'}')$. 

It follows that any composable action morphisms $\alpha(g; \mathrm{id}_{w_1}, ..., \mathrm{id}_{w_m})$, $\alpha(g'; \mathrm{id}_{w_1'}, ..., \mathrm{id}_{w_{m'}'})$ can be rewritten as action morphisms with a shared source and target sequence, and thus their composite can be expressed as a single action morphism of the right form. Therefore, composition is well-defined in $Z'$, and hence $Z'$ is a well-defined $\mathrm{E}G$-algebra.

By \cref{Gndef}, every morphism in $\mathbb{G}_n$ can be written uniquely as $[g ; \mathrm{id}_{x_1},...,\mathrm{id}_{x_m}] = \alpha(g;  \mathrm{id}_{x_1},...,\mathrm{id}_{x_m})$, for some $g \in G(m)$ and $x_1, ..., x_m \in \{ z_1, ..., z_n \}$ generators of $\mathbb{N}^{*n}$. Using this, we can define a map $\phi' : \mathbb{G}_n \to Z'$ which acts as $\phi$ does on objects and on morphisms by
\begin{eq*} \phi(\alpha(g ; \mathrm{id}_{x_1},...,\mathrm{id}_{x_m})) = \alpha(g ; \mathrm{id}_{\phi(x_1)},...,\mathrm{id}_{\phi(x_m)}) \end{eq*}
Now, since $Z$ is initial in $(\mathbb{G}_n \downarrow \mathrm{inv})$ each object $\psi : \mathbb{G}_n \to H$ has a corresponding map $u : Z \to H$ such that $u \phi = \psi$. We can use this to define a new map $u' : Z' \to H$ with $u' \phi' = \psi$ by simply letting $u'$ be the same as $u$ on objects and setting
\begin{eq*} u'(\alpha(g; \mathrm{id}_{w_1}, ..., \mathrm{id}_{w_m})) = \alpha(g; \mathrm{id}_{u'(w_1)}, ..., \mathrm{id}_{u'(w_m)}) \end{eq*}
However, this condition is a necessary part of $u'$ being a map of $\mathrm{E}G$-algebras, so we really had no choice about what to do with the morphisms. Thus $u'$ is the unique map such that $u' \phi' = \psi$, and hence $Z'$ is an initial object in $(\mathbb{G}_n \downarrow \mathrm{inv})$. But $Z'$ was subalgebra of $Z$, also an initial object, and this is only possible if in fact $Z' = Z$.
\end{proof}

Before moving on, notice that this result lets us immediately classify the connected components of $L\mathbb{G}_n$:

\begin{prop}\label{concomp} The connected components of $\mathbb{G}_n$ are $\mathbb{N}^n$, with the assignment $[ \,\, ] : \mathbb{N}^{*n} \to \mathbb{N}^n$ of objects to their component being the quotient map of abelianisation. Also, if $Z$ is an initial object in $(\mathbb{G}_n \downarrow \mathrm{inv})$ then the connected components of $Z$ are $\mathbb{Z}^n$, with its assignment of objects to components also given by abelianisation, and with the restriction of $\phi$ to components $\phi_\pi : \mathbb{N}^n \to \mathbb{Z}^n$ being the obvious inclusion. 
\end{prop}
\begin{proof}All morphisms in $\mathbb{G}_n$ are of the form $\alpha(g; \mathrm{id}_{w_1}, ..., \mathrm{id}_{w_m})$ for some $g \in G(m)$ and $w_i \in \mathbb{N}^{*n}$. Since these have source $w_1 \otimes ... \otimes w_m$ and target $w_{\pi(g)^{-1}(1)} \otimes ... \otimes w_{\pi(g)^{-1}(m)}$, we see that two objects can be in the same connected component only if they can expanded as a tensor product in ways that are permutations of one another. Moreover, for any two objects where this is true --- say source $w = w_1 \otimes ... \otimes w_m$ and target $w' = w_{\sigma^{-1}(1)} \otimes ... \otimes w_{\sigma^{-1}(m)}$ --- we can always find a map $\alpha(g; \mathrm{id}_{w_1}, ..., \mathrm{id}_{w_m})$  between them by choosing any $g$ with $\pi(g) = \sigma$, which we can do because $\pi$ is surjective. So two objects of $\mathbb{G}_n$ are in the same connected component if and only if their expansions are permutations of each others. Therefore, the canonical map $[ \,\, ] : \mathrm{Ob}(\mathbb{G}_n) \to \pi_0(\mathbb{G}_n)$ sending each object to its connected component is just the map which forgets about these permutations, making the free product on $\mathbb{N}^{*n}$ commutative. That is, it is the quotient map for the abelianisation $q : \mathbb{N}^{*n} \to (\mathbb{N}^{*n})^{ab}$, and so $\pi_0(\mathbb{G}_n) = \mathbb{N}^n$. 

Since all morphisms in $Z$ can also be written as $\alpha(g; \mathrm{id}_{w_1}, ..., \mathrm{id}_{w_m})$, the same proof works there too, giving $\pi_0(Z) = \mathbb{Z}^n$ and $[ \,\, ]_Z = q : \mathbb{Z}^{*n} \to \mathbb{Z}^n$. Also, by \cref{Zobj}, $\phi$ acts as an inclusion on objects, so we have the following commutative square:
\begin{eq*} \xymatrix{
\mathbb{N}^{*n} \ar[r]^{q} \ar[d]_{i} & \mathbb{N}^{n} \ar[d]^{\phi_\pi} \\
\mathbb{Z}^{*n} \ar[r]^{q} & \mathbb{Z}^{n} }.
\end{eq*}
Thus $\phi_\pi(q(x)) = q(x)$, and because $q$ is surjective this means that $\phi_\pi$ is an inclusion.
\end{proof}

\subsection{Refinement of $(\mathbb{G}_n \downarrow \mathrm{inv})$}

In the previous section, we learnt some important new features of the initial objects of $(\mathbb{G}_n \downarrow \mathrm{inv})$. Using this information, we can carefully make changes to the category we are finding initial objects in, hopefully in a way which will make the search for $L\mathbb{G}_n$ easier.

\begin{defn} Let $C(\mathbb{G}_n)$ denote a subcategory of the comma category $(\mathbb{G}_n \downarrow \mathrm{inv})$, defined in the following way. The objects of $C(\mathbb{G}_n)$ are those $\psi: \mathbb{G}_n \to X$ for which
\begin{itemize}
\item $X$ has $\mathbb{Z}^{\ast n}$ as its monoid of objects
\item the restriction of $\psi$ to objects, $\psi_{\mathrm{ob}}: \mathbb{N}^{\ast n} \to \mathbb{Z}^{\ast n}$, is an injective monoid homomorphism
\item every morphism of $X$ can be written as $\alpha_X(g; \mathrm{id}_{w_1}, ..., \mathrm{id}_{w_m})$, for some $g \in G(m)$ and $w_i \in \mathbb{Z}^{\ast n}$, not necessarily uniquely
\item the restriction of $\psi$ to connected components, $\psi_\pi: \mathbb{N}^n \to \mathbb{Z}^n$, is the abelianisation of $\psi_{\mathrm{ob}}$
\end{itemize}
Similarly, the morphisms of $C(\mathbb{G}_n)$ are those $f: X \to X'$ between its objects for which
\begin{itemize}
\item the restriction of $f$ to objects, $f_{\mathrm{ob}}: \mathbb{Z}^{\ast n} \to \mathbb{Z}^{\ast n}$, is injective
\item the restriction of $f$ to connected components, $f_\pi: \mathbb{Z}^n \to \mathbb{Z}^n$, is the abelianisation of $f_{\mathrm{ob}}$
\end{itemize}
\end{defn}

Note that by the exact same proof as \cref{concomp}, given any $\psi: \mathbb{G}_n \to X$ in $C(\mathbb{G}_n)$ the canonical map $[ \, \, ]: \mathrm{Ob}(X) \to \pi_0(X)$ will always be the quotient map of abelianisation $q: \mathbb{Z}^{\ast n} \to \mathbb{Z}^n$. In essence, the objects of $C(\mathbb{G}_n)$ are exactly those objects of $(\mathbb{G}_n \downarrow \mathrm{inv})$ which possess all of the properties that we proved the initial object $\phi$ must do. Consequently, it is not too surprising that $\phi$ retains its status as an initial object in this new category.

\begin{prop}\label{initialab} Let $\phi : \mathbb{G}_n \to Z$ be an initial object of $(\mathbb{G}_n \downarrow \mathrm{inv})$. Then it is also initial in $C(\mathbb{G}_n)$.
\end{prop}
\begin{proof}
By \cref{Zobj}, the objects of $Z$ are $\mathbb{Z}^{\ast n}$ and the restriction of $\phi$ to objects is the inclusion $\phi_{\mathrm{ob}} : \mathbb{N}^{\ast n} \to \mathbb{Z}^{\ast n}$. By \cref{allmapsaction}, every morphism of $Z$ can be written in the form $\alpha_Z(g; \mathrm{id}, ..., \mathrm{id})$, and by \ref{concomp} the restriction of $\phi$ to connected components is the inclusion $\phi_\pi = (\phi_{\mathrm{ob}})^{\mathrm{ab}} : \mathbb{N}^n \to \mathbb{Z}^n$. Thus, by design, $\phi$ is definitely an object of $C(\mathbb{G}_n)$. 

For any other object $\psi: \mathbb{G}_n \to X$ of $C(\mathbb{G}_n)$, we know that it is also an object of $(\mathbb{G}_n \downarrow \mathrm{inv})$, and so the initiality condition for $\phi$ tells us that there exists a unique $u : Z \to X$ in $(\mathbb{G}_n \downarrow \mathrm{inv})$ such that $\psi = u \phi$. As we saw in the proof of \cref{Zobj}, the fact that $\phi_{\mathrm{ob}}$ and $\psi_{\mathrm{ob}}$ are injective tells us that $u_{\mathrm{ob}}$ is injective when restricted to the free group generated by the image of $\phi$, $\langle \, \mathrm{im}(\phi_{\mathrm{ob}}) \, \rangle = \langle \, \mathbb{N}^{\ast n} \, \rangle = \mathbb{Z}^{\ast n}$. But since this is the entire source of $u_{\mathrm{ob}}$, it follows that $u$ is injective on objects. We also have that
\begin{eq*} u_\pi \phi_\pi \, = \, \psi_\pi \, = \, (\psi_{\mathrm{ob}})^{\mathrm{ab}} \, = \, (u_{\mathrm{ob}})^{\mathrm{ab}} (\phi_{\mathrm{ob}})^{\mathrm{ab}} \, =  \,  (u_{\mathrm{ob}})^{\mathrm{ab}} \phi_\pi \end{eq*}
This means that $u_ \pi$ is equal to $(u_{\mathrm{ob}})^{\mathrm{ab}}$ on the image of $\phi_\pi$. But by the same line of reasoning as before, the fact that $u_ \pi$ is a group homomorphism means that we actually get equality on the whole group $\langle \, \mathrm{im}(\phi_\pi) \, \rangle = langle \, \mathbb{N}^n \, \rangle = \mathbb{Z}^n$, and so this really says that they are equal everywhere. Lastly, $Z$ and $X$ must have all the properties required for them to be the source and target of an morphism in $C(\mathbb{G}_n)$, because they are already the targets of $\phi$ and $\psi$ respectively, and hence $u$ is a valid morphism from $\phi$ to $\psi$ in $C(\mathbb{G}_n)$. Clearly there can be no other such maps, as that would violate the initiality of $\phi$ in $(\mathbb{G}_n \downarrow \mathrm{inv})$, and therefore $\phi$ is also an initial object in $C(\mathbb{G}_n)$.
\end{proof}

Being able to search for the initial object in just $C(\mathbb{G}_n)$ rather than the whole of $(\mathbb{G}_n \downarrow \mathrm{inv})$ will make our task considerably easier. Part of the reason for this is obvious --- by excluding most of the objects and morphisms found in $(\mathbb{G}_n \downarrow \mathrm{inv})$, we've drastically reduced the number of commutative diagrams that will need to be checked in order to find $\phi$. But there is another  benefit to this approach. Choose an object $\psi : \mathbb{G}_n \to X$ of $C(\mathbb{G}_n)$, and consider the underlying monoidal category of $X$, the one with tensor product given by
\begin{eq*}\begin{array}{rll}
		x_{\pi(g)^{-1}(1)} \otimes ... \otimes x_{\pi(g)^{-1}(m)} & = & \alpha(g; x_1, ..., x_m) \\
		f_1 \otimes ... \otimes f_m & = & \alpha(e; f_1, ..., f_m)
		\end{array}
\end{eq*}
for objects $x_i$ and morphisms $f_i$. Notice that the $\mathrm{E}G$-action of $X$ on morphisms can always be split as
\begin{eq*}\begin{array}{rll}
		\alpha(g; f_1, ..., f_m) & = & \alpha(g; \mathrm{id}_{x_1}, ..., \mathrm{id}_{x_m}) \circ \alpha(e; f_1, ..., f_m) \\
		& = & \alpha(g; \mathrm{id}_{x_1}, ..., \mathrm{id}_{x_m}) \circ (f_1 \otimes ... \otimes f_m)
		\end{array}
 \end{eq*}
and so to recover the action of $X$ from its underlying monoidal category, the only additional information we need is the values of the action morphisms $\alpha(g; \mathrm{id}_{x_1}, ..., \mathrm{id}_{x_m})$. However, it turns out that the restrictions we placed on what $\mathrm{E}G$-algebras were in $C(\mathbb{G}_n)$ will end up forcing these $\alpha(g; \mathrm{id}_{x_1}, ..., \mathrm{id}_{x_m})$ to take a particular set of values, which we can recover from the underlying monoidal functor of $\psi$. 

To work toward proving this, we'll begin with a small result about $X$.

\begin{lem} Let $\psi: \mathbb{G}_n \to X$ be an object of $C(\mathbb{G}_n)$. If we denote the generating objects of the coproduct $X+X$ by $z_1, ..., z_n, z'_1, ..., z'_n$, then there is an inclusion $X \hookrightarrow X+X$ given by the assignment $z_i \mapsto z_i \otimes z'_i$, and the cokernel of this inclusion is $X$.
\end{lem}
\begin{proof}
First, we need to describe the coproduct algebra $X+X$. For any two algebras $A$, $B$, the algebra $A+B$ is the one where
\begin{itemize}
\item the object monoid is the coproduct of the object monoids of $A$ and $B$
\item the morphisms are the formal action moprhisms $\alpha_{A+B}(g; f_1, ..., f_m)$, with each $f_i$ being a morphism from $A$ or $B$, subject to the contraint that whenever all of the $f_i$ are from $A$, $\alpha_{A+B}(g; f_1, ..., f_m) = \alpha_A(g; f_1, ..., f_m)$, and likewise for B
\end{itemize}
For $X+X$, it follows that
\begin{eq*} \mathrm{Ob}(X+X) \, = \, \mathrm{Ob}(X)+\mathrm{Ob}(X) \, = \, \mathbb{Z}^{\ast n} \ast \mathbb{Z}^{\ast n} \, = \, \mathbb{Z}^{\ast 2n} \end{eq*}
and since every morphism in $X$ is of the form $\alpha_X(g; \mathrm{id}_{w_1}, ..., \mathrm{id}_{w_m})$ for $w_i \in \mathbb{Z}^{\ast n}$, the morphisms of $X+X$ will all be of the form $\alpha_{X+X}(g; \mathrm{id}_{w_1}, ..., \mathrm{id}_{w_m})$ for $w_i \in \mathbb{Z}^{\ast 2n}$. Now, consider the submonoid of $\mathbb{Z}^{\ast 2n}$ generated by the objects $z_1 \otimes z'_1, ..., z_n \otimes z'_n$ --- as these are all independent of one another, this will just be the free monoid on $n$ objects, $\mathbb{Z}^{\ast n}$. The full subcategory of $X+X$ on this submonoid with then consist of these object along with all possible morphisms $\alpha_{X+X}(g; \mathrm{id}, ..., \mathrm{id})$ between them, but because these can each be expanded in terms of generators, this is really all morphism of the form
\begin{eq*}\begin{array}{rll}
		\alpha_{X+X}(g; \mathrm{id}_{z_{k_1} \otimes z'_{k_1}}, ..., \mathrm{id}_{z_{k_m} \otimes z'_{k_m}}) & = &
		\end{array}
\end{eq*}
\end{proof}

\begin{prop}\label{initialmon} Objects and morphisms of $C(\mathbb{G}_n)$ are uniquely specified by their underlying monoidal functors. That is, two objects $\psi$, $\psi'$ of $C(\mathbb{G}_n)$ have the same underlying monoidal functor if and only if $\psi = \psi'$, and likewise for morphisms $f$, $f'$.
\end{prop}
\begin{proof}
Let $\psi: \mathbb{G}_n \to X$ and $\psi': \mathbb{G}_n \to X'$ be objects of $C(\mathbb{G}_n)$ which the same underlying monoidal functor. As noted before this proposition, the only way that $\psi'$ could differ from $\psi$ is that their source and target algebras might assign different values to actions of the form $\alpha(g; \mathrm{id}, ..., \mathrm{id})$. Since $\psi$ and $\psi'$ have the same source --- $\mathbb{G}_n$ with its canonical $\mathrm{E}G$-algebra structure --- all that remains to be checked are the targets, $X$ and $X'$.

One of the conditions that morphisms of the form $\alpha_X(g; \mathrm{id}, ..., \mathrm{id})$ must satisfy is
\begin{eq*} \psi \big( \, \alpha_{\mathbb{G}_n}(g; \mathrm{id}_{w_1}, ..., \mathrm{id}_{w_m}) \, \big) \, = \, \alpha_{X}(g; \mathrm{id}_{\psi'(w_1)}, ..., \mathrm{id}_{\psi'(w_m)}) \end{eq*}
for any objects $w_i$ in $\mathbb{G}_n$. Notice that if we know the underlying monoidal functor of $\psi$, this equation can be as thought of as defining part of $\alpha_X$ --- if we choose a collection of $w_i$ in $\mathbb{G}_n$ with $\psi'(w_i) = x_i$, then $\alpha_X(g; \mathrm{id}_{x_1}, ..., \mathrm{id}_{x_m})$ must have the value $\psi(\alpha_{\mathbb{G}_n}(g; \mathrm{id}_{w_1}, ..., \mathrm{id}_{w_m}))$. That is, the behaviour of $\alpha_X$ on $\mathrm{im}(\psi)$ is fixed by the underlying monoidal functor of $\psi$, and the same argument works for $\alpha_{X'}$ and $\mathrm{im}(\psi')$ too. But as $\psi$ and $\psi'$ shares the same underlying functor, this means that the actions of $X$ and $X'$ must agree on $\mathrm{im}(\psi)$:
\begin{eq*}\begin{array}{rll}
		\alpha_X(g; \mathrm{id}_{\psi(w_1)}, ..., \mathrm{id}_{\psi(w_m)}) & = & \psi \big( \, \alpha_{\mathbb{G}_n}(g; \mathrm{id}_{w_1}, ..., \mathrm{id}_{w_m}) \, \big) \\
		& = & \psi' \big( \, \alpha_{\mathbb{G}_n}(g; \mathrm{id}_{w_1}, ..., \mathrm{id}_{w_m}) \, \big) \\
		& = & \alpha_{X'}(g; \mathrm{id}_{\psi'(w_1)}, ..., \mathrm{id}_{\psi'(w_m)}) \\
		& = & \alpha_{X'}(g; \mathrm{id}_{\psi(w_1)}, ..., \mathrm{id}_{\psi(w_m)})
		\end{array}
\end{eq*}

In order to understand the behaviour of $\alpha_X$ outside of $\mathrm{im}(\psi)$, consider the generators $z_1, ..., z_n$ of the $\mathrm{Ob}(X) =  \mathbb{Z}^{\ast n}$. There is an obvious isomorphism between the $z_i$ and their inverses $z_i^*$, and we can extend this to a full $\mathrm{E}G$-algebra automorphism,
\begin{eq*} \begin{array}{rrcll}
		\delta & : & X & \to & X \\
		& : & z_i & \mapsto & z_i^* \\
		& : & z_i^* & \mapsto & z_i
		\end{array}
\end{eq*}
\begin{eq*} \delta \big( \, \alpha_X( \, g \, ; \, \mathrm{id}_{w_1}, ..., \mathrm{id}_{w_m} \, ) \, \big) \, := \, \alpha_X( \, g \, ; \, \mathrm{id}_{\delta(w_1)}, ..., \mathrm{id}_{\delta(w_m)} \, ) \end{eq*}
The first part of this definition is enough to determine the value that $\delta$ takes on any object of $X$, as they can be build from the values on generators and the fact that we need $\delta(x \otimes x') = \delta(x) \otimes \delta(x')$ for all objects $w, w'$. Likewise, because every morphism of $X$ can be written in the form $\alpha_X(g; \mathrm{id}_{w_1}, ..., \mathrm{id}_{w_m})$, not necessarily uniquely, the value of $\delta$ on morphisms will be given by the second part of the definition. This of course assumes that the statement is even well-defined, but since the assignment
\begin{eq*} \begin{tikzcd}
\alpha_X( \, g \, ; \, \mathrm{id}_{w_1}, ..., \mathrm{id}_{w_m} \, ) \ar[r, "\delta", mapsto] & \alpha_X( \, g \, ; \, \mathrm{id}_{\delta(w_1)}, ..., \mathrm{id}_{\delta(w_m)} \, )
\end{tikzcd} \end{eq*}
is clearly self-inverse
\begin{eq*} \begin{array}{rll}
		\alpha_X( \, g \, ; \, \mathrm{id}_{\delta(w_1)}, ..., \mathrm{id}_{\delta(w_m)} \, ) & \mapsto & \alpha_X( \, g \, ; \, \mathrm{id}_{\delta \delta(w_1)}, ..., \mathrm{id}_{\delta \delta(w_m)} \, ) \\
		& = & \alpha_X( \, g \, ; \, \mathrm{id}_{w_1}, ..., \mathrm{id}_{w_m} \, )
		\end{array}
\end{eq*}
it must be injective, and hence
\begin{eq*} \begin{array}{rll}
		\alpha_X( \, g \, ; \, \mathrm{id}_{w_1}, ..., \mathrm{id}_{w_m} \, ) & = & \alpha_{X'}( \, g' \, ; \, \mathrm{id}_{w'_1}, ..., \mathrm{id}_{w'_m} \, ) \\
		\implies \quad \delta\big( \, \alpha_X( \, g \, ; \, \mathrm{id}_{\delta(w_1)}, ..., \mathrm{id}_{\delta(w_m)} \, ) \, \big) & = & \delta\big( \, \alpha_{X'}( \, g' \, ; \, \mathrm{id}_{\delta(w'_1)}, ..., \mathrm{id}_{\delta(w'_m)} \, ) \, \big) \\
		\implies \quad \alpha_X( \, g \, ; \, \mathrm{id}_{\delta(w_1)}, ..., \mathrm{id}_{\delta(w_m)} \, ) & = & \, \alpha_{X'}( \, g' \, ; \, \mathrm{id}_{\delta(w'_1)}, ..., \mathrm{id}_{\delta(w'_m)} \, ) 
		\end{array}
\end{eq*}

With the automorphism $\delta$, we can now construct a new object, this time of $(\mathbb{G}_{2n} \downarrow (\_)_{\mathrm{inv}})$, via the composite
\begin{eq*} \begin{tikzcd}
\mathbb{G}_{2n} \ar[r, "\sim"] & \mathbb{G}_n + \mathbb{G}_n \ar[r, "\psi + \psi"] & X + X \ar[r, "\mathrm{id}_X + \delta"] & X + X \ar[r, "i + i"] & X
\end{tikzcd} \end{eq*}
which we will call $\psi_{2n}$. Notice that here we've made use of the fact that left adjoints preserve colimits, so that 
\begin{eq*} \mathbb{G}_{2n} \, := \, F(2n) \, = \, F(n) + F(n) \, = \, \mathbb{G}_n + \mathbb{G}_n \end{eq*}
Now, using the exact same process we just followed, we can also define an automorphism $\delta' : X' \to X'$, and then from it and $\psi'$ constuct another new object $\psi'_{2n}: \mathbb{G}_{2n} \to X'$. Since $X$ and $X'$ have the same underlying category, $\delta$ and $\delta'$ will act the same on objects, and similarly the fact that $\psi$ and $\psi'$ agree on objects means that $\psi_{2n}$ and $\psi'_{2n}$ will do too:
\begin{eq*} \begin{array}{rll}
		\psi_{2n}(z_i) & = &
			\begin{cases}
       				\psi(z_i) & \quad \text{if} \quad 1 \leq i \leq n \\
      				\psi(z_{i-n})^* & \quad \text{if} \quad n+1 \leq i \leq 2n \\
			\end{cases} \\
		& = &
			\begin{cases}
       				\psi'(z_i) & \quad \text{if} \quad 1 \leq i \leq n \\
      				\psi'(z_{i-n})^* & \quad \text{if} \quad n+1 \leq i \leq 2n \\
			\end{cases} \\
		& = & \psi'_{2n}(z_i)
		\end{array}
\end{eq*}
Furthermore, if we unpack the definition of $\psi_{2n}$ we see that the result of applying it to morphisms will depend on the value of expressions of the form $\psi(\alpha_{\mathbb{G}_n}(g; \mathrm{id}, ..., \mathrm{id}))$ and $\delta\psi(\alpha_{\mathbb{G}_n}(g; \mathrm{id}, ..., \mathrm{id}))$. The analagous fact will be true for $\psi'_{2n}$ as well, with the values of $\psi'(\alpha_{\mathbb{G}_n}(g; \mathrm{id}, ..., \mathrm{id}))$ and $\delta'\psi'(\alpha_{\mathbb{G}_n}(g; \mathrm{id}, ..., \mathrm{id}))$ instead. But these are actually the exact same terms --- since $\psi$ and $\psi'$ have the same underlying functor, we know that
\begin{eq*} \psi\big( \, \alpha_{\mathbb{G}_n}( \, g \, ; \, \mathrm{id_{w_1}}, ..., \mathrm{id_{w_m}} \, ) \, \big) \, = \, \psi'\big( \, \alpha_{\mathbb{G}_n}( \, g \, ; \, \mathrm{id_{w_1}}, ..., \mathrm{id_{w_m}} \, ) \, \big) \end{eq*}
and if we recall how we showed earlier that $\alpha_X$ and $\alpha_{X'}$ were equal on the the image of $\psi$, then
\begin{eq*} \begin{array}{rll}
 		\delta \psi \big( \, \alpha_{\mathbb{G}_n}( \, g \, ; \, \mathrm{id_{w_1}}, ..., \mathrm{id_{w_m}} \, ) \, \big) & = & \delta \big( \, \alpha_X( \, g \, ; \, \mathrm{id}_{\psi(w_1)}, ..., \mathrm{id}_{\psi(w_m)} \, ) \, \big) \\
		& = & \alpha_X( \, g \, ; \, \mathrm{id}_{\delta\psi(w_1)}, ..., \mathrm{id}_{\delta\psi(w_m)} \, ) \\
		& = & \alpha_X( \, g \, ; \, \mathrm{id}_{\delta'\psi'(w_1)}, ..., \mathrm{id}_{\delta'\psi'(w_m)} \, ) \\
		\end{array}
\end{eq*}

\begin{eq*} \begin{array}{rll}
		\psi_{2n}\big( \, \alpha_{\mathbb{G}_{2n}}(g; \mathrm{id}_{w_1}, ..., \mathrm{id}_{w_m}) \, \big) & = & \alpha_{\mathbb{G}_{2n}}(g; \mathrm{id}_{w_1}, ..., \mathrm{id}_{w_m})
		\end{array}
\end{eq*}

Now, since we know that $(\mathbb{G}_{2n} \downarrow (\_)_{\mathrm{inv}})$ has an initial object, $\phi_{2n}: \mathbb{G}_{2n} \to Z_{2n}$, there must exist a unique map $u: Z_{2n} \to X'$ such that $\psi'_{2n} = u \phi_{2n}$. Remembering that $\phi_{2n}$ has the properties
\begin{eq*}\phi_{2n}(w) = w, \quad \quad \phi_{2n} \big( \, \alpha_{\mathbb{G}_{2n}}(g; \mathrm{id}_{w_1}, ..., \mathrm{id}_{w_m}) \, \big) \, = \, \alpha_{Z_{2n}}(g; \mathrm{id}_{w_1}, ..., \mathrm{id}_{w_m}) \end{eq*}
we can conclude that $u$ satisfies
\begin{eq*}
u(z_i) \, = \, u\phi_{2n}(z_i) \, = \, \psi'_{2n}(z_i) \, = \,
\begin{cases}
       	\psi'(z_i) & \quad \text{if} \quad 1 \leq i \leq n \\
      	\psi'(z_{i-n})^* & \quad \text{if} \quad n+1 \leq i \leq 2n \\
\end{cases}
\end{eq*}
\begin{eq*} u \big( \, \alpha_{Z_{2n}}(g; \mathrm{id}_{w_1}, ..., \mathrm{id}_{w_m}) \, \big) \, = \, \alpha_{X'}(g; \mathrm{id}_{u(w_1)}, ..., \mathrm{id}_{u(w_m)}) \end{eq*}
Like before, we can use this second condition to fix the way that $\alpha_{X'}$ must act on the image of $u$, using only knowledge of the underlying monoidal functor of $u$. However, this time $\mathrm{im}(u)$ contains every object of $X'$, since the first condition on $u$ implies that
\begin{eq*} \mathrm{im}(u_{\mathrm{ob}})  \, = \, \langle \, \mathrm{im}(\psi'_{\mathrm{ob}}) \, \rangle \, = \, \langle \, \mathbb{N}^{\ast n} \, \rangle \, = \, \mathbb{Z}^{\ast n} \, = \, \mathrm{Ob}(X') \end{eq*}
Thus for any collection of objects $x_i$ in the underlying category of $X'$, we can always find some $w_i$ in $Z_{2n}$ with $u(w_i) = x_i$, meaning that
\begin{eq*}\begin{array}{rll}
		\alpha_{X'}(g; \mathrm{id}_{x_1}, ..., \mathrm{id}_{x_m}) & = & u \big( \, \alpha_{Z_{2n}}(g; \mathrm{id}_{w_1}, ..., \mathrm{id}_{w_m}) \, \big) \\
		& = & \alpha_{X}(g; \mathrm{id}_{x_1}, ..., \mathrm{id}_{x_m}) 
		\end{array}
\end{eq*}
It follows that $X$ and $X'$ are really the same $\mathrm{E}G$-algebra, and $\psi$ and $\psi'$ the same $\mathrm{E}G$-algebra map, and therefore the $\mathrm{E}G$-structure of an object of $(\mathbb{G}_n \downarrow \mathrm{inv})$ is completely determined by its underlying monoidal functor.

As for the morphisms of $(\mathbb{G}_n \downarrow \mathrm{inv})$, let $f: X \to Y$ be a morphism from $\psi: \mathbb{G}_n \to X$ to $\chi: \mathbb{G}_n \to Y$. Given the underlying monoidal functors of $\psi$ and $\chi$, the actions on $X$ and $Y$ are fixed, and this together with the underlying monoidal functor of $f$ is enough to determine $F$ as a map of $\mathrm{E}G$-algebras. Thus, the process of taking underlying monoidal functors of object and morphisms of $(\mathbb{G}_n \downarrow \mathrm{inv})$ is one-to-one, and so $(\mathbb{G}_n \downarrow \mathrm{inv}) \cong (\mathbb{G}_n \downarrow \mathrm{inv})^{\mathrm{mn}}$.
\end{proof}

This result has two very important consequences. The first is that the underlying monoidal functors of the initial objects of $C(\mathbb{G}_n)$ must be the initial objects of $C(\mathbb{G}_n)^{\mathrm{mn}}$. This will allow us to further use $C(\mathbb{G}_n)^{\mathrm{mn}}$ as a refinement of $(\mathbb{G}_n \downarrow \mathrm{inv})$. The second result is a more practical consideration.

\begin{cor}\label{initialact} It is possible to recontruct the entirety of $C(\mathbb{G}_n)$ from just the objects and morphisms of $C(\mathbb{G}_n)^{\mathrm{mn}}$ and $C_{\mathrm{gp}}(\mathbb{G}_{2n})^{\mathrm{mn}}$.
\end{cor}
\begin{proof}
When we were defining the value of $\alpha_{X'}(g; \mathrm{id}_{x_1}, ..., \mathrm{id}_{x_m})$ at the end of the previous proof, we needed to use the underlying monoidal functor of $u: Z_{2n} \to X$, which was defined in terms of the underlying monoidal functor of $\psi$, an object of $C(\mathbb{G}_n)^{\mathrm{mn}}$. But we also used the value of $\alpha_{Z_{2n}}(g; \mathrm{id}_{w_1}, ..., \mathrm{id}_{w_m})$, something not included in the data of any $C_{\mathrm{gp}}(\, \_ \,)^{\mathrm{mn}}$. However, notice that the way we defined $\psi'_2$ implies that
\begin{eq*} \mathrm{Ob}(X) = \langle \, \mathrm{im}(\psi_{\mathrm{ob}}) \, \rangle = \mathrm{im}\big( \, (\psi'_2)_{\mathrm{ob}} \, \big) = \mathrm{im}(u_{\mathrm{ob}} \phi_{\mathrm{ob}}) = u \big( \, \mathrm{im}(\phi_{\mathrm{ob}}) \, \big) \end{eq*}
Hence, when we had to pick some $w_i$ such that $u(w_i) = x_i$, we were always free to choose $w_i$ exclusively from $\mathrm{im}(\phi)$. 

This is important because we saw earlier in the proof that given any object $\psi: \mathbb{G}_{n} \to X$ of $C(\mathbb{G}_n)$, we can define the values of the action $\alpha_X$ on $\mathrm{im}(\psi)$ so long as we know the underlying monoidal functor of $\psi$. Applying this logic to $\phi: \mathbb{G}_{2n} \to Z_{2n}$, it follows that we can get all of the values of $\alpha_{Z_{2n}}(g; \mathrm{id}_{w_1}, ..., \mathrm{id}_{w_m})$ that we need just from knowing the underlying functor of $\phi$, which is an object of $C_{\mathrm{gp}}(\mathbb{G}_{2n})^{\mathrm{mn}}$. Therefore, we can fully reconstruct an object $\psi: \mathbb{G}_n \to X$ of $C(\mathbb{G}_n)$ from its underlying monoidal functor in $C(\mathbb{G}_n)^{\mathrm{mn}}$ and some extra data from $C_{\mathrm{gp}}(\mathbb{G}_{2n})^{\mathrm{mn}}$. Furthermore, once we've recovered all of the actions this way, we can reconstruct all of the morphism of $C(\mathbb{G}_n)$ from their underlying functors in $C(\mathbb{G}_n)^{\mathrm{mn}}$ too.
\end{proof}

\subsection{Initial algebras as monoidal groupoids}

Recall that objects $X$ in $C(\mathbb{G}_n)^{\mathrm{mn}}$ are always expressible as $Y_{\mathrm{inv}}$ for some monoidal category $Y$. From this fact, along with the definition of $\mathbb{G}_n$, it is clear that all of the structures we will be working with are not just categories, but groupoids. This is very convenient, as groupoids are often much simpler than general categories, and we can exploit this simplicity to help find initial algebras more easily. In particular, we will use the fact that for any connected component of a groupoid, all of the homsets between its objects are isomorphic. This will allow us to describe any groupoid $X$ by splitting it into two smaller pieces --- one encoding information about the objects of $X$, and the other about the morphisms of the components $X$. First, we'll look at the part which details the morphisms.

\begin{defn} Let $X$ be a monoidal groupoid. A \emph{skeleton} of $X$ is a full subcategory of $X$ which contains exactly one object from each of the connected components of $X$. \end{defn}

It is well known that the canonical inclusion of a skeleton, $i: \mathrm{sk}(X) \to X$, is part of an equivalence between the two. Indeed, the skeleton of $X$ is the smallest subcategory which is still equivalent to $X$. We shall denote weak inverse of $i$ by $R: X \to \mathrm{sk}(X)$. This inverse is actually always strict in one direction --- $R \circ i = \mathrm{id}_{\mathrm{sk}(X)}$. In the other direction we just have a natural isomorphism, which we will call $\rho: i \circ R \Rightarrow \mathrm{id}_X$. 

We will think about the objects of a particular skeleton of $X$ as being a chosen set of `representatives' for each of the connected components of $X$. For this reason we will normally be loose with notation and just write $s \in \mathrm{Ob}(X)$ when talking about $i(s) \in \mathrm{Ob}(X)$ for $s \in \mathrm{sk}(X)$. Under this interpretation, the functor $R$ sends each object $x$ to its isomorphism class' representative $R(x)$, and $\rho_x : R(x) \to x$ is an isomorphism which witnesses that they are indeed in the same class. Notice also that the naturality of $\rho$ serves to define the action of $R$ on morphisms:
\begin{eq*} R( \, f: x \to y \,) \, = \, \rho_y^{-1} \circ f \circ \rho_x \end{eq*}

Since $X$ is a (strict) monoidal category, the equivalence $(i, R, \rho, \mathrm{id})$ naturally induces on $\mathrm{sk}(X)$ the structure of a weak monoidal category:

\begin{defn}\label{boxtimes} Let $\boxtimes$ be the weak monoidal product defined on $\mathrm{sk}(X)$ by
\begin{eq*} \xymatrix{
\mathrm{sk}(X) \times \mathrm{sk}(X) \ar[d]_{i \times i} \ar[rr]^{\boxtimes} & & \mathrm{sk}(X) \\
X \times X \ar[rr]^{\otimes} & & X \ar[u]_{R} }.
\end{eq*}
This new product has unit object $R(I)$ and coherence data
\begin{eq*} \begin{tikzcd}
(s \boxtimes s') \boxtimes s'' \ar[d, equal] \ar[rr, "a^{\mathrm{sk}(X)}_{s,s',s''}"] & & s \boxtimes (s' \boxtimes s') \\
R\big( \, R(s \otimes s') \otimes s'' \, \big) \ar[d, "R( \rho_{s \otimes s'} \otimes \mathrm{id}_{s''})"'] & & R\big( \, s \otimes R(s' \otimes s'') \, \big) \ar[u, equal] \\
R\big( \, (s \otimes s') \otimes s'' \, \big) \ar[rr, "R(a^X_{s,s',s''})", equal] & & R\big( \, s \otimes (s' \otimes s'') \, \big) \ar[u, " R(\mathrm{id}_{s} \otimes \rho_{s' \otimes s''})^{-1}"']
\end{tikzcd} \end{eq*}
\begin{eq*} \begin{tikzcd}
R(I) \boxtimes s \ar[d, equal] \ar[rr, "l^{\mathrm{sk}(X)}_s"] & & s & & s \boxtimes R(I) \ar[d, equal] \ar[rr, "r^{\mathrm{sk}(X)}_s"] & & s \\
R\big( \, R(I) \otimes s \, ) \ar[d, "R(\rho_I \otimes \mathrm{id}_s)"'] & & & & R\big( \, s \otimes R(I) \, ) \ar[d, "R(\mathrm{id}_s \otimes \rho_I)"'] & & \\
R(I \otimes s) \ar[rr, "R(l^X_s)", equal] & & R(s) \ar[uu, equal] & & R(s \otimes I) \ar[rr, "R(r^X_s)", equal] & & R(s) \ar[uu, equal] 
\end{tikzcd} \end{eq*}
\end{defn}

Moreover, as this weak monoidal structure $\boxtimes$ was induced on $\mathrm{sk}(X)$ by the equivalence $(i, R, \rho, \mathrm{id})$, it follows immediately that $(i, R, \rho, \mathrm{id})$ is a weak monoidal equivalence with respect to $\boxtimes$. Explicitly:

\begin{cor}\label{iRdata} There exists coherence data
\begin{eq*} \begin{array}{rclrrcl}
		\mu^R_{x, x'} & = & R( \rho_x \otimes \rho_{x'} ) & \quad & \eta^R & = & \mathrm{id}_{R(I)} \\
		\mu^i_{s, s'} & = & \rho^{-1}_{s \otimes s'} & & \eta^i & = & \rho^{-1}_I \\
		\end{array}
\end{eq*}
making $R, i$ into weak monoidal functors and $\rho$ into a weakly monoidal natural transformation. 
\end{cor}

It is important to note that while the product $\boxtimes$ is strictly associative and unital
\begin{eq*} \begin{array}{rcl}
		(s \boxtimes s') \boxtimes s'' & = & R\big( \, R(s \otimes s') \otimes s'' \, \big)  \\
		& = & R(s \otimes s' \otimes s'') \\
		& = & R\big(\, s \otimes R(s' \otimes s'') \, \big) \\
		& = & s \boxtimes (s'' \boxtimes s'') 
		\end{array}
\end{eq*}
\begin{eq*} \begin{array}{rcllrcl}
		R(I) \boxtimes s & = & R\big( \, R(I) \otimes s \,) & \quad & s \boxtimes R(I) & = & R\big( \, s \otimes R(I) \, ) \\
		& = & R(I \otimes s) & & & = & R(s \otimes I) \\
		& = & R(s) & & & = & R(s) \\
		& = & s & & & = & s \\
		\end{array}
\end{eq*}
nevertheless the associator and unitors of $\mathrm{sk}(X)$ are in general \emph{not} identities.

Finally, because $\mathrm{sk}(X)$ contains one object for each of the elements of $\pi_0(X)$ there is an obvious functor, isomorphic on objects, between the two:
\begin{eq*} \begin{array}{rrrll}
		[ \, \_ \, ] & : & \mathrm{sk}(X) & \to & \pi_0(X) \\
		& : & s & \mapsto & [s] \\
		& : & f: s \to s & \mapsto & id_{[s]}
		\end{array}
\end{eq*}
This functor is clearly weak monoidal; the only morphisms we have to pick the coherence maps $\mu^{[ \, \_ \, ]}_{s, s'}$ and $\eta^{[ \, \_ \, ]}$ from are identities, which is well-defined because
\begin{eq*} [s] \otimes [s'] \, = \, [s' \otimes s'] \, = \, [ \, R(s' \otimes s') \, ] \, = \, [ s \boxtimes s' ], \quad \quad [I] \, = \, [ \, R(I) \, ] \end{eq*}
and since the coherence data $a, l, r$ of $\mathrm{sk}(X)$ will all be mapped onto identities too the conditions for a weak monoidal functor will be satisfied trivially.

Next, we will define the subcategory which will describe the objects of our monoidal groupoids.

\begin{defn} Let $X$ be a monoidal groupoid, and consider a new category which has the same objects as $X$, and has a unique morphism between two objects if and only if there is at least one morphism between them in $X$. We call this $\mathrm{po}(X)$, since it is always a posetal category. \end{defn}

Because $X$ is a groupoid $\mathrm{po}(X)$ must be one too, and it also inherits a strict monoidal product directly from $X$. Furthermore, $\mathrm{po}(X)$ has exactly the same objects and connected components that $X$ does, and so we can construct two obvious strict monoidal functors,
\begin{eq*} \begin{array}{rrrll}
		P & : & X & \to & \mathrm{po}(X) \\
		& : & x & \mapsto & x \\
		& : & f: x \to y & \mapsto & x \to y
		\end{array}
\end{eq*}
and
\begin{eq*} \begin{array}{rrrll}
		[ \, \_ \, ] & : & \mathrm{po}(X) & \to & \pi_0(X) \\
		& : & x & \mapsto & [x] \\
		& : & x \to y & \mapsto & id_{[x]}
		\end{array}
\end{eq*}
Here we denote the morphisms of $\mathrm{po}(X)$ by giving their unique source and target, rather than assigning them any particular names. 

Putting this together with what we had for $\mathrm{sk}(X)$, we can now express precisely how we are going to split our monoidal groupoids:

\begin{prop}\label{pullback} Let $X$ be a monoidal groupoid. Then for any choice of skeleton $\mathrm{sk}(X)$, the commutative diagram
\begin{eq*} \begin{tikzcd}
& X \ar[dl, "R"'] \ar[dr, "P"] & \\
\mathrm{sk}(X) \ar[dr, "\lbrack \, \_ \, \rbrack"'] & & \mathrm{po}(X) \ar[dl, "\lbrack \, \_ \, \rbrack"] \\
& \pi_0(X) &
\end{tikzcd} \end{eq*}
is a pullback square in the category of weak monoidal categories. That is, 
\begin{eq*} X \quad \cong \quad \mathrm{sk}(X) \times_{\pi_0(X)} \mathrm{po}(X) \end{eq*}
\end{prop}
\begin{proof}
First of all, we need to check that the above diagram actually commutes. Notice that $[ \, P( \, \_ \, ) \, ]$ and $[ \, R( \, \_ \, ) \, ]$ are both strict monoidal functors, since their coherence data lives in the category $\pi_0(X)$, which contains only identity morphisms. This means we only need to check that their underlying functors are equal, which is simple enough:
\begin{eq*} [ \, R(x) \, ] \, = \, [x] \, = \, [ \, P(x) \, ], \quad \quad \quad [ \, R(f: x \to y) \, ] \, = \, id_{[x]} \, = \, [ \, P(f) \, ]\end{eq*}
In other words, both  $[ \, P( \, \_ \, ) \, ]$ and $[ \, R( \, \_ \, ) \, ]$ are the obvious map $[ \, \_ \, ]: X \to \pi_0(X)$ sending objects to their connected components. 

Now, assume that we are given a pair of weak monoidal functors $S: Y \to \mathrm{sk}(X)$ and $Q: Y \to \mathrm{po}(X)$ which also form an appropriate commutative square --- that is, $[ \, S( \, \_ \, ) \, ] = [ \, Q( \, \_ \, ) \, ]$. We wish to construct a unique map $U: Y \to X$ which factors $S$ and $Q$ through $R$ and $P$ respectively:
\begin{eq*} \begin{tikzcd}
& Y \ar[dd, dashrightarrow, "U"] \ar[dddl, bend right, "S"'] \ar[dddr, bend left, "Q"] & \\
&& \\
& X \ar[dl, "R"'] \ar[dr, "P"] & \\
\mathrm{sk}(X) \ar[dr, "\lbrack \, \_ \, \rbrack"'] & & \mathrm{po}(X) \ar[dl, "\lbrack \, \_ \, \rbrack"] \\
& \pi_0(X) &
\end{tikzcd} \end{eq*}
Let $y$ be an object of $Y$. In order for the right-hand triangle in the above diagram to commute, we need that $Q(y) = PU(y) = U(y)$. If we take this to be the definition of $U$ on objects, then we see that the left-hand triangle will also commute on objects too, since
\begin{eq*} \begin{array}{rrccl}
		& [ \, S(y) \, ] & = & [ \, Q(y) \, ] & \\
		& & = & [ \, U(y) \, ] & \\
		\implies & S(y) & \cong & U(y), & S(y) \in \mathrm{sk}(X) \\
		\implies & S(y) & = & RU(y)
		\end{array}
\end{eq*}
Similarly, let $f: y \to y'$ be a morphism from $Y$. The right-hand triangle of the diagram commutes regardless of what $U$ does to morphisms, since
\begin{eq*} \begin{array}{rll}
		PU( \, f : y \to y' \, ) & = & P\big( \, U(f) : U(y) \to U(y') \, \big) \\
		& = & U(y) \to U(y') \\
		& = & Q(y) \to Q(y') \\
		& = & Q( \, f : y \to y' \, )
		\end{array}
\end{eq*}
In order for the left-hand triangle to commute as well, we need $S(f) = RU(f) = \rho_{y'}^{-1} U(f)\rho_y$, and we can again take this to be definitional, so that $U(f) := \rho_{y'} S(f) \rho_y^{-1}$.
 
Next, we must determine the coherence data for $U$. For the right-hand triangle we need
\begin{eq*} \eta^Q = P(\eta^U) \circ \eta^P, \quad \quad \quad \mu^Q = P(\mu^U) \circ \mu^P, \end{eq*}
but since morphisms in $\mathrm{po}(X)$ are uniquely specified by their source and target, these equalities necessarily hold. Similarly, the left-hand triangle gives
\begin{eq*} \begin{array}{rrlllrll}
		& \eta^S & = & R(\eta^U) \circ \eta^R & & \mu^S & = & R(\mu^U) \circ \mu^R \\
		& & = & \rho_{U(I)}^{-1} \, \eta^U \, \rho_{I} \, \eta^R & & & = & \rho_{U(\_ \otimes \_)}^{-1} \, \mu^U \, \rho_{U(\_) \otimes U(\_)} \, \mu^R \\
		&&&&&&& \\
		\implies & \eta^U & = & \rho_{U(I)} \, \eta^S \, (\eta^R)^{-1} \, \rho_{I}^{-1} & & \mu^U & = & \rho_{U(\_ \otimes \_)} \, \mu^S \, (\mu^R)^{-1} \, \rho_{U(\_) \otimes U(\_)}^{-1} \\
		& & = & \rho_{U(I)} \, \eta^S \, (\eta^R)^{-1} \, \rho_{I}^{-1} & & \mu^U & = & \rho_{U(\_ \otimes \_)} \, \mu^S \, (\mu^R)^{-1} \, \rho_{U(\_) \otimes U(\_)}^{-1} \\
		\end{array}
\end{eq*}

Since this definition of $U$ has been forced on us by the requirement that it fit into a certain commutative diagram, clearly $U$ is the unique functor with that property. However, to complete the proof we also need to verify that what we have constructed is actually a well-defined weak monoidal functor. To see that $U$ respects associators, consider the following diagram:
\begin{eq*} \xymatrix{
\big( \, RU(y) \boxtimes RU(y') \, \big) \boxtimes RU(y'') \ar[rr]^{a^{\mathrm{sk}(X)}} \ar[d]_{\mu^R \, \boxtimes \, \mathrm{id}} & & RU(y) \boxtimes \big( \, RU(y') \boxtimes RU(y'') \, \big) \ar[d]^{\mathrm{id} \, \boxtimes \, \mu^R} \\
R\big( \, U(y) \otimes U(y') \, \big) \boxtimes RU(y'') \ar[d]_{R(\mu^U) \, \boxtimes \, \mathrm{id}} \ar@/^6pc/[dd]^{\mu^R} & & RU(y) \boxtimes R\big( \, U(y') \otimes U(y'') \, \big) \ar[d]^{\mathrm{id} \, \boxtimes \, R(\mu^U)} \ar@/_6pc/[dd]_{\mu^R}  \\
RU(y \otimes y') \boxtimes RU(y'')  \ar@/_7pc/[dd]_{\mu^R} & & RU(y) \boxtimes RU(y' \otimes y'') \ar@/^7pc/[dd]^{\mu^R} \\
R\Big( \, \big( \, U(y) \otimes U(y') \, \big) \otimes U(y'') \, \Big) \ar[rr]^{R(a^X)} \ar[d]^{R(\mu^U \, \otimes \, \mathrm{id})} & & R\Big( \, U(y) \otimes \big( \, U(y') \otimes U(y'') \, \big) \, \Big) \ar[d]_{R(\mathrm{id} \, \otimes \, \mu^U)} \\
R\big( \, U(y \otimes y') \otimes U(y'') \, \big) \ar[d]_{R(\mu^U)} & & R\big( \, U(y) \otimes U(y' \otimes y'') \, \big) \ar[d]^{R(\mu^U)} \\
RU\big( \, (y \otimes y') \otimes y'' \, \big) \ar[rr]^{RU(a^Y)} & & RU\big( \, y \otimes (y' \otimes y'') \, \big) }
\end{eq*}
The topmost region in this diagram is just the associativity coherence condition for $R$, and hence it commutes. Likewise, the outside edges of this diagram commute because they form the associativity condition for $S$, via $\mu^S = R(\mu^U) \mu^R$. Lastly, the two areas on either side of the diagram commute by naturality of $\mu^R$. As a result of this, and the fact that every edge of the diagram is invertible, it follows that the bottom rectangle the diagram commutes too. Focusing on this area, we see that is is the image under $R$ of the associativity condition for $U$. But since $R$ just acts on morphisms by $f \mapsto \rho_{y'}^{-1} f \rho_y$, which is bijective, the full condition for $U$ is immediately recoverable. 

We can prove that $U$ respects unitors in a similar way. Consider the following diagram:
\begin{eq*} \xymatrix{
& RU(y) & \\
R(I) \boxtimes RU(y) \ar[ur]^{l^Y} \ar[dd]_{\eta^R \, \boxtimes \, \mathrm{id}} & & RU(I \otimes y) \ar[ul]_{RU(l^{\mathrm{sk}(X)})} \\
& R\big( \, I \otimes U(y) \, \big) \ar[uu]^{R(l^X)} \ar[dr]^{R(\eta^U \, \otimes  \, \mathrm{id})} & \\
R(I) \boxtimes RU(y) \ar[ur]^{\mu^R} \ar[dr]^{R(\eta^U) \, \otimes \mathrm{id}} & & R\big( \, U(I) \otimes U(y) \, \big) \ar[uu]_{R(\mu^U)} \\
& RU(I) \boxtimes RU(y) \ar[ur]^{\mu^R} & }
\end{eq*}
The top-left square is the left unitality coherence condition for $R$; the outer edges form the same coherence condition but for $S$ with $\eta^S = R(\eta^U) \eta^R$; and the bottom square follows from the naturality of $\mu^R$. Hence, all of those parts of the diagram commute, and since again all edges are invertible, we can conclude that the remaining top-right square also commutes. But this is just the image under $R$ of left unitality for $\eta^U$, and again we can obtain the original from this by bijectivity. The proof of the right unitality coherence condition for $U$ proceeds in exactly the same way, except involving $r$ instead of $l$. Therefore, $U$ is indeed a well-defined weak monoidal functor, and hence $X$ is the required pullback.
\end{proof}

For any of the objects $\psi : \mathbb{G}_n \to X$ of $C(\mathbb{G}_n)^{\mathrm{mn}}$, we now have a way of breaking down their source and target into smaller groupoids representing their connected components and automorphisms. We might wonder if we can do the same sort of thing to $\psi$ itself --- show that it is equivalent to several `smaller' maps running between corresponding terms of $\mathrm{sk}(\mathbb{G}_n) \times_{\pi_0(\mathbb{G}_n)} \mathrm{po}(\mathbb{G}_n)$ and $\mathrm{sk}(X) \times_{\pi_0(X)} \mathrm{po}(X)$. This would allow us to contruct the initial object of $C(\mathbb{G}_n)^{\mathrm{mn}}$ by finding the initial versions of these reduced maps, greatly simplifying the problem. It turns out that this procedure is relatively straightforward.

\begin{prop}\label{factor1} Let $X$ and $Y$ be monoidal groupoids with chosen skeletons $\mathrm{sk}(X)$ and $\mathrm{sk}(Y)$ respectively. Then for any weak monoidal functor $F : X \to Y$ there exists a unique pair of weak monoidal functors
\begin{eq*} F_{\mathrm{sk}} : \mathrm{sk}(X) \to \mathrm{sk}(Y), \quad \quad F_{\mathrm{po}} : \mathrm{po}(X) \to \mathrm{po}(Y) \end{eq*}
such that the diagrams
\begin{eq*} \begin{tikzcd}
X \ar[r, "F"] \ar[d, "R^X"'] & Y \ar[d, "R^Y"] & & X \ar[r, "F"] \ar[d, "P^X"'] & Y \ar[d, "P^Y"] \\
\mathrm{sk}(X) \ar[r, "F_{\mathrm{sk}}"] \ar[d, "\lbrack \, \_ \, \rbrack"'] & \mathrm{sk}(Y) \ar[d, "\lbrack \, \_ \, \rbrack"] & & \mathrm{po}(X) \ar[r, "F_{\mathrm{po}}"] \ar[d, "\lbrack \, \_ \, \rbrack"'] & \mathrm{po}(Y) \ar[d, "\lbrack \, \_ \, \rbrack"] \\
\pi_0(X) \ar[r, "F_\pi"] & \pi_0(Y) & & \pi_0(X) \ar[r, "F_\pi"] & \pi_0(Y)
\end{tikzcd} \end{eq*}
commute, where $F_{\pi}$ is the restriction of $F$ on connected components.
\end{prop}
\begin{proof}
To begin, recall that we saw in the proof of \cref{pullback} that $[ \, R^X \, ]$, $[ \, R^Y \, ]$, $[ \, P^X \, ]$, and $[ \, P^Y \, ]$ are just the canonical maps $\mathrm{Ob}(X) \to \pi_0(X)$ or $\mathrm{Ob}(Y) \to \pi_0(Y)$ sending objects to their connected components. It follows that the outside edges of both diagrams will commute by the definition of $F_{\pi}$, regardless of our choice of $F_{\mathrm{sk}}$ and $F_{\mathrm{po}}$.

Next, notice that if a particular choice of $F_{\mathrm{sk}}$ makes the top square of its diagram commute, then the bottom square will automatically do so too. This is because if we have such an $F_{\mathrm{sk}}$, a quick diagram chase yields
\begin{eq*} [ \, F_{\mathrm{sk}} R^X \, ] \, = \, F_{\pi}[ \, R^X \, ]  \end{eq*}
and precomposing this by $i^X$ reduces it to
\begin{eq*} [ \, F_{\mathrm{sk}} \, ] \, = \, [ \, F_{\mathrm{sk}} R^X  i^X \, ] \, = \, F_{\pi}[ \, R^X i^X \, ] \, = \, F_{\pi}[ \, \_ \, ] \end{eq*}
However, the exact same method of precomposing by $i^X$ shows that there is only one choice of $F_{\mathrm{sk}}$ for which the top square of the diagram commutes:  
\begin{eq*} F_{\mathrm{sk}} R^X \, = \, R^Y F \quad \implies \quad F_{\mathrm{sk}} \, = \, F_{\mathrm{sk}} R^X i^X \, = \, R^Y F i^X \end{eq*}
Therefore, this is the unique weak monoidal functor $F_{\mathrm{sk}}$ we are looking for.

Lastly, consider $F_{\mathrm{po}}$. The value it takes on objects follows immediately from the requirement that the top square of its diagram commutes:
\begin{eq*} F_{\mathrm{po}}(x) \, = \,  F_{\mathrm{po}} P^X(x) \, = \, P^Y F(x)  \, = \, F(x) \end{eq*}
Its action on morphisms is then fixed by the fact that morphisms of $\mathrm{po}(X)$ and $\mathrm{po}(Y)$ are uniquely determined by their sources and targets. Moreover, since $\mathrm{po}(Y)$ is a strictly monoidal category, the coherence data $\eta$, $\mu$ for the weak monoidal functor $F_{\mathrm{po}}$ will have components which are all automorphisms of certain objects. Thus, uniqueness of morphisms on a given source and target also tells us that these components are identities, and hence that $F_{\mathrm{po}}$ is strict.

With everything about $F_{\mathrm{po}}$ now known, the following is sufficient to show that the bottom square of its diagram commutes as well:
\begin{eq*} \begin{array}{rllrrlll}
		\left[ \, F_{\mathrm{po}}(x) \, \right] & = & [ \, F(x) \, ] & \quad \quad & [ \, F_{\mathrm{po}}(x \to x') \, ] & = & [ \, F(x) \to F(x') \, ] \\
		& = & F_{\pi}\big( \, [x] \, \big) & & & = & \mathrm{id}_{[ F(x) ]} \\
		& & & & & = & \mathrm{id}_{F_{\pi}([x])} \\
		& & & & & = & F_{\pi}(\mathrm{id}_{[x]}) \\
		& & & & & = & F_{\pi}\big( \, [x \to x'] \, \big) \\
		\end{array}
 \end{eq*}
Therefore, we have a found the unique $F_{\mathrm{sk}}$ and $F_{\mathrm{po}}$ with the required properties.
\end{proof}

\begin{prop} \label{factor2} Let $X$ and $Y$ be monoidal groupoids with chosen skeletons $\mathrm{sk}(X)$ and $\mathrm{sk}(Y)$ respectively. Then for any pair of weak monoidal functors
\begin{eq*} F_{\mathrm{sk}} : \mathrm{sk}(X) \to \mathrm{sk}(Y), \quad \quad F_{\mathrm{po}} : \mathrm{po}(X) \to \mathrm{po}(Y) \end{eq*}
that have the same restriction to connected components, $F_{\pi}$, there exists a unique weak monoidal functor $F : X \to Y$ such that the following diagrams commute:
\begin{eq*} \begin{tikzcd}
X \ar[r, "F"] \ar[d, "R^X"'] & Y \ar[d, "R^Y"] & & X \ar[r, "F"] \ar[d, "P^X"'] & Y \ar[d, "P^Y"] \\
\mathrm{sk}(X) \ar[r, "F_{\mathrm{sk}}"] & \mathrm{sk}(Y) & & \mathrm{po}(X) \ar[r, "F_{\mathrm{po}}"] & \mathrm{po}(Y)
\end{tikzcd} \end{eq*}
\end{prop}
\begin{proof}
The fact that $F_{\pi}$ is the underlying map on connected components for both $F_{\mathrm{sk}}$ and $F_{\mathrm{po}}$ means that
\begin{eq*} [ \, F_{\mathrm{sk}} \, ] \, = \, F_{\pi}[ \, \_ \, ] \, = \, [ \, F_{\mathrm{po}} \, ] \end{eq*}
and, by \cref{pullback}, we also know
\begin{eq*} [ \, R^X \, ] \, = \, [ \, P^X \, ] \end{eq*}
It follows from these that
\begin{eq*} [ \, F_{\mathrm{sk}} R^X \, ] \, = \, F_{\pi}[ \, R^X \, ] \, = \, F_{\pi}[ \, P^X \, ] \, = \, [ \, F_{\mathrm{po}} P^X \, ] \end{eq*}
or, in other words, the outside edges of the following diagram commute:
\begin{eq*} \begin{tikzcd}
& X \ar[dd, dashrightarrow, "F"] \ar[dddl, bend right, "F_{\mathrm{sk}} R^X"'] \ar[dddr, bend left, "F_{\mathrm{po}} P^X"] & \\
&& \\
& Y \ar[dl, "R^Y"'] \ar[dr, "P^Y"] & \\
\mathrm{sk}(Y) \ar[dr, "\lbrack \, \_ \, \rbrack"'] & & \mathrm{po}(Y) \ar[dl, "\lbrack \, \_ \, \rbrack"] \\
& \pi_0(Y) &
\end{tikzcd} \end{eq*}
But the bottom region of this diagram is a pullback square, again by \cref{pullback}, and so there must exist a unique map $F: X \to Y$ as shown making the top left and top right regions of the diagram commute, as required.
\end{proof}

Propositions \ref{factor1} and \ref{factor2} together tell us that we can break apart any object $\psi : \mathbb{G}_n \to X$ of $C(\mathbb{G}_n)^{\mathrm{mn}}$ into two simpler maps $\psi_{\mathrm{sk}}$ and $\psi_{\mathrm{po}}$, and that we can always recover $\psi$ from them again. Thus, if we want to find the initial objects $\phi : \mathbb{G}_n \to Z$ of $C(\mathbb{G}_n)^{\mathrm{mn}}$, it will suffice to calculate $\phi_{\mathrm{sk}}$ and $\phi_{\mathrm{po}}$. But as it happens, we have already proven enough about the initial algebra to get the second of these:

\begin{lem}\label{polem} Let $\phi : \mathbb{G}_n \to Z$ be the initial object of $C(\mathbb{G}_n)^{\mathrm{mn}}$. Then $\phi_{\mathrm{po}}$ is the inclusion
\begin{eq*} \bigsqcup_{w \in \mathbb{N}^n} \mathrm{E}\big( \, q_{\mathbb{N}^{\ast n}}^{-1}(w) \, \big) \quad \hookrightarrow \quad \bigsqcup_{w \in \mathbb{Z}^n} \mathrm{E}\big( \, q_{\mathbb{Z}^{\ast n}}^{-1}(w) \, \big) \end{eq*}
where the $q$ are the appropriate quotients of abelianisation, and the monoidal products are inherited from $\mathbb{N}^{\ast n}$ and $\mathbb{Z}^{\ast n}$ in the obvious way.
\end{lem}
\begin{proof}
We know from \cref{Zobj} that action of $\phi$ on objects is given by the inclusion map $\mathbb{N}^{\ast n} \to \mathbb{Z}^{\ast n}$, and in the proof of \cref{factor1} we saw that $\phi_{\mathrm{po}}$ is a strict monoidal functor which is completely determined by $\phi_{\mathrm{ob}}$. Thus, $\phi_{\mathrm{po}}$ is the inclusion $\mathrm{po}(\mathbb{G}_n) \to \mathrm{po}(Z)$. By \cref{concomp} the connected components of $\mathbb{G}_n$ are $\mathbb{N}^n$ and for $Z$ they are $\mathbb{Z}^n$, each assigned by abelianisation. Thus $\mathrm{po}(\mathbb{G}_n)$ and $\mathrm{po}(Z)$ are isomorphic as categories to the coproducts given in the lemma, with $\otimes$ induced by $\phi_{\mathrm{ob}}: \mathbb{N}^{\ast n} \to \mathbb{Z}^{\ast n}$.
\end{proof}

Therefore, the final step is to try to find $\phi_{\mathrm{sk}}$. For that purpose, we will define one last new category to work in.

\begin{defn} Fix a choice of skeleton $\mathrm{sk}(\mathbb{G}_n)$ of $\mathbb{G}_n$. Then we define the category $C_{\mathrm{sk}}(\mathbb{G}_n)$ in the following way:
\begin{itemize}
\item an object of $C_{\mathrm{sk}}(\mathbb{G}_n)$ is any weak monoidal functor $\chi: \mathrm{sk}(\mathbb{G}_n) \to \mathrm{sk}(X)$ whose target is a choice of skeleton for some monoidal category $X$ that is the target of at least one object $\psi: \mathbb{G}_n \to X$ of $C(\mathbb{G}_n)^{\mathrm{mn}}$
\item a morphism $f: \chi \to \chi'$ in $C_{\mathrm{sk}}(\mathbb{G}_n)$ between two objects with targets $\mathrm{sk}(X)$ and $\mathrm{sk}(X')$ respectively is a weak monoidal functor $f: \mathrm{sk}(X) \to \mathrm{sk}(X')$ such that $\chi' = f \circ \chi$
\end{itemize}
\end{defn}

This definition may seem a little strange at first, but the following lemma should show why it is a useful one:

\begin{lem}\label{Csklem} A functor $\chi$ is an object of $C_{\mathrm{sk}}(\mathbb{G}_n)$ if and only if there exists an object $\psi$ of $C(\mathbb{G}_n)^{\mathrm{mn}}$ such that $\psi_{\mathrm{sk}} = \chi$.
\end{lem}
\begin{proof}
It should be clear from the previous definition that if $\psi$ is in $C(\mathbb{G}_n)^{\mathrm{mn}}$ then $\psi_{\mathrm{sk}}$ is in $C_{\mathrm{sk}}(\mathbb{G}_n)$. For the converse, let $\chi: \mathrm{sk}(\mathbb{G}_n) \to \mathrm{sk}(X)$ be an object of $C_{\mathrm{sk}}(\mathbb{G}_n)$ and consider the composite
\begin{eq*} \begin{tikzcd}
\mathrm{Ob}(\mathbb{G}_n) \ar[r, "\lbrack \, \_ \, \rbrack"] & \pi_0(\mathbb{G}_n) \ar[d, equal] & & & \\
& \pi_0\big( \mathrm{sk}(\mathbb{G}_n) \big) \ar[r, "\chi_{\pi}"] & \pi_0\big( \mathrm{sk}(X) \big) \ar[r, "\lbrack \, \_ \, \rbrack^{-1}"] & \mathrm{Ob}\big(\mathrm{sk}(X) \big) \ar[r, "i^X"] & \mathrm{Ob}(X)
\end{tikzcd} \end{eq*}
Note that this makes sense because $\mathrm{sk}(X)$ is skeletal, and so $[ \, \_ \, ] : \mathrm{Ob}(\mathrm{sk}(X)) \to \pi_0(\mathrm{sk}(X))$ an isomorphism on objects. From this composite we can define a unique map $\chi': \mathrm{po}(\mathbb{G}_n) \to \mathrm{po}(X)$, since such maps are completely determined by their behaviour on objects. But because of how $\chi'$ was constructed, $\chi'_{\pi} = \chi_{\pi}$, and so we can apply \cref{factor2} to $\chi, \chi'$ to obtain an object $\psi : \mathbb{G}_n \to X$ of $C(\mathbb{G}_n)^{\mathrm{mn}}$ for which $\psi_{\mathrm{sk}} = \chi$.
\end{proof}

The exact same reasoning also shows that the morphism of $C_{\mathrm{sk}}(\mathbb{G}_n)$ are just the maps $f_{\mathrm{sk}}: \mathrm{sk}(X) \to \mathrm{sk}(Y)$ for each morphism $f: X \to Y$ in $C(\mathbb{G}_n)^{\mathrm{mn}}$. Now we can put this new definition to use.

\begin{prop}\label{initialsk} For any initial object $\phi: \mathbb{G}_n \to Z$ of $C(\mathbb{G}_n)^{\mathrm{mn}}$, the functor $\phi_{\mathrm{sk}}$ is an initial object of $C_{\mathrm{sk}}(\mathbb{G}_n)$.
\end{prop}
\begin{proof}
Let $\chi: \mathrm{sk}(\mathbb{G}_n) \to \mathrm{sk}(X)$ be an arbitrary object of $C_{\mathrm{sk}}(\mathbb{G}_n)$. If we wish to show that $\phi_{\mathrm{sk}}$ is initial, we need to construct the unique morphism $u: \mathrm{sk}(Z) \to \mathrm{sk}(X)$ in $C_{\mathrm{sk}}(\mathbb{G}_n)$ such that $\chi = u \circ \phi_{\mathrm{sk}}$.

To begin, choose any object $\psi: \mathbb{G}_n \to X$ of $C(\mathbb{G}_n)^{\mathrm{mn}}$ with the property that $\psi_{\mathrm{sk}} = \chi$. By \cref{Csklem} at least one such $\psi$ must exist. We can use the fact that $\phi$ is an initial object of $C(\mathbb{G}_n)^{\mathrm{mn}}$ to find a unique morphism $v: Z \to X$ such that $\psi = v \circ \phi$, and then we can apply \cref{factor1} to obtain from it a morphism $v_{\mathrm{sk}}: \mathrm{sk}(Z) \to \mathrm{sk}(X)$ of $C_{\mathrm{sk}}(\mathbb{G}_n)$. This is all summarised by the following diagram:
\begin{eq*} \begin{tikzcd}
& \mathbb{G}_n \ar[dd, "R^{\mathbb{G}_n}" near end] \ar[ld, "\phi"'] \ar[rd, "\psi"] & \\
Z \ar[dd, "R^Z"'] \ar[rr, "v" near start, crossing over]& & X \ar[dd, "R^X"] \\
& \mathrm{sk}(\mathbb{G}_n)\ar[ld, "\phi_{\mathrm{sk}}"'] \ar[rd, "\chi"] & \\
\mathrm{sk}(Z) \ar[rr, "v_{\mathrm{sk}}"]& & \mathrm{sk}(X)
\end{tikzcd} \end{eq*}
Every face of this diagram is known to commute except for the bottom one, and so it follows that
\begin{eq*} \chi R^{\mathbb{G}_n} \, = \, v_{\mathrm{sk}} \phi_{\mathrm{sk}} R^{\mathbb{G}_n} \end{eq*}
and hence
\begin{eq*} \chi \, = \, \chi R^{\mathbb{G}_n} i^{\mathbb{G}_n} \, = \, v_{\mathrm{sk}} \phi_{\mathrm{sk}} R^{\mathbb{G}_n} i^{\mathbb{G}_n} \, = \, v_{\mathrm{sk}} \phi_{\mathrm{sk}} \end{eq*}
Therefore, $v_{\mathrm{sk}}$ shows that there is at least one morphism of $C_{\mathrm{sk}}(\mathbb{G}_n)$ which satisfies the condition that we want $u$ to. 

To complete the proof, assume now that $u, u': \mathrm{sk}(Z) \to \mathrm{sk}(X)$ are both morphisms of $C_{\mathrm{sk}}(\mathbb{G}_n)$ with $\chi = u \circ \phi_{\mathrm{sk}} = u' \circ \phi_{\mathrm{sk}}$. From $u$, define a new map $w: \mathrm{po}(Z) \to \mathrm{po}(X)$ which is the unique such map that acts on objects as the composite
\begin{eq*} \begin{tikzcd}
\mathrm{Ob}(Z) \ar[r, "\lbrack \, \_ \, \rbrack"] & \pi_0(Z) \ar[d, equal] & & & \\
& \pi_0\big( \mathrm{sk}(Z) \big) \ar[r, "u_{\pi}"]  & \pi_0\big( \mathrm{sk}(X) \big) \ar[r, "\lbrack \, \_ \, \rbrack^{-1}"] & \mathrm{Ob}\big(\mathrm{sk}(X) \big) \ar[r, "i^X"] & \mathrm{Ob}(X)
\end{tikzcd} \end{eq*}
We'll also define a map $w': \mathrm{po}(Z) \to \mathrm{po}(X)$ from $u'$ in the precisely analogous way. Now, by design $w, w'$ act on connected components in the same way that $u, u'$ do respectively --- that is, $w_{\pi} = u_{\pi}$ and $w'_{\pi} = u'_{\pi}$. Thus by \cref{factor2} we can combine $u$ and $w$ into a unique morphism $v: Z \to X$ from $C(\mathbb{G}_n)^{\mathrm{mn}}$, and likewise get a unique $v': Z \to X$ from $u', w'$. Moreover, by applying \cref{factor1} to break $v$ and $v'$ back down again, we see that $v = v'$ if and only if $u = u'$. As before, we have diagrams
\begin{eq*} \begin{tikzcd}
& \mathbb{G}_n \ar[dd, "R^{\mathbb{G}_n}" near end] \ar[ld, "\phi"'] \ar[rd, "\psi"] & & & \mathbb{G}_n \ar[dd, "R^{\mathbb{G}_n}" near end] \ar[ld, "\phi"'] \ar[rd, "\psi"] & \\
Z \ar[dd, "R^Z"'] \ar[rr, "v" near start, crossing over]& & X \ar[dd, "R^X"] & Z \ar[dd, "R^Z"'] \ar[rr, "v'" near start, crossing over]& & X \ar[dd, "R^X"] \\
& \mathrm{sk}(\mathbb{G}_n)\ar[ld, "\phi_{\mathrm{sk}}"'] \ar[rd, "\chi"] & & & \mathrm{sk}(\mathbb{G}_n)\ar[ld, "\phi_{\mathrm{sk}}"'] \ar[rd, "\chi"] & \\
\mathrm{sk}(Z) \ar[rr, "u"]& & \mathrm{sk}(X) & \mathrm{sk}(Z) \ar[rr, "u'"]& & \mathrm{sk}(X)
\end{tikzcd} \end{eq*}
but this time all of the faces are known to commute except for the top ones. It follows from these that
\begin{eq*} R^X \psi \, = \, R^X v \phi \, = \, R^X v' \phi \end{eq*}
However, recall that we constructed $w$ by using a composite whose last part was the inclusion $i^X$, and as a result $\mathrm{im}(w) \subseteq \mathrm{po}(\mathrm{sk}(X)) \subseteq \mathrm{po}(X)$. Obviously $\mathrm{im}(u) \subseteq \mathrm{sk}(X) = \mathrm{sk}(\mathrm{sk}(X))$ as well, and so if we think of $u, w$ as being maps $u: \mathrm{sk}(\mathbb{G}_n) \to \mathrm{sk}(\mathrm{sk}(X))$ and $w: \mathrm{po}(\mathbb{G}_n) \to \mathrm{po}(\mathrm{sk}(X))$ we can apply \cref{factor2} and conclude that $\mathrm{im}(v) \subseteq \mathrm{sk}(X)$. Thus $R^X v$ is actually just $v$, and so --- after making all of the same arguments again for $u', w', v'$ --- we get that
\begin{eq*} R^X v \phi \, = \, R^X v' \phi \implies v \phi \, = \, v' \phi \end{eq*}
But by initiality of $\phi$ there should be only one morphism $f$ with the property that $v \circ \phi = f \circ \phi$, and so we can conclude that $v = v'$ and $u = u'$. 

Thus there is also at most one morphism $u$ in $C_{\mathrm{sk}}(\mathbb{G}_n)$ such that $\chi = u \circ \phi_{\mathrm{sk}}$. Therefore $u$ must be unique, and hence $\phi_{\mathrm{sk}}$ is indeed an initial object of $C_{\mathrm{sk}}(\mathbb{G}_n)$.
\end{proof}

\subsection{Initial algebras as a colimit}

Finally, we have reached the point where in order to find the initial object of $(\mathbb{G}_n \downarrow \mathrm{inv})$ it will suffice to find an initial object in the much more restrictive category $C_{\mathrm{sk}}(\mathbb{G}_n)$. With so much of the unnecessary structure of $(\mathbb{G}_n \downarrow \mathrm{inv})$ now removed, this task is at last appproachable.

\begin{defn} \label{Ddef} Let $D_n$ be a diagram in category of groups whose vertices are the endomorphism groups $\mathbb{G}_n(s, s)$ of the representing objects $s$ in $\mathrm{sk}(\mathbb{G}_n)$, and which has edges
\begin{eq*} \begin{array}{rrrll}
		\_ \boxtimes \mathrm{id}_{s'} & : & \mathbb{G}_n(s, s) & \to & \mathbb{G}_n(s \boxtimes s', s \boxtimes s') \\
		\mathrm{id}_{s'} \boxtimes \_ & : & \mathbb{G}_n(s, s) & \to & \mathbb{G}_n(s' \boxtimes s, s' \boxtimes s) 
		\end{array}
\end{eq*}
for all $s, s' \in \mathbb{N}^{\ast n} = \mathrm{Ob}\big(\mathrm{sk}(\mathbb{G}_n)\big)$.
\end{defn}

\begin{prop}\label{colimD} Let $\phi_{\mathrm{sk}} : \mathrm{sk}(\mathbb{G}_n) \to \mathrm{sk}(Z)$ be the initial object of $C_{\mathrm{sk}}(\mathbb{G}_n)$. Then $\mathrm{sk}(Z)$ is the weak monoidal category with underlying category $\mathbb{Z}^n \times \mathrm{colim}(D_n)$ and coherence data
\begin{eq*} \begin{array}{rll}
		a^{\mathrm{sk}(Z)}_{s, s', s''} & = & j_{s \boxtimes s' \boxtimes s''}(a^{\mathrm{sk}(\mathbb{G}_n)}_{s, s', s''}) \\
		l^{\mathrm{sk}(Z)}_s & = & j_s(l^{\mathrm{sk}(\mathbb{G}_n)}_s) \\
		r^{\mathrm{sk}(Z)}_s & = & j_s(r^{\mathrm{sk}(\mathbb{G}_n)}_s) \\
		\end{array}
\end{eq*} 
and $\phi_{\mathrm{sk}}$ is the weak monoidal functor
\begin{eq*}\begin{array}{rrrll}
		\phi_{\mathrm{sk}} & : & \mathrm{sk}(\mathbb{G}_n) & \to & \mathbb{Z}^n \times \mathrm{B} \, \mathrm{colim}(D_n) \\
		& : & s & \mapsto & s \\
		& : & f: s \to s & \mapsto & \big( \, \mathrm{id}_s, j_s(f) \, \big) \\
		& & \mu^\chi_{s, s'} & = & \mathrm{id}_{s \boxtimes s'} \\
		& & \eta^\chi & = & \mathrm{id}_{R(I)}
		\end{array}
\end{eq*}
where the $j_s$ are the canonical maps $\mathbb{G}_n(s, s) \to \mathrm{colim}(D_n)$.
\end{prop}
\begin{proof}
By unraveling its definition, recall an object $\chi$ of $C_{\mathrm{sk}}(\mathbb{G}_n)$ is a monoidal functor whose source is a chosen skeleton $\mathrm{sk}(\mathbb{G}_n)$ of $\mathbb{G}_n$, whose target is a skeleton $\mathrm{sk}(X)$ of some $\mathrm{E}$G-algebra $X$ with objects $\mathbb{Z}^{\ast n}$ assigned to connected components though abelianisation, and whose restriction $\chi_{\pi}$ is the inclusion $\mathbb{N}^n \to \mathbb{Z}^n$. In other words, $\phi_{\mathrm{sk}}$ is part of the initial diagram amongst those of the form
\begin{eq*} \begin{tikzcd}
\mathrm{sk}(\mathbb{G}_n) \ar[r, "\chi"] \ar[d, "\lbrack \, \_ \, \rbrack"'] & \mathrm{sk}(X) \ar[d, "\lbrack \, \_ \, \rbrack"] \\
\mathbb{N}^n \ar[r, hookrightarrow] & \mathbb{Z}^n
\end{tikzcd} \end{eq*}
where all the maps $[ \, \_ \, ]$ do is 
\begin{eq*} [s] \, = \, s \quad \quad [f: s \to s] \, = \, \mathrm{id}_s \end{eq*}
However, from now on it will prove more useful to adopt a different perspective, where we will stop thinking of objects like $\chi$ as weak monoidal functors, but instead as structure-preserving maps between collections of morphisms. Since the categories in question are skeletons, their sets of morphisms are just
\begin{eq*} \mathrm{Mor}\big( \, \mathrm{sk}(\mathbb{G}_n) \, ) \, = \, \bigsqcup_s \mathbb{G}_n(s, s), \quad \quad \mathrm{Mor}\big( \, \mathrm{sk}(X) \, ) \, = \, \bigsqcup_s X(s, s) \end{eq*}
where the disjoint unions are indexed over the representative objects $s$. These sets then possess extra structure which $\chi$ must respect --- the binary operations $\boxtimes$ and $\circ$, and the coherence data, $a_{s, s', s''}$, $l_s$, $r_s$.

Now, since the objects of $\mathrm{sk}(X)$ form the monoid $\mathbb{Z}^n$ under $\boxtimes$, each object $s$ has an inverse $s^*$, and so we have some isomorphisms of sets
\begin{eq*}\begin{array}{rll}
		X(s, s) & \to & X(R(I), R(I)) \\
		f & \mapsto & f \boxtimes \mathrm{id}_{s^*} 
		\end{array}
\end{eq*}
We can combine all of these into a larger isomorphism
\begin{eq*}\begin{array}{rrrll}
		\theta & : & \bigsqcup X(s, s) & \to & \mathbb{Z}^n \times X(R(I), R(I)) \\
		& & f: s \to s & \mapsto & (\mathrm{id}_s, f \boxtimes \mathrm{id}_{s^*})
		\end{array}
\end{eq*}
which then allows us to split the function $\chi: \bigsqcup \mathbb{G}_n(s, s) \to \bigsqcup X(s, s)$ into two independent pieces, namely
\begin{eq*}\begin{array}{rrrrrll}
		p_1 \circ \theta \circ \chi & =: & \chi_{\mathbb{Z}} & : & \bigsqcup \mathbb{G}_n(s, s) & \to & \mathbb{Z}^n \\
		& & & : & f: s \to s & \mapsto & \mathrm{id}_{s}
		\end{array}
\end{eq*}
and
\begin{eq*}\begin{array}{rrrrrll}
		p_2 \circ \theta \circ \chi & =: & \chi_I & : & \bigsqcup \mathbb{G}_n(s, s) & \to & X(R(I), R(I)) \\
		& & & : & f: s \to s & \mapsto & \chi(f) \boxtimes \mathrm{id}_{s^*}
		\end{array}
\end{eq*}
This will in turn separate the diagram for $\chi$ into two independent diagrams
\begin{eq*} \begin{tikzcd}
\bigsqcup_s \mathbb{G}_n(s, s) \ar[r, "\chi_{\mathbb{Z}}"] \ar[d, "\lbrack \, \_ \, \rbrack"'] & \mathbb{Z}^n \ar[d, "\lbrack \, \_ \, \rbrack", equal] & & \bigsqcup_s \mathbb{G}_n(s, s) \ar[r, "\chi_I"] \ar[d, "\lbrack \, \_ \, \rbrack"'] &  X(R(I), R(I)) \ar[d, "\lbrack \, \_ \, \rbrack"]\\
\mathbb{N}^n \ar[r, hookrightarrow] & \mathbb{Z}^n & & \mathbb{N}^n \ar[r, hookrightarrow] & \mathbb{Z}^n
\end{tikzcd} \end{eq*}
with the corresponding diagrams for $\phi_{\mathrm{sk}}$ being the initial ones of each kind. But diagram on the left only commutes for one value of $\chi_{\mathbb{Z}}$, so all such maps are equal and trivially. Also, since $R(I)$ is the unit of $\mathrm{sk}(\mathbb{G}_n)$ the map $[ \, \_ \, ]: X(R(I), R(I)) \to \mathbb{Z}^n$ will always send everything to 0, rendering the bottom part of the righthand diagram uninteresting. Therefore, to find $\phi_{\mathrm{sk}}$ all that remains is to construct the initial map among all of the $\chi_I$.

So, what effect do the properties of $\chi$ have on $\chi_I$? Firstly, preservation of composition is inherited directly:
\begin{eq*}\begin{array}{rll}
		\chi_I( \, f \circ f' : s \to s \, ) & = & \chi(f \circ f') \boxtimes \mathrm{id}_{s^*} \\
		& = & \big( \, \chi(f) \circ \chi(f') \, \big) \boxtimes (\mathrm{id}_{s^*} \circ \mathrm{id}_{s^*}) \\
		& = & \big( \, \chi(f) \boxtimes \mathrm{id}_{s^*} \, \big) \circ \big( \, \chi(f') \boxtimes \mathrm{id}_{s^*} \, \big)  \\
		& = & \psi_I(f) \circ \psi_I(f')
		\end{array}
\end{eq*}
In other words, $\chi_I$ can be seen as the disjoint union of family of group homomorphisms $\chi_s: \mathbb{G}_n(s, s) \to X(R(I), R(I))$, indexed by the objects of $\mathrm{sk}(X)$. 

Secondly, part of the coherence data for the weak monoidal functor $\chi$ is the natural isomorphism $\mu^{\chi}$, and so by naturality we have
\begin{eq*}\begin{array}{rll}
		\chi_I(f  \boxtimes f') & = & \chi(f \boxtimes f') \boxtimes \mathrm{id}_{(s \boxtimes s)^*} \\
		& = & \big( \, \mu^{\chi}_{s, s'} \circ \big( \, \chi(f) \boxtimes \chi(f') \, \big) \circ (\mu^{\chi}_{s, s'})^{-1} \, \big) \boxtimes \mathrm{id}_{(s \boxtimes s)^*} \\
		& = & \big( \, \mu^{\chi}_{s, s'} \boxtimes \mathrm{id}_{(s \boxtimes s)^*} \, \big) \circ \big( \, \chi(f) \boxtimes \chi(f') \boxtimes \mathrm{id}_{(s \boxtimes s)^*} \, \big) \circ \big( \, \mu^{\chi}_{s, s'} \boxtimes \mathrm{id}_{(s \boxtimes s)^*} \, \big)^{-1} 
		\end{array}
\end{eq*}
However, we can simplify this condition greatly, due to an important fact about $X(R(I), R(I))$. Recall that even though $\mathrm{sk}(X)$ is a weak monoidal category, its product $\boxtimes$ is strictly associative and unital, like $\circ$. These operations also have the same unit, $\mathrm{id}_{R(I)}$, and obey an interchange law,
\begin{eq*} (f \circ f') \boxtimes (g \circ g') \, = \, (f \boxtimes f') \circ (g \boxtimes g') \end{eq*}
and thus by applying the classic Eckmann-Hilton argument (see \cite{eckhil} for their original paper) we can conclude that $X(R(I), R(I))$ is a commutative monoid with $\boxtimes = \circ$. Indeed, this means that it is even an abelian group, since all morphisms in $X$ have compositional inverses. Using this, and the fact that $X$ is spacial (\cref{spacial}), we get
\begin{eq*}\begin{array}{rll}
		\chi_I(f  \boxtimes f') & = & \big( \, \mu^{\chi}_{s, s'} \boxtimes \mathrm{id}_{(s \boxtimes s)^*} \, \big) \circ \big( \, \chi(f) \boxtimes \chi(f') \boxtimes \mathrm{id}_{(s \boxtimes s)^*} \, \big) \circ \big( \, \mu^{\chi}_{s, s'} \boxtimes \mathrm{id}_{(s \boxtimes s')^*} \, \big)^{-1} \\
		& = & \big( \, \mu^{\chi}_{s, s'} \boxtimes \mathrm{id}_{(s \boxtimes s')^*} \, \big) \circ \big( \, \mu^{\chi}_{s, s'} \boxtimes \mathrm{id}_{(s \boxtimes s')^*} \, \big)^{-1} \circ \big( \, \chi(f) \boxtimes \chi(f') \boxtimes \mathrm{id}_{(s \boxtimes s')^*} \, \big) \\
		& = & \chi(f) \boxtimes \chi(f') \boxtimes \mathrm{id}_{(s \boxtimes s')^*} \\
		& = & \chi(f) \boxtimes \chi(f') \boxtimes \mathrm{id}_{(s')^*} \boxtimes \mathrm{id}_{s^*} \\
		& = & \chi(f) \boxtimes \mathrm{id}_{s^*} \boxtimes \chi(f') \boxtimes \mathrm{id}_{(s')^*} \\
		& = & \chi_I(f) \boxtimes \chi_I(f')
		\end{array}
\end{eq*}
So the effect that $\mu^{\chi}$ has on $\chi_I$ is to make it preserve $\boxtimes$. Moreover, since
\begin{eq*}\begin{array}{rll}
		\chi_I(f \boxtimes f') & = & \chi_I\big( \, (f \boxtimes \mathrm{id}_{s'}) \circ (\mathrm{id}_s \boxtimes f') \, \big) \\
		& = & \chi_I(f \boxtimes \mathrm{id}_{s'}) \circ \chi_I(\mathrm{id}_s \boxtimes f') \\
		& = & \chi_I(f \boxtimes \mathrm{id}_{s'}) \boxtimes \chi_I(\mathrm{id}_s \boxtimes f') \\
		\end{array}
\end{eq*}
we can recover the full condition $\chi_I(f \boxtimes f') = \chi_I(f) \boxtimes \chi_I(f')$ from two of its subcases,
\begin{eq*}\begin{array}{rllrrll}
		\chi_I(f \boxtimes \mathrm{id}_{s'}) & = & \chi_I(f) \boxtimes \chi_I(\mathrm{id}_{s'}) & \quad & \chi_I(\mathrm{id}_{s'} \boxtimes f) & = & \chi_I(\mathrm{id}_{s'}) \boxtimes \chi_I(f) \\
		& = & \chi_I(f) \boxtimes \mathrm{id}_{R(I)} & & & = & \mathrm{id}_{R(I)} \boxtimes \chi_I(f) \\
		& = & \chi_I(f) & & & = & \chi_I(f)
		\end{array} \end{eq*}
and so these are really the more fundamental property of $\chi_I$. Returning to the view of $\chi_I$ as a family of group homomorphisms again, what we've proven is that both $\chi_{s \boxtimes s'}(\_ \boxtimes \mathrm{id}_{s'})$ and $\chi_{s' \boxtimes s}(\mathrm{id}_{s'} \boxtimes \_)$ are equal to $\chi_s$. In the language of \cref{Ddef}, the pair $(X(R(I), R(I)), \chi_I)$ is a cocone of the diagram $D_n$.

The last few properties that $\chi_I$ will inherit come from the coherence conditions governing how $\chi$ and its natural isomorphisms $\mu^\chi, \eta^\chi$ interact with the data $a, l, r$ of $\mathrm{sk}(\mathbb{G}_n)$ and $\mathrm{sk}(X)$. However, it turns out that that these can be safely ignored when determining the initial object $\phi_{\mathrm{sk}}$. To see this, let $\chi': \mathrm{sk}(\mathbb{G}_n) \to \mathrm{sk}(X)'$ be the weak monoidal functor with the same underlying functor as $\chi$, but coherence data 
\begin{eq*} \begin{array}{rll}
		a^{\mathrm{sk}(X)'}_{s, s', s''} & = & \chi'(a^{\mathrm{sk}(\mathbb{G}_n)}_{s, s', s''}) \\
		l^{\mathrm{sk}(X)'}_s & = & \chi'(l^{\mathrm{sk}(\mathbb{G}_n)}_s) \\
		r^{\mathrm{sk}(X)'}_s & = & \chi'(r^{\mathrm{sk}(\mathbb{G}_n)}_s) \\
		\mu^\chi_{s, s'} & = & \mathrm{id}_{s \boxtimes s'} \\
		\eta^\chi & = & \mathrm{id}_{R(I)}
		\end{array}
\end{eq*} 
It is easy to verify that this is a well-defined object of $C_{\mathrm{sk}}(\mathbb{G}_n)$ which produces the same cocone $\chi_I$ of $D_n$ that $\chi$ does. Thus any weak monoidal functor $f: \mathrm{sk}(X)' \to \mathrm{sk}(X)$ such that $\chi = f \circ \chi'$ must have the identity as its underlying functor, and coherence data satisfying
\begin{eq*} \begin{array}{rclrrcl}
		\mu^f_{s, s'} & = & f(\mu^{\chi'}_{s, s'})^{-1} \circ \mu^{\chi}_{s, s'} & \quad & \eta^f_{s, s'} & = & f(\eta^{\chi'}_{s, s'})^{-1} \circ \eta^{\chi}_{s, s'} \\
		& = & \mu^\chi_{s, s'} & \quad & & = & \eta^\chi_{s, s'}
		\end{array}
\end{eq*} 
This weak monoidal functor is well-defined --- the proof involves similar steps to those taken during the proof of \cref{pullback} --- and so there exists a unique morphism from $\chi'$ to $\chi$ in $C_{\mathrm{sk}}(\mathbb{G}_n)$. Therefore, the unique morphism from the initial object $\phi_{\mathrm{sk}}$ onto $\chi$ can always be recovered from another map with the same cocone, and hence the cocones produced by the $\chi_I$ alone will provide sufficient information to determine $\phi_{\mathrm{sk}, I}$.

Furthermore, this argument also showns that for any cocone $\xi_s: \mathbb{G}_n(s, s) \to A$ of $D_n$, there exist some object of $C_{\mathrm{sk}}(\mathbb{G}_n)$ with $\xi_s$ as its associated cocone. Specifically, if we combine the $\xi_s$ into a functor 
\begin{eq*}\begin{array}{rrrll}
		\chi & : & \mathrm{sk}(\mathbb{G}_n) & \to & \mathbb{Z}^n \times \mathrm{B}A \\
		& : & s & \mapsto & s \\
		& : & f: s \to s & \mapsto & \big( \, \mathrm{id}_s, \xi_s(f) \, \big)
		\end{array}
\end{eq*}
then the weak monoidal functor $\chi'$ we get from it by the method given above will have $\chi'_s = \xi_s$. Therefore, $\phi_{\mathrm{sk}} : \mathrm{sk}(\mathbb{G}_n) \to \mathrm{sk}(Z)$ is the initial object of $C_{\mathrm{sk}}(\mathbb{G}_n)$, then the cocone obtained from $\psi_{\mathrm{sk}, I}$ is initial cocone of $D_n$ --- the colimit. In other words, the cocone $(Z(R(I), R(I)), \psi_{\mathrm{sk}, I})$ is just the pair $(\mathrm{colim}(D_n), j)$. Working backwards, from this cocone we can then reconstruct a $\phi_{\mathrm{sk}}: \mathrm{sk}(\mathbb{G}_n) \to \mathrm{sk}(Z)$ which produces it, again by setting
\begin{eq*} \begin{array}{rll}
		\mathrm{sk}(Z) & = & \mathbb{Z}^n \times \mathrm{B} \, \mathrm{colim}(D_n) \\
		a^{\mathrm{sk}(Z)}_{s, s', s''} & = & j_{s \boxtimes s' \boxtimes s''}(a^{\mathrm{sk}(\mathbb{G}_n)}_{s, s', s''}) \\
		l^{\mathrm{sk}(Z)}_s & = & j_s(l^{\mathrm{sk}(\mathbb{G}_n)}_s) \\
		r^{\mathrm{sk}(Z)}_s & = & j_s(r^{\mathrm{sk}(\mathbb{G}_n)}_s) \\
		\mu^\chi_{s, s'} & = & \mathrm{id}_{s \boxtimes s'} \\
		\eta^\chi & = & \mathrm{id}_{R(I)}
		\end{array}
\end{eq*} 
\end{proof}

With this proposition proven, the results in this chapter now collectively describe how to construct free $\mathrm{E}G$-algebras on $n$ invertible objects. Since the argument is obviously arranged in a rather piecemeal fashion, it would be best to restate the general conclusion all in one place.

\begin{thm}\label{freeinvalg} Let $\mathbb{G}_n$ be the free $\mathrm{E}G$-algebra on $n$ objects. Choose a skeleton of $\mathbb{G}_n$ and an equivalence $\rho$ between the two, and use them to define the diagram $D_n$ as in \cref{Ddef}, with colimit $(\mathrm{colim}(D_n), j)$. Also do the same with $\mathbb{G}_{2n}$, and let $u: \mathrm{colim}(D_{2n}) \to \mathrm{colim}(D_n)$ be the unique group homomorphism induced by the cocone of maps
\begin{eq*} \begin{tikzcd} \mathbb{G}_{2n}(w, w) \ar[r, "j + j^*"] & \mathrm{colim}(D_n) \end{tikzcd} \end{eq*}
Lastly, denote by $q_n$ the the quotient of abelianisation $\mathbb{Z}^{\ast n} \to \mathbb{Z}^n$, and define the function $h: \mathbb{Z}^{\ast n} \to \mathbb{N}^{\ast 2n}$ using minimal generator decompositions:
\begin{eq*} h(z_i) \, = \, z_i, \quad \quad h(z_i^*) \, = \, z_{i+n}, \quad \quad d\big( \, h(w) \, \big) \, = \, h\big( \, d(w) \, \big) \end{eq*}
Then the free $\mathrm{E}G$-algebra on $n$ invertible objects, $L\mathbb{G}_n$, is the algebra described by
\begin{eq*}\begin{array}{rll}
		& & \\
		\mathrm{Ob}(L\mathbb{G}_n) & = & \mathbb{Z}^{\ast n} \\
		& & \\
		 L\mathbb{G}_n(w, w') & = & \begin{cases}
     	  		\mathrm{colim}(D_n) & \quad \text{if} \quad q_n(w) \, = \, q_n(w') \\
      			\emptyset & \quad \text{otherwise}
			\end{cases} \\
		& & \\
		\alpha_{L\mathbb{G}_n}( \, e \, ; \, f_1, ..., f_m \, ) & = & f_1 \cdot ... \cdot f_m \\
		& & \\
		\alpha_{L\mathbb{G}_n}( \, g \, ; \, \mathrm{id}_{w_1}, ..., \mathrm{id}_{w_m} \, ) & = & u j_{q_{2n}(w)}\big( \, \rho_{w'}^{-1} \alpha_{\mathbb{G}_{2n}}( \, g \, ; \, \mathrm{id}_{h(w_1)}, ..., \mathrm{id}_{h(w_m)} \, ) \rho_{w} \, \big) \\
		& &
		\end{array}		
\end{eq*}
Note that this means that the underlying monoidal category of $L\mathbb{G}_n$ is
\begin{eq*}L\mathbb{G}_n \quad = \quad \mathrm{B} \, \mathrm{colim}(D_n) \times \bigsqcup_{w \in \mathbb{Z}^n} \mathrm{E}\big( \, q_{\mathbb{Z}^{\ast n}}^{-1}(w) \, \big) \end{eq*}
with the monoidal product inherited from $\mathbb{Z}^{\ast n}$.
\end{thm}
\begin{proof}
\cref{colimD} details the initial object of $C_{\mathrm{sk}}(\mathbb{G}_n)$. \cref{initialsk} tells us that this is this one of the functors $\phi_{\mathrm{sk}}$ obtained from the initial object $\phi_n$ of $C(\mathbb{G}_n)^{\mathrm{mn}}$ by applying \cref{factor1}. The other one, $\phi_{\mathrm{po}}$, is given by \cref{polem}. \cref{pullback} then lets us recover the target of $\phi_n$ as the pullback
\begin{eq*} \begin{array}{rll}
		 Z & \cong & \big( \, \mathbb{Z}^n \times \mathrm{B} \, \mathrm{colim}(D_n) \, \big) \times_{\mathbb{Z}^n} \Big( \, \bigsqcup_{\mathbb{Z}^n} \mathrm{E}\big( \, q_{\mathbb{Z}^{\ast n}}^{-1}(w) \, \big) \, \Big) \\
		& = & \mathrm{B} \, \mathrm{colim}(D_n) \times \bigsqcup_{\mathbb{Z}^n} \mathrm{E}\big( \, q_{\mathbb{Z}^{\ast n}}^{-1}(w) \, \big)
		\end{array}
\end{eq*}
and from Propositions \ref{initial}, \ref{initialab} and \ref{initialmon} we can conclude that this target is indeed the free algebra $L\mathbb{G}_n$. Similarly, after making any choice of skeleton for $\mathbb{G}_n$, \cref{factor2} can combine $\phi_{\mathrm{sk}}$ and $\phi_{\mathrm{po}}$ into $\phi$ by means of a pullback:
\begin{eq*}\begin{array}{rrrll}
		\phi_n & : & \mathbb{G}_n & \to & \mathrm{B} \, \mathrm{colim}(D_n) \times \bigsqcup_{\mathbb{Z}^n} \mathrm{E}\big( \, q_{\mathbb{Z}^{\ast n}}^{-1}(w) \, \big) \\
		& : & w & \mapsto & w \\
		& : & f: w \to w' & \mapsto & \Big( \, j_{q_{\mathbb{Z}^{\ast n}}(w)}\big( \, (\rho^{\mathbb{G}_n}_{w'})^{-1} f \rho^{\mathbb{G}_n}_{w} \, \big), \, w \to w' \, \Big)
		\end{array}
\end{eq*}
Finally, we can recontruct the action of $L\mathbb{G}_n$ using \cref{initialact}. Specifically, let $\phi'_n: \mathbb{G}_{2n} \to L\mathbb{G}_n$ be the composite
\begin{eq*} \begin{tikzcd}
\mathbb{G}_{2n} \ar[r, "\sim"] & \mathbb{G}_n + \mathbb{G}_n \ar[r, "\phi_n + \phi_n"] & L\mathbb{G}_n + L\mathbb{G}_n \ar[r, "\mathrm{id} + \delta"] & L\mathbb{G}_n + L\mathbb{G}_n \ar[r, "i + i"] & L\mathbb{G}_n
\end{tikzcd} \end{eq*}
On generating objects, this functor acts by
\begin{eq*}\begin{array}{rllll}
		\phi'_n(z_i) & = &
			\begin{cases}
       				z_i & \quad \text{if} \quad 1 \leq i \leq n \\
      				z_{i-n}^* & \quad \text{if} \quad n+1 \leq i \leq 2n \\
			\end{cases}
		\end{array}
\end{eq*}
and so we have $\phi'_n h(w) = w$ for any element $w$ of $\mathbb{Z}^{\ast n}$. Moreover, $\phi'_n$ is an object of $(\mathbb{G}_{2n} \downarrow (\_)_{\mathrm{inv}})$, and so if $\phi_{2n}$ is the initial object of that category then there must exist some $u: L\mathbb{G}_{2n} \to L\mathbb{G}_n$ such that $\phi'_n = u \circ \phi_{2n}$. The values that this $u$ takes on objects are given by
\begin{eq*} u(w) \, = \, u \phi_{2n}(w) \, = \, \phi'_n(w) \end{eq*}
and its values on morphisms are determined by the fact that the maps
\begin{eq*} \begin{tikzcd} \mathbb{G}_{2n}(w, w) \ar[r, "\sim"] & (\mathbb{G}_n + \mathbb{G}_n)(w, w) \ar[r, "j + j"] & \mathrm{colim}(D_n) + \mathrm{colim}(D_n) \ar[r, "\mathrm{id} + \_^*"] & \mathrm{colim}(D_n) \end{tikzcd} \end{eq*}
form a cocone over $D_{2n}$. Therefore,
\begin{eq*} \begin{array}{rll}
		\alpha_{L\mathbb{G}_n}( \, g \, ; \, \mathrm{id}_{w_1}, ..., \mathrm{id}_{w_m} \, ) & = & \alpha_{L\mathbb{G}_n}( \, g \, ; \, \mathrm{id}_{\phi'_n h(w_1)}, ..., \mathrm{id}_{\phi'_n h(w_m)} \, ) \\
		& = & \phi'_n \big( \, \alpha_{\mathbb{G}_{2n}}( \, g \, ; \, \mathrm{id}_{h(w_1)}, ..., \mathrm{id}_{h(w_m)} \, ) \, \big) \\
		& = & u \phi_{2n} \big( \, \alpha_{\mathbb{G}_{2n}}( \, g \, ; \, \mathrm{id}_{h(w_1)}, ..., \mathrm{id}_{h(w_m)} \, ) \, \big) \\
		& = & u j_{q_{2n}(w)}\big( \, \rho^{-1} \alpha_{\mathbb{G}_{2n}}( \, g \, ; \, \mathrm{id}_{h(w_1)}, ..., \mathrm{id}_{h(w_m)} \, ) \rho_{w} \, \big)
		\end{array}
\end{eq*}
\end{proof}

\subsection{Examples}

With \cref{freeinvalg} proven we can now finally achieve the primary goal of this chapter --- to describe the free braided monoidal category on $n$ invertible objects. In addition, this section will provide a few other simple applications of the theorem, in an effort to build up to the main result more gently. The definition of $L\mathbb{G}_n$ given in \ref{freeinvalg} is after all a little difficult to parse on first reading, because of the fairly abstract way it is presented, and hopefully the following concrete examples should allow the braided case to be properly understood.

\begin{prop} Let $G$ be any action operad, and denote by $G(\infty)$ the colimit of the sequence 
\begin{eq*} G(0) \, \hookrightarrow \, G(1) \, \hookrightarrow G(2) \, \hookrightarrow \, ... \end{eq*}
 Then the free $\mathrm{E}G$-algebra on a single invertible object is the category
\begin{eq*} L\mathbb{G}_1 \quad = \quad \mathrm{B}G(\infty)^{\mathrm{ab}} \times \mathbb{Z} \end{eq*}
equipped with the action
\begin{eq*}\alpha_{L\mathbb{G}_1}( \, g \, ; \, f_1, ..., f_m \, ) \, = \, g \cdot f_1 \cdot ... \cdot f_m \end{eq*}
\end{prop}
\begin{proof}
Each object of $\mathbb{G}_1$ is isomorphic only to itself, and so it is its own skeleton. We are therefore free to choose $\rho$ to be the identity transformation, in which case the product $\boxtimes$ in our `skeleton' will just be the normal $\otimes$. The diagram $D_1$ from \cref{Ddef} then has edges
\begin{eq*}\begin{tikzcd} \mathbb{G}_1(n) \ar[r, "\, \_ \, \otimes \mathrm{id}_m"] & \mathbb{G}_1(n + m), & \mathbb{G}_1(n) \ar[r, "\mathrm{id}_m \otimes \, \_ \,"] & \mathbb{G}_1(m+n)  \end{tikzcd}\end{eq*}
for each $m, n \in \mathbb{N}$. Given the structure of $\mathbb{G}_1$, these are really
\begin{eq*}\begin{tikzcd} G(n) \ar[r, "\, \_ \, \otimes \mathrm{id}_m"] & G(n + m), & \mathbb{G}_1(n) \ar[r, "\mathrm{id}_m \otimes \, \_ \,"] & G(m+n)  \end{tikzcd}\end{eq*}
which are all composites of the specific cases
\begin{eq*}\begin{tikzcd} G(n) \ar[r, "\, \_ \, \otimes \mathrm{id}_1"] & G(n + 1), & \mathbb{G}_1(n) \ar[r, "\mathrm{id}_1 \otimes \, \_ \,"] & G(n+1)  \end{tikzcd}\end{eq*}
Moreover, since we know that $\mathrm{colim}(D_1)$ is always an abelian group, and each of the above pairs has the same image under abelianisation, we can simply choose one of them to be the canonical inclusion $G(n) \hookrightarrow G(n+1)$, and then only abelianise after we've taken the colimit. Hence $L\mathbb{G}_1$ has the underlying category
\begin{eq*}\mathrm{Ob}(L\mathbb{G}_1) \, = \, \mathbb{Z}, \quad \quad L\mathbb{G}_1(m, n) \, = \, \begin{cases}
     	  		G(\infty)^{\mathrm{ab}} & \quad \text{if} \quad m \, = \, n \\
      			\emptyset & \quad \text{otherwise}
			\end{cases}
\end{eq*}
as required.

In order to find the action $\alpha_{L\mathbb{G}_1}$ as well, we now also need to consider some of the structure of $L\mathbb{G}_2$. In particular, 
\end{proof}

\subsection{The free algebra on $n$ weakly invertible objects}

Up until now, we've been working under the convention that by `invertible' objects we mean stictly invertible --- $x \otimes x^* = I$. As an additional exercise, we can ask ourselves how all of this would change if we permitted our objects to be only weakly invertible, that is $x \otimes x^* \cong I$. The situation is actually quite elegant, in that the effect of weakening in our objects can be offset completely by the effect of also weakening our algebra homomorphisms, such that we won't need to calculate any new free algebras other than those given by \cref{freeinvalg}. Before proving this though, we first to need to set out some definitions.

\begin{defn} Given an $\mathrm{E}G$-algebra $X$, we denote by $X_{\mathrm{wkinv}}$ the category whose
\begin{itemize}
\item objects are tuples $(x, x^*, \eta, \epsilon)$, where $x$ and $x^*$ are objects of $X$ and $\eta: I \to x^* \otimes x$ and $\epsilon : x \otimes x^* \to I$ are morphisms such that the composites
\begin{eq*} \xymatrix{
x \ar[r]^-{id \otimes \eta} & x \otimes x^* \otimes x \ar[r]^-{\epsilon \otimes id} & x &
x^* \ar[r]^-{\eta \otimes id} & x^* \otimes x \otimes x^* \ar[r]^-{id \otimes \epsilon} & x^* }
\end{eq*}
are identity morphisms.
\item maps $(f, f^*): (x, x^*, \eta_x, \epsilon_x) \to (y, y^*, \eta_y, \epsilon_y)$ are pairs $f: x \to y$, $f^* : x^* \to y^*$ of morphisms such that the diagrams
\begin{eq*} \xymatrix{
& I \ar[dl]_{\eta_x} \ar[dr]^{\eta_y} & & x \otimes x^* \ar[rr]^{f \otimes f^*} \ar[dr]_{\epsilon_x} & & y \otimes y^* \ar[dl]^{\epsilon_y} \\
x^* \otimes x \ar[rr]_{f^* \otimes f} & & y \otimes y^* & & I & }
\end{eq*}
commute.
\end{itemize}
\end{defn}

\begin{defn}\label{weakmonfunc} Let $(X, \alpha)$ and $(Y, \beta)$ be $\mathrm{E}G$-algebras. A \emph{weak $\mathrm{E}G$-algebra homorphism} between them is a weak monoidal functor $\psi: X \to Y$ such that all diagrams of the form
\begin{eq*} \xymatrix{
\psi( x_1 \otimes ... \otimes x_m) \ar[r]^-{\sim} \ar[d]_{\psi(\alpha(g; h_1, ... h_m))} &  \psi(x_1) \otimes ... \otimes \psi(x_m) \ar[d]^{\beta(g; \psi(h_1), ... \psi(h_m))} \\
\psi( y_{\pi(g)^{-1}(1)} \otimes ... \otimes y_{\pi(g)^{-1}(m)}) \ar[r]^-{\sim} &  \psi(y_{\pi(g)^{-1}(1)}) \otimes ... \otimes \psi(y_{\pi(g)^{-1}(m)}) }
\end{eq*}
commute.
\end{defn}

\begin{defn} We denote by $\mathrm{E}G\mathrm{Alg}_W$ the 2-category of $\mathrm{E}G$-algebras, weak $\mathrm{E}G$-algebra homomorphisms, and weak monoidal transformations.\end{defn}

Now we can properly express what we mean by the free algebras on weakly invertible objects being the same as those in the strict case.

\begin{thm} The algebra $L\mathbb{G}_n$ is also the free $\mathrm{E}G$-algebra on $n$ weakly invertible objects. Specifically, for any other $\mathrm{E}G$-algebra $X$ there is an equivalence of categories
\begin{eq*} \mathrm{E}G\mathrm{Alg}_W(L\mathbb{G}_n, X) \simeq (X_{\mathrm{wkinv}})^n \end{eq*}
\end{thm}
\begin{proof}
We begin by defining a functor $F : \mathrm{E}G\mathrm{Alg}_W(L\mathbb{G}_n, X) \to (X_{\mathrm{wkinv}})^n$. On weak maps, $F$ acts as 
\begin{eq*} F( \, \psi: L\mathbb{G}_n \to X \, ) = \big\{ \, ( \, \psi(z_i), \, \psi(z_i^*), \, I \xrightarrow{\sim} \psi(I) \xrightarrow{\sim} \psi(z_i^*)\psi(z_i), \, \psi(z_i)\psi(z_i^*) \xrightarrow{\sim} \psi(I) \xrightarrow{\sim} I \, ) \, \big\}_{i \in \{ 1, ..., n \} } \end{eq*}
where the $z_i$ are the generators of $\mathbb{Z}^{*n}$ and the isomorphisms are those given by $\psi$ being a weak moniodal functor. On weak monoidal transformations, $F$ acts as
\begin{eq*} F( \, \theta : \psi \to \chi \, ) = \big\{ \, ( \, \theta_{z_i}, \, \theta_{z_i^*} \, ) \, \big\}_{i \in \{ 1, ..., n \} }\end{eq*}
This choice does satisfy the condition on morphisms of $(X_{\mathrm{wkinv}})^n$, since we can build the required commuting diagrams out of smaller ones given by $\theta$ being a weak monoidal transfomation:
\begin{eq*} \xymatrix{
& I \ar[dl]_{\sim} \ar[dr]^{\sim} & & \psi(z_i) \otimes \psi(z_i^*) \ar[rr]^{\theta_{z_i} \otimes \theta_{z_i^*}} \ar[d]_{\sim} & & \chi(z_i) \otimes \chi(z_i^*) \ar[d]^{\sim} \\
\psi(I) \ar[d]_{\sim} \ar[rr]^{\theta_I} & & \chi(I) \ar[d]^{\sim} & \psi(I) \ar[dr]^{\sim} \ar[rr]^{\theta_I} & & \chi(I) \ar[dl]_{\sim} \\
\psi(z_i^*) \otimes \psi(z_i) \ar[rr]^{\theta_{z_i^*} \otimes \theta_{z_i}} & & \chi(z_i^*) \otimes \chi(z_i) & & I & }.
\end{eq*}

Now we need to check if $F$ is an equivalence of categories. First, let $\big\{ ( x_i, x_i^*, \eta_i, \epsilon_i ) \big\}_{i \in \{1, ..., n \} }$ be an arbitrary object of $(X_{\mathrm{wkinv}})^n$. We can construct a weak algebra map $\psi: L\mathbb{G}_n \to X$ from it as follows. Define
\begin{eq*} \psi(I) = I, \quad \psi(z_i) = x_i, \quad \psi(z_i^*) = x_i^* \end{eq*}
and choose the isomorphisms
\begin{eq*} \begin{array}{rllllll}
		\psi_I & : & I \to \psi(I) & = & \mathrm{id}_I & : & I \to I \\
		\psi_{z_i, z_i^*} & : & \psi(z_i) \otimes \psi(z_i^*) \to \psi(I) & = & \epsilon_i & : & x_i \otimes x_i^* \to I \\
		\psi_{z_i^*, z_i} & : & \psi(z_i^*) \otimes \psi(z_i) \to \psi(I) & = & \eta_i^{-1} & : & x_i^* \otimes x_i \to I
		\end{array} .
\end{eq*}
Then for any $w, w' \in \mathrm{Ob}(L\mathbb{G}_n)$ such that $d(w \otimes w') = d(w) \otimes d(w')$, where $d(-)$ is the minimal generator decomposition from \cref{mgd}, set 
\begin{eq*} \psi(w \otimes w') = \psi(w) \otimes \psi(w'), \quad \quad \psi_{w, w'} = \mathrm{id}_{\psi(w) \otimes \psi(w')} \end{eq*}
This is enough to determine the value of $\psi$ on all of the remaining objects, via successive decompositions. For the isomorphisms, first note that the ones we have already defined satisfy the associativity and unitality required of weak monoidal functors. Now consider some $w, w'$ with $d(w \otimes w') \neq d(w) \otimes d(w')$. The fact that they differ implies that tensoring $w$ with $w'$ causes some cancellation of inverses to occur where the end of one sequence meets the beginning of another. In particular, if we let $b$ be the last term in the minimal generator decomposition of $w$, and let $c = w'$, then we conclude that the length $d(b \otimes c)$ is smaller than the length of $d(c)$. Let $a$ be the product of the rest of $d(w)$, so that $a \otimes b = w$. Then we can use requirement for associativity,
\begin{eq*} \xymatrix{
\psi(a) \otimes \psi(b) \otimes \psi(c) \ar[rr]^{id \otimes \psi_{b, c}} \ar[d]_{\psi_{a, b} \otimes id} && \psi(a) \otimes \psi(b \otimes c) \ar[d]^{\psi_{a, b \otimes c}} \\
\psi(a \otimes b) \otimes \psi(c) \ar[rr]_{\psi_{a \otimes b, c}} && \psi(a \otimes b \otimes c) },
\end{eq*}
to define $\psi_{w, w'} = \psi{a\otimes b, c}$ in terms of three other isomorphisms that each have strictly smaller decompositions. Repeating this process will therefore eventually yield a definition in terms of our previous isomorphisms.

By \cref{allmapsaction}, every morphism in $L\mathbb{G}_n$ can be written as $\alpha(g; \mathrm{id}_{w_1}, ..., \mathrm{id}_{w_m})$ for some $g \in G(m)$, $w_i \in \mathbb{Z}^{*n}$. The action of $\psi$ on morphisms is thus determined by the diagram in \cref{weakmonfunc}, that is
\begin{eq*} \psi(\alpha(g; w_1, ... w_m)) \, = \, \psi_{\mathbf{w}_{\pi(g)^{-1}}} \circ \beta(\, g \, ; \, \mathrm{id}_{\psi(w_1)}, \, ..., \, \mathrm{id}_{\psi(w_m)}\, ) \circ \psi_{\mathbf{w}}^{-1}\end{eq*} 
However, morphisms do not have a unique representation of this form, so we must check that whenever we have different representations of the same morphism
\begin{eq*} \alpha(g; \mathrm{id}_{w_1}, ..., \mathrm{id}_{w_m}) = \alpha(g'; \mathrm{id}_{w_1'}, ..., \mathrm{id}_{w_{m'}'}) \end{eq*}
their diagrams give the same image under $\psi$. There are two cases to consider here;
\begin{eq*} \alpha(g; \mathrm{id}_{w_1}, ..., \mathrm{id}_{w_m}) = \alpha( \, g \otimes e_k \, ; \, \mathrm{id}_{w_1}, \, ..., \, \mathrm{id}_{w_m}, \, \mathrm{id}_{v_1}, \, ..., \, \mathrm{id}_{v_k} \, ) \end{eq*}
when $v_1 \otimes ... \otimes v_k = 0$, which comes from the edges of the colimit diagram $D_n$ in \cref{colimthm}; and
\begin{eq*} \begin{array}{rll}
		\alpha(g; \mathrm{id}_{w_1}, ..., \mathrm{id}_{w_m}) & = & \alpha(\, h \, ; \, \mathrm{id}_{w_1'}, \, ..., \, \mathrm{id}_{w_{m'}} \, ) \\
		&& \circ \, \, \alpha(\, j \, ; \, \mathrm{id}_{w_1''}, \, ..., \, \mathrm{id}_{w_{m''}''} \, ) \\
		&& \circ \, \, \alpha(\, h^{-1} \, ; \, \mathrm{id}_{w_1'}, \, ..., \, \mathrm{id}_{w_{m'}'} \, ) \\
		&& \circ \, \, \alpha(\, j^{-1} \, ; \, \mathrm{id}_{w_1''}, \, ..., \, \mathrm{id}_{w_{m''}''} \, ) \\
		& = & \mathrm{id}_{w_1 \otimes ... \otimes w_m} 
		\end{array}
\end{eq*}
for $ \alpha(\, h \, ; \, \mathrm{id}_{w_1'}, \, ..., \, \mathrm{id}_{w_{m'}} \, ), \alpha(\, j \, ; \, \mathrm{id}_{w_1''}, \, ..., \, \mathrm{id}_{w_{m''}''} \, ) \in \mathbb{G}_n(w_1 \otimes ... \otimes w_m,  w_1 \otimes ... \otimes w_m)$, which comes from the abelianisation of the vertices of $D_n$. All other ways for a morphism to have different representations must be generated by successive examples of these cases, since otherwise they wouldn't be coequalised by the colimit in \cref{colimthm}. In the first case we just have
\begin{eq*} \begin{array}{rl}
		& \psi( \, \alpha( \, g \otimes e_k \, ; \, \mathrm{id}_{w_1}, \, ..., \, \mathrm{id}_{w_m}, \, \mathrm{id}_{v_1}, \, ..., \, \mathrm{id}_{v_k} \, ) \, ) \\
		= & \psi_{\mathbf{w}_{\pi(g)^{-1}}, \mathbf{v}} \circ \beta(\, g \otimes e_k \, ; \, \mathrm{id}_{\psi(w_1)}, \, ..., \, \mathrm{id}_{\psi(w_m)}, \, \mathrm{id}_{\psi(v_1)}, \, ..., \, \mathrm{id}_{\psi(v_k)} \, ) \circ \psi_{\mathbf{w}, \mathbf{v}}^{-1} \\
		= & \big( \psi_{\mathbf{w}_{\pi(g)^{-1}}} \otimes \psi_{\mathbf{v}} \big) \circ \big( \beta( g ; \mathrm{id}_{\psi(w_1)}, ..., \mathrm{id}_{\psi(w_m)}) \otimes \mathrm{id}_{\psi(\mathbf{v})} \big) \circ \big( \psi_{\mathbf{w}}^{-1} \otimes \psi_{\mathbf{v}}^{-1} \big) \\
		= & \big( \psi_{\mathbf{w}_{\pi(g)^{-1}}} \circ \beta( g ; \mathrm{id}_{\psi(w_1)}, ..., \mathrm{id}_{\psi(w_m)}) \circ \psi_{\mathbf{w}}^{-1} \big) \otimes \big( \psi_{\mathbf{v}} \circ \mathrm{id}_{\psi(\mathbf{v})} \circ \psi_{\mathbf{v}}^{-1} \big) \\
		= & \psi_{\mathbf{w}_{\pi(g)^{-1}}} \circ \beta( g ; \mathrm{id}_{\psi(w_1)}, ..., \mathrm{id}_{\psi(w_m)}) \circ \psi_{\mathbf{w}}^{-1} \\
		=& \psi( \, \alpha(g; \mathrm{id}_{w_1}, ..., \mathrm{id}_{w_m}) \, )
		\end{array}
\end{eq*}
as required. The second case is more subtle. We begin by expanding
\begin{eq*} \begin{array}{rl}
		& \psi( \, \alpha( \, g \, ; \, \mathrm{id}_{w_1}, \, ..., \, \mathrm{id}_{w_m} \, ) \\
		= & \psi( \, \alpha(\, h \, ; \, \mathrm{id}_{w_1'}, \, ..., \, \mathrm{id}_{w_{m'}} \, ) \, ) \\
		& \circ \, \, \psi( \, \alpha(\, j \, ; \, \mathrm{id}_{w_1''}, \, ..., \, \mathrm{id}_{w_{m''}''} \, ) \, ) \\
		& \circ \, \, \psi( \, \alpha(\, h^{-1} \, ; \, \mathrm{id}_{w_1'}, \, ..., \, \mathrm{id}_{w_{m'}'} \, ) \, ) \\
		&\circ \, \, \psi( \, \alpha(\, j^{-1} \, ; \, \mathrm{id}_{w_1''}, \, ..., \, \mathrm{id}_{w_{m''}''} \, ) \, ) \\
		= & \psi_{\mathbf{w'}} \circ \beta(\, h \, ; \, \mathrm{id}_{\psi(w_1')}, \, ..., \, \mathrm{id}_{\psi(w_{m'})} \, ) \circ \psi_{\mathbf{w'}}^{-1} \\
		& \circ \, \, \psi_{\mathbf{w''}} \circ\beta(\, j \, ; \, \mathrm{id}_{\psi(w_1'')}, \, ..., \, \mathrm{id}_{\psi(w_{m''}'')} \, ) \circ \psi_{\mathbf{w''}}^{-1} \\
		& \circ \, \, \psi_{\mathbf{w'}} \circ \beta(\, h^{-1} \, ; \, \mathrm{id}_{\psi(w_1')}, \, ..., \, \mathrm{id}_{\psi(w_{m'}')} \, ) \circ \psi_{\mathbf{w'}}^{-1}  \\
		&\circ \, \, \psi_{\mathbf{w''}} \circ \beta(\, j^{-1} \, ; \, \mathrm{id}_{\psi(w_1'')}, \, ..., \, \mathrm{id}_{\psi(w_{m''}'')} \, ) \circ \psi_{\mathbf{w''}}^{-1} \\
		\end{array}
\end{eq*}
Here the objects $w_i, w_i', w_i''$ are all in $\mathbb{G}_n \subseteq L\mathbb{G}_n$, and so we know their minimal generator decompositions are also in $\mathbb{G}_n$. It follows that $d(w_i \otimes w_j) = d(w_i) \otimes d(w_j)$ for all $i,j$, and hence by our definition of $\psi$ we have $\psi(w_i \otimes w_j) = \psi(w_i) \otimes \psi(w_j)$ and also $\psi_{\mathbf{w}_{\sigma}} = id$ for any permuation $\sigma$ --- and the same for $\mathbf{w'}$ and $\mathbf{w''}$. Also, note that since we are working in $\mathbb{G}_n(w_1 \otimes ... \otimes w_m,  w_1 \otimes ... \otimes w_m)$, all of the action morphisms in the above composite have the same source and target, $\psi(w_1 \otimes ...\otimes w_m)$. This object is weakly invertible, because each of the $w_i$ are invertible. However, the automorphisms of any weakly invertible object are isomorphic to the automorphisms of the unit object, as in the proof of \cref{zerotree}, and hence form an abelian group, by an Eckmann-Hilton argument like in the proof of \cref{colimthm}. Therefore we may permute these action morphisms freely, and so
\begin{eq*} \begin{array}{rl}
& \psi( \, \alpha( \, g \, ; \, \mathrm{id}_{w_1}, \, ..., \, \mathrm{id}_{w_m} \, ) \\
		= & \beta(\, h \, ; \, \mathrm{id}_{\psi(w_1')}, \, ..., \, \mathrm{id}_{\psi(w_{m'})} \, ) \\
		& \circ \, \, \beta(\, h^{-1} \, ; \, \mathrm{id}_{\psi(w_1')}, \, ..., \, \mathrm{id}_{\psi(w_{m'}')} \, )  \\
		& \circ \, \, \beta(\, j \, ; \, \mathrm{id}_{\psi(w_1'')}, \, ..., \, \mathrm{id}_{\psi(w_{m''}'')} \, ) \\
		& \circ \, \, \beta(\, j^{-1} \, ; \, \mathrm{id}_{\psi(w_1'')}, \, ..., \, \mathrm{id}_{\psi(w_{m''}'')} \, ) \\
		= & \mathrm{id}_{\psi(w_1) \otimes ... \otimes \psi(w_m)} \\
		= & \psi_{\mathbf{w}} \circ \beta(\, e_m \, ; \, \mathrm{id}_{\psi(w_1)}, \, ..., \, \mathrm{id}_{\psi(w_{m})} \, ) \circ \psi_{\mathbf{w}}^{-1}
		\end{array}
\end{eq*}
as required.

With $\psi$ now fully defined, notice that
\begin{eq*} \begin{array}{rll}
		F(\psi) & = & \big\{ \, ( \, \psi(z_i), \, \psi(z_i^*), \, I \xrightarrow{\sim} \psi(I) \xrightarrow{\sim} \psi(z_i^*)\psi(z_i), \, \psi(z_i)\psi(z_i^*) \xrightarrow{\sim} \psi(I) \xrightarrow{\sim} I \, ) \, \big\}_{i \in \{ 1, ..., n \} } \\
		& = & \big\{ \, ( \, x_i, \, x_i^*, \, \eta_i, \, \epsilon_i \, ) \, \big\}_{i \in \{ 1, ..., n \} } \\
		\end{array}
\end{eq*}
which was our arbitrary object in $(X_{\mathrm{wkinv}})^n$. Therefore, $F$ is surjective on objects.

Next, choose an arbitrary monoidal transformation $\theta : \psi \to \chi$ from $\mathrm{E}G\mathrm{Alg}_W(L\mathbb{G}_n, X)$. By naturality, for any $w, w' \in \mathrm{Ob}(L\mathbb{G}_n)$ we have that
\begin{eq*} \xymatrix{
\psi(w) \otimes \psi(w') \ar[r]^-{\sim} \ar[d]_{\theta_w \otimes \theta_{w'}} & \psi(w \otimes w') \ar[d]^{\theta_{w \otimes w'}} \\
\chi(w) \otimes \chi(w') \ar[r]^-{\sim} & \chi(w \otimes w') }
\end{eq*}
or equivalently, $\theta_{w \otimes w'} = \chi_{w, w'} \circ (\theta_w \otimes \theta_{w'}) \circ \psi_{w, w'}^{-1}$. It follows from this that the components of $\theta$ are generated by the components on the generators of $\mathrm{Ob}(L\mathbb{G}_n)$, namely $\{ \, ( \, \theta_{z_i}, \, \theta_{z_i^*} \, ) \, \}_{i \in \{ 1, ..., n \} }$. But this is just $F(\theta)$, and thus any monoidal transformation $\theta$ is determined uniquely by its image under $F$, or in other words $F$ is faithful.

Finally, let $\psi, \chi$ be objects of $\mathrm{E}G\mathrm{Alg}_W(L\mathbb{G}_n, X)$, and choose an arbitrary map $\{ \, ( \, f_i, \, f^*_i \, ) \, \}_{i \in \{ 1, ..., n \} } : F(\psi) \to F(\chi)$ from $(X_{\mathrm{wkinv}})^n$. We can use this to construct a monoidal transformation $\theta : \psi \to \chi$ via the reverse of process we just used. Specifically, if we define
\begin{eq*} \theta_I = \chi_I \circ \psi_I^{-1}, \quad \quad \theta_{z_i} =  f_i, \quad \quad \theta_{z_i^*} = f_i^*\end{eq*}
then these will automatically form the naturality squares
\begin{eq*} \xymatrix{
\psi(z_i) \otimes \psi(z_i^*) \ar[rr]^-{\psi_{z_i, z_i^*}} \ar[dd]_{f_i \otimes f_i^*} & & \psi(I) \ar[d]^{\psi_I^{-1}} & \psi(z_i^*) \otimes \psi(z_i) \ar[rr]^-{\psi_{z_i^*, z_i}} \ar[dd]_{f_i^* \otimes f_i} & & \psi(I) \ar[d]^{\psi_I^{-1}} \\
& & I \ar[d]^{\chi_I} & & & I \ar[d]^{\chi_I}\\
\chi(z_i) \otimes \chi(z_i^*) \ar[rr]^-{\chi_{z_i, z_i^*}} & & \chi(I) & \chi(z_i^*) \otimes \chi(z_i) \ar[rr]^-{\chi_{z_i^*, z_i}} & & \chi(I)}
\end{eq*}
since these are just the conditions for $\{ \, ( \, f_i, \, f^*_i \, ) \, \}_{i \in \{ 1, ..., n \} }$ to be a map $F(\psi) \to F(\chi)$ in $(X_{\mathrm{wkinv}})^n$. Repeatedly applying the naturality condition $\theta_{w \otimes w'} = \chi_{w, w'} \circ (\theta_w \otimes \theta_{w'}) \circ \psi_{w, w'}^{-1}$ will then generate all of the other components of $\theta$, in a way that clearly satisfies naturality. Thus we have a well-defined monoidal transformation $\theta : \psi \to \chi$, and applying $F$ to it gives
\begin{eq*} \begin{array}{rll}
		F(\theta) & = & \big\{ \, ( \, \theta_{z_i}, \, \theta_{z_i^*} \, ) \, \big\}_{i \in \{ 1, ..., n \} } \\
		& = & \big\{ \, ( \, f_i, \, f_i^* \, ) \, \big\}_{ i \in \{ 1, ..., n \} },
		\end{array}
\end{eq*}
our arbitrary map. Therefore $F$ is full and, putting this together with the previous results, is an equivalence of categories.
\end{proof}

\begin{thebibliography}{9}

\bibitem{graphicalmon}
Peter Selinger.
\textit{A survey of graphical languages for monoidal categories}.
http://www.mscs.dal.ca/~selinger/papers/graphical.pdf

\bibitem{operadborel} 
Nick Gurski. 
\textit{Operads, tensor products, and the categorical Borel construction}. 
 arXiv:1508.04050 [math.CT].

\bibitem{bct}
Tom Leinster.
\textit{Basic Category Theory}
Cambridge Studies in Advanced Mathematics, Vol. 143, Cambridge University Press, Cambridge, 2014.
https://arxiv.org/pdf/1612.09375v1.pdf

\bibitem{eckhil}
Eckmann, B.; Hilton, P. J. 
\textit{Group-like structures in general categories. I. Multiplications and comultiplications}
Mathematische Annalen, 145 (3): 227–255

\end{thebibliography}


\end{document}