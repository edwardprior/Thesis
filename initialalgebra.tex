\chapter{Free invertible algebras as initial objects}
\label{initialalgebra}

In this chapter we will start to consider how to construct free $\mathrm{E}G$-algebras on some number of invertible objects. Specifically, we will begin by showing that such algebras are the initial objects of a particular comma category, in accordance with some well known properties of adjunctions and their units. Using this initial object prespective will allow us to recover all of the data associated with the objects of a given free invertible algebra --- what those objects are, how they act under tensor product, and which pairs of objects form the source and target of at least one morphism. Unfortunately, a concrete decription of the morphisms themselves will ultimately remain elusive. We can get tantalisingly closer though, and an examiniation of the exact way that this method fails will provide the neccessary insight to motivate a more successful approach in \cref{coeqalgebra}.

\section{The free algebra on $n$ invertible objects}

We saw in \cref{freealg} that the existence of a free $\mathrm{E}G$-algebra on $n$ objects can be proven by taking the left adjoint of a 2-functor which forgets about the algebra structure. Now we want to extend this idea into the realm of algebras on invertible objects. For the analogous approach, we will need to find a new 2-functor that lets us forget about non-invertible objects, and then hopefully we can find its left adjoint too, and use it to freely add inverses to $\mathbb{G}_n$. First though, we need to make this concept of `forgetting non-invertible objects' a little more precise.

\begin{defn} Given an $\mathrm{E}G$-algebra $X$, we denote by $X_{\mathrm{inv}}$ the sub-$\mathrm{E}G$-algebra containing all invertible objects in $X$ and the isomorphisms between them. \end{defn}

Note that this is indeed a well-defined $\mathrm{E}G$-algebra. If $x_1, ..., x_m$ are invertible objects with inverses $x_1^*, ..., x_m^*$, then $\alpha(g; x_1, ..., x_m)$ is an invertible object with inverse $\alpha(g; x_m^*, ..., x_1^*)$, since 
\begin{eq*} \begin{array}{ll}
		& \alpha(g; x_1, ..., x_m) \, \otimes \, \alpha(g; x_m^*, ..., x_1^*) \\
		= & \big( \, x_{\pi(g)^{-1}(1)} \otimes ... \otimes x_{\pi(g)^{-1}(m)} \, \big) \, \otimes  \, \big( \, x_{\pi(g)^{-1}(m)}^* \otimes ... \otimes x_{\pi(g)^{-1}(1)}^* \, \big) \\
		= & I \\
		& \\
		& \alpha(g; x_m^*, ..., x_1^*) \, \otimes \, \alpha(g; x_1, ..., x_m) \\
		= & \big( \, x_{\pi(g)^{-1}(m)}^* \otimes ... \otimes x_{\pi(g)^{-1}(1)}^* \, \big) \, \otimes \, \big( \, x_{\pi(g)^{-1}(1)} \otimes ... \otimes x_{\pi(g)^{-1}(m)} \, \big) \\
		= & I
		\end{array}
\end{eq*}
Likewise, if $f_1, ..., f_m$ are isomorphisms from invertible objects $x_1, ..., x_m$ to invertible objects $y_1, ..., y_m$, then $\alpha(g; f_1, ..., f_m)$ is a map from the invertible object $\alpha(g; x_1, ..., x_m)$ to the invertible object $\alpha(g; y_1, ..., y_m)$, and it has an inverse $\alpha(g^{-1}; f_{\pi(g)(1)}^{-1}, ..., f_{\pi(g)(m)}^{-1})$, since
\begin{eq*} \begin{array}{ll}
		& \alpha\big( \, g^{-1} \, ; \, f_{\pi(g)(1)}^{-1}, \, ..., \, f_{\pi(g)(m)}^{-1} \, \big) \, \circ \, \alpha( \, g \, ; \, f_1, ..., f_m \,) \\
		= & \alpha\big( \, g^{-1}g \, ; \, f_1^{-1} f_1, \, ..., \, f_m^{-1} f_m \, \big) \\
		= & \mathrm{id}_{x_1 \otimes ... \otimes x_m} \\
		& \\
		& \alpha( \, g \, ; \, f_1, ..., f_m \,) \, \circ \, \alpha\big( \, g^{-1} \, ; \, f_{\pi(g)(1)}^{-1}, \, ..., \, f_{\pi(g)(m)}^{-1} \, \big) \\
		= & \alpha\big( \, gg^{-1} \, ; \, f_{\pi(g)(1)} f_{\pi(g)(1)}^{-1}, \, ..., \, f_{\pi(g)(m)} f_{\pi(g)(m)}^{-1} \, \big) \\
		= & \mathrm{id}_{y_{\pi(g)(1)} \otimes ... \otimes y_{\pi(g)(m)}}
		\end{array}
\end{eq*}
Clearly then, $X_{\mathrm{inv}}$ is the correct algebra for our new forgetful 2-functor to send $X$ to. Knowing this, we can contruct the rest of the functor fairly easily.

\begin{prop} \label{invprop} The assignment $X \mapsto X_{\mathrm{inv}}$ can be extended to a 2-functor $(\_)_{\mathrm{inv}}: \mathrm{E}G\mathrm{Alg}_S \to \mathrm{E}G\mathrm{Alg}_S$.
\end{prop}
\begin{proof}
Let $F: X \to Y$ be a (strict) map of $\mathrm{E}G$-algebras. If $x$ is an invertible object in $X$ with inverse $x^*$, then $F(x)$ is an invertible object in $Y$ with inverse $F(x^*)$, by
\begin{eq*} F(x) \otimes F(x^*) = F(x \otimes x^*) = F(I) = I \end{eq*}
\begin{eq*} F(x^*) \otimes F(x) = F(x^* \otimes x) = F(I) = I \end{eq*}
Since $F$ sends invertible objects to invertible objects, it will also send isomorphisms of invertible objects to isomorphisms of invertible objects. In other words, the map $F: X \to Y$ can be restricted to a map $F_{\mathrm{inv}} : X_{\mathrm{inv}} \to Y_{\mathrm{inv}}$. Moreover, we have that
\begin{eq*} (F \circ G)_{\mathrm{inv}}(x) = F \circ G(x) = F_{\mathrm{inv}} \circ G_{\mathrm{inv}}(x) \end{eq*}
\begin{eq*} (F \circ G)_{\mathrm{inv}}(f) = F \circ G(f) = F_{\mathrm{inv}} \circ G_{\mathrm{inv}}(f) \end{eq*}
and so the assignment $F \mapsto F_{\mathrm{inv}}$ is clearly functorial. Next, let $\theta : F \Rightarrow G$ be an $\mathrm{E}G$-monoidal natural transformation. Choose an invertible object $x$ from $X$, and consider the component map of its inverse, $\theta_{x^*} : F(x^*) \to G(x^*)$. Since $\theta$ is monoidal, we have $\theta_{x^*} \otimes \theta_x = \theta_I = I$ and $\theta_x \otimes \theta_{x^*} = I$, or in other words that $\theta_{x^*}$ is the monoidal inverse of $\theta_x$. We can use this fact to construct a compositional inverse as well, namely $\mathrm{id}_{F(x)} \otimes \theta_{x^*} \otimes \mathrm{id}_{G(x)}$, which can be seen as follows:
\begin{eq*}  \begin{array}{rll}
		\big( \mathrm{id}_{F(x)} \otimes \theta_{x^*} \otimes \mathrm{id}_{G(x)} \big)  \circ \theta_x & = & \theta_x \otimes \theta_{x^*} \otimes \mathrm{id}_{G(x)} \\
		& = &  \mathrm{id}_{G(x)} \\
		&& \\
		\theta_x \circ  \big( \mathrm{id}_{F(x)} \otimes \theta_{x^*} \otimes \mathrm{id}_{G(x)} \big) & = & \mathrm{id}_{F(x)} \otimes \theta_{x^*} \otimes \theta_x \\
		& = &  \mathrm{id}_{F(x)} \\
		\end{array} 
\end{eq*}
Therefore, we see that all the components of our transformation on invertible objects are isomorphisms, and hence we can define a new transformation $\theta_{\mathrm{inv}}: F_{\mathrm{inv}} \Rightarrow G_{\mathrm{inv}}$ whose components are just $(\theta_{\mathrm{inv}})_x = \theta_x$. The assignment $\theta \mapsto \theta_{\mathrm{inv}}$ is also clearly functorial, and thus we have a complete 2-functor $(\_)_{\mathrm{inv}}: \mathrm{E}G\mathrm{Alg}_S \to \mathrm{E}G\mathrm{Alg}_S$.
\end{proof}

\begin{prop} \label{invadj} The 2-functor $(\_)_{\mathrm{inv}}: \mathrm{E}G\mathrm{Alg}_S \to \mathrm{E}G\mathrm{Alg}_S$ has a left adjoint, $L: \mathrm{E}G\mathrm{Alg}_S \to \mathrm{E}G\mathrm{Alg}_S$.
\end{prop}
\begin{proof} To begin, consider the 2-monad $\mathrm{E}G(\_)$. This is a finitary monad, that is it preserves all filtered colimits, and it is a 2-monad over $\mathrm{Cat}$, which is locally finitely presentable. It follows from this that $\mathrm{E}G\mathrm{Alg}_S$ is itself locally finitely presentable. Thus if we want to prove $(\_)_{\mathrm{inv}}$ has a left adjoint, we can use the Adjoint Functor Theorem for locally finitely presentable categories, which amounts to showing that $(\_)_{\mathrm{inv}}$ preserves both limits and filtered colimits.
\begin{itemize}
\item Given an indexed collection of $\mathrm{E}G$-algebras $X_i$, the $\mathrm{E}G$-action of their product $\prod X_i$ is defined componentwise. In particular, this means that the tensor product of two objects in $\prod X_i$ is just the collection of the tensor products of their components in each of the $X_i$. An invertible object in $\prod X_i$ is thus simply a family of invertible objects from the $X_i$ --- in other words, $(\prod X_i)_{\mathrm{inv}} = \prod (X_i)_{\mathrm{inv}}$.
\item Given maps of $\mathrm{E}G$-algebras $F: X \to Z$, $G : Y \to Z$, the $\mathrm{E}G$-action of their pullback $X \times_Z Y$ is also defined componentwise. It follows that an invertible object in $X \times_Z Y$ is just a pair of invertible objects $(x, y)$ from $X$ and $Y$, such that $F(x) = G(y)$. But this is the same as asking for a pair of objects $(x, y)$ from $X_{\mathrm{inv}}$ and $Y_{\mathrm{inv}}$ such that $F_{\mathrm{inv}}(x) = G_{\mathrm{inv}}(y)$, and hence $(X \times_Z Y)_{\mathrm{inv}} = X_{\mathrm{inv}} \times_{Z_{\mathrm{inv}}} Y_{\mathrm{inv}}$.
\item Given a filtered diagram $D$ of $\mathrm{E}G$-algebras, the $\mathrm{E}G$-action of their colimit $\mathrm{colim}(D_n)$ is defined in the following way: use filteredness to find an algebra which contains (representatives of the classes of) all the things you want to act on, then apply the action of that algebra. In the case of tensor products this means that $[x]\otimes[y] = [x \otimes y]$, and thus an invertible object in $\mathrm{colim}(D_n)$ is just (the class of) an invertible object in one of the algebras of $D$. In other words, $\mathrm{colim}(D_n)_{\mathrm{inv}} = \mathrm{colim}(D_{\mathrm{inv}})$.
\end{itemize}
Preservation of products and pullbacks gives preservation of limits, and preservation of limits and filtered colimits gives the result.
\end{proof}

With this new 2-functor $L: \mathrm{E}G\mathrm{Alg}_S \to \mathrm{E}G\mathrm{Alg}_S$, we now have the ability to `freely add inverses to objects' in any $\mathrm{E}G$-algebra we want. The algebra $L\mathbb{G}_n$ is then a clear candidate for our free algebra on $n$ invertible objects, and indeed the proof of this is very simple.

\begin{thm} There exists a free $\mathrm{E}G$-algebra on $n$ invertible objects. Specifically, the algebra $L\mathbb{G}_n$ is such that for any other $\mathrm{E}G$-algebra $X$, we have an isomorphism of categories
\begin{eq*} \mathrm{E}G\mathrm{Alg}_S(L\mathbb{G}_n, X) \quad \cong \quad (X_{\mathrm{inv}})^n \end{eq*}
\end{thm}
\begin{proof}
Using the adjunction from \cref{invadj} along with the one from \cref{freealg}, we see that
\begin{eq*}\begin{array}{rll}
		 U(X_{\mathrm{inv}})^n & = & \mathrm{Cat}(\{z_1, ..., z_n\}, U(X_{\mathrm{inv}}) ) \\
		& \cong & \mathrm{E}G\mathrm{Alg}_S( F(\{z_1, ..., z_n\}), X_{\mathrm{inv}}) \\
		& \cong & \mathrm{E}G\mathrm{Alg}_S( LF(\{z_1, ..., z_n\}), X)
\end{array}
 \end{eq*}
As before, $X_{\mathrm{inv}}$ and $U(X_{\mathrm{inv}})$ are obviously isomorphic as categories, and so \( LF(\{z_1, ..., z_n\}) = L\mathbb{G}_n \) satisfies the requirements for the free algebra on $n$ invertible objects.
\end{proof}

\section{$L\mathbb{G}_n$ as an initial algebra}

We have now proven that a free $\mathrm{E}G$-algebra on $n$ invertible objects indeed exists. But this fact on its own is not very helpful. To be able to actually use the free algebra $L\mathbb{G}_n$, we need to know how to contruct it explicitly, in terms of its objects and morphisms. We could do this by finding a detailed characterisation of the 2-functor $L$, and then applying this to our explicit description of $\mathbb{G}_n$ from \cref{Gndef}. However, this would probably take far more effort than is required, since it would involve determining the behaviour of $L$ in many situtations that we aren't interested in. Also, we wouldn't be leveraging $\mathbb{G}_n$'s status as a free algebra to make the calculations any easier. We will try a different strategy instead, one that begins by noticing a special property of the functor $L$.

\begin{prop} \label{linveql} For any $\mathrm{E}G$-algebra $X$, we have $L(X)_{\mathrm{inv}} = L(X)$.
\end{prop}
\begin{proof}
From the definition of adjunctions, the isomorphisms
\begin{eq*}\mathrm{E}G\mathrm{Alg}_S(LX , Y) \quad \cong \quad \mathrm{E}G\mathrm{Alg}_S(X, Y_{\mathrm{inv}}) \end{eq*}
are subject to certain naturality conditions. Specifically, given $F: X' \to X$ and $G: Y \to Y'$ we get a commutative diagram
\begin{eq*} \begin{tikzcd}
\mathrm{E}G\mathrm{Alg}_S(LX , Y) \ar[dd, "G \circ \_ \circ LF"'] \ar[r, "\sim"] & \mathrm{E}G\mathrm{Alg}_S(X, Y_{\mathrm{inv}}) \ar[dd, "G_{\mathrm{inv}} \circ \_ \circ F"] \\
& \\
\mathrm{E}G\mathrm{Alg}_S(LX' , Y') \ar[r, "\sim"] & \mathrm{E}G\mathrm{Alg}_S(X', Y'_{\mathrm{inv}})
\end{tikzcd} \end{eq*}
Consider the case where $F$ is the identity map $\mathrm{id}_X : X \to X$ and $G$ is the inclusion $j: L(X)_{\mathrm{inv}} \to L(X)$. Note that because $j$ is an inclusion, the restriction $j_{\mathrm{inv}}: (L(X)_{\mathrm{inv}})_{\mathrm{inv}} \to L(X)_{\mathrm{inv}}$ is also an inclusion, but since $((\_)_{\mathrm{inv}})_{\mathrm{inv}} = (\_)_{\mathrm{inv}}$, we have that $j_{\mathrm{inv}} = \mathrm{id}$. It follows that
\begin{eq*} \begin{tikzcd}
\mathrm{E}G\mathrm{Alg}_S(LX , LX_{\mathrm{inv}}) \ar[dd, "j \circ \_"'] \ar[r, "\sim"] & \mathrm{E}G\mathrm{Alg}_S(X, LX_{\mathrm{inv}}) \ar[dd, equal] \\
& \\
\mathrm{E}G\mathrm{Alg}_S(LX , LX) \ar[r, "\sim"] & \mathrm{E}G\mathrm{Alg}_S(X, LX_{\mathrm{inv}})
\end{tikzcd} \end{eq*}
Therefore, for any map $f: LX \to LX$ there exists a unique $g: LX \to LX_{\mathrm{inv}}$ such that $j \circ g =f$. But this means that for any such $f$, we must have $\mathrm{im}(f) \subseteq L(X)_{\mathrm{inv}}$, and so in particular $L(X) = \mathrm{im}(\mathrm{id}_{LX}) \subseteq L(X)_{\mathrm{inv}}$. Since $L(X)_{\mathrm{inv}} \subseteq L(X)$ by definition, we obtain the result.
\end{proof}

This result is not especially surprising. Intuitively, it just says that when you freely add inverses to an algebra, every object ends up with an inverse. But the upshot of this is that we now have another way of thinking about $L(X)$: as the target object of the unit of our adjunction, $\eta_X: X \to L(X)_{\mathrm{inv}}$. This means that we don't really need to know the entirety of $L$ in order to determine the free algebra $L\mathbb{G}_n$, just its unit. To find this unit directly, we can turn to the following fact about adjunctions, for which a proof can be found in Lemma 2.3.5 of Leinster's \textit{Basic Category Theory} \cite{bct}.

\begin{prop}\label{initial} Let $F \dashv G: A \to B$ be an adjunction with unit $\eta$. For any object $a$ in $A$, let $(a \downarrow G)$ denote the comma category whose objects are pairs $(b, f)$ consisting of an object $\mathrm{B}$ from $\mathrm{B}$ and a morphism $f: a \to G(b)$ from $A$, and whose morphisms $h: (b, f) \to (b', f')$ are morphisms $f: b \to b'$ from $\mathrm{B}$ such that $G(f) \circ f = f'$. Then the pair $\big(F(a), \eta_a: a \to GF(a) \big)$ is an initial object of $(a \downarrow G)$.
\end{prop}

\begin{cor} $\eta_{\mathbb{G}_n}: \mathbb{G}_n \to (L\mathbb{G}_n)_{\mathrm{inv}} = L\mathbb{G}_n$ is an initial object of $(\mathbb{G}_n \downarrow \mathrm{inv})$.
\end{cor}

Being able to view $L\mathbb{G}_n$ as the initial object in the comma category $(\mathbb{G}_n \downarrow \mathrm{inv})$ will prove immensely useful in the coming sections. This is because it lets us think about the properties of $L\mathbb{G}_n$ in terms of maps $\psi: \mathbb{G}_n \to X_{\mathrm{inv}}$, and this is exactly the context where we can exploit $\mathbb{G}_n$'s status as a free algebra. As a result, its worth taking some time to think about what exactly this map $\eta_{\mathbb{G}_n}$ is.

\begin{lem} The initial object $\eta_{\mathbb{G}_n}: \mathbb{G}_n \to L\mathbb{G}_n$ is the obvious inclusion of the free $\mathrm{E}G$-algebra on $n$ objects into the free $\mathrm{E}G$-algebra on $n$ invertible objects. That is, $\eta_{\mathbb{G}_n}$ is the algebra map defined by
\begin{eq*} \begin{array}{rrrcl}
			\eta_{\mathbb{G}_n} & : & \mathbb{G}_n & \to & L\mathbb{G}_n \\
			& : & F(\{z_1, ..., z_n\}) & \to & LF(\{z_1, ..., z_n\}) \\
			& : & z_i & \mapsto & z_i
		\end{array}
\end{eq*}
\end{lem}
\begin{proof}
Consider the $n$-tuple $(z_1, ..., z_n)$ in $(\mathbb{G}_n)^n$. Clearly the image of $(z_1, ..., z_n)$ under the functor $L$ is just the object $(z_1, ..., z_n)$ in the algebra 
\begin{eq*} L\big( \, (\mathbb{G}_n)^n \, \big) \quad = \quad (L\mathbb{G}_n)^n \quad = \quad LF(\{z_1, ..., z_n\})^n \end{eq*}
But the image of $(z_1, ..., z_n) \in (\mathbb{G}_n)^n$ under the isomorphism
\begin{eq*} \mathrm{E}G\mathrm{Alg}_S( \, \mathbb{G}_n , \mathbb{G}_n \, ) \quad \cong \quad (\mathbb{G}_n)^n \end{eq*}
is just the identity map $\mathrm{id}_{\mathbb{G}_n}$. Thus by functoriality of $L$, the map $L(\mathrm{id}_{\mathbb{G}_n}) = \mathrm{id}_{L\mathbb{G}_n}$ must be the one which corresponds to the $n$-tuple $(z_1, ..., z_n) \in (L\mathbb{G}_n)^n$ image via the isomorphism
\begin{eq*} \mathrm{E}G\mathrm{Alg}_S( \, L\mathbb{G}_n , L\mathbb{G}_n \, ) \quad \cong \quad (L\mathbb{G}_n)^n \end{eq*}
Furthermore, the $\mathbb{G}_n$ component of the unit $\eta$ is by definition the image of the identity map $\mathrm{id}_{L\mathbb{G}_n}$ under the isomorphism
\begin{eq*}\mathrm{E}G\mathrm{Alg}_S( \, L\mathbb{G}_n , L\mathbb{G}_n \, ) \quad \cong \quad \mathrm{E}G\mathrm{Alg}_S( \, \mathbb{G}_n, L\mathbb{G}_n \, ) \end{eq*}
Hence it follows that $\eta_{\mathbb{G}_n}$ is the map that corresponds to $(z_1, ..., z_n)$ via
\begin{eq*} \mathrm{E}G\mathrm{Alg}_S( \, \mathbb{G}_n, L\mathbb{G}_n \, ) \quad \cong \quad (L\mathbb{G}_n)^n \end{eq*}
which is exactly the definition given in the statement of the lemma.
\end{proof}

Before moving on, we'll make a small change in notation. From now on, rather than writing objects in $(\mathbb{G}_n \downarrow \mathrm{inv})$ as maps $\psi: \mathbb{G}_n \to Y_{\mathrm{inv}}$, we will instead just let $X = Y_{\mathrm{inv}}$ and speak of maps $\psi: \mathbb{G}_n \to X$. This is purely to prevent the notation from becoming cluttered, and shouldn't be a problem so long as we always remember that the targets of these maps only ever contain invertible objects and morphisms. We'll also drop the subscript from $\eta_{\mathbb{G}_n}$, since it is the only component of the unit we'll ever use.

\section{The objects of $L\mathbb{G}_n$}

So now we know that $L\mathbb{G}_n$ is an initial object in the category $(\mathbb{G}_n \downarrow \mathrm{inv})$. But what does this actually tell us? After all, we do not currently have a method for finding initial objects in an arbitrary collection of $\mathrm{E}G$-algebra maps. Because of this, we'll have to approach the problem step-by-step, using the initiality of $\eta$ to extract different pieces of information about the algebra $L\mathbb{G}_n$ as we go. We'll begin by tring to find its objects.

\begin{defn}\label{Obdef} Denote by $\mathrm{Ob}: \mathrm{E}G\mathrm{Alg}_S \to \mathrm{Mon}$ be the functor that sends $\mathrm{E}G$-algebras $X$ to their monoid of objects $\mathrm{Ob}(X)$, and algebra maps $F: X \to Y$ to their underlying monoid homomorphism $\mathrm{Ob}(F): \mathrm{Ob}(X) \to \mathrm{Ob}(Y)$. \end{defn}

In order to find $\mathrm{Ob}(L\mathbb{G}_n)$, we'll need to make use of an important result about the nature of $\mathrm{Ob}$.

\begin{defn}\label{Edef} Recall that given a monoid $M$, the monoidal category $\mathrm{E}M$ is the one whose monoid of objects is $M$ and which has a unique isomorphism between any two objects. We can view $\mathrm{E}M$ as not just a category but an $\mathrm{E}G$-algebra, by letting the action on morphisms take the only possible values it can, given the required source and target. Similarly, for any monoid homomorphisms $h: M \to M'$ we can define a map of $\mathrm{E}G$-algebras
\begin{eq*} \begin{array}{rlrll}
		\mathrm{E}h & : & \mathrm{E}M & \to & \mathrm{E}M' \\
		& : & m & \mapsto & h(m) \\
		& : & m \to m' & \mapsto & h(m) \to h(m')
		\end{array}
\end{eq*}
This definition of $\mathrm{E}h$ respects composition and identities, and so together with $\mathrm{E}M$ it describes a functor $\mathrm{E}: \mathrm{Mon} \to \mathrm{E}G\mathrm{Alg}_S$.
 \end{defn}

\begin{prop}\label{Obadj} $\mathrm{E}$ is a right adjoint to the functor $\mathrm{Ob}$. 
\end{prop}
\begin{proof}
For any $\mathrm{E}G$-algebra $X$, a map $F: X \to \mathrm{E}M$ is determined entirely by its restriction to objects, the monoid homomorphism $\mathrm{Ob}(F) : \mathrm{Ob}(X) \to M$. This is because functorality of $F$ ensures that any map $x \to x'$ in $X$ must be sent to a map $F(x) \to F(x')$ in $\mathrm{E}M$, and by the definition of $\mathrm{E}$ there is always exactly one of these to choose from. In other words, we have an isomorphism between the homsets
\begin{eq*} \mathrm{E}G\mathrm{Alg}_S( \, X, \, \mathrm{E}M \, ) \quad \cong \quad \mathrm{Mon}( \, \mathrm{Ob}(X), \, M \, ) \end{eq*}
Additionally, this isomorphism is natural in both coordinates. That is, for any $G: X \to X'$ in $\mathrm{E}G\mathrm{Alg}_S$ and $h : M \to M'$ in $\mathrm{Mon}$, the diagram
\begin{eq*} \begin{tikzcd}
\mathrm{E}G\mathrm{Alg}_S(X, \mathrm{E}M) \ar[dd, "\mathrm{E}h \circ \_ \circ G"'] \ar[r, "\sim"] & \mathrm{Mon}(\mathrm{Ob}(X), M) \ar[dd, "h \circ \_ \circ \mathrm{Ob}(G)"] \\
& \\
\mathrm{E}G\mathrm{Alg}_S(X', \mathrm{E}M') \ar[r, "\sim"] & \mathrm{Mon}(\mathrm{Ob}(X'), M')
\end{tikzcd} \end{eq*}
commutes, because
\begin{eq*} \mathrm{Ob}( \, \mathrm{E}h \circ F \circ G \, ) \quad = \quad \mathrm{Ob}(Eh) \circ \mathrm{Ob}(F) \circ \mathrm{Ob}(G) \quad = \quad h \circ \mathrm{Ob}(F) \circ \mathrm{Ob}(G) \end{eq*}
Therefore, $\mathrm{Ob} \dashv \mathrm{E}$.
\end{proof}

What \cref{Obadj} is essentially saying is that the functor $\mathrm{Ob}$ provides a way for us to move back and forth between the categories $\mathrm{E}G\mathrm{Alg}_S$ and $\mathrm{Mon}$. By applying this reasoning to the universal property of the initial object $\eta$, we can then determine the value of $\mathrm{Ob}(L\mathbb{G}_n)$ in terms of a new universal property of $\mathrm{Ob}(\eta)$ in the category $\mathrm{Mon}$. In particular, the algebras in $(\mathbb{G}_n \downarrow \mathrm{inv})$ are those whose objects are all invertible, and so the induced property of $\mathrm{Ob}(\eta)$ will end up saying something about the relationship between $\mathrm{Ob}(\mathbb{G}_n)$ and groups --- those monoids whose elements are all invertible.

\begin{defn} Let $M$ be a monoid, $M^{\mathrm{gp}}$ a group, and $i: M \to M^{\mathrm{gp}}$ a monoid homomorphism between them. Then we say that $M^{\mathrm{gp}}$ is the \emph{group completion} of $M$ if for any other group $H$ and homomorphism $h: M \to H$, there exists a unique homomorphism $u: M^{\mathrm{gp}} \to H$ such that $u \circ i = h$.
\end{defn}

There are several different ways to actually calculate the group completion of a monoid. One is to use that fact that $M^{\mathrm{gp}}$ is the group whose group presentation is the same as the monoid presentation of $M$. That is, if $M$ is the quotient of the free monoid on generators $\mathcal{G}$ by the relations $\mathcal{R}$, then $M^{\mathrm{gp}}$ is the quotient of the free \emph{group} on generators $\mathcal{G}$ by relations $\mathcal{R}$. This makes finding the completion of free monoids particularly simple.

\begin{prop}\label{Zobj} The object monoid of $L\mathbb{G}_n$ is $\mathbb{Z}^{*n}$, the group completion of the object monoid of $\mathbb{G}_n$. The restriction of $\eta$ on objects, $\mathrm{Ob}(\eta)$, is then the obvious inclusion $\mathbb{N}^{*n} \hookrightarrow \mathbb{Z}^{*n}$.
\end{prop}
\begin{proof}
Let $H$ be a group, and $h: \mathrm{Ob}(\mathbb{G}_n) \to H$ a monoid homomorphism. By \cref{Obadj} we have an isomorphism of homsets
\begin{eq*} \mathrm{E}G\mathrm{Alg}_S( \, \mathbb{G}_n, \, \mathrm{E}H \, ) \quad \cong \quad \mathrm{Mon}( \, \mathrm{Ob}(\mathbb{G}_n), \, H \, ) \end{eq*}
Denote by $h': \mathbb{G}_n \to \mathrm{E}H$ the map of $\mathrm{E}G$-algebras corresponding to $h$ under this isomorphism. Since $H$ is a group, every object in $\mathrm{E}H$ is invertible, and so $h'$ is an object of $(\mathbb{G}_n \downarrow \mathrm{inv})$. Thus, by initiality of $\eta$, there must exist a unique map $u: L\mathbb{G}_n \to \mathrm{E}G$ making the lefthand traingle below commute:
\begin{eq*} \begin{tikzcd}
\mathbb{G}_n \ar[dd, "\eta"'] \ar[ddrr, "h'"] & & & & \mathrm{Ob}(\mathbb{G}_n) \ar[dd, "\mathrm{Ob}(\eta)"'] \ar[ddrr, "h"] & & \\
& & & & & & \\
L\mathbb{G}_n \ar[rr, "u"'] & & \mathrm{E}H & & \mathrm{Ob}(L\mathbb{G}_n) \ar[rr, "\mathrm{Ob}(u)"'] & & H
\end{tikzcd} \end{eq*}
It follows that the righthand triangle --- which is the image of the first under $\mathrm{Ob}$ --- also commutes. Hence for any group $H$ and homomorphism $h: \mathrm{Ob}(\mathbb{G}_n) \to H$, there is at least one map which factors $h$ through $\mathrm{Ob}(\eta)$.

But now let $v: \mathrm{Ob}(L\mathbb{G}_n) \to H$ be any homomorphism such that $v \circ \mathrm{Ob}(\eta) = h$. If $v': L\mathbb{G}_n \to \mathrm{E}H$ is the image of $v$ under the adjunction isomorphism, then by naturality $v' \circ \eta = h'$, a property that was supposed to be unique to $u$. Thus $v = \mathrm{Ob}(u)$, and so there is actually only one possible map which factors $h$ through $\mathrm{Ob}(\eta)$. 

Therefore every homomorphism from $\mathrm{Ob}(\mathbb{G}_n)$ onto a group factors uniquely through the $\mathrm{Ob}(L\mathbb{G}_n)$, or in other words $\mathrm{Ob}(L\mathbb{G}_n)$ is the group completion of the monoid $\mathrm{Ob}(\mathbb{G}_n)$. Since by \cref{Gnobj} the object monoid of $\mathbb{G}_n$ is $\mathbb{N}^{\ast n}$, the free monoid on $n$ generators, we can conclude that
\begin{eq*} \mathrm{Ob}(L\mathbb{G}_n) \, = \, \mathrm{Ob}(\mathbb{G}_n)^{\mathrm{gp}} \, = \, (\mathbb{N}^{\ast n})^{\mathrm{gp}} \, = \, \mathbb{Z}^{\ast n} \end{eq*}
the free group on $n$ generators. Moreover, the map $\mathrm{Ob}(\eta)$ is then the inclusion of $\mathrm{Ob}(\mathbb{G}_n)$ into its completion, which is just $\mathbb{N}^{*n} \hookrightarrow \mathbb{Z}^{*n}$.
\end{proof}

\section{The connected components of $L\mathbb{G}_n$}

The core result of \cref{Zobj} --- that $\mathrm{Ob}(L\mathbb{G}_n)$ is the group completion of $\mathrm{Ob}(\mathbb{G}_n)$ --- makes concrete the sense in which the functor $L$ represents `freely adding inverses' to objects. Extending this same logic to connected components as well, it would seem reasonable to expect that $\pi_0(L\mathbb{G}_n)$ is the group completion of $\pi_0(\mathbb{G}_n)$ as well. This is indeed the case, and the proof proceeds in a way completely analagous to \cref{Zobj}. 

First, we want to show that the process of taking connected components forms part of an adjunction. To do this we are going to need a category from which we can draw the kind of structures that can act as the components of an $\mathrm{E}G$-algebra. Exactly which category this should be will depend on our choice of action operad $G$, or more precisely its underlying permutations.

\begin{defn} For a given action operad $G$, denote by $\mathrm{im}(\pi)\mathrm{-Mon}$ the full subcategory of $\mathrm{Mon}$ on those monoids whose multiplication is invariant under the permutations in $\mathrm{im}(\pi)$. That is, a monoid $M$ is in $\mathrm{im}(\pi)\mathrm{-Mon}$ if and only if
\begin{eq*} m_1, ..., m_n \in M, \, g \in G(n) \quad \implies \quad m_1 ... m_n \, = \, m_{\pi(g)^{-1}(1)} ... m_{\pi(g)^{-1}(n)} \end{eq*}
\end{defn}

Of course, by \cref{surjortriv} there are really only two examples of such a $\mathrm{im}(\pi)\mathrm{-Mon}$. If the underlying permutations of $G$ are trivial, then $\mathrm{im}(\pi)\mathrm{-Mon}$ is just the whole of the category $\mathrm{Mon}$; if  instead $G$ is crossed then we are asking for monoids whose multiplication is invariant under arbitrary permutations from $\mathrm{S}$, and so $\mathrm{im}(\pi)\mathrm{-Mon}$ is just the category of \emph{commutative} monoids, $\mathrm{CMon}$. Regardless, when we are working with an arbitrary action operad $G$, the category $\mathrm{im}(\pi)\mathrm{-Mon}$ is exactly the collection of possible connected components that we were looking for.

\begin{lem}\label{pi0} Let $G$ be an action operad and $\mathrm{im}(\pi)$ its underlying permutation action operad. Then there is a functor
\begin{eq*} \pi_0: \mathrm{E}G\mathrm{Alg}_S \to \mathrm{im}(\pi)\mathrm{-Mon} \end{eq*}
which sends each algebra $X$ to its monoid of connected components $\pi_0(X)$, and sends each map of algebras $F: X \to Y$ to its restriction to connected components $\pi_0(F): \pi_0(X) \to \pi_0(Y)$.
\end{lem}
\begin{proof}
Let $x_1, ..., x_n$ be an arbitrary collection of objects from the algebra $X$, and $g$ an element of the group $G(n)$. Then the action of $G$ guarantees the existence of a morphism
\begin{eq*} \alpha(g; \mathrm{id}_{x_1}, ..., \mathrm{id}_{x_n}) \, : \, x_1 \otimes ... \otimes x_n \to x_{\pi(g^{-1})(1)} \otimes ... \otimes x_{\pi(g^{-1})(n)} \end{eq*}
By definition the source and target of this morphism belong to the same connected component, and hence
\begin{eq*} \begin{array}{rll}
			[ \, x_1 \otimes ... \otimes x_n \, ] & = & [ \, x_{\pi(g^{-1})(1)} \otimes ... \otimes x_{\pi(g^{-1})(n)} \, ] \\
			\implies \quad [x_1] \otimes ... \otimes [x_n] & = & [x_{\pi(g^{-1})(1)}] \otimes ... \otimes [x_{\pi(g^{-1})(n)}]
		\end{array} 
\end{eq*}
But since the $x_i$ are just arbitrary objects of $X$, the components $[x_i]$ are an arbitrary collection of elements from $\pi_0(X)$, and likewise for the group element $g$ and the permutation $\pi(g)$. Therefore multiplication in the monoid $\pi_0(X)$ is invariant under all permutations in the images of the homomorphisms $\pi_n: G(n) \to S_n$, and thus $\pi_0(X)$ is an object of $\mathrm{im}(\pi)\mathrm{-Mon}$, as required. Well-definedness of the functor $\pi_0$ on morphisms then follows immediately from the fullness of $\mathrm{im}(\pi)\mathrm{-Mon}$.
\end{proof}

Now that we have a functor which represents the act of finding the connected component monoid of an algebra, we need another functor heading in the opposite direction, so that we can construct an adjunction between them.

\begin{defn} There exists an inclusion of 2-categories $\mathrm{D}: \mathrm{Set} \hookrightarrow \mathrm{Cat}$ which allows us to view any set $S$ as a \emph{discrete category}, one whose objects are just the elements of $S$ and whose morphisms are all identities. If the given set also happens to be a monoid $M$, then there is an obvious way to see the discete category $\mathrm{D}M$ as a monoidal category, and so we have a similar inclusion $\mathrm{D}: \mathrm{Mon} \hookrightarrow \mathrm{MonCat}$. Finally, for any action operad $G$ and object $M$ of the category $\mathrm{im}(\pi)\mathrm{-Mon}$, there is a unique way to assign an $\mathrm{E}G$-action to the discete category $\mathrm{D}M$. This works because for any elements $m_1, ..., m_n \in M$ and $g \in G(n)$, the morphism $\alpha(g; \mathrm{id}_{m_1}, ..., \mathrm{id}_{m_n})$ must have source and target 
\begin{eq*} m_1 \otimes ... \otimes m_n  \quad = \quad m_{\pi(g^{-1})(1)} \otimes ... \otimes m_{\pi(g^{-1})(m)} \end{eq*}
and therefore it can only be the morphism $\mathrm{id}_{m_1 \otimes ... \otimes m_n}$. This choice of action yields one last inclusion $\mathrm{CMon} \hookrightarrow \mathrm{E}G\mathrm{Alg}_S$, which we shall also call $\mathrm{D}$. \end{defn}

\begin{prop}\label{concompadj} $\mathrm{D}$ is a right adjoint to the functor $\pi_0$. 
\end{prop}
\begin{proof}
Consider a map of $F: X \to \mathrm{D}C$ from some $\mathrm{E}G$-algebra $X$ onto the discrete $\mathrm{E}G$-algebra for a monoid $M$ in $\mathrm{im}(\pi)\mathrm{-Mon}$. For any $f: x \to x'$ in $X$, the morphism $F(f)$ must be an identity map in $\mathrm{D}M$, since these are the only morphisms that $\mathrm{D}M$ has. It follows that $x$ and $x'$ being in the same connected component will imply $F(x) = F(x')$, and so $F$ is determined entirely by its restriction to connected components, the monoid homomorphism $\pi_0(F) : \pi_0(X) \to M$. In other words, we have an isomorphism between the homsets
\begin{eq*} \mathrm{E}G\mathrm{Alg}_S( \, X, \mathrm{D}M \, ) \quad \cong \quad \mathrm{im}(\pi)\mathrm{-Mon}( \, \pi_0(X), M \, ) \end{eq*}
This isomorphism is natural in both coordinates, since for any $G: X \to X'$ in $\mathrm{E}G\mathrm{Alg}_S$ and $h : M \to M'$ in $\mathrm{im}(\pi)\mathrm{-Mon}$, 
\begin{eq*} \pi_0( \, \mathrm{D}h \circ F \circ G \, ) \quad = \quad \pi_0(\mathrm{D}h) \circ \pi_0(F) \circ \pi_0(G) \quad = \quad h \circ \pi_0(F) \circ \pi_0(G) \end{eq*}
and so the diagram
\begin{eq*} \begin{tikzcd}
\mathrm{E}G\mathrm{Alg}_S(X, \mathrm{D}M) \ar[dd, "\mathrm{D}h \circ \_ \circ G"'] \ar[rr, "\sim"] & & \mathrm{im}(\pi)\mathrm{-Mon}\big( \, \pi_0(X), M \, \big) \ar[dd, "h \circ \_ \circ \pi_0(G)"] \\
& & \\
\mathrm{E}G\mathrm{Alg}_S(X', \mathrm{D}M') \ar[rr, "\sim"] & & \mathrm{im}(\pi)\mathrm{-Mon}\big( \, \pi_0(X'), M' \, \big) 
\end{tikzcd} \end{eq*}
commutes. Therefore, $\pi_0 \dashv \mathrm{D}$.
\end{proof}

Now we can utilise \cref{concompadj} to draw out a universal property of $\pi_0(L\mathbb{G}_n)$, just as we did with $\mathrm{Ob}(L\mathbb{G}_n)$ in \cref{Obadj}.

\begin{prop}\label{Zconcomp} The connected components of $L\mathbb{G}_n$ are the group completion of the connected components of $\mathbb{G}_n$. Also, the restriction of $\eta$ onto connected components, $\pi_0(\eta)$, is the canonical map $\pi_0(\mathbb{G}_n) \to \pi_0(\mathbb{G}_n)^{\mathrm{gp}}$ associated with that group completion.
\end{prop}
\begin{proof}
Let $H$ be a group which is also an object of $\mathrm{im}(\pi)\mathrm{-Mon}$, and let $h: \pi_0(\mathbb{G}_n) \to H$ be a monoid homomorphism. By \cref{concompadj} there is a homset isomorphism
\begin{eq*} \mathrm{E}G\mathrm{Alg}_S( \, \mathbb{G}_n, \, \mathrm{D}H \, ) \quad \cong \quad \mathrm{im}(\pi)\mathrm{-Mon}( \, \pi_0(\mathbb{G}_n), \, H \, ) \end{eq*}
and thus some $\mathrm{E}G$-algebra map $h': \mathbb{G}_n \to \mathrm{D}H$ corresponding to $h$. As $H$ is a group, every object of $\mathrm{D}H$ is invertible, and so $h'$ is an object of $(\mathbb{G}_n \downarrow \mathrm{inv})$. It follows that there exists a unique map $u: L\mathbb{G}_n \to \mathrm{D}M$ which factors $h'$ through the initial object $\eta$:
\begin{eq*} \begin{tikzcd}
\mathbb{G}_n \ar[dd, "\eta"'] \ar[ddrr, "h'"] & & & & \pi_0(\mathbb{G}_n) \ar[dd, "\pi_0(\eta)"'] \ar[ddrr, "h"] & & \\
& & & & & & \\
L\mathbb{G}_n \ar[rr, "u"'] & & \mathrm{D}H & \quad & \pi_0(L\mathbb{G}_n) \ar[rr, "\pi_0(u)"'] & & H
\end{tikzcd} \end{eq*}
Applying the functor $\pi_0$ everywhere, we see that $\pi_0(u)$ must also factor $h$ through the homomorphism $\pi_0(\eta)$. Moreover, $\pi_0(u)$ is the only map with this property, since for any other map $v: \pi_0(L\mathbb{G}_n) \to H$ with $v \circ \pi_0(\eta) = h$, its image under the adjunction isomorphism $v': L\mathbb{G}_n \to \mathrm{D}H$ would have $v' \circ \eta = h'$ by naturality, which would mean that it was actually $u$. Therefore, any monoid homomorphism $\pi_0(\mathbb{G}_n) \to H$ will factor uniquely through $\pi_0(L\mathbb{G}_n)$, so long as $H$ is a group. 

Now consider another monoid homomorphism $k: \pi_0(\mathbb{G}_n) \to K$, where this time $K$ is still a group but not neccessarily in $\mathrm{im}(\pi)\mathrm{-Mon}$. From \cref{pi0}, we know that $\pi_0(\mathbb{G}_n)$ is still an object of $\mathrm{im}(\pi)\mathrm{-Mon}$, and from this we can conclude that the image $\mathrm{im}(k)$ will be too:
\begin{eq*} \begin{array}{rcrcl}
			 x_1, ..., x_m \in \pi_0(\mathbb{G}_n), \, g \in G(n) & \implies & x_1 \otimes ... \otimes x_m & = & x_{g(1)} \otimes ... \otimes x_{g(m)} \\
			& \implies & k( \, x_1 \otimes ... \otimes x_m \, ) & = & k( \, x_{g(1)} \otimes ... \otimes x_{g(m)} \, ) \\
			& \implies & k(x_1) \otimes ... \otimes k(x_m) & = & k(x_{g(1)}) \otimes ... \otimes k(x_{g(m)})
		\end{array}
\end{eq*}
Also, since $\mathrm{im}(k)$ is a submonoid of the group $K$, it is a group as well. Thus if we denote by $k_{\mathrm{im}}: \mathrm{Ob}(\mathbb{G}_n) \to \mathrm{im}(k)$ the restriction of $k$ to it image, then $k_{\mathrm{im}}$ is a map in $\mathrm{im}(\pi)\mathrm{-Mon}$ out of $\mathrm{Ob}(\mathbb{G}_n)$ and onto a group, and therefore by what we showed earlier there exists a unique homomorphism $v: \mathrm{Ob}(L\mathbb{G}_n) \to \mathrm{im}(k)$ with the property $v \circ \pi_0(\eta) = k_{\mathrm{im}}$. Composing this $v$ with the inclusion $i: \mathrm{im}(k) \hookrightarrow K$, we see that
\begin{eq*} i \circ v \circ \pi_0(\eta) \, = \, i \circ k_{\mathrm{im}} \, = \, k \end{eq*}
and $i \circ v$ must be the only map for which this is true, for restricting this equation back on $\mathrm{im}(k)$ yields the unique property of $v$ again. Thus $\pi_0(\eta)$ will actually take any homomorphism from $\mathrm{Ob}(\mathbb{G}_n)$ onto a group and factor it through $\pi_0(L\mathbb{G}_n)$ in a unique way, not just those homomorphisms in $\mathrm{im}(\pi)\mathrm{-Mon}$. In other words, 
\begin{eq*} \pi_0(L\mathbb{G}_n) \quad = \quad \pi_0(\mathbb{G}_n)^{\mathrm{gp}} \end{eq*}
and $\pi_0(\eta)$ is the canonical map of this group completion.
\end{proof}

As we've said before, this result is a reflection of the fact that the functor $L$ is trying to add inverses the objects of $\mathbb{G}_n$ freely, that is, with as little effect on the rest of the algebra as possible. Indeed, if we happen to know whether or not our action operad $G$ is crossed then we can now calculate exactly what the effect on the components will be.

\begin{cor}\label{crossconcomp} If $G$ is a crossed action algebra then
\begin{itemize} \itemsep0em
\item the connected components of $L\mathbb{G}_n$ are the monoid $\mathbb{Z}^n$
\item the restriction of $\eta$ to components is the obvious inclusion $\mathbb{N}^n \hookrightarrow \mathbb{Z}^n$
\item the assignment of objects to their component is given by the quotient map of abelianisation $\mathrm{ab}: \mathbb{Z}^{\ast n} \to \mathbb{Z}^n$
\end{itemize}
If instead $G$ is non-crossed, then
\begin{itemize} \itemsep0em
\item the connected components of $L\mathbb{G}_n$ are the monoid $\mathbb{Z}^{\ast n}$
\item the restriction of $\eta$ to components is the obvious inclusion $\mathbb{N}^{\ast n} \hookrightarrow \mathbb{Z}^{\ast n}$
\item the assignment of objects to their component is $\mathrm{id}_{\mathbb{Z}^{\ast n}}$
\end{itemize}
\end{cor}
\begin{proof}
Combining \cref{Zconcomp,Gnconcomp}, we see that
\begin{eq*} \pi_0(L\mathbb{G}_n) \quad = \quad \pi_0(\mathbb{G}_n)^{\mathrm{gp}} \quad = \quad \begin{cases}
													\quad (\mathbb{N}^n)^{\mathrm{gp}} \quad = \quad \mathbb{Z}^n & \text{if $G$ is crossed} \\
													\quad (\mathbb{N}^{\ast n})^{\mathrm{gp}} \quad = \quad \mathbb{Z}^{\ast n} & \text{otherwise}
														\end{cases}
\end{eq*}
Moreover, \cref{Zconcomp} says that restriction of $\eta$ to connected components, $\pi_0(\eta)$, will be the homomorphism associated with these group completion, which means the inclusion $\mathbb{N}^n \hookrightarrow \mathbb{Z}^n$ when $G$ is crossed and $\mathbb{N}^{\ast n} \hookrightarrow \mathbb{Z}^{\ast n}$ when it is not.

Next, by \cref{Gnconcomp} we know that the map $[ \, \_ \, ] : \mathrm{Ob}(\mathbb{G}_n) \to \pi_0(\mathbb{G}_n)$ sending objects of $\mathbb{G}_n$ to their connected component is either the quotient map of abelianisation $\mathbb{N}^{\ast n} \to \mathbb{N}^n$ or the identity on $\mathbb{N}^{\ast n}$, depending on whether or not it is crossed. If we also use $[ \, \_ \, ]$ to denote the map sending objects of $L\mathbb{G}_n$ to their components, it then follows from functoriality of $\eta$ that the corresponding choice of the followings two diagrams will commute:
\begin{eq*} \begin{tikzcd}
\mathbb{N}^{\ast n} \ar[dd, hookrightarrow, "\lbrack \, \_ \, \rbrack"'] \ar[rr, hookrightarrow, "\mathrm{Ob}(\eta)"] & & \mathbb{Z}^{\ast n} \ar[dd, "\lbrack \, \_ \, \rbrack"] & \quad & \mathbb{N}^{\ast n} \ar[dd, equals, "\lbrack \, \_ \, \rbrack"'] \ar[rr, hookrightarrow, "\mathrm{Ob}(\eta)"] & & \mathbb{Z}^{\ast n} \ar[dd, "\lbrack \, \_ \, \rbrack"] \\
& & & & \\
\mathbb{N}^n \ar[rr, hookrightarrow, "\pi_0(\eta)"'] & & \mathbb{Z}^n & & \mathbb{N}^{\ast n} \ar[rr, hookrightarrow, "\pi_0(\eta)"'] & & \mathbb{Z}^{\ast n}
\end{tikzcd} \end{eq*}
Using the values of $[ \, \_ \, ]$ from \cref{Gnconcomp}, $\mathrm{Ob}(\eta)$ from \cref{Zobj}, and $\pi_0(\eta)$ from earlier in this proof, it follows that for any generator $z_i$ of $\mathbb{Z}^{\ast n}$, 
\begin{eq*} [z_i] \quad = \quad [\mathrm{Ob}(\eta)(z_i)] \quad = \quad \pi_0(\eta)([z_i]) \quad = \quad \pi_0(\eta)(z_i) \quad = \quad z_i \end{eq*}
But this description of $[ \, \_ \, ]: \mathrm{Ob}(L\mathbb{G}_n) \to \pi_0(L\mathbb{G}_n)$ on generators is either the definition of the quotient map $\mathrm{ab}: \mathbb{Z}^{\ast n} \to (\mathbb{Z}^{\ast n})^{\mathrm{ab}}$ or the identity $\mathrm{id}: \mathbb{Z}^{\ast n} \to \mathbb{Z}^{\ast n}$, depending on the value of target monoid, as required.
\end{proof}

\section{The morphisms of $L\mathbb{G}_n$}  

Now that we understand the objects and connected components of the algebra $L\mathbb{G}_n$, the next most obvious thing to look for are its morphisms, $\mathrm{Mor}(L\mathbb{G}_n)$. It would be nice to construct this collection in the same way we constructed $\mathrm{Ob}(L\mathbb{G}_n)$ and $\pi_0(L\mathbb{G}_n)$, by applying the left adjoint of some adjunction to the initial map $\eta$. Before we can do this however, we need to ask ourselves a question. What sort of mathematical object is $\mathrm{Mor}(L\mathbb{G}_n)$, exactly?

Given a pair of morphisms $f: x \to y, f': y' \to z$ in an $\mathrm{E}G$-algebra $X$, there are two basic binary operations we can perform. First, we can take their tensor product $f \otimes f'$, and this together with the unit map $\mathrm{id}_{I}$ imbues $\mathrm{Mor}(X)$ with the structure of a monoid. Second, if we have $y = y'$ then we can form the composite morphism $f' \circ f$. However, these two operations are not as different as they first appear.

\begin{lem} \label{tenscomp} Let $f: x \to y$ and $f': y \to z$ be morphisms in some monoidal category, and $y$ is an invertible object of that category. Then
\begin{eq*} f' \circ f \quad = \quad f' \otimes \mathrm{id}_{y*} \otimes f \end{eq*}
\end{lem}
\begin{proof}
By the interchange law for monoidal categories,
\begin{eq*}\begin{array}{rll}
			f' \circ f & = & (f' \otimes \mathrm{id}_I) \circ (\mathrm{id}_I \otimes f) \\
			& = & (f' \otimes \mathrm{id}_{y*} \otimes \mathrm{id}_y) \circ (\mathrm{id}_y \otimes \mathrm{id}_{y*} \otimes f) \\
			& = & (f' \circ \mathrm{id}_y) \otimes (\mathrm{id}_{y*} \circ \mathrm{id}_{y*}) \otimes (\mathrm{id}_y \circ f) \\
			& = & f' \otimes \mathrm{id}_{y*} \otimes f 
		\end{array}
\end{eq*}
\end{proof}

In other words, composition along invertible objects in $X$ is completely determined by the monoidal structure of $X$. In the case of $L\mathbb{G}_n$, where every object is invertible, this means that if we understand $\mathrm{Mor}(L\mathbb{G}_n)$ as a monoid then we will be able to recover the operation $\circ$ in its entirety. For that reason, we will choose to ignore composition of elements of $\mathrm{Mor}(X)$ for the time being, and focus on its status as a monoid.

Now we try to proceed as we did before, by showing that $\mathrm{Mor}(X)$ is part of an adjunction.

\begin{defn} Let $\mathrm{Mor} : \mathrm{MonCat} \to \mathrm{Mon}$ be the functor which sends algebras $X$ to their monoid of morphisms $\mathrm{Mor}(X)$, and sends algebra maps $F: X \to Y$ to the monoid homomorphism
\begin{eq*} \begin{array}{rlrll}
			\mathrm{Mor}(F) & : & \mathrm{Mor}(X) & \to & \mathrm{Mor}(Y) \\
			& : & f: x \to x' & \mapsto & F(f) : F(x) \to F(x') \\
		\end{array}
\end{eq*}
\end{defn}

\begin{defn} For a given abelian group $A$, let $C(A)$ represent the monoidal category defines as follows:
\begin{itemize} \itemsep0em
\item The objects of $C(A)$ are the monoid $A$, with the monoid multiplication as the tensor product and the identity element $e$ as the monoidal unit.
\item For any two objects $a, a' \in A$, the homset $C(A)(a, a')$ is isomorphic to the underlying set of $A$.
\item From the above, the morphisms of $C(A)$ will clearly be
\begin{eq*} \mathrm{Mor}\big( \, C(A) \, \big) \, = \, A \times A^2 \, = \, A^3 \end{eq*}
when viewed as a set, but this equality also holds as monoids, so that the tensor product is defined componentwise using the monoid multiplication of $A$.
\item For any two composable morphisms $(a, b, b')$, $(a', b', b'')$ of $C(A)$, their composite is the morphism
\begin{eq*} (a', b', b'') \circ (a, b, b') \, = \, \big( \, a(b')^*a', \, b, \, b'' \, \big) \end{eq*}
\end{itemize}
Likewise, for any group homomorphism $h: A \to A'$ between abelian groups, denote by $C(h) : C(A) \to C(A')$ the obvious monoidal functor which acts like $h$ on objects and $h^3$ on morphisms. This defines a functor $C: \mathrm{Ab} \to \mathrm{MonCat}$ from the category of abelian groups onto the category of monoidal categories.
\end{defn}

Intuitively, $C(A)$ is the the monoidal category that we can build out of $A$ by using the trick we discussed before for extracting composition from the tensor product, $f' \circ f = f' \otimes \mathrm{id}_{y*} \otimes f$. This is why we had to choose $A$ to be a group, as this can only work when all of the objects of $C(A)$ are invertible. Notice also that commutativity is required in order for $C(A)$ to be a well-defined monoidal category, since we need its operations to obey an interchange law, and thus
\begin{eq*} \begin{array}{rll}
			(aa', e, e) & = & (a, e, e) \otimes (a', e, e) \\
			& = & \big( \, \mathrm{id}_e \circ (a, e, e) \, \big) \otimes \big( \, (a', e, e) \circ \mathrm{id}_e \, \big) \\
			& = & \big( \, (a', e, e) \otimes \mathrm{id}_e \, \big) \circ \big( \, \mathrm{id}_e \otimes (a, e, e) \, \big) \\
			& = & (a', e, e) \circ (a, e, e) \\
			& = & (a'a, e, e) 
		\end{array}
\end{eq*}
This is the classic Eckmann-Hilton argument.

\begin{prop}\label{Moradj} $C$ is a right adjoint to the functor $\mathrm{Mor}( \, \_ \, )^{\mathrm{gp}, \mathrm{ab}} : \mathrm{MonCat} \to \mathrm{Ab}$.
\end{prop} 
\begin{proof}
Let $X$ be a monoidal category, $A$ an abelian group, and $F: X \to C(A)$ a monoidal functor. For any morphism $f: x \to y$ in $X$, by functoriality $F$ will send it onto some morphism $F(f)$ in the homset $C(A)(F(x), F(y))$. However, every homset of $C(A)$ is isomorphic to a copy of $A$, and so clearly there is some sense in which $F$ contains a map $\mathrm{Mor}(X) \to A$. Specifically, if we define $\epsilon_A$ to be the projection
\begin{eq*} \begin{array}{rlrll}
			\epsilon_A & : & \mathrm{Mor}\big( \, C(A) \, \big) & \to & A \\
			& : & A^3 & \to & A \\
			& : & (a, b, b') & \mapsto & a
		\end{array}
\end{eq*}
then we can use the functor $\mathrm{Mor}$ and the map $\epsilon_A$ to form the following composite map:
\begin{eq*} \begin{tikzcd}
\mathrm{Mor}(X) \ar[rr, "\mathrm{Mor}(F)"] & & \mathrm{Mor}\big( \, C(A) \, \big) \ar[rr, "\epsilon_A"] & & A
\end{tikzcd} \end{eq*}
Then, since $A$ is not just a monoid but an abelian group, we can factor the homomorphism $\mathrm{Mor}(F)$ through the group completion of $\mathrm{Mor}(X)$, and then through the abelianisation of that group, at last yielding a map 
\begin{eq*} F' \, := \, \epsilon_A \circ \big( \, \mathrm{Mor}(F)^{\mathrm{gp}} \, \big)^{\mathrm{ab}} \, : \, \big( \, \mathrm{Mor}(X)^{\mathrm{gp}} \, \big)^{\mathrm{ab}} \to A \end{eq*}
This $\epsilon$ will be the counit of our adjunction, with the assignment $F \mapsto F'$ being one direction of the eventual homset adjunction.

Conversely, let $\eta_X$ be the monoidal functor defined by
\begin{eq*} \begin{array}{rlrll}
			\eta_X & : & X & \to & C\big( \, \mathrm{Mor}(X)^{\mathrm{gp, ab}} \, \big) \\
			& : & x & \mapsto & \mathrm{ag}(\mathrm{id}_x) \\
			& : & f: x \to y & \mapsto & \big( \, \mathrm{ag}(f), \mathrm{ag}(\mathrm{id}_x), \mathrm{ag}(\mathrm{id}_y) \, \big)
		\end{array}
\end{eq*}
where for brevity's sake we're using $\mathrm{ag}$ to refer to $\mathrm{ab} \circ \mathrm{gp}$, the composite of the group completion map $\mathrm{Mor}(X) \to \mathrm{Mor}(X)^{\mathrm{gp}}$ with the quotient of abelianisation $\mathrm{Mor}(X)^{\mathrm{gp}} \to \mathrm{Mor}(X)^{\mathrm{gp, ab}}$. Then any homomorphism $h: \mathrm{Mor}(X)^{\mathrm{gp, ab}} \to A$ can be used to construct a monoidal functor $h' : X \to C(A)$ as follows:
\begin{eq*} \begin{tikzcd}
X \ar[rr, "\eta_X"] & & C\big( \, \mathrm{Mor}(X)^{\mathrm{gp, ab}} \, \big) \ar[rr, "C(h)"] & & C(A)
\end{tikzcd} \end{eq*} 
Moreover, if we compare this $\eta$ with $\epsilon$ then we see that the composites
\begin{eq*} \begin{tikzcd}
\mathrm{Mor}(X)^{\mathrm{gp, ab}} \ar[d,"\mathrm{Mor}(\eta_X)^{\mathrm{gp, ab}}"] & & C(A) \ar[d, "\eta_{C(A)}"] \\
\mathrm{Mor}\Big( \, C\big( \, \mathrm{Mor}(X)^{\mathrm{gp, ab}} \, \big) \, \Big)^{\mathrm{gp, ab}} \ar[d, equals] & & C\Big( \, \mathrm{Mor}\big( \, C(A) \, \big)^{\mathrm{gp, ab}} \, \Big) \ar[d, equals] \\
\mathrm{Mor}\Big( \, C\big( \, \mathrm{Mor}(X)^{\mathrm{gp, ab}} \, \big) \, \Big) \ar[d, "\epsilon_{\mathrm{Mor}(X)^{\mathrm{gp, ab}}}"] & & C\Big( \, \mathrm{Mor}\big( \, C(A) \, \big)\, \Big) \ar[d, "C(\epsilon_A)"] \\
\mathrm{Mor}(X)^{\mathrm{gp, ab}} & & C(A) \\
\end{tikzcd} \end{eq*}
must be the respective identity maps:
\begin{eq*} \begin{array}{rll} 
			\epsilon_{\mathrm{Mor}(X)^{\mathrm{gp, ab}}} \circ \mathrm{Mor}(\eta_X)^{\mathrm{gp, ab}}\big( \, \mathrm{ag}(f: x \to y) \, \big) & = & \epsilon_{\mathrm{Mor}(X)^{\mathrm{gp, ab}}}\big( \, \mathrm{ag}(f), \mathrm{ag}(\mathrm{id}_x), \mathrm{ag}(\mathrm{id}_y) \, \big) \\
			& = & \mathrm{ag}(f)
		\end{array}
\end{eq*} 
\begin{eq*} \begin{array}{rll} 
			C(\epsilon_A) \circ \epsilon_{\mathrm{Mor}(X)^{\mathrm{gp, ab}}}(a) & = & C(\epsilon_A)(a, a, a) \\
			& = & a \\
			& & \\
			C(\epsilon_A) \circ \epsilon_{\mathrm{Mor}(X)^{\mathrm{gp, ab}}}(a, b, b') & = & C(\epsilon_A)\big( \, (a,b,b'), \, (b,b,b), \, (b',b',b') \, \big) \\
			& = & (a, b, b')
		\end{array}
\end{eq*} 
In other words, $\eta_X$ and $\epsilon_A$ really are the unit and counit of an adjunction $\mathrm{Mor}(\, \_ \,)^{\mathrm{gp, ab}} \dashv C$, whose isomorphism of homsets
\begin{eq*} \mathrm{MonCat}( \, X, C(A) \, ) \quad \cong \quad \mathrm{Ab}( \, \mathrm{Mor}(X)^{\mathrm{gp, ab}}, A \, ) \end{eq*}
is given by the assigments $F \mapsto F'$ and $h \mapsto h'$.
\end{proof}

\cref{Moradj} seems at first glance very similar to \cref{Obadj,concompadj}. However, our goal was to discover the relationship between the morphisms of $\mathbb{G}_n$ and $L\mathbb{G}_n$, paralleling what we did in \cref{Zobj,Zconcomp}, and in that regard the adjunction $\mathrm{Mor}( \, \_ \, )^{\mathrm{gp}, \mathrm{ab}} \dashv C$ falls short in two very important ways. 

\begin{enumerate}
\item What we really wanted was an adjunction involving $\mathrm{E}G\mathrm{Alg}_{S}$, not $\mathrm{MonCat}$. This is because $\eta$ is an initial object in $(\mathbb{G}_n \downarrow \mathrm{inv})$, and so we only know how to use it to factor algebra maps $\mathbb{G}_n \to X_{\mathrm{inv}}$, and not general monoidal functors. 
\item We would rather have had the other side of the adjunction be the category $\mathrm{Mon}$ instead of $\mathrm{Ab}$. After all, we wanted to use this adjunction to calculate the monoid $\mathrm{Mor}(L\mathbb{G}_n)$, and not the group-completed, abelianised version. 
\end{enumerate}

Unfortunately, this adjunction seems to be the best we can do. We already saw that we need $A$ to be an abelian group for $C(A)$ to have composition and interchange, and further for an arbitrary abelian group that we want to be the morphisms of an algebra there does not seem to be a general method for assigning it an $\mathrm{E}G$-action. But all is not lost. It turns out that this approach is fixable, though first we will need to reframe the problem somewhat.