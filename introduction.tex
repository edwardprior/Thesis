\chapter{Introduction}
 
The central goal of this paper is to determine how one can construct free monoidal categories over invertible objects, for as many different kinds of monoidal category as possible. This will be achieved by framing the problem in terms of the theory of action operads, and then gradually exploring the features possessed by their algebras.

The motivation for this topic came from earlier work by the author which attempted to produce a classification theorem for 3-groups. In general, $n$-groups are a higher dimensional categorification of the standard notion of a group. While a group can be seen as a monoid in which all elements are invertible, a 2-group is a monoidal category in which all objects and morphisms are invertible in the appropriate sense, a 3-group is a monoidal 2-category with all data invertible, and so on. Much has already been written on the subject of 2-groups \cite{hda5}, including a theorem which classifies them completely in terms of group cohomology. The original intention of the author --- which will hopefully still form the basis of a future paper --- was to generalise this classification theorem to work for 3-groups, by taking each step in the proof and replacing it with a version using concepts from one dimension up. In particular, to replace the sections that involved group cohomology it would be necessary to develop a theory of braided 2-group cohomology. A cohomology of \emph{symmetric} 2-groups already exists \cite{picard}, but proving that it is well-defined involves exploiting certain facts about symmetric monoidal categories, ones that do not immediately transfer to the braided case.   

Thus the key to resolving the whole issue is to understand the behaviour of braided monoidal categories whose objects are all invertible. Indeed, it would suffice to know how to construct the \emph{free} braided monoidal category on $n$ invertible objects, for any value of $n \in \mathbb{N}$, but this in turn is fairly tricky. Over the course of the following chapters we shall see how to accomplish this task, as well as how to find the analogous free entity for a large class of similar structures, what we will call the $G$-monoidal categories. These include the familiar symmetric monoidal categories, but also more unusual cases, such as ribbon braided monoidal categories. 

First, we shall spend most of \cref{actionoperad} covering definitions and results from the existing literature which will be relevant for reaching our objective. After beginning with a quick review of the concepts of monoidal categories and operads, we will introduce the main objects of study for this paper, the so-called `action operads'. First appearing with extra restrictions as `categorical operads' in the thesis of Nathalie Wahl \cite{ribbon1}, before being studied later in full generality by Alex Corner and Nick Gurski \cite{ogge}, action operads are kind of operad which generalise the notion of a group action upon a set. We will see how many common examples of operads-with-extra-structure --- including the founding example of operad theory, the symmetric operad \cite{gils} --- can be united into a single framework by viewing them as $G$-operads, ones that are acted on by some suitable action operad $G$. The translation operad $\mathrm{E}G$ will also be introduced at this point, as a way to categorify certain aspects of a given action operad $G$. Following on from the discussion of $G$-operads will be a look into what the appropriate algebras for these operads should be. In particular, we will see how they differ slightly from the more typical definition of an operad algebra, due to an additional equivariance condition. During this we will see that a certain monoidal structure, present in all action operads $G$, will be inherited by the algebras of both $G$ and $\mathrm{E}G$. Then at last all of the work in this chapter will come to a head in \cref{Gmonthm}, a result of Gurski \cite{operadborel}, where we learn that algebras of the $G$-operad $\mathrm{E}G$ are equivalent to kind of monoidal category, one equipped with extra permutative structure dictated by the nature of the action operad $G$. These are the $G$-monoidal categories, and thus by framing our questions about free braided monoidal categories in the language of action operad algebras, we will be able to produce results which are applicable to a much wider range of situations. 

Next, \cref{initialalgebra} will begin our investigation into the free $\mathrm{E}G$-algebras. We will open with a look at $\mathbb{G}_n$, the free algebra on some number $n \in \mathbb{N}$ of not-necessarily invertible objects. After providing a description for $\mathbb{G}_n$, we will also be able to surmise the existence of free $\mathrm{E}G$-algebra on $n$ invertible objects, denoted $L\mathbb{G}_n$, through the use of some monad theory. Then we shall see how this $L\mathbb{G}_n$ can be viewed as the initial object in a certain comma category of algebras, when paired with the obvious map between free algebras $\eta: \mathbb{G}_n \to L\mathbb{G}_n$. From this initial algebra perspective it will be possible for us to extract several important pieces of information about the structure of $L\mathbb{G}_n$, using a technique where we exploit the properties of adjoint functors. First, by showing that the previously mentioned translation functor $\mathrm{E}$ forms an adjunction with the object monoid functor $\mathrm{Ob}$, we will demonstrate that the objects of $L\mathbb{G}_n$ are the group completion of the objects of $\mathbb{G}_n$. Likewise, forming an adjunction between discrete category functor $\mathrm{D}$ and the connected component functor $\pi_0$ will let us prove that the components of $L\mathbb{G}_n$ are the group completion of $\pi_0(\mathbb{G}_n)$. However, a way of using this method to find the morphisms of $L\mathbb{G}_n$ will remain elusive. The closest we can get is by showing that the delooping functor $\mathrm{B}$ is right adjoint to a certain functor $\mathrm{M}( \, \_ \,)^{\mathrm{ab}} : \mathrm{MonCat} \to \mathrm{CMon}$, which describes what we will call the `collapsed morphisms' of a given monoidal category. In order to salvage this approach, we must therefore try to translate the defining property of $L\mathbb{G}_n$ into one that works solely within the category $\mathrm{MonCat}$, and then also prove that both the algebra structure and the true morphisms of a given $\mathrm{E}G$-algebra can be recovered from these new collapsed morphisms. This task will form the majority of the remaining three chapters. 

\cref{colimalgebra} will bring a couple of new ways for us to think about the algebra $L\mathbb{G}_n$. Instead of viewing it as part of an initial object like in \cref{initialalgebra}, we will instead show that it forms the target of a coequaliser map $q: \mathbb{G}_{2n} \to L\mathbb{G}_n$, whose source now has twice as many generating objects as before. The simplest way to do this involves exhibiting $q$ as the cokernel of an algebra map $\delta: \mathbb{G}_{2n} \to \mathbb{G}_{2n}$, which is designed in such a way that the additional $n$ generators of $\mathbb{G}_{2n}$ will get sent by $q$ onto the inverses of the $n$ generators of $L\mathbb{G}_n$. Through this new perspective we will learn several important facts about the action $\alpha$ of $L\mathbb{G}_n$, including how we will eventually be able to reconstruct it from $L\mathbb{G}_n$'s monoid of morphisms, once we finally understand them. This insight will then indicate how we can subtly change the coequaliser diagram for $q$, so that the preservation of the $\mathrm{E}G$-action is now a consequence of the way that we have built it, rather than just an automatic feature of $q$ being an algebra map. In other words, we will have demonstrated that the underlying monoidal functor of $q$ is also a coequaliser, and thus have found a property which marks the free algebra $L\mathbb{G}_n$ as special within the world of monoidal categories. This is exactly what we need in order to leverage the left adjoint status of the functor $\mathrm{M}( \, \_ \,)^{\mathrm{ab}}$, since it lives over the category $\mathrm{MonCat}$ and commutes with all colimits, like coequalisers. With a little work, our approach will then yield a description of the abelian group of collapsed morphisms $\mathrm{M}(L\mathbb{G}_n)^{\mathrm{gp},\mathrm{ab}}$ as a quotient of the larger group of collapsed morphisms of $\mathbb{G}_{2n}$.

In \cref{morphisms}, we will see how to use the information that we've accumulated up to this point to build the morphisms of $L\mathbb{G}_n$. The idea is that the invertibility of the objects in this category will let us split the monoid $\mathrm{Mor}(L\mathbb{G}_n)$ into two relevant pieces. The first is a subgroup $(s \times t)(L\mathbb{G}_n)$, which encodes all of the ordered pairs of objects that appear as the source and target data of at least one morphism. The fact that there is such a subgroup --- that we can choose a representative morphism for each source/target pair in a way which respects the tensor product of $L\mathbb{G}_n$ --- is a consequence of the way that the morphisms of the free algebra $\mathbb{G}_n$ are structured. Specifically, the source and target monoid $(s \times t)(\mathbb{G}_n)$ is free, which lets us easily construct an inclusion $(s \times t)(\mathbb{G}_n) \to \mathrm{Mor}(L\mathbb{G}_n)$, whose image under the coequaliser $q$ then forms the required inclusion for $(s \times t)(L\mathbb{G}_n)$. By comparison, the second subgroup that we need is much simpler, as it is just the homset of endomorphisms of the unit object, $L\mathbb{G}_n(I, I)$. Together, these two subgroups  shape the whole of $\mathrm{Mor}(L\mathbb{G}_n)$, in the sense that the latter is a semidirect product of the former. Moreover, under certain circumstances which will include all of the motivating examples for this research, this semidirect product is actually direct. This will allow us to easily perform abelianisations, group completions, and repeated quotients of $\mathrm{Mor}(L\mathbb{G}_n)$ until we arrive at the same the collapsed $\mathrm{M}(L\mathbb{G}_n)^{\mathrm{gp}, \mathrm{ab}}$ we had before, after which we will have successfully described a path from the morphisms of the free algebra $\mathbb{G}_{2n}$ to those of the invertible $L\mathbb{G}_n$. The rest of the chapter will then be concerned with simplifying this description, by carrying out some calculations that do not change for different instances of $L\mathbb{G}_n$. This will include an investigation into the way that action operads and the monoids we've built out of them will act under group completion and abelianisation.

Finally, in \cref{mainthm} we will compile all of the major results of the previous chapters into a single account of the free $\mathrm{E}G$-algebra on $n$ invertible objects. The only piece of data still missing at this stage will be the action $\alpha$, but a method for recovering it will have already been established back in \cref{colimalgebra}, so this will not present any further challenges. \cref{freeinvalgG1,freeinvalgc} are the focal point of the thesis, providing a step-by-step construction of the algebra $L\mathbb{G}_n$ for all values of $n \in \mathbb{N}$ and all action operads $G$. The remainder of the paper will then consist of applications of these theorems to specific examples of free $G$-monoidal categories on invertible objects --- the symmetric, the braided, and the ribbon braided.  




















